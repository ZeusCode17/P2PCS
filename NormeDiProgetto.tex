\documentclass[a4paper]{article}
% packages 
\usepackage[T1]{fontenc}
\usepackage[utf8]{inputenc}
\usepackage[english,italian]{babel}
\usepackage{microtype}
\usepackage{booktabs}
\usepackage{hyperref}
\usepackage{csquotes}
\usepackage{fancyhdr}
\usepackage{graphicx}
\setcounter{section}{0}


\hypersetup{
    linkbordercolor={0 1 1}
}


% new commands
\newcommand{\GroupName} {Gruppo ZeusCode }
\newcommand{\GroupEmail}{\href{mailto:zeuscode17@gmail.com}{zeuscode17@gmail.com }}
\newcommand{\ProjectName} {Progetto GaiaGo }
\newcommand{\ProjectAim} {CarSharing Peer to Peer }
\newcommand{\ProjectVersion} {v 1.0.0}

%fine inclusioni
%\graphicspath{ {} }
\title{Norme di Progetto}
\author{\GroupName}
\date{2019}

\begin{document}

%\begin{figure}
	%\centering
%	\includegraphics[width=90mm]{logo.jpg}
%\end{figure}

\maketitle
\newpage
\tableofcontents
\newpage
\section {Introduzione}
\subsection {Scopo del documento}
Questo documento  ha lo scopo di definire le regole di base che tutti i membri di \GroupName devono rispettare nello svolgimento del progetto, così da garantire uniformità in tutto il materiale. Verrà utilizzato un approccio incrementale, volto a normare passo passo ogni decisione descussa e concordata tra tutti i membri del gruppo. Ciascun componente è obbligato a prendere visione di tale documento e a rispettare le norme in esso descritte allo scopo di perseguire la coesione all'interno del team.
\subsection {Scopo del prodotto}
Il capitolato C5 ha per obiettivo l'arricchimento delle funzionalità dell'app GaiaGo già esistente, inserendo un nuovo servizio di Car Sharing Peer to Peer.
Il servizio, dunque, intende offrire la possibilità di condividere la propria macchina con altre persone amiche o meno.
\subsection {Glosssario}
All’interno del documento sono presenti termini che presentano significati ambigui a seconda del contesto.  Per evitare questa ambiguità è stato creato un documento di nome  Glossario  che  conterrà  tali  termini  con  il  loro  significato  specifico.   Per
segnalare  che  un  termine  del  testo  è  presente  all’interno  del
Glossario  v  1.0.0 verrà aggiunta una G a pedice a fianco del termine.
\subsection {Riferimenti}
\subsubsection {Riferimenti normativi}

\begin{itemize}
	\item  \textbf{Standard ISO/IEC 12207:1995:}\href{https://www.math.unipd.it/~tullio/IS-1/2009/Approfondimenti/ISO_12207-1995.pdf}{https://www.math.unipd.it/~tullio/IS-1/2009/Approfondimenti/ISO_12207-1995.pdf};
	\item \textbf{Capitolato d'appalto C5 - Piattaforma peer-to-peer car sharing:} \href{https://www.math.unipd.it/~tullio/IS- 1/2018/Progetto/C6.pdf}{https://www.math.unipd.it/~tullio/IS-1/2018/Progetto/C6.pdf}
\end{itemize}

\subsubsection {Riferimenti informativi}
\begin{itemize}
	\item \textbf{Piano di Progetto:}
	\item \textbf{Piano di Qualifica:}
	\item \textbf{Software Engineering - Ian Sommerville - 10th Edition: \\}(formato cartaceo);
\end{itemize}
\section {Processi primari}
	\subsection {Fornitura}
	In questa sezione vengono trattate le norme che i membri del \GroupName sono tenuti a rispettare al fine di proporsi e diventare fornitori nei confronti della Proponente GaiaGo e dei committenti Prof. Tullio Vardanega e Prof. Riccardo Cardin nell'ambito della progettazione, sviluppo e consegna del prodotto P2PCS.
		\subsubsection {Studio di fattibilità}
		In seguito alla formazione dei gruppi del secondo lotto avvenuta Venerdì 1 marzo 2019 presso l'aula 1C150 di Torre Archimede è stata convocata una riunione interna al gruppo per discutere sui vari Capitolati d'appalto. \\ Una volta stabilita la scelta del capitolato per il quale proporsi come fornitori, gli analisti hanno condotto un'ulteriore e approfondita attività di analisi dei rischi e delle oppurtunità culminata con la redazione del documento \textit{Studio di fattibilità v. 1.0.0}. Tale documento include le motivazioni che hanno portato il \GroupName a proporsi come fornitore per il prodotto indicato e riporta per ciascun capitolato: TODO
	\subsection {Sviluppo}
		\subsubsection {Analisi dei requisiti}
		\subsubsection {Progettazione}
		\subsubsection {Codifica}
		\subsubsection {Procedure}
		\subsubsection {Strumenti}
\section {Processi di Supporto}
\section {Processi Organizzativi}






\end{document}