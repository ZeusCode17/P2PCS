\documentclass[a4paper]{article}
% packages 
\usepackage[T1]{fontenc}
\usepackage[utf8]{inputenc}
\usepackage[english,italian]{babel}
\usepackage{microtype}
\usepackage{booktabs}
\usepackage{hyperref}
\usepackage{csquotes}
\usepackage{fancyhdr}
\usepackage{graphicx}
\setcounter{section}{0}


\hypersetup{
    linkbordercolor={0 1 1}
}


% new commands
\newcommand{\GroupName} {Gruppo Zeus Code}
\newcommand{\GroupEmail}{\href{mailto:zeuscode17@gmail.com}{zeuscode17@gmail.com}}
\newcommand{\ProjectName} {Progetto GaiaGo}
\newcommand{\ProjectAim} {CarSharing Peer to Peer}
\newcommand{\ProjectVersion} {v 1.0.0}

%fine inclusioni
%\graphicspath{ {} }
\title{Norme di Progetto}
\author{\GroupName}
\date{2019}

\begin{document}

\begin{figure}
	\centering
	\includegraphics[width=90mm]{logo.jpg}
\end{figure}

\maketitle
\newpage
\tableofcontents
\newpage
\section{Introduzione}
\subsection{Scopo del documento}
Questo documento  ha lo scopo di definire le regole di base che tutti i membri di \GroupName devono rispettare nello svolgimento del progetto, così da garantire uniformità in tutto il materiale. Verrè utilizzato un approccio incrementale, volto a normare passo passo ogni decisione descussa e concordare tra tutti i membri del gruppo. Ciascun componente è obbligato a prendere visione di tale documento e a rispettare le norme in esso descritte allo scopo di perseguire la coesione all'interno del team.
\subsection{Scopo del prodotto}
Il capitolato C5 ha per obiettivo l'arricchimento delle funzionalità dell'app GaiaGo già esistente, inserendo un nuovo servizio di Car Sharing Peer to Peer.
Il servizio, dunque, intende offrire la possibilità di condividere la propria macchina con altre persone amiche o meno.
\subsection{Glosssario}
All’interno del documento sono presenti termini che presentano significati ambigui a seconda del contesto.  Per evitare questa ambiguit`a `e stato creato un documento di  nome  Glossario  che  conterr`a  tali  termini  con  il  loro  significati  specifico.   Per
segnalare  che  un  termine  del  testo  `e  presente  all’interno  del
Glossario  v  1.0.0 verr`a aggiunta una G a pedice a fianco del termine.
\subsection{Riferimenti}
\subsubsection{Riferimenti normativi}

\begin{itemize}
	\item  \textbf{Standard ISO/IEC 12207:1995: \\}\href{https://www.math.unipd.it/~tullio/IS-1/2009/Approfondimenti/ISO_12207-1995.
pdf}{https://www.math.unipd.it/~tullio/IS-1/2009/Approfondimenti/ISO_12207-1995.pdf;}
	\item \textbf{Capitolato d'appalto C5 - Piattaforma peer-to-peer car sharing:  \\}\href{https://www.math.unipd.it/~tullio/IS- 1/2018/Progetto/C6.pdf}{https://www.math.unipd.it/~tullio/IS-1/2018/Progetto/C6.pdf
;}
\end{itemize}
\subsubsection{Riferimenti informativi}
\begin{itemize}
	\item \textbf{Piano di Progetto:}
	\item \textbf{Piano di Qualifica:}
	\item \textbf{Software Engineering - Ian Sommerville - 10th Edition: \\}(formato cartaceo);
\end{itemize}
\section{Processi primari}
\section{Processi di Supporto}
\section{Processi Organizzativi}






\end{document}