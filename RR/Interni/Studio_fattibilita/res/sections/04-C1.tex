\subsection{Capitolato C1 - Butterfly}
\subsubsection{Informazioni generali}
\begin{itemize}
\item
\textbf{Nome}: Butterfly: monitor per processi CI/CD;
\item
\textbf{Proponente}: Imola Informatica;
\item
\textbf{Committente}: Prof. Tullio Vardanega e Prof. Riccardo Cardin.
\end{itemize}
\subsubsection{Descrizione}
Il progetto Butterfly propone lo sviluppo di una piattaforma di notifica che raccolga le segnalazioni provenienti dai vari applicativi utilizzati dall’azienda e le riporti nella forma desiderata dall’ utilizzatore finale.

\subsubsection{Finalità del progetto}
Il prodotto finale utilizza un pattern Producer-Consumer che raccoglie le varie segnalzioni mandate dalle applicazioni e le indirizza attraverso i canali scelti dall’utilizzatore.L’azienda propone una soluzione a quattro componenti, così strutturate:
\begin{itemize}
	\item \textbf{Producers}: raccolgono le segnalazioni provenienti dalle varie applicazioni e le pubblicano, sotto forma di messaggio, all’interno dello specifico topic;
	\item \textbf{Broker}: strumento che istanzia e gestisce i topic;
	\item \textbf{Consumers}: componenti che hanno il compito di abbonarsi ai topic adeguati, recuperarne i messaggi ed inviarli verso i destinatari finali. I componenti richiesti hanno come finalità l’invio di segnalazioni attraverso Telegram, Slack e Email;
    \item \textbf{Componente custom specifico}: funzione che permette, attraverso dei metadati relativi agli utenti, di inviare le informazioni solo a chi interessato.
\end{itemize}
\subsubsection{Tecnologie interessate}
Per lo sviluppo dei componenti applicativi, l'azienda proponente consiglia:
\begin{itemize}
	\item \textbf{Java}, \textbf{Python}\glo, \textbf{Node.js}\glo: alternative di linguaggi per lo sviluppo dell'applicativo suggerite dal proponente; 
	\item\textbf{Apache Kafka}\glo : software open-source per la gestione delle operazioni tra i vari client, da utilizzare come Broker;
	\item \textbf{Docker}\glo: per creare i container relativi alle diverse componenti;
	\item \textbf{API\glosp Redmine}\glo,  \textbf{GitLab}\glo,  \textbf{SonarQube}\glo,  \textbf{Telegram}\glo, \textbf{Slack}\glo: utilizzate per potersi interfacciarsi con omonime applicazioni.
\end{itemize}
Ulteriori richieste del proponente:
\begin{itemize}
	\item Rispettare i 12 fattori esposti dal documento “The Twelve-Factor App”;
	\item Fornire API REST per tutte le componenti utilizzate; 
	\item Utilizzo di test unitari e d’integrazione, test di sistema sull’intero sistema.
\end{itemize}
\subsubsection{Aspetti positivi}
\begin{itemize}
	\item Le tecnologie proposte hanno larga diffusione nel mondo lavorativo ed
	 approfondire la conoscenza su di esse è un aspetto apprezzato dal gruppo;
	\item Java è materia di studio nel nostro corso di laurea, per cui il
	 capitolato\glosp offre la possibilità di migliorare la padronanza di questo
	 linguaggio.
\end{itemize}

\subsubsection{Criticità e fattori di rischio}
\begin{itemize}
	\item Lo sviluppo del componente Producer permetterebbe solamente l'apprendimento di aspetti marginali delle tecnologie coinvolte;
	\item Il lavoro per la raccolta dati appare ripetitivo 
	 e le API\glosp da utilizzare sembrano altamente specifiche per il progetto. Probabilmente queste conoscenze acquisite saranno poco spendibili nel futuro, specie se comparate alle offerte di altri capitolati;
	\item L'interesse da parte del gruppo di lavoro per questo capitolato\glosp si è dimostrato scarso.
\end{itemize}

\subsubsection{Conclusioni}
Lo scopo del capitolato\glosp non è risultato molto stimolante, in quanto lo
sviluppo di alcune componenti sembra caratterizzato da attività ripetitive.
Inoltre, il dover apprendere tecnologie per le quali è richiesta solamente 
l'integrazione di un sottoinsieme di funzionalità, ha demotivato il gruppo nella scelta di questo progetto.
