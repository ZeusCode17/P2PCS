\section{Verbale della riunione}
\begin{itemize}
	\item Scelta del capitolato\glosp C5 in base all'esposizione raccolta, da parte di ogni componente, 
		  degli aspetti positivi e negativi dei progetti candidati;
	\item Scelta del nome del gruppo tramite composizione di una lista di alternative;
	\item Scelta del logo del gruppo;
	\item Confronto tra i componenti del gruppo per capire gli strumenti necessari allo svolgimento del progetto, 
		  sia per la parte organizzativa sia per quella implementativa;
	\item Elenco di tali strumenti:
		\begin{itemize}
			\item \textbf{Git}: come strumento di controllo del versionamento;
			\item \textbf{GitHub}\glo: come piattaforma di hosting\glosp web e come spazio per la condivisione di materiale tramite repository\glo;
			\item \textbf{Telegram}\glo: come mezzo per gestire la comunicazione e condivisione di materiale consultabile tra i componenti
			del gruppo dove abbiamo concordato quando incontrarsi per affrontare alcuni punti iniziali quali:
			\begin{itemize}
				\item \textbf{Studio di Fattibilità};
				\item \textbf{Norme di Progetto};
				\item \textbf{Incontro con l'azienda}.
			\end{itemize}
			\item \textbf{TexStudio}: come applicazione per leggere e scrivere file in latex per lo svolgimento della documentazione richiesta.
		\end{itemize} 
\end{itemize} 
\pagebreak
\section{Riepilogo delle decisioni}

	%\renewcommand{\arraystretch}{1.5}
	\rowcolors{2}{pari}{dispari}
	
	\begin{longtable}{ >{\centering}p{0.20\textwidth} >{}p{0.70\textwidth}}
		\caption{Decisioni della riunione interna del 2019-03-05}\\	
		\rowcolorhead
		\textbf{\color{white}Codice} 
		& \centering\textbf{\color{white}Decisione} 
		\tabularnewline 
		\endfirsthead
		VI\_1.1 & Scelto il capitolato\glosp C5.
		
		\tabularnewline 
		VI\_1.2 & Scelto \textit{Zeus Code} come nome del gruppo.
		
		\tabularnewline 
		VI\_1.3 & Scelto il logo per rappresentare il gruppo.
	
		\tabularnewline 
		VI\_1.4 & Scelto Git come software di controllo di versione.
		
		\tabularnewline 
		VI\_1.5 & Scelto GitHub\glosp come servizio di hosting\glosp per il progetto.
		
		\tabularnewline 
		VI\_1.6 & Scelto Telegram\glosp come canale comunicativo.
	
		\tabularnewline 
		VI\_1.7 & Scelto TexStudio come strumento di scrittura e lettura dei file latex per la documentazione.
	
	\end{longtable}
	




