\section{Processi primari}
\subsection{Fornitura}
\subsubsection{Scopo}
Il processo\glosp di fornitura determina le procedure e le risorse necessarie allo svolgimento del progetto. Una volta comprese le richieste del proponente, \textit{GaiaGo}, e aver stilato uno \textit{Studio di Fattibilità}, si potrà procedere con il fine di soddisfare ognuna di queste richieste. Sarà necessario concordare con il proponente un contratto per la consegna del prodotto. Verrà sviluppato un \textit{Piano di Progetto} da seguire fino alla consegna del materiale prodotto.
	Il processo di fornitura è composto dalle seguenti fasi:
	\begin{itemize}
		\item avvio;
		\item messa a punto di risposte alle richieste;
		\item contrattazione;
		\item pianificazione;
		\item esecuzione e controllo;
		\item revisione e valutazione;
		\item consegna e completamento.
	\end{itemize}
	\subsubsection{Aspettative}
	Al fine di avere un rapporto costante ed efficacie con il proponente, il gruppo si impegna a:
	\begin{itemize}
		\item determinare aspetti chiave per far fronte ai bisogni del proponente;
		\item stilare requisiti e vincoli sui processi;
		\item stimare le tempistiche di lavoro;
		\item stimare i costi;
		\item promuovere una verifica continua;		
		\item accordarsi sulla qualifica del prodotto.
	\end{itemize}
	\subsubsection{Gestione della Qualità}
	Il fornitore si impegna ad utilizzare strategie di verifica e validazione con l'obiettivo di garantire efficacia e qualità nei processi e nei prodotti. I progettisti hanno a carico la scelta di queste strategie e le descrivono nel \textit{Piano di qualifica} v1.0.0. I verificatori invece si occupano di documentare l'esito delle verifiche e prove effettuate seguendo quanto deciso nel \textit{Piano di qualifica} v1.0.0.
	\subsubsection{Descrizione}
	Questa sezione tratta le norme che i membri del gruppo \textit{Zeus Code} devono rispettare in tutte le fasi di progettazione, sviluppo e consegna del prodotto \textit{P2PCS}, al fine di diventare fornitori nei confronti del proponente \textit{GaiaGo} e dei committenti Prof. Tullio Vardanega e Prof. Riccardo Cardin.
	\subsubsection{Attività}
	Per attuare un'implementazione soddisfacente del processo di fornitura si devono eseguire le seguenti attività:
\begin{itemize}
\item \textbf{identificazione delle opportunità} : il fornitore individua una proponente che rappresenti un'organizzazione avente necessità di un prodotto o servizio;
\item \textbf{contrattazione tra fornitore e acquirente} : il fornitore deve:
\begin{itemize}
\item condurre un'analisi dei requisiti nella richiesta del proponente tenendo conto delle politiche organizzative e altri regolamenti;
\item produrre una risposta alle richieste dell'acquirente;
\item preparare una risposta alla richiesta di proposta.
\end{itemize}
Il fornitore dovrebbe condurre una revisione dei requisiti nella richiesta di proposta tenendo conto delle politiche organizzative e di altri regolamenti.
\item \textbf{accordo contrattuale} : il fornitore contratta con il proponente, richiedendo eventuali modifiche al contratto facenti parte del meccanismo di controllo, al fine di trovare un accordo al fine di fornire un prodotto o un servizio software.
\item \textbf{esecuzione del contratto} : il fornitore deve:
\begin{itemize}
\item revisionare i requisiti di acquisizione per definire un framework per l'organizzazione e per assicurare la qualità del software o del prodotto finito.
\item definire o scegliere un modello di ciclo di vita adatto allo scopo e alla complessità del progetto. Questo modello è composto da diverse fasi, delle quali va definito l'obiettivo e il prodotto.
\item stabilire i requisiti dei piani per organizzare e assicurare il progetto. Questi devono includere le risorse necessarie e il coinvolgimento dell'acquirente.
\item una volta stabiliti i requisiti di pianificazione, considerare le diverse opzioni per lo sviluppo del prodotto o servizio software a fronte di un'analisi dei rischi associata ad ogni opzione. Le opzioni includono:
\begin{enumerate}
\item sviluppo del software utilizzando risorse interne;
\item sviluppo del software appoggiandosi a terzi;
\item ottenimento di un software standard da risorse interne esterne;
\item combinazione delle opzioni precedenti.
\end{enumerate}
\item sviluppare e documentare il piano di progetto basandosi sui requisiti di pianificazione e sulle opzioni scelte. Alcuni degli aspetti da considerare sono:
\begin{enumerate}
\item la struttura organizzativa del progetto e autorità e responsabilità di ogni unità organizzativa, organizzazioni esterne incluse;
\item ambiente di ingegnerizzazione la quale include ambiente di testing, librerie, dotazione, strutture , standard, procedure e strumenti;
\item la struttura funzionale del processo e attività, inclusi i prodotti software, servizi software e gli oggetti non consegnabili, tenendo conto di budget, personale, risorse fisiche, dimensione del software e tabelle di marcia associate ai task;
\item organizzazione della caratteristiche di qualità del prodotto software.
\item organizzazione della sicurezza e di altri requisiti fondamentali del software;
\item gestione di aziende terze, compreso di selezione e interazioni dei terzi con l'acquirente;
\item controllo della qualità;
\item verifica e validazione, incluso l'avvicinamento con l'agente di verifica e validazione, se specificato;
\item coinvolgimento dell'acquirente in forma di revisioni, verifiche, incontri informali, segnalazioni, modifiche e cambiamenti; implementazione, approvazione, accettazione e accesso alle strutture;
\item coinvolgimento dell'utente, in termini di esercizi sul settaggio dei requisiti, dimostrazioni del prototipo e valutazioni;
\item gestione dei rischi, cioè la gestione delle aree del progetto che includono potenziali rischi tecnici, di costo o di tempo;
\item politica di sicurezza, cioè le regole sulle informazioni necessarie sull'accesso ad esse, su tutti i livelli di organizzazione del progetto;
\item gestione delle certificazioni richieste come regolamenti, certificazioni richieste, proprietario, utilizzo, proprietà, garanzia e diritti di licenza;
\item mezzi per l'organizzazione del tempo, del tracciamento e delle segnalazioni;
\item formazione del personale.     
\end{enumerate}
\item implementare e eseguire il piano di progetto.
\item monitorare e controllare il progresso e la qualità dei prodotti software o servizi del progetto attraverso il ciclo di vita contrattato. Questo dovrebbe essere una attività iterativa costante la quale garantisce per:
\begin{enumerate}
\item monitoraggio dei progressi dal punto di vista tecnico, dei costi, delle scadenze e sulle segnalazioni dello stato di avanzamento del progetto
\item identificazione, tracciamento, analisi e risoluzione dei problemi.
\end{enumerate}
\item gestire e controllare le aziende terze seguendo quanto predisposto nel processo di acquisizione. Il fornitore seguire tutti i requisiti di contratto necessari a garantire la qualità del software prodotto in accordo con quanto specificato nel contratto e nel piano di progetto.
\item eseguire verifica e validazione per dimostrare che il prodotti software e i processi soddisfino pienamente i rispettivi requisiti.
\item rendere disponibili all'acquirente i report di valutazione, le revisioni, le verifiche, i test e la risoluzione ai problemi come specificato da contratto.
\item garantire l'accesso dell'acquirente al fornitore o alle parti terze coinvolte per la revisione dei prodotti software come specificato da contratto e nel piano di progetto.
\item eseguire attività di garanzia della qualità.
\end{itemize}
\item \textbf{consegna e supporto del prodotto/servizio} : il fornitore deve consegnare il prodotto o servizio software e offrire assistenza su di esso (tra cui la possibile installazione del prodotto) come specificato nel contratto.
\item \textbf{conclusione della fornitura} : il fornitore deve accettare e conoscere il pagamento o altre considerazioni accettate; inoltre trasferisce ogni responsabilità sul prodotto o servizio all'acquirente, come previsto da contratto.
\end{itemize}
	\begin{comment}
		\paragraph{Studio di Fattibilità} \mbox{}\\ \mbox{}\\
		Il responsabile di progetto ha l'onere di organizzare riunioni tra i membri del gruppo per discutere sui capitolati proposti.
		Lo \textit{Studio di Fattibilità}, redatto dagli analisti, indica per ogni capitolato\glo:
		\begin{itemize}
			\item \textbf{Informazioni generali}: vengono elencate le informazioni di base, come il nome del progetto, il proponente e il committente;
			\item \textbf{Descrizione e finalità del progetto}: viene fatta una presentazione del capitolato in generale, una descrizione delle caratteristiche principali richieste per il prodotto e viene definito l'obiettivo che si vuole raggiungere;
			\item \textbf{Tecnologie interessate}: viene fatto un elenco delle tecnologie richieste per lo svolgimento, che rientrano nel dominio tecnologico;
			\item \textbf{Aspetti positivi, criticità e fattori di rischio}: vengono esposte le considerazioni fatte dal gruppo sugli aspetti positivi e sui fattori di rischio del capitolato;
			\item \textbf{Conclusioni}: vengono esposte le ragioni per la quale il gruppo ha deciso di accettare o scartare il capitolato.
		\end{itemize}
		\paragraph{Piano di Progetto} \mbox{}\\ \mbox{}\\
		Il responsabile, con l'aiuto degli amministratori, redige un \textit{Piano di Progetto} da seguire durante il corso del progetto. Questo documento contiene:
		\begin{itemize}
			\item \textbf{Analisi dei rischi}: vengono analizzati nel dettaglio i rischi che potranno presentarsi e vengono esposte le misure e le modalità attraverso le quali i rischi vengono contenuti o mitigati. Viene anche fornita la probabilità con la quale questi possono presentarsi e il livello di gravità per ciascuno;
			\item \textbf{Modello di sviluppo}: viene descritto il modello di sviluppo\glosp che è stato scelto, indispensabile per la pianificazione;
			\item \textbf{Pianificazione}: vengono pianificate le attività da eseguire nelle diverse fasi del progetto e vengono stabilite le loro scadenze temporali;
			\item \textbf{Preventivo e consuntivo}: viene data una stima di lavoro necessaria per ciascuna fase proponendo così un preventivo per il costo totale del progetto. Viene anche tracciato, un consuntivo di periodo relativo all'andamento rispetto a ciò che è stato preventivato.
		\end{itemize}
		\paragraph{Piano di Qualifica} \mbox{}\\ \mbox{}\\
		I verificatori dovranno redigere un documento, detto \textit{Piano di Qualifica} contenente le strategie da adottare per garantire la qualità del materiale prodotto dal gruppo, e dei processi attuati. Il piano è così suddiviso:
		\begin{itemize}
			\item \textbf{Qualità di processo}: vengono identificati dei processi dagli standard, stabiliti degli obiettivi, escogitate delle strategie per attuarli e individuate le metriche per misurarli e controllarli;
			\item \textbf{Qualità di prodotto}: vengono identificati gli attributi più rilevanti per il prodotto, definiti degli obiettivi per raggiungerli e delle metriche per misurarli;
			\item \textbf{Specifiche dei test}: definiscono una serie di test attraverso i quali il prodotto passa per garantire che soddisfi i requisiti;
			\item \textbf{Standard di qualità}: vengono esposti gli standard di qualità scelti;
			\item \textbf{Valutazioni per il miglioramento:} vengono riportati i problemi e le relative soluzioni nel ricoprire un determinato ruolo e nell'uso degli strumenti scelti;
			\item \textbf{Resoconto delle attività di verifica:} per ogni attività si riportano i risultati delle metriche calcolate in forma di resoconto.
		\end{itemize}
		\end{comment}
		\subsubsection{Strumenti}
		Di seguito sono elencati gli strumenti utilizzati durante il processo di fornitura.
		\paragraph{Microsoft Excel} \mbox{}\\ \mbox{}\\
		Software della suite Microsoft Office per realizzare fogli elettronici. Usato per fare calcoli, produrre diagrammi, istogrammi e areogrammi, creare tabelle e grafici.
		\paragraph{Gantt Project} \mbox{}\\ \mbox{}\\
		Per assistere i responsabili di progetto nella pianificazione, nell'assegnazione delle risorse, nella verifica del rispetto dei tempi, nella gestione dei budget e nell'analisi dei carichi di lavoro attraverso la creazioni di diagrammi di Gantt\glo. \\
		\centerline{\url{https://www.ganttproject.biz/}}
		\pagebreak
		\begin{comment}
		\textbf{(questa ultima sezione è da inserire nella fase successiva)}
		\subsubsection{Collaudo e consegna del prodotto}
		Al fine di consegnare il prodotto terminato il gruppo deve effettuare un collaudo in presenza del proponente e dei committenti. Precedentemente a questo test il gruppo deve assicurare correttezza, completezza e affidabilità per ogni parte del materiale consegnato, permettendo così che tutti i requisiti obbligatori siano soddisfatti e l'esecuzione dei test abbiano un esito positivo. In seguito al collaudo finale il responsabile di progetto consegna il prodotto su un supporto fisico.
		\end{comment}
     
\subsection{Sviluppo}
	\subsubsection{Scopo}
	Il processo\glosp contiene le attività e i compiti da svolgere, al fine di realizzare il prodotto finale richiesto dal proponente.
	\subsubsection{Aspettative}
	Le aspettative sono le seguenti:
	\begin{itemize}
		\item fissare gli obiettivi di sviluppo;
		\item fissare i vincoli tecnologici;
		\item fissare i vincoli di design;
		\item realizzare un prodotto finale che superi i test, che soddisfi i requisiti e le richieste del proponente.
	\end{itemize}
	\subsubsection{Descrizione}
	Il processo di sviluppo si articola in:
	\begin{itemize}
		\item \textit{Analisi dei Requisiti};
		\item Progettazione;
		\item Codifica.
	\end{itemize}
	\subsubsection{Attività}
		\paragraph{Analisi dei Requisiti} \mbox{}\\ \mbox{}\\
			\textbf{Scopo} \newline \newline
			Gli analisti hanno il compito di redigere il documento di
			\textit{Analisi dei Requisiti} che individua ed elenca dunque, i requisiti.
			Lo scopo dei requisiti è quello di:
			\begin{itemize}
				\item definire lo scopo del lavoro;
				\item fornire ai progettisti riferimenti precisi ed affidabili;
				\item fissare le funzionalità e i requisiti concordati col cliente;
				\item fornire  una  base  per  raffinamenti  successivi  al  fine  di  garantire  un miglioramento continuo del prodotto e del processo di sviluppo;
				\item fornire ai verificatori riferimenti per l'attività di controllo dei test;
				\item calcolare la mole di lavoro per tracciare dei riferimenti per una stima dei costi.
			\end{itemize}
			\textbf{Aspettative} \newline \newline
			Obiettivo dell'attività è la creazione della documentazione formale contenente tutti i
			requisiti richiesti dal proponente. \newline \newline
			\textbf{Descrizione} \newline \newline
			I requisiti si raccolgono secondo modalità predefinite:
			\begin{itemize}
				\item lettura del capitolato\glo, analisi e approfondimento dello stesso;
				\item confronto con il proponente;
				\item confronto tra membri del team di progetto;
				\item analisi di uno o più casi d'uso.  \\
			\end{itemize}
			\noindent
			\textbf{Casi d'uso} \newline \newline
			Rappresenta un diagramma che esprime un comportamento,
			offerto o desiderato, sulla base di risultati osservabili.
			La struttura dei casi d'uso è così suddivisa:
			\begin{itemize}
				\item codice identificativo;
				\item titolo;
				\item diagramma UML\glo;
				\item attori primari;
				\item attori secondari;
				\item descrizione;
				\item scenario principale;
				\item scenario alternativo (se presente);
				\item inclusioni(se presenti);
				\item estensioni(se presenti);
				\item specializzazioni(se presenti);
				\item precondizione;
				\item postcondizione. \\
			\end{itemize}
			\noindent
			\textbf{Codice identificativo dei casi d'uso} \newline \newline
			Il codice di ogni caso d'uso seguirà questo formalismo: \newline \newline
			\centerline{\textbf{UC[codice\_padre].[codice\_figlio]}} \\
			Dove:
			\begin{itemize}
				\item \textbf{codice\_padre}: numero che identifica univocamente i casi d'uso;
				\item \textbf{codice\_figlio}: numero progressivo che identifica i sottocasi. Può a sua volta includere altri livelli. \\
			\end{itemize}
			%\pagebreak
			%esempio di caso d'uso?(immagine)
			\noindent
			\textbf{Requisiti} \newline \newline
			Ogni requisito è composto dalla seguente struttura:
			\begin{itemize}
				\item \textbf{codice identificativo}: ogni codice identificativo è univoco e conforme alla seguente codifica: \\
				\centerline{\textbf{R[Tipologia][Importanza][Codice]}} \\ \\
				Il significato delle cui voci è:
				\begin{itemize}					
					\item \textbf{Tipologia}: ogni requisito può assumere uno dei seguenti valori:
					\begin{itemize}
						\item \textit{F}: funzionale;
						\item \textit{V}: vincolo.
						\item \textit{Q}: qualitativo;
						\item \textit{P}: prestazionale;		
					\end{itemize}
				\item \textbf{Importanza}: ogni requisito può assumere uno dei seguenti valori:
				\begin{itemize}
					\item \textit{O}: requisito obbligatorio, ovvero irrinunciabile per gli stakeholder\glo;
					\item \textit{D}: requisito desiderabile, ovvero non strettamente necessari ma a valore aggiunto riconoscibile;
					\item \textit{F}: requisito facoltativo, ovvero relativamente utile oppure contrattabile più avanti nel progetto;	
				\end{itemize}
					\item \textbf{Codice}: è un identificatore univoco del requisito in forma gerarchica padre/figlio.
				\end{itemize}
				\item \textbf{classificazione}: viene riportata l'importanza del requisito. Sebbene questa sia un'informazione ridondante ne facilita la lettura;
				\item \textbf{descrizione}: descrizione breve ma completa del requisito, meno ambigua possibile;
				\item \textbf{fonti}: ogni requisito può derivare da una o più tra le seguenti opzioni:
				\begin{itemize}
					\item \textit{capitolato\glo}: si tratta di un requisito individuato dalla lettura del capitolato;
					\item \textit{interno}: si tratta di un requisito che gli analisti hanno ritenuto opportuno aggiungere;
					\item \textit{caso d'uso}: il requisito è estrapolato da uno o più casi d'uso. In questo caso deve essere riportato il codice univoco del caso d'uso;
					\item \textit{verbale}: si tratta di un requisito individuato in seguito ad una richiesta di chiarimento con il proponente. Tali informazioni sono riportate nei verbali in cui ogni requisito individuato è segnato da un codice presente nella tabella dei tracciamenti. \\
				\end{itemize}
			\end{itemize}

			\noindent{\textbf{UML}} \newline \newline
			I diagrammi UML\glosp devono essere realizzati usando la versione del linguaggio v2.0.

		\paragraph{Progettazione} \mbox{}\\ \mbox{}\\
			\textbf{Scopo} \newline \newline
			L'attività di progettazione definisce, in funzione dei requisiti specificati nel documento \textit{Analisi dei Requisiti}, le caratteristiche del prodotto software richiesto. Il compito di questa fase è di definire una soluzione del problema che sia soddisfacente per tutti gli stakeholder. La progettazione segue il procedimento inverso rispetto all'\textit{Analisi dei Requisiti} che divide il problema in parti per capirne completamente il dominio applicativo. La progettazione, infatti, rimette insieme le parti specificando le funzionalità dei sottosistemi in modo da ricondurre ad un'unica possibile soluzione. \newline \newline
			\textbf{Aspettative} \newline \newline
			Il processo\glosp ha come risultato la realizzazione dell’architettura del sistema. \newline \newline
			\textbf{Descrizione} \newline \newline
			Le parti principali sono due:
			\begin{itemize}
				\item \textbf{Technology baseline}\glo: contiene le specifiche della progettazione ad alto livello del prodotto e delle sue componenti, l'elenco dei diagrammi UML\glosp che saranno utilizzati per la realizzazione dell'architettura e i test di verifica;
				\item \textbf{Product baseline}\glo: dettaglia ulteriormente l'attività di progettazione, integrando ciò che è riportato nella Technology baseline. Inoltre definisce i test necessari alla verifica.
			\end{itemize}
			\textbf{Technology baseline} \newline \newline
			Redatta dal progettista, dovrà includere:
			\begin{itemize}
				\item \textbf{Diagrammi UML\glo}:
				\begin{itemize}
					\item diagrammi delle classi;
					\item diagrammi dei package;
					\item diagrammi di attività;
					\item diagrammi di sequenza.
				\end{itemize}
				\item \textbf{Tecnologie utilizzate}: devono essere descritte le tecnologie adottate specificandone l'utilizzo nel progetto, i vantaggi e gli svantaggi;
				\item \textbf{Design pattern}: devono essere descritti i design pattern\glosp utilizzati per realizzare l'architettura. Ogni design pattern deve essere accompagnato da una descrizione ed un diagramma, che ne esponga il significato e la struttura;
				\item \textbf{Tracciamento delle componenti}: ogni requisito deve riferirsi al componente che lo soddisfa;
				\item \textbf{Test di integrazione}: l'unione delle parti, intese come classi di verifica, permette di verificare che ogni componente del sistema funzioni nella maniera voluta.
			\end{itemize}
			\textbf{Product baseline} \newline \newline
			A carico del progettista c'è anche la Product baseline che si sofferma su diversi aspetti tra i quali:
			\begin{itemize}
				\item \textbf{definizione delle classi}: ogni classe deve essere descritta in modo da spiegarne in maniera esaustiva lo scopo e le funzionalità, evitando ridondanze;
				\item \textbf{tracciamento delle classi}: ogni requisito deve essere tracciato in modo da garantire che per ognuno esista una classe che lo soddisfi. Questa operazione è fondamentale per permettere di risalire alle classi ad esso associate;
				\item \textbf{test di unità}: devono essere definiti al fine di verificare che le parti funzionino individualmente nel modo stabilito.
			\end{itemize}
		\paragraph{Codifica} \mbox{}\\ \mbox{}\\
			\textbf{Scopo} \newline \newline
			Questa attività ha come scopo quello di normare l'effettiva realizzazione del prodotto software richiesto. In questa fase si concretizza la soluzione attraverso la programmazione. I programmatori dovranno attenersi a queste norme durante la fase di programmazione ed implementazione. \newline \newline
			\textbf{Aspettative} \newline \newline
			Obiettivo dell'attività è la creazione di un prodotto software conforme alle richieste	prefissate con il proponente.
			L'uso di norme e convenzioni in questa fase, è fondamentale per permettere la generazione di codice leggibile ed uniforme,  agevolare le fasi di manutenzione,  verifica e validazione e migliorare la qualità di prodotto. \newline \newline
			\textbf{Descrizione} \newline \newline
			La scrittura del codice dovrà rispettare quanto stabilito nella documentazione di prodotto. Dovrà perseguire gli obiettivi di qualità definiti all'interno del documento \textit{Piano di Qualifica v1.0.0} per poter garantire una buona qualità del codice. \newline \newline
			\textbf{Stile di codifica} \newline \newline
			Al fine di garantire uniformità nel codice del progetto, ciascun membro del gruppo è
			tenuto a rispettare le seguenti norme:
			\begin{itemize}
				\item \textbf{Indentazione}: i blocchi innestati devono essere correttamente indentati, usando per ciascun livello di indentazione quattro (4) spazi (fanno eccezione i commenti). Al fine di assicurare il rispetto di questa regola si consiglia di configurare adeguatamente il proprio editor o IDE;
				\item \textbf{Parentesizzazione}: è richiesto di inserire le parentesi di delimitazione dei costrutti in linea e non al di sotto di essi;
				\item \textbf{Scrittura dei metodi}: è desiderabile, ove possibile, mantenere i metodi brevi (poche righe di codice);
				\item \textbf{Univocità dei nomi}: classi, metodi, variabili devono avere un nome univoco	ed esplicativo al fine di evitare ambiguità e incomprensione;
				\item \textbf{Classi}: i nomi delle classi devono iniziare sempre con una lettera maiuscola;
				\item \textbf{Costanti}: i nomi delle costanti devono essere scritte usando solo maiuscole;
				\item \textbf{Metodi}: i nomi dei metodi devono iniziare con una lettera minuscola. Nel caso
				siano composti da più parole, quelle successive devono iniziare con una lettera maiuscola (CamelCase\glo{});
				\item \textbf{Lingua}: il codice, come anche i commenti, deve essere scritto in lingua inglese.
			\end{itemize}
			Come supporto alla programmazione del codice scritto in Kotlin\glosp si consiglia di seguire la documentazione fornita da \textit{JetBrains}\glosp e \textit{Android}\glo. \newline \newline
			%\subparagraph{Intestazione} \mbox{}\\
			\textbf{Ricorsione} \newline \newline
			L'uso della ricorsione va evitato quanto più possibile in  quanto  potrebbe
			indurre  ad  una  maggiore  occupazione  di  memoria  rispetto  a  soluzioni
			iterative.
	\subsubsection{Strumenti}
	Di seguito sono elencati gli strumenti utilizzati dal gruppo durante il progetto per il processo di sviluppo.	
		\paragraph{PragmaDB} \mbox{}\\ \mbox{}\\
		Programma usato per il tracciamento dei requisiti, fondamentale dunque per la stesura del documento \textit{Analisi dei Requisiti v1.0.0}. \newline
		\centerline{\url{https://github.com/StefanoMunari/PragmaDB}}
		\paragraph{Draw.io} \mbox{}\\ \mbox{}\\
		Per la produzione di diagrammi UML\glosp viene utilizzato Draw.io in quanto offre molte agevolazioni per la produzione veloce dei diagrammi e risulta semplice da usare. \newline
		\centerline{\url{https://www.draw.io/}}
		\centerline{\url{https://www.jetbrains.com/idea/}}
		\paragraph{Android Studio} \mbox{}\\ \mbox{}\\
		Android Studio è un ambiente di sviluppo integrato per lo sviluppo per la piattaforma Android\glo. Il linguaggio utilizzato in questo ambiente è Kotlin\glo.
		 \newline
		\centerline{\url{https://developer.android.com/studio}}
		\paragraph{Adobe Photoshop} \mbox{}\\ \mbox{}\\
		Adobe Photoshop è un software per l'elaborazione di immagini, sarà utilizzato per progettare la GUI\glo.
		\newline
		\centerline{\url{https://www.adobe.com/it/products/photoshop.html}}
		\paragraph{Amazon Elastic Compute Cloud} \mbox{}\\ \mbox{}\\
		Amazon Elastic Compute Cloud è servizio di cloud computing fornito da Amazon Web Services.
		\newline
		\centerline{\url{https://aws.amazon.com/it/ec2/}}
		\paragraph{Travis CI} \mbox{}\\ \mbox{}\\
		Travis CI è un tool utilizzato per la Continuos Integration\glo. \'E stato scelto perché Open-Source\glosp e gratuito.
		\newline
		\centerline{\url{https://travis-ci.org/}}
		\begin{comment}
			\begin{figure}[H]
			\includegraphics[width=0.99\linewidth]{res/images/""}
			\caption{Software per la codifica}
			\end{figure} 
		\end{comment}
