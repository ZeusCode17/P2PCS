\section{Processi Organizzativi}
	\subsection{Gestione Organizzativa}
		\subsubsection{Scopo}

		Lo scopo di questo processo è definire le linee guida che sono raccolte nel documento \textit{Piano di Progetto}. In particolare:
		\begin{itemize}
			\item definire un modello di sviluppo\glosp comune a tutti i membri del gruppo; 
			\item creare un modello organizzativo volto alla prevenzione e correzzione di errori;
			\item pianificare il lavoro in base alle scadenze;
			\item calcolare il piano economico in base al ruolo coperto;
			\item effettuare il bilancio finale sulle spese.
		\end{itemize}

		\subsubsection{Aspettative}
		Le aspettative del processo sono:
		\begin{itemize}
			\item ottenere una pianificazione efficace delle attività da svolgere;
			\item suddividere per ruoli i membri del gruppo così da poter ricoprire tutte le attività;
			\item garantire un controllo diretto su ogni parte del progetto.
		\end{itemize}
		\subsubsection{Descrizione}
		Viene trattata la gestione dei seguenti argomenti:
		\begin{itemize}
			\item ruoli di progetto;
			\item gestione delle comunicazioni;
			\item pianificazione degli incontri;
			\item gestione degli strumenti di coordinamento;
			\item gestione dei rischi;
		\end{itemize}
		\subsubsection{Ruoli di progetto}%--------start-------------
		Ogni membro del gruppo ricopre un ruolo che viene assegnato a rotazione, le attività svolte da ogni ruolo sono definite in modo chiaro nel documento \textit{Piano di Progetto}.
		I ruoli che ogni componente è tenuto a rappresentare sono descritti in generale di seguito.
			\paragraph{Responsabile di progetto} \mbox{}\\ \mbox{}\\
			Il responsabile di progetto è incaricato di gestire le comunicazioni, fa da referente sia per il committente\glosp che per il fornitore.\newline
			Il responsabile si occupa di:
			\begin{itemize}
				\item pianificazione delle attività di progetto;
				\item  gestione e coordinamento tra membri del team;
				\item studio ed analisi dei rischi;
				\item approvare la documentazione.
			\end{itemize}
			\paragraph{Amministratore di progetto} \mbox{}\\ \mbox{}\\
			L'amministratore coordina l'ambiente di lavoro, assumendosi la responsbilità di gestire la capacità operativa.\newline
			Egli si fa carico dei seguenti aspetti:
			\begin{itemize}
				\item amministra i servizi di supporto, come documentazione e strumenti;
				\item risolvere problemi legati alla gestione dei processi;
				\item effettua controlli volti alla correzzione,verifica,aggiornamento e approvazione della documentazione;
				\item controlla versionamento e configuarazione dei prodotti.
			\end{itemize}
			\paragraph{Analista} \mbox{}\\ \mbox{}\\
			L'analista è la figura incaricata di studiare il problema indicato nel modo più approfondito possibile, così da poter fornire eventuali strumenti e metodologie per affrontarlo. 
			Partecipa per un periodo limitato di tempo, è di grande importanza durante la stesura del documento \textit{Analisi dei requisiti}.\newline
			Le sue responsabilità sono:
			\begin{itemize}
				\item studio del dominio del problema e della sua complessità;
				\item analisi delle richieste implicite ed esplicite;
				\item stesura dei documenti: \textit{Analisi dei Requisiti} e \textit{Studio di Fattibilità}.
			\end{itemize}
			\paragraph{Progettista} \mbox{}\\ \mbox{}\\
			Il progettista ha il compito di trovare una soluzione ai problemi rilevati dall'analista, fornendo aspetti tecnici e tecnologici coerenti.\newline
			Il progettista deve:
			\begin{itemize}
				\item applicare soluzioni note ed ottime.
				\item operare scelte che portino ad una soluzione efficiente rispetto ai requisiti, considerando costi e risorse;
				\item sviluppare l'architettura seguendo un insieme di best practice per ottenere un progetto solido e facilmente mantenibile.
			\end{itemize}
			\paragraph{Programmatore} \mbox{}\\ \mbox{}\\
			Il programmatore è la figura responsabile delle attività di codifica e delle componenti necessarie per effettuare le prove di verifica.
			Il programmatore si occupa di:
			\begin{itemize}
				\item implementare le decisioni del Progettista;
				\item scrivere codice che rispetti le metriche predefinite, sia versionato e documentato;
				\item creare e gestire componenti di supporto per la verifica e validazione del codice.
			\end{itemize}
			\paragraph{Verificatore} \mbox{}\\ \mbox{}\\
			Il verificatore ha il compito di supervisionare il prodotto del lavoro degli altri mebri del team, sia esso codice o documentazione. Segue delle linee guida volte al controllo e correzione di errori presenti nel documento \textit{Norme di progetto}, nonché alla propria esperienza e capacità di giudizio.\\
			Il verificatore deve:
			\begin{itemize}
				\item controllare la conformità del prodotto in ogni suo stadio di vita;
				\item sagnalare al Responsabile di progetto eventuali problemi causati dalla violazione del documento \textit{Norme di progetto}.
				\item segnalare errori meno importanti all'autore dell'oggetto in questione per un eventuale correzzione.
			\end{itemize}
		\subsubsection{Procedure}
		Insieme di linee guida che i membri del gruppo seguiranno durante lo sviluppo del progetto.
			\paragraph{Gestione delle comunicazioni} \mbox{}\\ \mbox{}\\
			\textbf{Comunicazioni interne} \newline \newline
			Le comunicazioni interne sono gestite utilizzando il sistema di messaggistica instantanea Telegram. Questo servizio, tramite l'uso del simbolo \# seguito dal nome dell'argomento affrontato, richiama l'attenzione dei soli membri coinvolti. Vi è la possibilità di integrare dei Bot\glosp, di inviare file di grandi dimensioni e messaggi audio. \newline \newline
			\textbf{Comunicazioni esterne} \mbox{}\\ \mbox{}\\
			Le comunicazioni con soggetti esterni al gruppo,quali commitente e proponente, sono di competenza del responsabile. Gli strumenti predefiniti sono la posta elettronica, utilizzando l'inidrizzo di posta elettronica del gruppo, di cui tutti i membri hanno le credenziali di accesso  \url{zeuscode17@gmail.com}.
			Per le comunicazioni con \textit{Gaiago} si utilizza il servizio Google Meet per le riunioni. In caso di assenza di uno o più membri del gruppo un membro presente, a turno, assume l'onere di creare un riassunto scritto per gli altri membri.
			\newline
			\paragraph{Gestione degli incontri} \mbox{}\\ \mbox{}\\
			\textbf{Incontri interni del team} \newline \newline
			Le riunioni interne del team sono organizzate dal responsabile in accordo con i membri del gruppo. Viene usato Google Calendar se si deve effettuare un incontro importante ed è necessaria la presenza di tutti i membri, in questo modo ogni membro comunica i giorni liberi e si decide. In caso di incontri di minore importanza si utilizza Telegram per organizzarsi con gli orari \newline \newline %adfsdfdsfsdfsdfsdfsdfsdf
			\textbf{Verbali di riunioni interne} \newline \newline
			Ad ogni riunione interna corrisponde un \textit{Verbale}. Questo sarà redatto da un segretario, persona nominata dal responsabile, che dovrà tenere nota delle discussioni fatte e delle decisioni prese. \newline \newline
			\textbf{Incontri esterni del team} \newline \newline
			Il responsabile ha il compito di comunicare e organizzare gli incontri con proponente e committente. Se un membro del gruppo, il proponente o il committente ritiene necessario organizzare un incontro allora il responsabile decide una data, in accordo tra le due parti, e la comunica tramite i canali sopra citati.
			\newline \newline
			\textbf{Verbali di riunioni esterne} \newline \newline
			Come per le riunioni interne, anche per le esterne viene redatto un \textit{Verbale}. La struttura delle due tipologie è analoga, ma le riunioni esterne presentano una maggiore criticità per la presenza di persone esterne al gruppo, quali il committente e il proponente. 
			\paragraph{Gestione degli strumenti di coordinamento} \mbox{}\\ \mbox{}\\
			\textbf{Tickecting} \newline \newline
			Il ticketing consente ai membri di avere chiaro in ogni momento quali attività sono in corso; permette al responsabile di progetto di assegnare compiti ai membri del team e di controllare l'andamento delle attività. Inoltre permette ai membri del team di conoscere e gestire il proprio carico di lavoro.\newline Lo strumento di ticketing scelto è Trello: consiste in una lavagna virtuale online dove sono esposti i task. Ad ogni task sono associati una data di scadenza e un insieme di membri assegnatari. Ogni compito passa attraverso i seguenti stati:
			\begin{itemize}
				\item da fare (to do);
				\item in lavorazione (doing);
				\item in revisione;
				\item completato (done).
			\end{itemize}

			\noindent{La scelta di Trello è stata determinata dalla sua facilità di apprendimento, di controllo, dall'usabilità e dalla trasparenza che fornisce. La gestione dei ticket dev'essere scrupolosa, perché la lavagna di Trello tende ad affollarsi velocemente.}

			\paragraph{Gestione dei rischi} \mbox{}\\ \mbox{}\\
			Il responsabile di progetto ha il compito di rilevare i rischi e di renderli noti, documentando quest'attività nel \textit{Piano di Progetto}. La procedura da seguire per la gestione dei rischi è la seguente:
			\begin{itemize}
				\item individuare nuovi problemi e monitorare i rischi già previsti;
				% PLACEHOLDER: sostituire la frase sottostante, in particolare 'riscontro'
				\item registrare ogni riscontro previsto dei rischi nel \textit{Piano di Progetto};
				\item aggiungere i nuovi rischi individuati nel \textit{Piano di Progetto};
				\item ridefinire, se necessario, le strategie di gestione dei rischi.

			\end{itemize}
			
			\noindent
			\subparagraph{Codifica dei rischi} \mbox{}\\ \mbox{}\\
				Le tipologie di rischi sono così codificate:
				\begin{itemize}
					\item \textbf{RT}: Rischi Tecnologici;
					\item \textbf{RO}: Rischi Organizzativi;
					\item \textbf{RI}: Rischi Interpersonali.
				\end{itemize}

		\subsubsection{Strumenti}
		Il gruppo, nel corso del progetto, ha utilizzato o utilizzerà i seguenti strumenti:
		\begin{itemize}
			\item \textbf{Telegram\glo}: strumento di messaggistica utilizzato inizialmente per la gestione del gruppo;
			\item \textbf{Slack\glo}: per la comunicazione interna del team ed eventuali comunicazioni col proponente;
			\item \textbf{Trello}: per assegnare determinate attività e imporre una scadenza;
			\item \textbf{Git}: sistema di controllo di versionamento;
			\item \textbf{Gitflow, GitKraken}: interfacce per utilizzare Git più comodamente sul proprio desktop;
			\item \textbf{GitHub}: per il versionamento e il salvataggio in remoto di tutti i file riguardanti il progetto.
			\item \textbf{Google Drive}: utilizzato per la stesura di file che sono soggetti a molti cambiamenti e devono essere visibili a tutti nella loro versione più aggiornata, come ad esempio il \textit{Glossario};
			\item \textbf{Google Calendar}: per facilitare il lavoro al responsabile, ogni settimana ciascun membro indica quando non è disponibile, in modo da semplificare l'organizzazione degli incontri;
			\item \textbf{Skype}: servizio che offre possibilità di fare videoconferenze e chiamate VoIP, utilizzato per il primo contatto con il proponente;
			\item \textbf{Hangouts}: servizio che offre la possibilità di fare videoconferenze e chiamate VoIP, utilizzato per parlare con il proponente e per alcuni incontri interni;
			\item \textbf{Sistemi operativi}: i requisiti non indicano la necessità di usare un sistema operativo specifico, verranno quindi utilizzati Windows, Linux e Mac OS dai diversi membri del team. %niente mac perché viva la povertà!
		\end{itemize}
	