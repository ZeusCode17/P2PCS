\section {Introduzione}
\subsection {Scopo del documento}
Questo documento  ha lo scopo di definire le regole di base che tutti i membri di Zeus Code devono rispettare nello svolgimento del progetto, così da garantire uniformità in tutto il materiale. Verrà utilizzato un approccio incrementale, volto a normare passo passo ogni decisione descussa e concordata tra tutti i membri del gruppo. Ciascun componente è obbligato a prendere visione di tale documento e a rispettare le norme in esso descritte allo scopo di perseguire la coesione all'interno del team.
\subsection {Scopo del prodotto}
Il capitolato C5 ha per obiettivo l'arricchimento delle funzionalità dell'app GaiaGo già esistente, inserendo un nuovo servizio di Car Sharing Peer to Peer per la piattaforma Android.
Il servizio, dunque, intende offrire la possibilità di condividere la propria macchina con altre persone amiche o meno sfrutttando almeno 5 core drive del framework Octalysis (G) allo scopo di motivare l'utente a condividere la propria auto con altri.
\subsection {Glosssario}
Al fine di evitare ambiguità e facilitare la comprensione dei documenti formali viene incluso il Glossario v1.0.0, in cui saranno presenti acronimi, abbreviazioni e termini tecnici. Ogni termine presente nel glossario sarà marcato con una G a pedice.
\subsection {Riferimenti}
\subsubsection {Riferimenti normativi}

\begin{itemize}
	\item  \textbf{Standard ISO/IEC 12207:1995:}\href{https://www.math.unipd.it/~tullio/IS-1/2009/Approfondimenti/ISO_12207-1995.pdf}{https://www.math.unipd.it/~tullio/IS-1/2009/Approfondimenti/ISO\textunderscore12207-1995.pdf};
	\item \textbf{Capitolato d'appalto C5 - Piattaforma peer-to-peer car sharing:} \href{https://www.math.unipd.it/~tullio/IS- 1/2018/Progetto/C6.pdf}{https://www.math.unipd.it/~tullio/IS-1/2018/Progetto/C6.pdf}
\end{itemize}

\subsubsection {Riferimenti informativi}
\begin{itemize}
	\item \textbf{Piano di Progetto:}
	\item \textbf{Piano di Qualifica:}
	\item \textbf{Software Engineering - Ian Sommerville - 10th Edition: \\}(formato cartaceo);
	\item \textbf{Octalysis - Gamification framework \\}
	\href{https://yukaichou.com/gamification-examples/octalysis-complete-gamification-framework/}{https://yukaichou.com/gamification-examples/octalysis-complete-gamification-framework/} 
\end{itemize}