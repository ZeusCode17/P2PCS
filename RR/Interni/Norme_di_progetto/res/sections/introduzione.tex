\section {Introduzione}
\subsection {Scopo del documento}
Questo documento  ha lo scopo di definire le regole di base che tutti i membri di \textit{ZeusCode} devono rispettare nello svolgimento del progetto, così da garantire uniformità in tutto il materiale. Verrà utilizzato un approccio incrementale\glo, volto a normare passo passo ogni decisione discussa e concordata tra tutti i membri del gruppo. Ciascun componente è obbligato a prendere visione di tale documento e a rispettare le norme in esso descritte allo scopo di perseguire la coesione\glosp all'interno del team.
\subsection {Scopo del prodotto}
Il capitolato\glosp C5 ha per obiettivo l'arricchimento delle funzionalità dell'app \textit{GaiaGo} già esistente, inserendo un nuovo servizio di \textit{Peer-to-Peer\glosp Car Sharing} per il piattaforma Android.
Il servizio, dunque, intende offrire la possibilità di condividere la propria macchina con altre persone amiche o meno sfruttando almeno 5 core drive del framework\glosp Octalysis\glosp allo scopo di motivare l'utente a condividere la propria auto con altri.
\subsection {Glossario}
Al fine di evitare ambiguità e facilitare la comprensione dei documenti formali viene incluso il Glossario v1.0.0, in cui saranno presenti acronimi, abbreviazioni e termini tecnici. Ogni termine presente nel glossario sarà marcato con una G a pedice.
\subsection {Riferimenti}
\subsubsection {Riferimenti normativi}

\begin{itemize}
	\item  \textbf{Standard ISO/IEC 12207:1995:}\newline
	\url{https://www.math.unipd.it/~tullio/IS-1/2009/Approfondimenti/ISO_12207-1995.pdf};
	\item \textbf{Capitolato\glosp d'appalto C5 - Piattaforma Peer-to-Peer\glosp car sharing:}\newline 
	\url{https://www.math.unipd.it/~tullio/IS-1/2018/Progetto/C5.pdf};
\end{itemize}

\subsubsection {Riferimenti informativi}
\begin{itemize}
	\item \textbf{Piano di Progetto:} \textit{Piano di Progetto v1.0.0}
	\item \textbf{Piano di Qualifica:} \textit{Piano di Qualifica v1.0.0}
	\item \textbf{Software Engineering - Ian Sommerville - 10th Edition: \\}(formato cartaceo);
	\item \textbf{Octalysis\glosp - Gamification\glosp framework\glosp \\}
	\href{https://yukaichou.com/gamification-examples/octalysis-complete-gamification-framework/}{https://yukaichou.com/gamification-examples/octalysis-complete-gamification-framework/}
	\item \textbf{Movens\glosp - Mobility Platform \\}
	\href{http://www.henshingroup.com/overview/}{http://www.henshingroup.com/overview} 
\end{itemize}