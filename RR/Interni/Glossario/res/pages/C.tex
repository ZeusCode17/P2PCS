\subsection*{\quad$C\quad$}
\subsubsection*{CamelCase}
\index{CamelCase}
Pratica di scrivere una parola unendone delle altre lasciando le loro iniziali maiuscole.

\subsubsection*{Capitolato}
\index{Capitolato}
Il capitolato è un documento tecnico, in genere allegato ad un contratto di appalto che intercorre tra il cliente ed un fornitore in cui vengono indicate modalità, costi e tempi di realizzazione dell'oggetto del contratto.

\subsubsection*{Changelog}
\index{Changelog}
Registro delle modifiche di un documento. Deve riportare la release di intervento, l'autore, la data, la descrizione, le motivazioni ed eventualmente la versione.

\subsubsection*{Coesione}
\index{Coesione}
Stretta unione di parti che concorrono alla stessa funzionalità, allo stesso obiettivo: le parti coese sono tutto il necessario e nulla di superfluo.

\subsubsection*{Comportamento emergente}
\index{Comportamento emergente}
Nuovo comportamento del software nato dall'integrazione delle parti, spesso non previsto dagli sviluppatori delle singole parti.

\subsubsection*{Continuous delivery}
\index{Continuous delivery}
Definisce la metodologia attraverso cui i team producono software in cicli brevi, assicurando che il software possa essere rilasciato in modo affidabile in qualsiasi momento e farlo manualmente.

\subsubsection*{Continuous integration}
\index{Continuous integration}
 È una pratica di sviluppo che consiste nell'allineamento frequente dagli ambienti di lavoro degli sviluppatori verso l’ambiente condiviso. Questa pratica permette di minimizzare problemi di integrazione corposa.

\subsubsection*{Copertura}
\index{Copertura}
Nell'analisi dei requisiti, è la quantità di requisiti soddisfatti. Nei test del codice, è la quantità di righe percorse durante il test. Si ricerca copertura massima.

\subsubsection*{Core-drive}
\index{Core-drive}
Nel framework\glosp Octalysis\glosp sono otto i core-drives (unità principali) che motivano la scelta progettuale per la gamification\glo.

