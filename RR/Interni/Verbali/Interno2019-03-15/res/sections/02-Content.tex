\section{Verbale della riunione}
\begin{itemize}
	\item Trattate le problematiche relative alla miglior gestione del lavoro in team e per tenere traccia delle nostre attività, arrivando alla seguente conclusione:
	\begin{itemize}
		\item Slack\glo: come mezzo per gestire la comunicazione, in quanto permette di dividere le discussioni in argomento e a tal proposito si è deciso di creare i canali:
		\begin{itemize}
			\item studiofattibilità: per confrontarsi nella stesura dell'omonimo documento;
			\item normeprogetto: per definire le regole, gli strumenti e le convenzioni che si andranno ad adottare per lo svolgimento del progetto;
			\item general: per comunicare informazioni relative ad altri argomenti;
			\item analisirequisiti: per confrontarsi nella stesura dei casi d'uso in utilizzo per il progetto;
			\item pianoprogetto: per confrontarsi su linee guida, da utilizzare quale riferimento di confronto nel corso dell’esecuzione del progetto, oltre che per misurare e controllare le prestazioni del progetto;
			\item pianoqualifica: per confrontarsi nella stesura dell'omonimo documento;
			\item zeusgithub: per notificarci le attività svolte dai singoli membri su GitHub;
			\item zeustrello: per notificarci le attività svolte dai singoli membri su Trello;
		\end{itemize}  
		\item Trello\glo: come tool di collaborazione strutturato in bacheche condivise e personali, con scadenze e assegnazione dei task, integrabile anche in Slack;
		\item Mega: come spazio per la condivisione di materiale consultabile come guide e paper per approfondire i temi trattati nel capitolato scelto.
	\end{itemize}
	\item Suddivisione della stesura del documento \textit{Analisi dei Requisiti v1.0.0}. 
\end{itemize} 
\pagebreak
\section{Riepilogo delle decisioni}

	%\renewcommand{\arraystretch}{1.5}
	\rowcolors{2}{pari}{dispari}
	
	\begin{longtable}{ >{\centering}p{0.20\textwidth} >{}p{0.70\textwidth}}
		\caption{Decisioni della riunione interna del 2019-03-15}\\	
		\rowcolorhead
		\textbf{\color{white}Codice} 
		& \centering\textbf{\color{white}Decisione} 
		\tabularnewline 
		\endfirsthead
		VI\_1.1 & Scelto Slack\glosp come mezzo per gestire la comunicazione tramite canali appositi per argomento;
		
		\tabularnewline 
		VI\_1.2 & Scelto Trello\glosp come tool di collaborazione strutturato in bacheche condivise e personali;
		
		\tabularnewline 
		VI\_1.3 & Scelto Mega come spazio per la condivisione di materiale consultabile;
	
		\tabularnewline 
		VI\_1.4 & Suddivisione della stesura del documento \textit{Analisi dei Requisiti v1.0.0}.
	
	\end{longtable}
	




