\section{Verbale della riunione}
\begin{itemize}
	\item Conferenza telefonica col proponente del capitolato scelto che ha portato alle seguenti conclusioni:
	\begin{itemize}
		\item Scelta del linguaggio di sviluppo per l'applicazione Android tra Java o Kotlin\glo;
		\item Scelta sull'utilizzo o meno della piattaforma Henshin\glosp attraverso Movens\glo;
		\item Implementazione di una strategia di gamification\glosp attraverso i core-drive\glosp di Octalysis\glo;
		\item Definita la struttura del codice come punto per le norme di progetto;
		\item Rimodellati i requisiti obbligatori ed opzionali;
		\item Definito come implementare il Peer-to-Peer\glosp e come dev'essere l'applicazione finita.   
	\end{itemize}
\end{itemize}
\pagebreak
\section{Riepilogo delle decisioni}

	%\renewcommand{\arraystretch}{1.5}
	\rowcolors{2}{pari}{dispari}
	
	\begin{longtable}{ >{\centering}p{0.20\textwidth} >{}p{0.70\textwidth}}
		\caption{Decisioni della riunione interna del 2019-03-11}\\	
		\rowcolorhead
		\textbf{\color{white}Codice} 
		& \centering\textbf{\color{white}Decisione} 
		\tabularnewline 
		\endfirsthead
		VI\_1.1 & Scelta di Kotlin\glosp come linguaggio di sviluppo per l'applicazione Android;
		
		\tabularnewline 
		VI\_1.2 & Scelta sull'utilizzo della piattaforma Henshin attraverso Movens\glo;
		
		\tabularnewline 
		VI\_1.3 & Scelta di una strategia di gamification\glosp attraverso alcuni dei core-drive\glosp di Octalysis\glo;
	
		\tabularnewline 
		VI\_1.4 & Definita struttura del codice come punto per le norme di progetto;
		
		\tabularnewline 
		VI\_1.5 & Scelta per il Peer-to-Peer\glosp di scambiare l'auto di persona;
		
		\tabularnewline 
		VI\_1.6 & Definita a grandi linee come dev'essere l'applicazione finita.	
	\end{longtable}
	




