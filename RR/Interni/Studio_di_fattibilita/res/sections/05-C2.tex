\subsection{Capitolato C2 - Colletta} 
\subsubsection{Informazioni Generali}
\begin{itemize}
	\item \textbf{Nome}: Colletta: piattaforma raccolta dati di analisi di testo;
	\item \textbf{Proponente}: Mivoq S.r.l.;
	\item \textbf{Commitente}: Tullio Vardanega, Riccardo Cardin.
\end{itemize}
\subsubsection{Descrizione}
L'obiettivo del progetto Colletta è la creazione di una piattaforma collaborativa di raccolta dati in cui gli utenti possono svolgere esercizi grammaticali in diverse lingue. Tali dati devono essere resi disponibili e facilmente consultabili per gli sviluppatori e ricercatori, i quali hanno il fine di insegnare ad un elaboratore a svolgere i medesimi esercizi attraverso tecniche di autoapprendimento.
\subsubsection{Finalità del progetto}
Il risultato finale sarà un'applicazione con tre attori principali ai quali verranno fornite funzionalità diverse:
\begin{itemize}
	\item \textbf{Insegnanti}: Dovranno poter inserire gli esercizi in modo rapido e intuitivo. Gli esercizi inseriti verranno risolti in modo automatico dall'applicazione, fornendo un risultato immediato all'insegnante, il quale dovrà poi validare e modificare, se necessario, le risposte;
	\item \textbf{Allievi}: Dovranno poter eseguire il test in modo pratico e ricevere subito una valutazione. I test possono essere scelti tra quelli già presenti oppure creati al momento(in questo caso verrà utilizzata la soluzione automatica per valutare). Ogni studente verrà ricompensato con un sistema a punti per aver svolto un esercizio e potrò visualizzare lo storico degli esercizi svolti;
	\item \textbf{Sviluppatori}: Dovranno poter consultare i dati prodotti al fine di utilizzarli per l'apprendimento automatico. Allo sviluppatore dovranno essere fornite più di una versione delle risposte fornite, con lo storico delle modifiche effettuate.
\end{itemize}
\subsubsection{Tecnologie interessate}
\begin{itemize}
	\item \textbf{Hunpos e Freeling}: Entrambi software per il Part of Speech(PoS) tagging, cioè l'etichettatura delle varie parti di una frase;
	\item \textbf{Firebase Storage}: Suggerito per l'immagazzinamento dei dati prodotti dall'applicazione;
	\item \textbf{Web/Mobile programming}: Da utilizzare in modo esclusivo per la realizzazione dell'applicazione. Nessuna tecnologia specifica preposta dall'azienda sotto questo punto di vista.
\end{itemize}
\subsubsection{Aspetti positivi}
\begin{itemize}
	\item Nessun vincolo da parte dell'azienda sulle tecnologie da utilizzare, quindi grande libertà;
	\item Requisiti obbligatori in numero molto ridotto, si ha così una maggiore flessibilità;
	\item Utilizzo di Firebase,sistema utilizzato da molte aziende in ambito professionale;
	\item Implementazione dell'applicazione su Web oppure Mobile, campi non trattati (se non in minima parte) nel nostro corso di studi. 
\end{itemize}
\subsubsection{Criticità e fattori di rischio}
\begin{itemize}
\item Le troppe tecnologie nuove da utilizzare potrebbero portare ad un pesante lavoro di studio di esse prima di poterle utilizzare;
\item La non conoscenza della grammatica delle varie lingue potrebbe portare ad una elevata difficoltà implementativa della soluzione.
\end{itemize}
\subsubsection{Conclusioni}
Questo capitolato\glosp è stato trovato molto interessante da tutto il gruppo ma al momento della valutazione gli slots disponibili erano già esauriti.