\subsection{Capitolato C3 - G\&B}
\subsubsection{Informazioni generali}
\begin{itemize}
	\item \textbf{Nome}: G\&B: monitoraggio intelligente di processi DevOps\glo;
	\item \textbf{Proponente}: Zucchetti;
	\item \textbf{Committente}: Prof. Tullio Vardanega e Prof. Riccardo Cardin.
\end{itemize}
\subsubsection{Descrizione}
Il capitolato\glosp prevede la realizzazione di un plug-in\glosp per monitorare, tramite l'utilizzo di Grafana\glo, un sistema DevOps\glo, cioè un sistema in cui chi produce il software e chi lo usa collaborano strettamente. Perché la collaborazione sia efficace è necessario che si applichi, a tale sistema di monitoraggio, reti Bayesiane al flusso dei dati ricevuti per allarmi o segnalazioni tra gli operatori del servizio Cloud e la linea di produzione del software il tutto visualizzato tramite grafici che permetteranno di analizzare e controllare tali notifiche.
\subsubsection{Finalità del progetto}
La struttura del plug-in\glosp verrà scritta in linguaggio JavaScript che leggerà da un file JSON\glosp la definizione della rete Bayesiana\glosp e permetterà di associare dei nodi della rete, con informazioni di probabilità, ad un flusso di dati presente in nel sistema di monitoraggio. La rete riceverà il flusso, ad intervalli predefiniti o con continuità, e verranno eseguiti dei calcoli modificando così le probabilità dei nodi. Sia il flusso di dati che la rete verranno monitorati tramite un'apposita dashboard\glosp visualizzando il tutto attraverso dei grafici. Opzionalmente il capitolato\glosp consiglia la possibilità di	un'eventuale generazione di allarmi/notifiche che valutano l'andamento dei dati visualizzati in quel momento.
\subsubsection{Tecnologie interessate}
\begin{itemize}
	\item \textbf{Grafana\glosp}: software Open-Source\glosp per il monitoraggio di sistemi che, ricevuti dati, consente di raccoglierli in grafici che si possono visualizzare, analizzare, misurare e controllare; 
	\item \textbf{JavaScript}: linguaggio di programmazione richiesto per costruire il plug-in\glosp di Grafana e per definire la rete di Bayes\glosp in formato JSON\glo;
	\item \textbf{Rete di Bayes}: rete di nodi che contengono informazioni di probabilità. Quando si verifica un evento significativo le probabilità dei nodi si aggiornano di conseguenza.
\end{itemize}
\subsubsection{Aspetti positivi}
\begin{itemize}
	\item Il proponente si presenta come la prima software house italiana e quindi crea interesse nel gruppo;
	\item Il documento fornito per la spiegazione del capitolato è chiaro e i requisiti sono ben definiti.
\end{itemize}
\subsubsection{Criticità e fattori di rischio}
\begin{itemize}
	\item Non vi sono molte tecnologie da apprendere se non l'utilizzo del sistema di monitoraggio tramite Grafana\glo;
	\item Il solo apprendimento di quest'ultimo software non ha suscitato motivo di interesse al gruppo di lavoro.
\end{itemize}
\subsubsection{Conclusioni}
Il capitolato\glosp presenta dei punti di sviluppo molto interessanti, tuttavia i posti disponibili per l'appalto sono stati esauriti dal primo lotto.