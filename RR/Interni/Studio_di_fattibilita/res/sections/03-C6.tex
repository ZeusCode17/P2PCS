\subsection{Capitolato C6 - Soldino}

\subsubsection{Informazioni generali}
\begin{itemize}
	\item \textbf {Nome}: Soldino: piattaforma Ethereum per pagamenti IVA;
	\item \textbf {Proponente}: Red Babel;
	\item \textbf {Committente}: Prof. Tullio Vardanega e Prof. Riccardo Cardin.
\end{itemize}

\subsubsection{Descrizione}
Il capitolato\glosp C6 richiede di sviluppare un sistema, gestito dal Governo\glo, volto alla gestione dell'IVA tramite la blockchain\glosp Ethereum\glo. I proprietari di partita IVA registrati potranno acquistare/vendere beni e servizi.
Il Governo è in grado di coniare e distribuire la moneta utilizzata nelle transazioni. I cittadini potranno fare acquisti tramite la moneta coniata dal governo.   

\subsubsection{Finalità del progetto}
Lo scopo ultimo di \textit{Soldino} è quello di fornire, tramite un sito web, un insieme di ÐApps\glosp che lavorano su EVM\glosp (Etherium Virtual Machine). Il Governo\glosp e le aziende possono eseguire le solite azioni di contabilizzazione legate all'IVA\glosp (gestione pagamenti, tassi di cambio...).  
\subsubsection{Tecnologie interessate}
\begin{itemize}
	\item \textbf{Ethereum}\glo: blockchain\glosp che serve per approvare le transazioni effettuate sulla piattaforma e ad archiviarle su un sistema distribuito. 
	\item \textbf{ÐApps}\glo: applicazione decentralizzata che utilizza la blockchain di Ethereum, è composta da più parti possibilmente separate ed ogni sua parte è in grado di eseguire il proprio lavoro indipendentemente; 
	\item \textbf{Ethereum Virtual Machine (EVM)}\glo: macchina virtuale che permette di verificare ed eseguire il codice sulla blockchain assicurando che venga eseguito nello stesso modo su qualsiasi macchina;
	\item \textbf{Smart Contracts}\glo: dove risiede il codice vero e proprio utilizzato dalle ÐApp;
	\item \textbf{Solidity}: linguaggio che permette la scrittura di Smart Contracts su EVM;
	\item \textbf{MetaMask}\glo: add-on del browser che permette la gestione dei propri account su rete Ethereum. Serve inoltre a verificare l'identità degli utenti e validare le transazioni;
	\item \textbf{Web3}: API utilizzata per effettuare chiamate ad un nodo remoto di Ethereum;
	\item\textbf{Ropsten}: rete di test che utilizza lo stesso insieme di protocolli di Ethereum, utile a testare le ÐApp;
	\item\textbf{Truffle}: ambiente di sviluppo che permette la scrittura di Smart Contracts e implementa automaticamente i relativi test;
	\item\textbf{ESlint}: utilizzato per l'analisi sintattica del codice, utilizzato soprattuto per trovare pattern problematici o codice che non aderisce ad una linea guida;
	\item \textbf{JavaScript, HTML, Redux\glo, SCSS\glo, React\glo}: insieme di framework\glosp e linguaggi utilizzati per creare il front end.
\end{itemize}

\subsubsection{Aspetti positivi} 
\begin{itemize}
	\item l'impiego di tecnologie quali React\glo, Reduxglo e SCSS\glosp permetterebbe al gruppo di acquisire conoscenze molto utili soprattutto in un futuro ambito lavorativo;	
	\item l'idea di base ha piacevolmente colpito il gruppo, un eventuale riutilizzo del valore aggiunto, tramite una blockchain\glo, ci è sembrata un idea allettante;	
	\item il gruppo era inoltre molto interessato anche al solo trattamento della blockchain e di ciò che ne fa parte(criptovaluta, EVM\glo...) senza l'aspetto riguardante il trattamento dell'IVA.
\end{itemize}

\subsubsection{Criticità e fattori di rischio}
\begin{itemize}
	\item l'impiego di un consistente numero di nuove tecnologie prevede un carico di studio non indifferente, vista anche la scarsità di documentazione presente sul web;	
	\item la distanza fisica della sede di \textit{Red Babel} potrebbe influire in modo negativo sulla comunicazione tra gruppo e proponente/riferente; 	
	\item a differenza dello scorso anno, l'interesse verso le criptovalute è molto diminuito e di conseguenza anche il gruppo ha deciso di spostarsi verso una realtà più solida.  
\end{itemize}

\subsubsection{Conclusioni}
Nonostante ci sia stato un forte interesse iniziale verso il capitolato\glo, in quanto utilizzava nuove tecnologie molto interessanti, a seguito di una analisi più oggettiva riguardante appunto queste ultime, si è scelto di spostarsi verso una realtà più concreta e che si avvicini ai nostri interessi più che alla nostra curiosità.