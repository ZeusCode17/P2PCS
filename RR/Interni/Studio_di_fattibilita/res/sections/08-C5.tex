\section{Capitolato Scelto C5 - P2PCS}
\subsection{Informazioni generali}
\begin{itemize}
\item \textbf{Nome}: P2PCS: piattaforma di Peer-to-Peer\glosp car sharing;
\item \textbf{Proponente}: GaiaGo S.r.l;
\item \textbf{Committente}: Prof. Tullio Vardanega e Prof. Riccardo Cardin.
\end{itemize}
\subsection{Descrizione}
Questo capitolato propone di integrare un'applicazione Android con l'obiettivo di fornire agli utenti la possibilità di condividere il proprio veicolo con altri utenti (car sharing) e ad invogliarne l'utilizzo attraverso delle strategie di gamification\glosp. 
I due vantaggi principali sono i seguenti:
\begin{itemize}
	\item \textbf{Proprietario del veicolo}: avrà un vantaggio economico in quanto farà fruttare il proprio mezzo quando non deve utilizzarlo, prestandolo ad altri utenti;
	\item \textbf{Usufruente}: avrà il vantaggio di avere a disposizione un veicolo per potersi spostare pagando solamente le ore effettive di utilizzo.
\end{itemize}
\subsection{Finalità del progetto}
L'applicazione si baserà su un calendario nel quale il proprietario di un veicolo potrà indicare in che giorni e orari il suo mezzo sarà disponibile. Ogni utente potrà cercare un mezzo libero nella propria zona, prenotarlo e ritirarne le chiavi.
\subsection{Tecnologie interessate}
\begin{itemize}
	\item \textbf{Node.js\glosp}: framework\glosp Open-Source lato server basato su JavaScript con un modello asincrono di I/O guidato da eventi;
	\item \textbf{Google Cloud}: per la gestione del database;
	\item \textbf{Octalysis\glosp}: framework\glosp per integrare una strategia di gamification\glosp volta a rendere più accattivante e aggiornata l'applicazione;
	\item \textbf{Movens\glosp}: piattaforma Open-Source\glosp che fornisce funzionalità di gestione di servizi nelle smart cities. Copre diversi livelli tecnologici come: dispositivi fisici, connetività e servizi applicativi;	
	\item \textbf{Android Studio}: framework\glosp per lo sviluppo di applicazioni Android.
	\item \textbf{Java}, \textbf{Kotlin\glosp}: linguaggi utlizzati per lo sviluppo di applicazioni Android.
\end{itemize}
\subsection{Aspetti positivi}
\begin{itemize}
	\item Acquisizione di nuove competenze nel campo della programmazione mobile e nello sviluppo di un'architettura Peer-to-Peer\glosp;
	\item Apprendimento di nuovi linguaggi e piattaforme, come Node.js\glo, Kotlin\glosp e Movens\glosp.
	\item Comprendere la teoria della gamification\glosp e applicarla all'interno di un'applicazione.
\end{itemize}
\subsection{Criticità e fattori di rischio}
\begin{itemize}
	\item La concorrenza propone già delle valide alternative e far emergere l'applicazione nel mercato e renderla un prodotto superiore richiederà molto impegno;
	\item La cessione del proprio veicolo ad utenti terzi può portare diffidenza verso il servizio proposto dall'applicazione.
\end{itemize}
\subsection{Conclusioni}
Il gruppo ha espresso un giudizio principalmente positivo verso questo capitolato\glosp,
in quanto convinto di poter proporre un prodotto superiore alla concorrenza e in grado di soddisfare le richieste di mercato e del committente.