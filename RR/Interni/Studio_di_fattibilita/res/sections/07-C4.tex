\subsection{Capitolato C4 - MegAlexa}
\subsubsection{Informazioni generali}
\begin{itemize}
	\item \textbf{Nome}: MegAlexa, arricchitore di skill di Amazon Alexa;
	\item \textbf{Proponente}: ZERO12;
	\item \textbf{Committente}: Prof. Tullio Vardanega e Prof. Riccardo Cardin.
\end{itemize}
\subsubsection{Descrizione}
La sfida lanciata dall’azienda proponente consiste di progettare una skill\glosp di Alexa, l’assistente virtuale prodotto da Amazon, in cui gli utenti tramite un applicativo Web o Mobile (Android o iOS) siano in grado di avviare Workflow\glo.
\subsubsection{Finalità del progetto}
Far sì che un qualsiasi utente che possiede Amazon Alexa possa, attraverso l’applicativo realizzato tramite micro-funzioni già fornite, crearsi una routine con le informazioni che vuole tramite un comando vocale personalizzato.
\subsubsection{Tecnologie interessate}
\begin{itemize}
	\item \textbf{Amazon Alexa}: l'assistente digitale di Amazon;
	\item \textbf{Lambda (AWS)}: servizio di elaborazione serverless per l'esecuzione del proprio codice; 
	\item \textbf{API Gateway (AWS)}: servizio API per la comunicazione con Lambda;
	\item \textbf{Aurora Serverless (AWS)}: offre capacità di database;
	\item \textbf{Node.js\glo}: piattaforma per esecuzione di codice JavaScript;
	\item \textbf{HTML5, CSS3 e JavasScript}: linguaggi da utilizzare per l'implementazione dell'interfaccia web;
	\item \textbf{Bootstrap\glo}: framework\glosp front end, consigliato dal proponente;
	\item \textbf{Android e iOS}: linguaggi come \textit{Kotlin\glosp} (Android) o \textit{Swift\glosp} (iOS) sono consigliati dal proponente.
\end{itemize}
\subsubsection{Aspetti positivi}
\begin{itemize}
	\item Il proponente offre delle lezioni al fine di introdurre il gruppo alle nuove tecnologie da utilizzare nello sviluppo del progetto e lasciando poi piena libertà di sviluppo dei servizi Google per la realizzazione dell’assistente virtuale;
	\item Grande presenza nel web di documentazione dettagliata che rende più semplice l’apprendimento di tali tecnologie. In particolare, Amazon fornisce Alexa Skills Kit (raccolta di API\glo, strumenti, documentazioni ed esempi di codice).
\end{itemize}
\subsubsection{Criticità e fattori di rischio}
\begin{itemize}
	\item È obbligatorio che le shortcuts\glosp siano multilingua. Echo al momento supporta le lingue: inglese, francese, tedesco, italiano, giapponese e spagnolo. Tuttavia, possiamo realizzare in modo esaustivo solamente la versione italiana ed inglese, viste le nostre limitate conoscenze linguistiche;
	\item Sono già presenti, nel web, tecnologie per la realizzazione di skills\glosp in grado di avviare dei workflow\glosp personalizzati, anche se in modo piuttosto grezzo. Infatti, la stessa applicazione di Alexa permette di creare sequenze di azioni precedentemente selezionate.
\end{itemize}
\subsubsection{Conclusioni}
Nonostante tale capitolato\glosp sia interessante dal punto di vista delle nuove tecnologie che stanno prendendo piede in questo momento e per le competenze curricolari che potrebbe comportare, il gruppo si è mostrato più stimolato verso un altro progetto non meno allettante.