\section{Pianificazione}
La pianificazione del gruppo \textit{ZeusCode} è stata costruita sulla base delle scadenze presentate nelola sottosezione 1.5 di questo documento. Seguendo quelle scadenze è stato deciso di suddividere lo sviluppo in cinque fasi:
\begin{enumerate}
	\item \textbf{Analisi};
	\item \textbf{Consolidamento dei requisiti};
	\item \textbf{Progettazione architetturale};
	\item \textbf{Progettazione di dettaglio e codifica};
	\item \textbf{Validazione e collaudo}.
\end{enumerate}
Ogni fase viene suddivisa in attività che verranno realizzate durante il 
periodo stabilito per la fase stessa, come riportato nei corrispettivi diagrammi di Gantt\glo. 
\subsection{Analisi}
\textit{Periodo: dal 2018-03-01 al 2019-04-12}\\
L'inizio del periodo di questa fase coincide con la data di formazione del 
gruppo e la fine coincide con la data ultima per la consegna dei documenti relativi alla revisione dei requisiti. Questa fase è stata scomposta nelle seguenti sotto attività:
\begin{itemize}
	\item \textbf{Individuazione degli strumenti}: questa attività consiste nel scegliere gli strumenti che saranno utilizzati per la comunicazione, per la stesura dei documenti e per il versionamento, lo sviluppo e la verifica del software;
	\item \textbf{Norme di Progetto}: comprende la stesura di una serie di regole per lo svolgimento del progetto, relative al prodotto da realizzare e ai processi da adottare. Il documento Norme di Progetto viene redatto dall'Amministratore per conto del Responsabile di progetto. La produzione di questo documento deve essere prioritaria per garantire uniformità e correttezza nelle successive fasi del progetto;
	\item \textbf{Studio di fattibilità}: questa attività consiste nella redazione da parte degli Analisti dello Studio di Fattibilità, che contiene un’analisi dei vari capitolati proposti, in modo da determinare quale di essi verrà scelto. Questa attività è da considerarsi bloccante per l'attività di Analisi dei Requisiti;
	\item \textbf{Analisi dei Requisiti}: durante questa attività viene eseguito uno studio approfondito dei requisiti del capitolato scelto nell'attività di studio di fattibilità e il relativo documento viene composto dagli Analisti;
	\item \textbf{Piano di Progetto}: il Responsabile analizza le attività necessarie e le loro scadenze per la buona riuscita del progetto e l’Amministratore analizza i rischi nei quali il gruppo \textit{ZeusCode} può incombere durante il progetto. Inoltre vengono suddivise le risorse disponibili per l’intera durata del progetto
e viene calcolato il preventivo per la realizzazione del progetto. Questa attività comporta anche la stesura del documento Piano di Progetto;
	\item \textbf{Piano di Qualifica}: 	
	in questa attività si individuano le strategie di verifica e validazione adottate e le metodologie attraverso le quali si garantisce la qualità del prodotto. Comprende la stesura del documento Piano di Qualifica da parte dell'Amministratore;
	\item \textbf{Glossario}: tutti i termini considerati possibilmente ambigui vengono definiti nel documento Glossario, che verrà aggiornato in modo incrementale fino al completamento della documentazione.
\end{itemize}

\begin{figure}[H]
	\includegraphics[width=0.99\linewidth]{res/images/gantt_analisi.jpg}
	\caption{Diagramma di Gantt della fase di Analisi}
\end{figure}


\subsection{Consolidamento dei requisiti}
\textit{Periodo: dal 2019-04-12 al 2019-04-19} \\
Questa fase comincia con la fine della fase di Analisi e termina il 
giorno della presentazione della Revisione dei Requisiti. Le attività 
di questa fase sono:
\begin{itemize}
	\item \textbf{Consolidamento}: questa attività ha lo scopo di consolidare e 
	migliorare i requisiti ottenuti nella fase precedente;
	\item \textbf{Preparazione alla presentazione}: durante questa attività 
	viene preparato il materiale necessario alla presentazione del 2019-01-21;
	\item \textbf{Incremento e Verifica}: se necessario vengono migliorati i 
	documenti prodotti nella fase precedente;
	\item \textbf{Approfondimento personale}: ogni componente del gruppo dovrà 
	dedicare almeno 15 ore di studio e approfondimento delle tecnologie 
	necessarie alle prossime fasi e alla realizzazione del prodotto. Questa 
	attività verrà gestita in modo autonomo dai membri del gruppo, quindi non 
	sarà riportata nel diagramma di Gantt~\ref{fig:gantt_con} sottostante.
\end{itemize}

\begin{figure}[H]
	\includegraphics[width=0.99\linewidth]{res/images/gantt_cons.jpg}
	\caption{Diagramma di Gantt della fase di Consolidamento dei requisiti}
	\label{fig:gantt_con}
\end{figure}

%-----------------Sottosezione Progettazione Architetturale---------------------
\subsection{Progettazione architetturale}
\textit{Periodo: dal 2019-04-20 al 2019-05-10} \\
Questa fase comincia il giorno successivo alla presentazione e la fine coincide con la data di consegna Revisione di 
Progettazione. In questo periodo verrà individuata una soluzione architetturale 
tale per cui i requisiti richiesti vengano soddisfatti.
\begin{itemize}
	\item \textbf{Technology Baseline}: viene redatto l'
	Allegato Tecnico  nel quale vengono 
	individuati i design 
	pattern\glosp che verranno adottati per lo sviluppo. Inoltre il documento 
	include il tracciamento dei requisiti.\\
	Infine viene codificato il \textbf{Proof of Concept}\glosp il 
	quale viene presentato o condiviso tramite repository al committente e 
	proponente in una data da definirsi;
	\item \textbf{Incremento e Verifica}: se necessario vengono migliorati i 
	documenti prodotti nelle fasi precedenti.
\end{itemize}

\begin{figure}[H]
	\includegraphics[width=0.99\linewidth]{res/images/gantt_pa.jpg}
	\caption{Diagramma di Gantt della fase di Progettazione architetturale}
\end{figure}


%-------------------Sottosezione Progettazione di Dettaglio---------------------
\subsection{Progettazione di dettaglio e codifica}
\textit{Periodo: dal 2019-05-17 al 2019-06-10}
L'inizio di questa fase è il giorno della scadenza della \textit{Revisione di 
Progettazione} e la data di fine coincide con la data di consegna dei documenti 
in vista della \textit{Revisione di Qualifica}. Le attività di questa fase sono:
\begin{itemize}
	\item \textbf{Product Baseline}: a seguito della \textit{Technology 
	Baseline} l'architettura individuata in essa viene scomposta nelle sue unità,
	che sono analizzate in profondità per fornire i 
	dettagli necessari alla loro codifica e verifica. A supporto 
	di ciò viene redatto l'Allegato Tecnico di supporto alla Product Baseline;
	\item \textbf{Codifica}: questa attività consiste nella scrittura del 
	codice e della sua verifica con modalità e strumenti definiti nel 
	\textit{Piano di Qualifica v2.0.0}
	\item \textbf{Manuale Utente}: viene redatto il documento \textit{Manuale 
	Utente} atto a fornire istruzioni e indicazioni per l'utilizzo del prodotto;
	\item \textbf{Incremento e Verifica}: se necessario vengono migliorati i 
	documenti prodotti nelle fasi precedenti.
\end{itemize}

\begin{figure}[H]
	\includegraphics[width=0.99\linewidth]{res/images/gantt_pd.jpg}
	\caption{Diagramma di Gantt della fase di Progettazione di dettaglio e codifica}
\end{figure}
\pagebreak


\subsection{Validazione e collaudo}
\textit{Periodo: dal 2019-06-10 al 2019-07-15 } \\
L'inizio di questa fase coincide con la data di consegna dei documenti per la 
\textit{Revisione di Qualifica}, mentre la data di fine coincide con la 
consegna in vista della \textit{Revisione di Accettazione}. Durante questo periodo 
si eseguiranno ulteriori attività di validazione e verifica. Le attività 
previste sono: 
\begin{itemize}
	\item \textbf{Validazione e Collaudo}: per la parte di collaudo si 
	eseguiranno ulteriori test sul prodotto, in modo da garantirne la 
	correttezza e stabilità. Per la parte di validazione, verrà 
	valutata la coerenza del prodotto e dei requisiti specificati nel documento 
	\textit{Analisi dei Requisiti} nella sua ultima versione;
	\item \textbf{Manuale Sviluppatore}: viene redatto il documento \textit{Manuale Sviluppatore} atto a fornire tutte le informazioni necessarie al mantenimento, manutenzione e ampliamento del prodotto finale;
	\item \textbf{Incremento e Verifica}: se necessario vengono migliorati i 
	documenti prodotti nelle fasi precedenti.
\end{itemize}
\begin{figure}[H]
	\includegraphics[width=0.99\linewidth]{res/images/gantt_val.jpg}
	\caption{Diagramma di Gantt della fase di Validazione e collaudo}
\end{figure}