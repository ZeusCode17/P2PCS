\section{Consuntivi di periodo}
Di seguito verranno indicate le spese effettivamente sostenute, considerando sia quelle per ruolo sia quelle per persona. Il bilancio potrà risultare:
\begin{itemize}
	\item \textbf{Positivo:} se il preventivo supera il consuntivo;
	\item \textbf{Pari:} se il consuntivo e il preventivo sono pari;
	\item \textbf{Negativo:} se il consuntivo supera il preventivo.
\end{itemize}

\subsection{Periodo di Analisi}
Le ore di lavoro sostenute in questa fase sono da considerarsi come ore di investimento per l'approfondimento personale. Esse sono quindi non rendicontate.

\begin{table}[H]
				\centering\renewcommand{\arraystretch}{1.5}
				\caption{Consuntivo di periodo della fase di Analisi}
				\vspace{0.2cm}
                \begin{tabular}{c c c}
                               
                \rowcolorhead
                 {\colorhead \textbf{Ruolo}} &
                 {\colorhead \textbf{Ore}} & 
                 {\colorhead \textbf{Costo}} \\
				
                \rowcolorlight
                 {\colorbody Responsabile} & {\colorbody 38 (+2)} & 
                 {\colorbody \EUR{1.140,00} (+\EUR{60,00})}  
				\\
				
				\rowcolordark
                 {\colorbody Amministratore} & {\colorbody 25 (+7)} & 
                 {\colorbody \EUR{500,00} (+\EUR{140,00})}
				\\	
				
				\rowcolorlight
                 {\colorbody Analista} & {\colorbody 71 (-4)} & 
                 {\colorbody \EUR{1.775,00} (-\EUR{100,00})} 
				\\
				
				\rowcolordark
                 {\colorbody Progettista} & {\colorbody 19
                 (+0)} & 
                 {\colorbody \EUR{418,00} (+\EUR{0,00})} 
				\\
				
				\rowcolorlight
                 {\colorbody Programmatore} & {\colorbody -} & 
                 {\colorbody -} 
				\\
				
				\rowcolordark
                 {\colorbody Verificatore} & {\colorbody 57 (+3)} & 
                 {\colorbody \EUR{855,00} (+\EUR{45,00})} 
				\\
				
				\rowcolorlight
                 {\colorbody \textbf{Totale Preventivo}} & {\colorbody \textbf{210}} & 
                 {\colorbody \textbf{\EUR{4.688,00}}} 
				\\
				
				
				\rowcolordark
                 {\colorbody \textbf{Totale Consuntivo}} & {\colorbody \textbf{218}} & 
                 {\colorbody \textbf{\EUR{4.833,00}}} 
				\\
				
				
				\rowcolorlight
                 {\colorbody \textbf{Differenza}} & {\colorbody \textbf{8}} & 
                 {\colorbody \textbf{\EUR{+145,00}}} 
				\\
				
                

                \end{tabular}
                
\end{table}

\subsubsection{Conclusioni}
Come emerge dai dati riportati nella tabella soprastante, che presenta le ore relative al consuntivo della fase di Analisi, è stato necessario investire più tempo del previsto nei ruoli di \textit{Responsabile}, \textit{Amministratore} e \textit{Verificatore}.  Al contempo però sono risultate sufficienti
un numero inferiore di ore per il ruolo di \textit{Analista}. 
Di seguito sono elencate le cause dei ritardi sopracitati:
\begin{itemize}
	\item \textbf{Amministratori:} la ricerca e configurazione dei software atti alla produzione e alla gestione del progetto ha richiesto più tempo del previsto;
	\item \textbf{Responsabile:} si è reso necessario un monte ore maggiore per la coordinazione generale del progetto, causato dall'inesperienza dei membri del gruppo; 
	\item \textbf{Verificatore:} a causa dell'inesperienza dei membri del gruppo si sono verificati diversi errori e mancanze durante la stesura della documentazione che hanno richiesto un maggiore monte ore di verifica;
	\item \textbf{Analista:} al contrario di quanto preventivato il totale delle ore di analisi è risultato inferiore, questo grazie alla facile comprensione dei requisiti richiesti che ha permesso una rapida stesura di quest'ultimi.
\end{itemize}

\subsubsection{Preventivo a finire}
Il risultato del periodo è complessivamente di 8 ore lavorative oltre il previsto e di una
spesa aggiunta di \EUR{+145,00}, che però facendo parte del periodo di investimento
non influirà sul totale rendicontato.

\subsection{Periodo di Consolidamento dei requisiti}
Le ore di lavoro sostenute durante questo periodo sono successive alla fase di analisi. Diverse ore sono state impiegate per lo studio personale delle tecnologie che andremo ad utilizzare e per questo non vengono riportate nella tabella sottostante e non sono rendicontate.


\begin{table}[H]
	\centering\renewcommand{\arraystretch}{1.5}
	\caption{Consuntivo di periodo della fase di Consolidamento dei requisiti}
	\vspace{0.2cm}
	\begin{tabular}{c c c}
		
		\rowcolorhead
		{\colorhead \textbf{Ruolo}} &
		{\colorhead \textbf{Ore}} & 
		{\colorhead \textbf{Costo}} \\
		
		\rowcolorlight
		{\colorbody Responsabile} & {\colorbody 5 (+0)} & 
		{\colorbody \EUR{150,00} (+\EUR{0,00})}  
		\\
		
		\rowcolordark
		{\colorbody Amministratore} & {\colorbody 3 (+0)} & 
		{\colorbody \EUR{60,00} (+\EUR{0,00})}
		\\	
		
		\rowcolorlight
		{\colorbody Analista} & {\colorbody 12 (0)} & 
		{\colorbody \EUR{300,00} (+\EUR{0,00})} 
		\\
		
		\rowcolordark
		{\colorbody Progettista} & {\colorbody -} & 
		{\colorbody \EUR{0,00} (+\EUR{0,00})} 
		\\
		
		\rowcolorlight
		{\colorbody Programmatore} & {\colorbody -} & 
		{\colorbody -} 
		\\
		
		\rowcolordark
		{\colorbody Verificatore} & {\colorbody 10 (+0)} & 
		{\colorbody \EUR{150,00} (+\EUR{0,00})} 
		\\
		
		\rowcolorlight
		{\colorbody \textbf{Totale Preventivo}} & {\colorbody \textbf{30}} & 
		{\colorbody \textbf{\EUR{660,00}}} 
		\\
		
		
		\rowcolordark
		{\colorbody \textbf{Totale Consuntivo}} & {\colorbody \textbf{30}} & 
		{\colorbody \textbf{\EUR{660,00}}} 
		\\
		
		
		\rowcolorlight
		{\colorbody \textbf{Differenza}} & {\colorbody -} & 
		{\colorbody -} 
		\\
		
		
		
	\end{tabular}
	
\end{table}

\subsubsection{Conclusioni}
Come emerge dai dati riportati nella tabella soprastante, che presenta le ore relative al consuntivo della fase di Consolidamento dei requisiti, è stato rispettato il monte ore preventivato. Questo grazie all'esperienza acquisita durante la fase precedente e alla breve durata del periodo, che ha permesso una migliore gestione delle risorse.

\subsubsection{Preventivo a finire}
Il risultato del consuntivo di periodo coincide col monte ore preventivato inoltre, facendo parte del periodo rendicontato, non è necessario eseguire nessuna modifica o accorgimenti ai futuri periodo o al preventivo.

\subsection{Periodo di Progettazione e Codifica per la Technology Baseline}
Le ore di lavoro sostenute durante questo periodo sono successive alla fase di analisi. Diverse ore sono state impiegate per lo studio personale delle tecnologie che andremo ad utilizzare e per questo non vengono riportate nella tabella sottostante e non sono rendicontate.


\begin{table}[H]
	\centering\renewcommand{\arraystretch}{1.5}
	\caption{Consuntivo di periodo della fase di Progettazione e Codifica per la Technology Baseline}
	\vspace{0.2cm}
	\begin{tabular}{c c c}
		
		\rowcolorhead
		{\colorhead \textbf{Ruolo}} &
		{\colorhead \textbf{Ore}} & 
		{\colorhead \textbf{Costo}} \\
		
		\rowcolorlight
		{\colorbody Responsabile} & {\colorbody 10 (+0)} & 
		{\colorbody \EUR{300,00} (+\EUR{0,00})}  
		\\
		
		\rowcolordark
		{\colorbody Amministratore} & {\colorbody 17 (+10)} & 
		{\colorbody \EUR{540,00} (+\EUR{200,00})}
		\\	
		
		\rowcolorlight
		{\colorbody Analista} & {\colorbody 29 (-5)} & 
		{\colorbody \EUR{625,00} (-\EUR{100,00})} 
		\\
		
		\rowcolordark
		{\colorbody Progettista} & {\colorbody 41 (-20)} & 
		{\colorbody \EUR{462,00} (-\EUR{440,00})} 
		\\
		
		\rowcolorlight
		{\colorbody Programmatore} & {\colorbody 26 (+35)} & 
		{\colorbody \EUR{915,00} (+\EUR{525,00})} 
		\\
		
		\rowcolordark
		{\colorbody Verificatore} & {\colorbody 45 (-20)} & 
		{\colorbody \EUR{375,00} (-\EUR{300,00})} 
		\\
		
		\rowcolorlight
		{\colorbody \textbf{Totale Preventivo}} & {\colorbody \textbf{168}} & 
		{\colorbody \textbf{\EUR{3332,00}}} 
		\\
		
		
		\rowcolordark
		{\colorbody \textbf{Totale Consuntivo}} & {\colorbody \textbf{168}} & 
		{\colorbody \textbf{\EUR{3192,00}}} 
		\\
		
		
		\rowcolorlight
		{\colorbody \textbf{Differenza}} & {\colorbody -} & 
		{\colorbody +140} 
		\\
		
		
		
	\end{tabular}
	
\end{table}

\subsubsection{Conclusioni}
Come emerge dai dati riportati nella tabella soprastante, che presenta le ore relative al consuntivo della fase di Progettazione e Codifica per la Technology Baseline, la progettazione ha subito una sostanziale modifica. Tale scostamento è dovuto all'idea iniziale di presentare una completa progettazione architetturale del prodotto. Tuttavia ci siamo concentrati maggiormente sul creare delle solide fondamenta per lo sviluppo dell'applicazione attraverso la progettazione e codifica del Proof of Concept\glo. Di seguito sono riportate in dettaglio i vari scostamenti orari dei ruoli interessati:
\begin{itemize}
\item \textbf{Amministratori:} successivamente alla fase di testing dell'editor prestabilito sono insorte svariate problematiche riguardanti il sistema di Continuos Integration\glosp, che ha richiesto un notevole monte ore per essere operativo;
\item \textbf{Progettista:} visto il cambiamento dell'obbiettivo finale di questo periodo il monte ore per la fase di progettazione è diminuito in quanto il progettista ha dovuto occuparsi solo di alcune parti del prodotto finale; 
\item \textbf{Verificatore:} a seguito della diminuzione del monte ore della fase di progettazione anche la fase di verifica ha subito una diminuzione del monte ore totali preventivate;
\item \textbf{Programmatore:} al contrario di quanto preventivato il totale delle ore di programmazione ha subito un notevole aumento, ciò è dovuto alla necessità di integrare le modifiche da noi apportate con il codice dell'applicazione già fornito dal proponente. La difficoltà di tale integrazione è scaturita dalla difficoltà di comprensione di alcune parti del codice fornito prive di documentazione e/o semplice spiegazione.
\end{itemize}

\subsubsection{Preventivo a finire}
Il bilancio economico è positivo, sono stati risparmiati \EUR{140,00}. Tali fondi verranno reinvestiti nelle fasi future per la realizzazione di alcuni requisiti opzionali.
