\section{Modello di sviluppo}
La scelta del modello di sviluppo è una delle realtà fondamentali per la realizzazione di un progetto software.
Avendo una visione abbastanza chiara dell'intero progetto e, allo stesso tempo, una limitata conoscenza dei requisiti specifici, si è scelto di seguire il \textbf{modello incrementale}.

\subsection{Modello incrementale}
Il modello di sviluppo incrementale permette la suddivisione del progetto in più sottoinsiemi,
ognuno di questi sottoinsiemi incorpora una funzionalità diversa che, se necessario, verrà migliorata ad ogni incremento dell'intero 
sistema.  \\
Ad ogni incremento del sistema è consentita la modifica, aggiunta ed eliminazione di requisiti in base alle esigenze progettuali, 
è necessario un colloquio diretto con il proponente per approvare tali cambiamenti; \\
Utilizzando questo modello di sviluppo il versionamento del sistema è reso semplice e intuitivo, in quanto ogni modifica è facilmente tracciabile da un incremento all'altro e se ne possono valutare direttamente i difetti o benefici.\\
I vantaggi predisposti dal modello incrementale sono i seguenti:
\begin{itemize}
	\item gli incrementi sono disposti in base alle funzionalità con priorità decrescente, partendo da quelle con priorità e impatto maggiori
	così da avere subito un riscontro diretto;
	\item ogni incremento genera un risultato che può essere valutato dal proponente, approvandone i benefici o evidenziandone i difetti;
	\item redne sempre disponibile una recente baseline\glosp per una eventuale rivalutazione, senza dover ripercorrere tutti i passi effettuati fino ad ora dall'inizio dello sviluppo;
	\item tutti gli errori sono limitati al singolo incremento;
	\item le modifiche e le correzioni sono molto economiche in quanto di facile reperibilità;
	\item le fasi di test sono mirate al corrente incremento quindi più efficienti.
\end{itemize}
Sono presenti anche degli aspetti negativi per quanto riguarda il modello incrementale, primo tra tutti la rapida degradazione del codice. Per ovviare a questo svantaggio verrà utilizzato il refactoring\glosp del codice. I vantaggi che il refactoring persegue riguardano in genere un miglioramento della leggibilità, della manutenibilità, della riusabilità e dell'estendibilità del codice e la riduzione della sua complessità generale.


