\section{Introduzione}
\subsection{Scopo del documento}
Questo documento ha l’obiettivo di identificare e dettagliare la pianificazione del gruppo \textit{ZeusCode} relativa allo sviluppo del progetto \textit{P2PCS}. In particolare il documento tratta i seguenti argomenti:
\begin{itemize}
	\item analisi dei rischi;
	\item descrizione del modello di sviluppo adottato;
	\item ripartizione dei compiti tra i membri del gruppo;
	\item stima dei costi e delle risorse necessarie.
\end{itemize}
\subsection{Scopo del prodotto}
Lo scopo del prodotto è quello di sviluppare un piattaforma di Car Sharing Peer-to-Peer per l'applicazione Android sviluppato da \textit{GaiaGo} per il servizio di Car Sharing Condominiale con l'aiuto del piattaforma \textit{Movens}\glosp di \textit{Henshin}\glo. Il Car Sharing Peer to Peer è la possibilità di condividere la propria macchina con altre persone amiche o meno. La parte centrale del
capitolato è quella di realizzare un'app che sfrutti i meccanismi di gamification\glosp; a questo fine
verrà utilizato il \textit{framework}\glosp \textit{Octalysis}\glo.
\subsection{Glossario}
Con l'obiettivo di evitare ridondanze e ambiguità di linguaggio, i termini tecnici e gli acronimi
utilizzati nei documenti verranno definiti e descritti riportandoli nel documento Glossario v1.0.0 .
I vocaboli riportati vengono indicati con una 'G' a pedice.
\subsection{Riferimenti}
\subsubsection{Normativi}
\begin{itemize}
	\item \textbf{Norme di Progetto}: \textit{Norme di Progetto v1.0.0};
	\item \textbf{Regolamento organigramma e specifica tecnico-economica}: \\
	\url{https://www.math.unipd.it/~tullio/IS-1/2018/Progetto/RO.html}.
\end{itemize}

\subsubsection{Informativi}
\begin{itemize}
	\item \textbf{Capitolato d'appalto C5 - \textit{P2PCS}: piattaforma di peer-to-peer car sharing}: \\
	\url{https://www.math.unipd.it/~tullio/IS-1/2018/Progetto/C5.pdf};
	\item \textbf{Slide L05 del corso Ingegneria del Software - Ciclo di vita 
		del software}:\\
	\url{https://www.math.unipd.it/~tullio/IS-1/2018/Dispense/L05.pdf};
	\item \textbf{Slide L06 del corso Ingegneria del Software - Gestione di 
	Progetto}: \\
	\url{https://www.math.unipd.it/~tullio/IS-1/2018/Dispense/L06.pdf};
	\item \textbf{Software Engineering - Ian Sommerville - 10$^{th}$ Edition, 
	2010.}
\end{itemize}

\hypertarget{scadenze}{\subsection{Scadenze}}
Il gruppo \textit{ZeusCode} si impegna a rispettare le seguenti scadenze per lo 
sviluppo del progetto \textit{P2PCS}:

\begin{itemize}
	\item \textbf{Revisione dei Requisiti}: 2019-04-19;
	\item \textbf{Revisione di Progettazione}: 2019-06-17;
	\item \textbf{Revisione di Qualifica}: 2019-07-15;
	\item \textbf{Revisione di Accettazione}: 2019-08-26.
\end{itemize}