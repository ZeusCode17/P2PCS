\section{Verbale della riunione}
\begin{itemize}
	\item Fare di \textit{GaiaGo} un servizio facendo pagare meno gli utenti che lo sfruttano, ma dandogli sconti per esempio sulla spesa o su vari altri acquisti;
	\item Elenco delle proposte di gamification\glo:
		\begin{itemize}
			\item Ruota della fortuna (Lucky Spin) e/o regalo misterioso (Mistery Box) contenenti premi che l'utente potrà ricevere gratuitamente dopo aver concluso una prenotazione. Premi che vanno da non prendere nulla ad ottenere un viaggio gratis con \textit{GaiaGo};
			\item tutoria guidato all'utilizzo dell'applicazione e/o dell'eventuale minigioco, presente all'interno di quest'ultima, da fare prima o dopo la registrazione;
			\item invio di un "codice amico", come invito da parte di un utente, da utilizzare nella fase di registrazione all'applicazione con conseguente aumento di punti per entrambe le parti;
			\item creare una sezione "utenti preferiti" dove visualizzare gli utenti che ci sono piaciuti e che potremmo ancora contattare in caso di bisogno; 
			\item Mini gioco: 
				\begin{itemize}
					\item formato da un avatar modificabile per tutti gli utenti, mentre per chi aggiunge il suo veicolo, introduzione di un garage in cui la macchina, tramite gettoni e premi, possa venire modificata nell'aspetto e nel colore;
					\item mappa statica con edifici sbloccabili solo a raggiungimento di livelli prestabiliti dove l'utente possa potenziarsi utilizzando sempre più l'applicazione;
					\item utilizzo di materiali per creare oggetti con tempi e costi predefiniti. 
				\end{itemize}
			\item notifiche e gestione notifiche che avvisano qualsiasi cosa succeda nell'applicazione come per esempio l'inutilizzo di quest'ultima, etc;
			\item classifica utenti in base ai punti esperienza ottenuti e alle valutazioni ricevute oppure alle prestazioni della macchina nel mini gioco, tutte queste da decidere se classificarle per regione, città, paese;
			\item progress bar;
			\item punti esperienza/gettoni accumulati in base a quanto viene utilizzata l'applicazione;
			\item rewards giornalieri;
			\item recensioni agli utenti;
			\item easter egg;
			\item mileston unlock\glosp che sono semplici obbiettivi da completare all'interno dell'applicazione che fruttano premi;
		\end{itemize}
	\item Utilizzare AWS\glosp come servizio alternativo alla piattaforma Movens\glosp per il back end\glosp dell'applicazione.
\end{itemize} 
\pagebreak
\section{Riepilogo delle decisioni}

	%\renewcommand{\arraystretch}{1.5}
	\rowcolors{2}{pari}{dispari}
	
	\begin{longtable}{ >{\centering}p{0.20\textwidth} >{}p{0.70\textwidth}}
		\caption{Decisioni della riunione interna del 2019-04-09}\\	
		\rowcolorhead
		\textbf{\color{white}Codice} 
		& \centering\textbf{\color{white}Decisione} 
		\tabularnewline 
		\endfirsthead
		VI\_1.1 & Scelto AWS\glosp come piattaforma di back end\glosp in sostituzione a Movens\glo.
		
		\tabularnewline 
		VI\_1.2 & Ricevuto feedback delle proposte con varie soluzioni per alcune di esse.
		
		\tabularnewline 
		VI\_1.3 & Scelto di inserire la lucky spin come proposta di gamification\glo.
	
		\tabularnewline 
		VI\_1.4 & Scelto di inserire il tutoria dell'applicazione una volta registrati.
		
		\tabularnewline 
		VI\_1.5 & Scelto di inserire il "codice amico" come metodo di invito amici all'utilizzo dell'applicazione.
		
		\tabularnewline 
		VI\_1.6 & Scelto di inserire la sezione utenti preferiti;
		
		\tabularnewline 
		VI\_1.7 & Scelto di inserire il mini gioco e notifiche riguardati l'applicazione e tutto quello che ne consegue;
	\end{longtable}
	




