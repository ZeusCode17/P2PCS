\section{Valutazioni per il miglioramento}
In questa sezione viene riportata la valutazione fatta dal gruppo riguardo il 
lavoro svolto finora. Lo scopo di questa scelta è trattare i problemi sorti e
procedere alla loro più efficiente risoluzione in modo tale che non si verifichino
in futuro. \\
Verano dunque tracciati problemi riguardanti i seguenti ambiti:

\begin{itemize}
	\item \textbf{Organizzazione}: in cui vengono analizzati i problemi riguardanti 
		l'organizzazione e la comunicazione all'interno del gruppo;
	\item \textbf{Ruoli}: in cui vengono analizzati i problemi riguardanti il 
		corretto svolgimento di un ruolo;
	\item \textbf{Strumenti di lavoro}: in cui vengono analizzati i problemi riguardanti 
		l'uso degli strumenti scelti.
\end{itemize}

\noindent Ogni problema viene sollevato sulla base dell'autovalutazione dei membri del 
gruppo, poiché non vi è una persona esterna che possa dare una valutazione
oggettiva. 
Questa sezione verrà aggiornata con l'avanzamento del lavoro riportando nuove 
problematiche, nel caso in cui queste dovessero verificarsi.

\subsection{Revisione dei requisiti}

\subsubsection{Valutazioni sull'organizzazione}
\begin{itemize}	
		
		\item 2019-03-7 \textbf{Incontro di gruppo} \\
		\begin{itemize}
			\item \textbf{Problema:} si è riscontrata una certa difficoltà 				nell'organizzare gli incontri in modo tale che fossero presenti tutti i componenti;
			\item \textbf{Soluzione:} si è deciso di utilizzare un calendario condiviso per scegliere il giorno
		in cui tutto il team potesse essere presente. 
		\end{itemize}
		
		\item 2019-03-8 \textbf{Incontro di gruppo} \\
		\begin{itemize}
			\item \textbf{Problema:} si è riscontrata una certa difficoltà nel trovare un'aula libera nella giornata di disponibilità di tutti i membri del gruppo;
			\item \textbf{Soluzione:} si è deciso di organizzare l'incontro fisico solo nel caso di necessità di discussione e confronto. Organizzare chiamata di gruppo su Google Meet in caso di indisponibilità delle aule. 
		\end{itemize}
						
						
		\item 2019-03-11 \textbf{Incontro con il proponente} \\
		\begin{itemize}
			\item \textbf{Problema:} poiché l'azienda proponente ha sede a Milano, il primo incontro è stato fatto
		via Hangouts e c'è stata un'iniziale difficoltà nel trovare una sede in cui 
		trovarsi con il gruppo per effettuare la videochiamata senza essere disturbati;
			\item \textbf{Soluzione:} abbiamo organizzato la chiamata con il proponente nella giornata di disponibilità dell'aula P1D del complesso Paolotti. Inoltre, in seguito alla chiamata il proponente ci ha garantito una risposta rapida alle email, perciò si scelto l'email come mezzo principale di comunicazione con il proponente, evitando così il problema della disponibilità delle aule.
		\end{itemize}
		
		\item 2019-03-18 \textbf{Compiti individuali} \\
		\begin{itemize}
			\item \textbf{Problema:} si è riscontrata una certa difficoltà nel tracciare e sincronizzare i compitini all'interno del gruppo;
			\item \textbf{Soluzione:} si è deciso di usare Trello\glosp per tenere traccia del lavoro che ogni membro sta svolgendo al momento e segnalazione immediata sul canale Telegram\glosp del gruppo di eventuali problemi riscontrati.
		\end{itemize}
		 		
\end{itemize}

\subsubsection{Valutazione sui ruoli}
\begin{itemize}

		\item 2019-03-12 \textbf{Rivestire il ruolo di \textit{Amministratore}} \\
		\begin{itemize}
			\item \textbf{Problema:} si è riscontrato un iniziale problema con la configurazione dei comandi \LaTeX{} per la creazione dei template dei documenti, e con la suddivisione delle cartelle di lavoro;
			\item \textbf{Soluzione:} per evitare ritardi sul lavoro e per risolvere il problema in fretta, parte del gruppo si è riunita e ha dedicato circa un'ora di lavoro per la configurazione dei template e relativa suddivisione in cartelle.
		\end{itemize}	
		
		\item 2019-03-15 \textbf{Rivestire il ruolo di \textit{Responsabile}} \\
		\begin{itemize}
			\item \textbf{Problema:} a causa dell'inesperienza, chi ha lavorato come \textit{Responsabile} ha avuto discrete
		difficoltà nella suddivisione bilanciata delle ore tra i membri provocando 
		diverse ridistribuzioni delle ore;
			\item \textbf{Soluzione:} per evitare eventuali ritardi nelle consegne, il gruppo ha deciso di dedicare 
		del tempo per analizzare meglio la mole di lavoro e compiere così una più
		accurata distribuzione delle ore.
		\end{itemize}
		
						
		\item 2019-03-21 \textbf{Rivestire il ruolo di \textit{Amministratore} - Codice rischio: RO2} \\
		\begin{itemize}
			\item \textbf{Problema:} la scelta e la configurazione del software per il tracciamento dei requisiti
		ha richiesto più ore del previsto;
			\item \textbf{Soluzione:} per evitare ritardi sul lavoro, chi ha svolto il ruolo di \textit{Verificatore}
		e ha avanzato ore, ha affiancato gli \textit{Amministratori} per completare 
		i loro compiti.
		\end{itemize}
		
\end{itemize}
\subsubsection{Valutazioni sugli strumenti di lavoro}
\begin{itemize}
				
		\item 2019-03-7 \textbf{\LaTeX{}} \\
		\begin{itemize}
			\item \textbf{Problema:} a causa dell'inesperienza di alcuni membri del gruppo all'utilizzo di questo strumento, si sono riscontrate diverse difficoltà con i vari comandi di \LaTeX{}, soprattutto per la costruzione delle tabelle;
			\item \textbf{Soluzione:} per risolvere in breve tempo questa problematica, si è deciso di affiancare
		ai membri meno pratici coloro che sapevano già utilizzare i comandi di \LaTeX{}, dando
		così la possibilità ai primi di imparare e permettere ai secondi di non 
		subire grossi rallentamenti nel lavoro.
		\end{itemize}		
		
\end{itemize}
\newpage
\subsection{Revisione di progettazione}

\subsubsection{Valutazioni sull'organizzazione}
\begin{itemize}	
		\item 2019-04-26 \textbf{Compiti individuali} \\
		\begin{itemize}
			\item \textbf{Problema:} a causa dell'inesperienza tecnologica, con conseguente aumento delle ore necessarie per gli Amministratori, sono insorti altri problemi di coordinamento all'interno del gruppo;
			\item \textbf{Soluzione:} nei momenti di difficoltà, per ridurre i tempi di ritardo, gli Amministratori sono stati affiancati dai Verificatori.
		\end{itemize}
	
\end{itemize}

\subsubsection{Valutazione sui ruoli}

\begin{itemize}
\item 2019-03-29 \textbf{Rivestire il ruolo di \textit{Amministratore}} \\
		\begin{itemize}
			\item \textbf{Problema:} a causa dell'inesperienza tecnologica del gruppo, gli Amministratori hanno riscontrato gravi difficoltà nella configurazione delle tecnologie necessarie per lo svolgimento del software;
			\item \textbf{Soluzione:} il gruppo si impegna ad aumentare le ore di formazione necessarie per la comprensione delle tecnologie.
		\end{itemize}	
		
		
\end{itemize}
\subsubsection{Valutazioni sugli strumenti di lavoro}
\begin{itemize}
				
		\item 2019-05-02 \textbf{AWS} \\
		\begin{itemize}
			\item \textbf{Problema:} a causa dell'inesperienza di tutti i membri del gruppo all'utilizzo di questo strumento, si sono riscontrate grandi difficoltà nel configurare il server E2C di AWS;
			\item \textbf{Soluzione:} a causa dei tempi stringenti e dall'impossibilità del gruppo di risolvere i problemi, si è scelto di scartare l'utilizzo di AWS e valutare altre possibilità che hanno portato alla scelta di utilizzare il database realtime Firebase.
		\end{itemize}		
		\item 2019-05-02 \textbf{Gitlab - Continuous Integration - Codice rischio: RT1} \\
		\begin{itemize}
			\item \textbf{Problema:} a causa dell'inesperienza di tutti i membri, si sono riscontrare diverse difficoltà nella configurazione della CI (Continuos Integration\glo);
			\item \textbf{Soluzione:} il problema rimane ancora irrisolto. Il gruppo si impegna a cercare ulteriore documentazione e materiale formativo necessario a risolvere il problema.
		\end{itemize}
\end{itemize}