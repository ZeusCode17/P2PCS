\section{Introduzione}
\subsection{Premessa}
Si prevede di lavorare sul presente documento per l'intera durata del progetto. 
Molti dei contenuti del documento sono di natura instabile. \newline
Alcune metriche scelte non sono applicabili nella fase iniziale, e solo con il loro utilizzo pratico si può valutarne l'effettiva utilità. Anche i processi selezionati possono essere soggetti a cambiamenti, rivelandosi insufficienti o inadeguati agli scopi del progetto e al modo di lavorare del team.\newline 
Parti del documento sono prodotte in fasi temporali successive, come l'appendice sul resoconto delle verifiche.\newline 
Per tutte queste ragioni, il documento è prodotto in maniera incrementale, e i suoi contenuti iniziali sono da considerarsi incompleti: subiranno significative aggiunte e modifiche nel tempo.
\subsection{Scopo del documento}
Il seguente documento si propone di mostrare le strategie di verifica e validazione\glosp adottate al fine di garantire la qualità di processo e di prodotto desiderata. \newline
Per raggiungere tali obiettivi, il sistema qui descritto viene utilizzato in continuo sui processi in corso e sulle attività svolte. In questo modo è quindi possibile rilevare e correggere eventuali anomalie, riducendo la possibilità che possano creare ulteriori problemi e sprechi di risorse. 
\subsection{Scopo del prodotto}
Lo scopo del prodotto è quello di realizzare una applicazione Android che fornisca un servizio peer-to-peer\glosp per la condivisione delle auto. L'applicazione intende utilizzare concetti di gamification per invogliare gli utenti ad utilizzare il servizio offerto.
% PLACEHOLDER espandere
\subsection{Glossario}
Al fine di evitare possibili ambiguità relative al linguaggio utilizzato nei documenti formali, viene fornito il \textit{Glossario v1.0.0}. In questo documento vengono definiti e/o descritti tutti i termini con un significato particolare. Tali termini sono contrassegnati da una 'G' a pedice.
\subsection{Riferimenti}
\subsubsection{Riferimenti normativi}
\begin{itemize}

\item \textbf{Capitolato d'appalto C5 - P2PCS: Peer-to-peer car sharing}: \\ \url{https://www.math.unipd.it/~tullio/IS-1/2018/Progetto/C5.pdf}
%verbali normativi
\end{itemize}
\subsubsection{Riferimenti informativi}
\begin{itemize}
% Guide?(vedi Pro-tech)
\item \textbf{ISO/IEC 9126}: \\* \url{https://en.wikipedia.org/wiki/ISO/IEC_9126}
\item \textbf{ISO/IEC 12207}: \\* 
\url{https://www.math.unipd.it/~tullio/IS-1/2009/Approfondimenti/ISO\_12207-1995.pdf}
\item \textbf{Indice di Gulpease}: \\* \url{https://it.wikipedia.org/wiki/Indice_Gulpease}
\item \textbf{Schedule Variance e metriche correlate}:\\* \url{https://www.smartsheet.com/hacking-pmp-how-calculate-schedule-variance}
\end{itemize}