\section{Resoconto attività di verifica}
In questa sezione vengono descritti e analizzati gli esiti delle attività
di verifica svolte su tutti i documenti che vengono consegnati nelle varie 
revisioni di avanzamento del progetto.

\subsection{Revisione dei Requisiti}

%\subsubsection{Analisi statica dei documenti}
%L'analisi dei documenti mediante \textit{Walkthrough}\glo{} ha portato 
%all'individuazione di alcuni errori frequenti a partire dai quali è stata 
%stilata una lista di controllo interna.

\subsubsection{Esiti verifiche automatizzate}
Nelle tabelle seguente vengono riportati gli indici Gulpease\glosp e gli indici di Gunning Fog\glosp di tutti
i documenti prodotti finora.

	\rowcolors{2}{pari}{dispari}
	
		
	\begin{longtable}{ >{\centering}p{0.40\textwidth} >{\centering}p{0.25\textwidth}
			 >{\centering}p{0.2075\textwidth}}
		\caption{Esiti verifiche automatizzate RR - Indice di Gulpease} \\
		%\hline
		\rowcolorhead
		\centering\textbf{\color{white}Documento} 
		& \centering\textbf{\color{white}Indice Gulpease} 
		& \centering\textbf{\color{white}Esito}
		\tabularnewline %\hline 
		\endfirsthead
			
	
		\textit{Analisi dei Requisiti v1.0.0} & 52.32 & Superato
		
		\tabularnewline 
		\textit{Glossario v1.0.0} & 100 & Superato
				
		\tabularnewline 
		\textit{Norme di Progetto v1.0.0} & 57.61 & Superato
		
		\tabularnewline 
		\textit{Piano di Progetto v1.0.0} & 53.39 & Superato
		
		\tabularnewline 
		\textit{Piano di Qualifica v1.0.0} & 56.87 & Superato	
		
		\tabularnewline 
		\textit{Studio di Fattibilità v1.0.0} & 54.93 & Superato
		
		\tabularnewline 
		\textit{Verbale Esterno 2019-03-14 v1.0.0} & 80 & Superato
		
		\tabularnewline 
		\textit{Verbale Esterno 2019-03-25 v1.0.0} & 72 & Superato
		
		\tabularnewline 
		\textit{Verbale Esterno 2019-04-10 v1.0.0} & 69 & Superato
		
		\tabularnewline 
		\textit{Verbale Interno 2019-03-06 v1.0.0} & 79 & Superato
		
		\tabularnewline 
		\textit{Verbale Interno 2019-03-13 v1.0.0} & 77 & Superato
		
		\tabularnewline 
		\textit{Verbale Interno 2019-03-18 v1.0.0} & 71 & Superato
	\end{longtable}
	\subparagraph{Considerazioni}
	Tutti i documenti hanno un esito positivo. La media dei valori è di 68.54, un valore intermedio tra il limite inferiore stabilito di [40] e il limite inferiore preferibile di [80]. \\
	I valori più alti sono riscontrati nei verbali e glossario, mentre i valori più bassi sono presenti nell'\texttt{AdR} e nel \texttt{PdP} ma data la loro natura tecnica, riteniamo soddisfacente l'esito della verifica.
	
	
	\rowcolors{2}{pari}{dispari}
	\begin{longtable}{ >{\centering}p{0.40\textwidth} >{\centering}p{0.25\textwidth}
			 >{\centering}p{0.2075\textwidth}}
		\caption{Esiti verifiche automatizzate RR - Indice di Gunning Fog} \\
		%\hline
		\rowcolorhead
		\centering\textbf{\color{white}Documento} 
		& \centering\textbf{\color{white}Indice Gunning Fog} 
		& \centering\textbf{\color{white}Esito}
		\tabularnewline %\hline 
		\endfirsthead
			
	
		\textit{Analisi dei Requisiti v1.0.0} & 13.06 & Superato
		
		\tabularnewline 
		\textit{Glossario v1.0.0} & 12.32 & Superato
				
		\tabularnewline 
		\textit{Norme di Progetto v1.0.0} & 11.70  & Superato
		
		\tabularnewline 
		\textit{Piano di Progetto v1.0.0} & 12.29 & Superato
		
		\tabularnewline 
		\textit{Piano di Qualifica v1.0.0} & 12.71 & Superato	
		
		\tabularnewline 
		\textit{Studio di Fattibilità v1.0.0} & 11.28 & Superato
		
		\tabularnewline 
		\textit{Verbale Esterno 2019-03-14 v1.0.0} & 14.81 & Superato
		
		\tabularnewline 
		\textit{Verbale Esterno 2019-03-25 v1.0.0} & 15.59 & Superato
		
		\tabularnewline 
		\textit{Verbale Esterno 2019-04-10 v1.0.0} & 15.77  & Superato
		
		\tabularnewline 
		\textit{Verbale Interno 2019-03-06 v1.0.0} & 13.96 & Superato
		
		\tabularnewline 
		\textit{Verbale Interno 2019-03-13 v1.0.0} & 15.44 & Superato
		
		\tabularnewline 
		\textit{Verbale Interno 2019-03-18 v1.0.0} & 15.62 & Superato
	\end{longtable}
	\subparagraph{Considerazioni}
	Tutti i documenti hanno superato la verifica con un esito inferiore al limite massimo imposto di 16. 
	Con una media di 13,71, in cui i documenti più rilevanti hanno un esito vicino al valore preferibile di 12, riteniamo che la verifica abbia riportato dei risultati soddisfacenti. 
	
\subsubsection{Esiti verifiche non automatizzate}

	\subparagraph{Verifica delle metriche di processo}
	Di seguito viene riportata la tabella con gli esiti delle metriche di processo
	\rowcolors{2}{pari}{dispari}	
	\begin{longtable}{ >{\centering}p{0.15\textwidth} >{\centering}p{0.3\textwidth}
			 >{\centering}p{0.195\textwidth} >{\centering}p{0.195\textwidth}}
		\caption{Esiti delle metriche di processo - Revisione dei Requisiti} \\
		%\hline
		\rowcolorhead
		\centering\textbf{\color{white}Codice metrica} 
		& \centering\textbf{\color{white}Breve descrizione} 
		& \centering\textbf{\color{white}Valore} 
		& \centering\textbf{\color{white}Esito}
		\tabularnewline %\hline 
		\endfirsthead
		
		\texttt{MPR1}  & Percentuale di requisiti obbligatori soddisfatti & Non calcolabile & Non superato
		\tabularnewline 
		
		\texttt{MPR2} & Coupling Between Object classes & Non calcolabile & Non superato
		\tabularnewline
		
		\texttt{MPR3} & Planned Value & 4.688,00 & Superato
		\tabularnewline
		
		\texttt{MPR4} & Actual Cost & 4.833,00 & Superato
		\tabularnewline
		
		\texttt{MPR5} & Earned Value & 4,688.00 & Superato
		\tabularnewline
		
		\texttt{MPR6} & Budget at Completion & 4.688,00 & Superato
		\tabularnewline
		
		\texttt{MPR7} & Cost Variance & -145.00 & Non superato
		\tabularnewline
		
		\texttt{MPR8} & Schedule Variance & 0,00 & Superato 
		\tabularnewline
		
		\texttt{MPR9} & Code Coverage & Non calcolabile & Non superato
		\tabularnewline
		
		\texttt{MPR10} & Indice Gunning fog (media) & 13,71 & Superato
		\tabularnewline
		
		\texttt{MPR11} & Indice di Gulpease (media) & 68,54 & Superato
		\tabularnewline
		
		\texttt{MPR12} & Correttezza ortografica & 0 & Superato
		\tabularnewline
		
		\texttt{MPR13} & Percentuale di metriche soddisfatte & 88.89 & Superato
		\tabularnewline
		
	\end{longtable}
	\subparagraph{Considerazioni}
	Ci sono delle metriche non calcolabili in questa fase. Tali metriche si riferiscono all'analisi del codice e alla sua progettazione che verrà svolta nelle fasi successive.
	Risulta invece negativa la metrica di Cost Variance, in conseguenza dell'impiego di 8 ore di lavoro aggiuntive rispetto a quanto previsto. \\
	La percentuale di metriche soddisfatte tiene conto solo delle metriche calcolabili, quindi di un totale di 9 metriche di cui 8 soddisfatte; se si ritiene che le metriche non calcolabili siano da considerare non superate, la percentuale di metriche soddisfatte risulta comunque superata, per un valore di 66.67.
	\subparagraph{Verifica delle metriche di prodotto} 

	Le metriche di prodotto non possono essere valutate al momento.


\subsubsection{Esiti dei test}
	Al momento i test non sono implementati, perciò non verificabili. 
	
\subsubsection{Esito della Revisione dei Requisiti}
	