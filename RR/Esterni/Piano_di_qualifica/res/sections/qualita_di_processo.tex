\section{Qualità di processo}
Sono stati inizialmente scelti, come riferimento, i processi proposti dallo standard ISO/IEC/IEEE 12207:1995 e successivamente semplificati o adattati secondo le esigenze.

%\subsection{Processo di Costruzione del Software}. Opzione per il futuro.
	\subsection{Processo di Fornitura}
	Lo scopo del processo di fornitura è fornire un prodotto e/o servizio all'acquirente che soddisfi i requisiti accordati.
	Per garantire la qualità del processo di fornitura sono state adottate le attività proposte dallo standard ISO/IEC 12207:1995 [§5.2] che stabilisce una linea guida per:
	\begin{itemize}
		\item produrre una riposta alle richieste degli stakeholders\glo ;
		\item stabilire un accordo con la proponente per lo sviluppo, il mantenimento, il funzionamento, consegna e installazione del prodotto e/o del servizio;
		\item sviluppare un prodotto e/o servizio che soddisfa i requisiti concordati;
		\item consegnare il prodotto sviluppato.
	\end{itemize}
	\subsubsection{Obiettivi}
	\begin{itemize}
		\item effettuare un'analisi delle richieste del capitolato e valutare la sua fattibilità;
		\item definire una proposta in risposta alle richieste del proponente;
		\item negoziare e stipulare un contratto con la proponente per la fornitura del prodotto;
		\item avviare le attività di pianificazione che comprendono:
			\begin{itemize}
				\item definire un framework\glosp per la gestione del progetto e per garantire la qualità del prodotto;
				\item valutare le risorse necessarie per lo svolgimento del progetto;
				\item valutare e documentare i rischi di progetto;
				\item valutare e documentare i requisiti per garantire la qualità dei processi e prodotti;
				\item stabilire i metodi di verifica e validazione.
			\end{itemize}
	\end{itemize}
	\subsubsection{Strategia}
	\begin{itemize}
		\item interagire con la proponente per comprendere a fondo i requisiti del capitolato commissionato;
		\item condurre le analisi necessarie per adempiere agli obiettivi sopra descritti e tenerne traccia in appositi documenti;
		\item interagire continuamente con la proponente per ottenere feedback sulle decisioni riguardanti il prodotto.
	\end{itemize}
	\subsection{Processo di Sviluppo}
	Il processo di sviluppo contiene tutte le attività tipiche dello sviluppo software. Alcune di esse sono molto importanti per cui vengono approfondite in: 
	\begin{itemize}
		\item analisi dei requisiti;
		\item progettazione architetturale;
		\item progettazione di dettaglio.
	\end{itemize}
	\subsubsection{Analisi dei Requisiti}
	Durante l'analisi dei requisiti le informazioni raccolte dalle varie fonti sono trasformate in forma di casi d'uso e requisiti.
	Questa forma fornisce una descrizione dettagliata del sistema e definisce il funzionamento e le caratteristiche di ogni sua parte.
		\paragraph{Obiettivi}
		\begin{itemize}
			\item formulare la definizione di casi d'uso e requisiti;
			\item ottenere la loro approvazione;
			\item tracciare il loro cambiamento nel tempo.
		\end{itemize}		
		\paragraph{Strategia}
		\begin{itemize}
			\item considerare lo scopo del progetto e le richieste degli stakeholder\glo;
			\item esprimere ciò in forma di requisiti, classificati in obbligatori/desiderabili/facoltativi;
			\item valutare il corpo dei requisiti e negoziare cambiamenti se necessario;
			\item ottenere la loro approvazione da parte del proponente;
			\item disporre del tracciamento dei requisiti del sistema.
		\end{itemize}
		
		\paragraph{Metriche}
			\subparagraph{PROS: Percentuale di requisiti obbligatori soddisfatti} Indica la percentuale di requisiti obbligatori soddisfatti.
			\begin{itemize}
				\item misurazione: (valore percentuale) $ PROS = \frac{requisiti\ obbligatori\ soddisfatti}{requisiti\ obbligatori\ totali}$;
				\item valore preferibile: $100\%$;
				\item valore accettabile: $100\%$.
			\end{itemize}
	
			
	\subsubsection{Progettazione di Dettaglio}
	La progettazione di design, tramite una suddivisione delle macro-component, in componenti più piccole evidenzia i seguenti benefici:
	\begin{itemize}
		\item facilmente comprensibili;
		\item strettamente collegati ai requisiti funzionali;
		\item implementabili da un singolo programmatore.
	\end{itemize}
		\paragraph{Obiettivi}
		\begin{itemize}
			\item tradurre i requisiti in unità di codice (moduli);
			\item facilitare il lavoro dei programmatori con compiti individuali relativi ai singoli moduli;
			\item produrre un sistema software da raffinare ma già eseguibile;
			\item mantenere il tracciamento tra requisiti e componenti.
		\end{itemize}
		\paragraph{Strategia}
		\begin{itemize}
			\item scomporre le componenti architetturali in componenti piccole;
			\item implementare le micro-componenti individuate.
		\end{itemize}
		\paragraph{Metriche}
			\subparagraph{CBO - Coupling Between Object classes} 
			Grado di accoppiamento tra le classi: una classe è accoppiata quando i propri metodi usano metodi o variabili di istanza definiti in un'altra classe. 
			\begin{itemize}
				\item misurazione: (valore intero) $CBO$;
				\item valore preferibile: $0 \leq CBO \leq 2$;
				\item valore accettabile: $0 \leq CBO \leq 6$.
			\end{itemize}
		
\subsection{Processi di Supporto}
	\subsubsection{Pianificazione}
	La pianificazione è un'attività fondamentale della gestione di progetto. Consiste nel disporre in modo efficiente le risorse a disposizione, monitorarle nel tempo ed essere pronti ai cambiamenti. La pianificazione si esprime specificatamente all'interno del Piano di Progetto.
		\paragraph{Obiettivi}
		\begin{itemize}
			\item avere a disposizione piani e obiettivi ben definiti;
			\item aver definito ruoli, responsabilità, obblighi e autorità a cui rispondere;
			\item aver allocato le risorse e i beni necessari;
			\item attivare il piano per sostenere il progetto.
		\end{itemize}	
		\paragraph{Strategia}
		\begin{itemize}
			\item produrre la pianificazione delle attività;
			\item mantenerla aggiornata mentre esse vengono svolte;
			\item usarla come riferimento e supporto.
		\end{itemize}		
		\paragraph{Metriche}
		\subparagraph{PV: Planned Value}
		Valore del lavoro pianificato per realizzare le attività di progetto fino al momento del calcolo.
		 \textit{(Metrica di utilità per il calcolo di $SV$ spiegata di seguito.)}
			\begin{itemize}
				\item  misurazione: $BAC \cdot \%\ di\ lavoro\ pianificato\ $;
				\item  valore preferibile: $ \geq 0$;
				\item  valore accettabile: $ \geq 0$.
			\end{itemize}	
		\subparagraph{AC: Actual Cost}
			Costo effettivamente sostenuto fino al momento del calcolo.
			\begin{itemize}
				\item  misurazione: numero intero;
				\item  valore preferibile: $0 \leq AC < PV$;
				\item  valore accettabile: $0 \leq AC \leq budget\ totale\ $.
			\end{itemize}
		\subparagraph{EV: Earned Value}
		Valore del lavoro realizzato fino al momento del calcolo.
		 \textit{(Metrica di utilità per il calcolo di $SV$ e $CV$ spiegate di seguito.)}
			\begin{itemize}
				\item  misurazione: $BAC \cdot \%\ di\ lavoro\ completato\ $;
				\item  valore preferibile: $ \geq 0$;
				\item  valore accettabile: $ \geq 0$.
			\end{itemize}
		\subparagraph{BAC: Budget at Completion}
		Valore previsto per la realizzazione del progetto (valore iniziale previsto).
		\begin{itemize}
			\item misurazione: numero intero;
			% PLACEHOLDER: Aggiornare i 2 item qui sotto
			\item valore preferibile: $pari\ al\ preventivo$;
			\item valore accettabile: $preventivo -5\% \leq BAC \leq preventivo + 5\%$. 
		\end{itemize}
		\subparagraph{CV: Cost Variance}
			Indicatore di produttività. Una CV positiva indica che si sta rispettando il budget, viceversa se negativo.
			\begin{itemize}
				\item misurazione: $CV = EV - AC$;
				\item valore preferibile: $ > 0$;
				\item valore accettabile: $ \geq 0$.
			\end{itemize}
		\subparagraph{SV: Schedule Variance}
			Indica se si è in linea, in anticipo o in ritardo nello svolgimento del progetto rispetto alla pianificazione.
			\begin{itemize}
				\item misurazione: $SV = EV - PV$;
				\item valore preferibile: $ > 0$;
				\item valore accettabile: 0.
			\end{itemize}
			
	\subsubsection{Verifica}
	Il processo consiste nella ricerca e correzione di anomalie e difetti nei processi e prodotti del progetto, mediante tecniche predefinite e ove possibile automatiche.
		\paragraph{Obiettivi}
		\begin{itemize}
			\item individuare e correggere le anomalie;
			\item provare che il sistema soddisfi i requisiti.
		\end{itemize}	
		\paragraph{Strategia}
		\begin{itemize}
			\item individuare tecniche e strumenti di verifica da applicare;
			\item affinarli con l'esperienza;
		\end{itemize}	
		\paragraph{Metriche}
			\subparagraph{CC: Code Coverage}
				Indica il numero di righe di codice percorse dai test durante la loro esecuzione. 
				\begin{itemize}
					\item misurazione: (valore percentuale) $CC = \frac{linee\ di\ codice\ percorse}{linee\ di\ codice\ totali}$;
					\item valore preferibile: $100\%$;
					\item valore accettabile: $75\%$.
				\end{itemize}
			% a grana più fine ci sarebbero: function coverage, loop coverage, branch coverage

	\subsubsection{Documentazione}
	La documentazione consiste nella produzione di informazioni riguardanti il progetto e nella loro gestione. I documenti sono prodotti a supporto di tutte le attività di progetto.
		\paragraph{Obiettivi}
			Si costruisce la documentazione affinché costituisca un body of knowledge\glosp % PLACEHOLDER: glossario
			che raccoglie la conoscenza in modo:
			\begin{itemize}
				\item completo e non ambiguo;
				\item modulare e fatto di parti coese;
				\item trasparente, adatto alla trasmissione delle informazioni;
				\item disponibile esternamente.
			\end{itemize}
		\paragraph{Strategia}
		I documenti sono:
		\begin{itemize}
			\item prodotti in concomitanza con tutte le attività di sviluppo;
			\item prodotti in modo collaborativo;
			\item supportati da un glossario;
			\item supportati dalle \textit{Norme di Progetto};
			\item prodotti con strumenti software adatti alla collaborazione e alla modularità in \LaTeX{};
			\item ospitati in una repository\glosp pubblica su GitHub\glo.
		\end{itemize}
		\paragraph{Metriche}
			\subparagraph{Indice Gunning fog}
			Indice della leggibilità del testo. Quanto più alto è il valore dell'indice tanto più difficile è capire il documento.					\newline
			Ha dei limiti: uno di questi è la sua definizione di \textit{parola complessa}, intesa come una parola di tre o più sillabe eccetto alcuni suffissi comuni.
			\begin{itemize}
				\item misurazione: (valore intero) $ I_{GF} = 0.4 \cdot
				(
				\frac{numero\ di\ parole}{numero\ di\  frasi}
				+ 100 \cdot
				\frac{numero\ di\ parole\ complesse}{numero\ di\  frasi}
				) $;
				\item valore preferibile: $ \leq 12$;
				\item valore accettabile: $ \leq 16$.
			\end{itemize}
			\subparagraph{Indice di Gulpease}
			Indice della leggibilità del testo. Valuta la lunghezza delle parole e delle frasi rispetto al numero totale di lettere. 
			\begin{itemize}
				\item misurazione: (valore intero da 0 a 100)\newline 	
				$I_G = 89+ \frac{(300 \cdot numero\ di\ frasi - 10 \cdot numero\ di\ lettere)}{numero\ di\ parole}$;	
				\item valore preferibile: $80 < I_G < 100$;
				\item valore accettabile: $40 < I_G < 100$.
			\end{itemize}
			\subparagraph{Correttezza ortografica}
			Tutti i documenti non devono assolutamente contenere errori grammaticali o errori ortografici. 
			\begin{itemize}
				\item misurazione: (valore intero) numero di errori grammaticali o ortografici per documento;
				\item valore preferibile: 0;
				\item valore accettabile: 0.
			\end{itemize}
			
\subsection{Processi Organizzativi}
	\subsubsection{Gestione della Qualità}
	Tale processo ha per scopo il raggiungimento di un grado soddisfacente di qualità nel progetto. Esso fornisce:
	\begin{itemize}
		\item obiettivi da perseguire;
		\item strumenti tecnici, come procedure e/o metriche.
	\end{itemize}
		\paragraph{Obiettivi}
			\begin{itemize}
				\item garantire che i prodotti e processi rispettino gli standard di qualità richiesti.
			\end{itemize}
		\paragraph{Strategia}
		\begin{itemize}
			\item pianificare le proprie azioni perseguendo gli obiettivi;
			\item utilizzare gli strumenti forniti per misurare e monitorare i risultati;
			\item reagire ai risultati aggiornando obiettivi, strategia e strumenti.
		\end{itemize}
		\paragraph{Metriche}
			\subparagraph{PMS: Percentuale di metriche soddisfatte}
			La percentuale di metriche soddisfatte valuta quante metriche raggiungono soglie accettabili sul numero totale delle metriche calcolate. Una bassa percentuale di soddisfazione può indicare poca qualità, metriche inadeguate o mancata correttezza nel calcolo.
			\begin{itemize}
				\item misurazione: $\frac{numero\ di\ metriche\ soddisfatte}{numero\ di\ metriche\ totali} $;
				\item valore preferibile: $ \geq 80\%$;
				\item valore accettabile: $ \geq 60\%$.
			\end{itemize}
		