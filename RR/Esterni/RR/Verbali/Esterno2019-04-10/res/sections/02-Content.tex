\section{Verbale della riunione}
\begin{itemize}
	\item Fare di \textit{GaiaGo} un servizio che incentivi gli utenti (proprietari) a condividere la propria auto ad un prezzo minimo. In cambio riceveranno vari premi come sconti per l'e-commerce, viaggi gratuiti con \textit{GaiaGo} e oggetti virtuali da usare all'interno dell'applicazione;
	\item Elenco delle proposte di gamification\glo:
		\begin{itemize}
			\item ruota della fortuna (Lucky Spin) e/o regalo misterioso (Mistery Box) contenenti premi che l'utente potrà ricevere gratuitamente dopo aver concluso una prenotazione. Nella ruota della fortuna ci può essere anche la possibilità di ottenere un viaggio gratis con \textit{GaiaGo} oppure la possibilità di non vincere nulla;
			\item tutorial guidato all'utilizzo dell'applicazione e/o dell'eventuale minigioco, presente all'interno di quest'ultima, da fare prima o dopo la registrazione;
			\item invio di un "codice amico", come invito da parte di un utente, da utilizzare nella fase di registrazione all'applicazione con conseguente aumento di punti per entrambe le parti;
			\item creare una sezione "utenti preferiti" dove inserire gli utenti favoriti con i quali si vorrebbe ricontrattare; 
			\item Mini gioco: 
				\begin{itemize}
					\item formato da un avatar modificabile disponibile per tutti gli utenti, mentre per chi aggiunge il suo veicolo, introduzione di un garage in cui la macchina, tramite gettoni e premi, possa venire modificata nell'aspetto e nel colore;
					\item mappa statica con edifici sbloccabili solo a raggiungimento di livelli prestabiliti, dove l'utente possa potenziarsi utilizzando sempre più l'applicazione;
					\item utilizzo di materiali per creare oggetti con tempi e costi predefiniti. 
				\end{itemize}
			\item notifiche e gestione notifiche che avvisino dell'inizio o completamento di vari eventi nell'applicazione, tra cui anche l'inutilizzo di quest'ultima;
			\item classifica utenti in base ai punti esperienza ottenuti, alle valutazioni ricevute oppure in base alle prestazioni della macchina nel mini gioco. Decidere se classificarle per regione, città, paese;
			\item progress bar;
			\item punti esperienza/gettoni accumulati in base a quanto viene utilizzata l'applicazione;
			\item rewards giornalieri;
			\item recensioni agli utenti;
			\item easter egg;
			\item milestone: semplici obbiettivi da completare all'interno dell'applicazione che fruttano premi;
		\end{itemize}
	\item Utilizzare AWS\glosp come servizio alternativo alla piattaforma Movens\glosp per il back end\glosp dell'applicazione.
\end{itemize} 
\pagebreak
\section{Riepilogo delle decisioni}

	%\renewcommand{\arraystretch}{1.5}
	\rowcolors{2}{pari}{dispari}
	
	\begin{longtable}{ >{\centering}p{0.20\textwidth} >{}p{0.70\textwidth}}
		\caption{Decisioni della riunione interna del 2019-04-09}\\	
		\rowcolorhead
		\textbf{\color{white}Codice} 
		& \centering\textbf{\color{white}Decisione} 
		\tabularnewline 
		\endfirsthead
		VE\_3.1 & Scelto AWS\glosp come piattaforma di back end\glosp in sostituzione a Movens\glo.
		
		\tabularnewline 
		VE\_3.2 & Ricevuto feedback delle proposte con varie soluzioni per alcune di esse.
		
		\tabularnewline 
		VE\_3.3 & Scelto di inserire la Lucky Spin\glosp come proposta di gamification\glo.
	
		\tabularnewline 
		VE\_3.4 & Scelto di inserire il tutorial dell'applicazione una volta registrati.
		
		\tabularnewline 
		VE\_3.5 & Scelto di inserire il "codice amico" come metodo di invito amici all'utilizzo dell'applicazione.
		
		\tabularnewline 
		VE\_3.6 & Scelto di inserire la sezione utenti preferiti.
		
		\tabularnewline 
		VE\_3.7 & Scelto di inserire il mini gioco e notifiche riguardati l'applicazione e tutto quello che ne consegue.
	\end{longtable}
	




