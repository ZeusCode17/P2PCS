\section{Verbale della riunione}
\begin{itemize}
	\item Core-drive\glosp di gamification\glosp riguardo alla parte social con opportunità di una chat di messaggistica supportata dalla piattaforma esterna Firebase\glo;
	\item Elenco delle proposte di gamification al proponente per avere un feedback riguardante:
		\begin{itemize}
			\item classifica utenti in base ai punti esperienza ottenuti e alle valutazioni ricevute;
			\item progress bar per incentivare il completamento dei dati personali dell'account;
			\item punti esperienza/gettoni accumulati in base a quanto viene utilizzata l'applicazione (km percorsi, obiettivi raggiunti, completamento della progress bar, challenges superate);
			\item avatar personalizzabile con abiti acquistabili con i gettoni accumulati;
			\item rewards giornalieri: si ricevono dei gettoni e/o abiti per l'avatar anche per la frequenza di utilizzo dell'applicazione;
			\item chat per la comunicazione tra utenti;
			\item challenge: gli utenti sono stimolati dal prestare/prendere in prestito i veicoli in quanto sono presenti varie challenge (ad esempio: percorri  20km, presta la macchina a 5 utenti);
			\item come già detto gli utenti possono dare un rating agli utenti con cui hanno avuto un contatto;
			\item (opzionale) mini gioco: si tratta di un garage in cui è possibile acquistare veicoli diversi e personalizzarli;
			\item easter egg: personalizzazioni di avatar e veicolo nel minigioco aggiuntive in base alle festività.
		\end{itemize}
	\item Compilazione foglio Excel con schema per valutare i core-drive di gamification;
	\item Presenza o meno di requisiti prestazionali da rispettare per lo sviluppo dell'applicazione;
	\item Versione di Android\glosp da supportare per il progetto.
\end{itemize} 
\pagebreak
\section{Riepilogo delle decisioni}

	%\renewcommand{\arraystretch}{1.5}
	\rowcolors{2}{pari}{dispari}
	
	\begin{longtable}{ >{\centering}p{0.20\textwidth} >{}p{0.70\textwidth}}
		\caption{Decisioni della riunione interna del 2019-03-22}\\	
		\rowcolorhead
		\textbf{\color{white}Codice} 
		& \centering\textbf{\color{white}Decisione} 
		\tabularnewline 
		\endfirsthead
		VE\_2.1 & Scelto di non utilizzare una chat di messaggistica all'interno dell'applicazione.
		
		\tabularnewline 
		VE\_2.2 & Ricevuto feedback delle proposte con varie soluzioni per alcune di esse.
		
		\tabularnewline 
		VE\_2.3 & Scelto programma Office Excel per la compilazione dello schema per i core-drive\glosp di gamification\glo.
	
		\tabularnewline 
		VE\_2.4 & Presenza di un requisito prestazionale riguardante l'impiego di tempo per il caricamento dell'applicazione.
		
		\tabularnewline 
		VE\_2.5 & Nessun vincolo riguardo alla versione Android\glosp da utilizzare per il progetto.
	\end{longtable}
	




