\subsection*{\quad$C\quad$}
\subsubsection*{CamelCase}
\index{CamelCase}
Pratica di scrivere una parola unendone delle altre lasciando le loro iniziali maiuscole.

\subsubsection*{Capitolato}
\index{Capitolato}
Il capitolato è un documento tecnico, in genere allegato ad un contratto di appalto e parte di esso. Serve a definire le specifiche tecniche delle opere che andranno ad eseguirsi per effetto del contratto stesso.

\subsubsection*{Changelog}
\index{Changelog}
Tabella delle modifiche di un documento. Indica la versione del documento, la data di modifica, chi ha prodotto
la modifica e una descrizione della stessa.

\subsubsection*{Coesione}
\index{Coesione}
Stretta unione di parti che concorrono alla stessa funzionalità, allo stesso obiettivo: le parti coese sono tutto il necessario e nulla di superfluo.

\subsubsection*{Comportamento emergente}
\index{Comportamento emergente}
Nuovo comportamento del software nato dall'integrazione delle parti, spesso non previsto dagli sviluppatori delle singole parti.

\subsubsection*{Continuous delivery}
\index{Continuous delivery}
Definisce la metodologia attraverso la quale si hanno brevi cicli di rilascio. In questo modo è possibile rilasciare in modo affidabile il software in qualsiasi momento.

\subsubsection*{Continuous integration}
\index{Continuous integration}
E’ una pratica di sviluppo che consiste nell’integrare frequentemente le modifiche locali degli sviluppatori al
ramo condiviso principale. Questa pratica permette di minimizzare problemi di integrazione corposa.

\subsubsection*{Copertura}
\index{Copertura}
Nell'analisi dei requisiti, è la quantità di requisiti soddisfatti. Nei test del codice, è la quantità di righe percorse durante il test. Si ricerca copertura massima.


