\subsection*{\quad$S\quad$}
\subsubsection*{Sass}
\index{Sass}
Syntactically Awesome StyleSheets, è un'estensione del linguaggio CSS che permette di utilizzare variabili, di creare funzioni e di organizzare il foglio di stile in più file. 
Sigla per "Syntactically Awesome Style Sheets", è un linguaggio interpretato o compilato in CSS dotato di una sintassi che semplifica ed estende le funzionalità offerte da CSS. Supporta anche la nuova sintassi SCSS\glo.

\subsubsection*{SCSS}
\index{SCSS}
Sassy CSS, sintassi per l'estensione del linguaggio CSS, che ne aumenta le funzionalità e l'espressività.

\subsubsection*{Server-side}
\index{Server-side}
Lato server: fa riferimento a operazioni compiute dal server in un ambito client-server. 

\subsubsection*{Shortcut}
\index{Shortcut}
Scorciatoia: una maniera alternativa e più veloce per svolgere la stessa attività.


\subsubsection*{Slack}
\index{Slack}
Sistema di messaggistica cloud-based orientato al mondo lavorativo. Permette la creazioni di diverse aree di lavoro, suddivise in canali che trattano un particolare topic.

\subsubsection*{Smart contract}
\index{Smart contract}
Protocolli per facilitare, attuare e verificare la negoziazione di un contratto in versione digitale. Permettono di ottenere lo stesso valore di un contratto reale senza l'ausilio di un garante esterno. Le transazioni che avvengono con questo protocollo sono tracciabili e irreversibili. Uno smart contract rappresenta del codice che può essere eseguito.

\subsubsection*{Snake case}
\index{Snake case}
Convenzione per i nomi dei file. Utilizza "underscore" come carattere separatore, solitamente con le prime lettere delle singole parole in minuscolo, e la prima lettera dell'intero identificatore minuscola o maiuscola.
Nell'accezione del team, le lettere sono tutte minuscole e le preposizioni non si omettono.

\subsubsection*{Solidity}
\index{Solidity}
Linguaggio di programmazione orientato agli oggetti per la scrittura di smart contracts\glosp. Viene utilizzato per l'implementazione di contratti intelligenti su varie piattaforme blockchain\glo.

\subsubsection*{SonarQube}
\index{SonarQube}
Piattaforma open source che permette di svolgere una continua revisione del codice caricato in una repository, svolgendo analisi statica per individuare possibili bug e vulnerabilità in sicurezza.

\subsubsection*{Staging}
\index{Staging}
Fase successiva alla fase di sviluppo del software che consiste nell'assemblare tutte le componenti e testarle in un server per svolgere verifiche e test. Se il software possiede il comportamento desiderato allora può passare alla fase di produzione.

\subsubsection*{Stand-up}
\index{Stand-up}
Riunioni giornaliere di breve durata in cui le persone che partecipano ad un progetto espongono i progressi raggiunti e le attività pianificate per la giornata. Sono uno degli elementi principali dei metodi di sviluppo Agile\glo. 

\subsubsection*{Stub}
\index{Stub}
Nel test di unità, lo stub è un'unità fantoccio, usata a supporto l'interfaccia dell'unità da testare. Ad esempio, se l'unità da testare fa una chiamata a una funzione non ancora definita, lo stub funge funzione chiamata e ritorna un risultato conforme senza eseguire calcoli.

\subsubsection*{Surge.sh}
\index{Surge.sh}
Web server che offre il servizio di hosting per siti web statici.

