\subsection*{\quad$A\quad$}
\subsubsection*{Accomplishment}
\index{Accomplishment}
Core drive\glosp di Octalysis\glo, in cui le persone sono guidate da un senso di crescita verso un obiettivo e la sua realizzazione.

\subsubsection*{Activity}
\index{Activity}
Sono quelle classi scritte in linguaggio Java che compongono una applicazione Android e subiscono una interazione diretta con l'utente. All'avvio di ogni applicazione viene eseguita una activity, la quale può eseguire delle operazioni e può anche aprire/eseguire altre activity. Le attività create estendono la classe Activity da cui ereditano proprietà e metodi.

\subsubsection*{Add-on}
\index{Add-on}
Software aggiuntivo capace di ampliare le funzionalità di un applicativo.

\subsubsection*{Android}
\index{Android}
Sistema operativo per dispositivi mobili sviluppato da Google Inc. e basato sul kernel Linux; è un sistema embedded progettato principalmente per smartphone e tablet, con interfacce utente specializzate per televisori (Android TV), automobili (Android Auto), orologi da polso (Wear OS), occhiali (Google Glass), e altri.

\subsubsection*{Apache Kafka}
\index{Apache Kafka}
Piattaforma Open-Source\glosp di streaming distribuito, usata per stream processing, monitoraggio di dati e traffico web, messaggistica e varie operazioni su file d i log.

\subsubsection*{API}
\index{API}
Con Application Programming Interface (API) si indica un insieme di procedure e funzioni proposte ai programmatori per facilitare lo sviluppo di un programma.

\subsubsection*{Area di staging}
\index{Area di staging}
L'area di staging è un'area di memoria intermedia utilizzata per l'elaborazione dei dati durante l'estrazione, la trasformazione e il caricamento di un processo.

\subsubsection*{Artefatti}
\index{Artefatti}
Un artefatto è un sottoprodotto che viene realizzato durante lo sviluppo software. Sono artefatti i casi d'uso, i diagrammi delle classi, i modelli UML\glo, il codice sorgente e la documentazione varia.

\subsubsection*{AWS}
\index{AWS}
Amazon Web Services (AWS) è un insieme di servizi di cloud computing che compongono la piattaforma on demand offerta dall’azienda Amazon.

