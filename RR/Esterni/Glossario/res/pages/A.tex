\subsection*{\quad$A\quad$}
\subsubsection*{Activity}
\index{Activity}

\subsubsection*{Add-on}
\index{Add-on}
Software aggiuntivo capace di ampliare le funzionalità di un applicativo.

\subsubsection*{Agile}
\index{Agile}
Metodologia che si riferisce a u n insieme di metodi di sviluppo del software proponendo un approccio meno strutturato e focalizzato sull'obiettivo di consegnare al cliente, in tempi brevi e frequentemente, software funzionante e di qualità.

\subsubsection*{Apache Kafka}
\index{Apache Kafka}
Piattaforma Open-Source\glosp di streaming distribuito, usata per stream processing, monitoraggio di dati e traffico web, messaggistica e varie operazioni su file d i log.

\subsubsection*{API}
\index{API}
Con Application Programming Interface (API) si indica un insieme di procedure e funzioni proposte ai programmatori per facilitare lo sviluppo di un programma.

\subsubsection*{Approccio incrementale}
\index{Approccio incrementale}

\subsubsection*{Area di staging}
\index{Area di staging}
L'area di staging è un'area di memoria intermedia utilizzata per l'elaborazione dei dati durante l'estrazione, la trasformazione e il caricamento di un processo.

\subsubsection*{Artefatti}
\index{Artefatti}
Un artefatto è un sottoprodotto che viene realizzato durante lo sviluppo software. Sono artefatti i casi d'uso, i diagrammi delle classi, i modelli UML\glo, il codice sorgente e la documentazione varia.

\subsubsection*{Attore}
\index{Attore}
Entità che interagisce con il sistema per svolgere delle attività. Può essere utente umano, altri sistemi software, dispositivi hardware e così via.

\subsubsection*{AWS}
\index{AWS}
Amazon Web Services (AWS) è un insieme di servizi di cloud computing che compongono la piattaforma on demand offerta dall’azienda Amazon.

