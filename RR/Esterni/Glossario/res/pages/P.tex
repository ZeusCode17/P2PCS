\subsection*{\quad$P\quad$}
\subsubsection*{Partizione di equivalenza}
\index{Partizione di equivalenza}
Nei test case, insieme di valori che sono trattati uniformemente: valori validi, valori di contorno della partizione o valoro non validi.

\subsubsection*{Peer-to-peer}
\index{Peer-to-peer}
Un sistema distribuito dove non c'è distinzione tra client e server. I computer nel sistema possono operare sia come cliente sia come server. 

\subsubsection*{Plug-in}
\index{Plug-in}
Componente aggiuntivo che può essere aggiunto a un'altro software per ampliarne le funzionalità. Di solito può essere eseguito in modo indipendente.

\subsubsection*{PoS tagging}
\index{PoS tagging}
Part-of-Speech tagging, è il processo di marcatura di una parola in un testo (corpus) come corrispondente a una particolare Part of Speech\glosp , in base sia alla sua definizione che al suo contesto.

\subsubsection*{Product Baseline}
\index{Product Baseline}
Indica un punto d’arrivo tecnico dal quale non si retrocede e serve da base per gli avanzamenti futuri. La baseline è fatta di elementi di configurazione versionati. Può essere cambiata solo tramite procedure di controllo di cambiamento.

\subsubsection*{Proof of Concept}
\index{Proof of Concept}
È un prototipo software che dimostra che il progetto è fattibile conformemente alle richieste.

\subsubsection*{Pull request}
\index{Pull request}
Funzionalità di GitHub. Permette di comunicare agli altri membri del team i cambiamenti caricati nella repository\glosp di GitHub.

\subsubsection*{Python}
\index{Python}
Linguaggio di programmazione general purpose, interpretato, ad alto livello. Gode di ampia diffusione per la sua semplicità e l'ampia collezione di librerie che lo supportano.

