\subsection*{\quad$G\quad$}
\subsubsection*{Gamification}
\index{Gamification}
Riutilizzo di concetti ed elementi tipici dei giochi in applicazioni di diverso contesto. Utilizzando questi principi si ottengono sia maggiore coinvolgimento degli utenti dell'applicazione, sia maggior produttività ed organizzazione nel lavoro.

\subsubsection*{Gantt}
\index{Gantt}
Il diagramma di Gantt è uno strumento di supporto alla gestione dei progetti. Usato principalmente nelle attività di project management, è costruito partendo da un asse orizzontale, che rappresenta l'arco temporale totale del progetto, suddiviso in fasi incrementali; e da un asse
verticale che rappresenta le mansioni o attività che costituiscono il progetto.

\subsubsection*{Gigacore}
\index{Gigacore}
Boilerplate\glosp per lo sviluppo di applicazioni che fanno uso di React\glo, Redux\glosp e Sass\glo.

\subsubsection*{GitHub}
\index{GitHub}
Servizio di hosting\glosp per progetti software. Il sito è principalmente utilizzato dagli sviluppatori, che caricano il codice sorgente dei loro programmi e lo rendono scaricabile dagli utenti. Questi ultimi possono interagire con lo sviluppatore tramite un sistema di issue tracking, pull request e commenti che permette di migliorare il codice del repository\glosp risolvendo bug o aggiungendo funzionalità.

\subsubsection*{GitLab}
\index{GitLab}
Sistema di gestione repository online, che integra servizi quali issue tracking e l’integrazione di pipeline volte a continuos delivery\glosp e continuos integration\glo.

\subsubsection*{Grafana}
\index{Grafana}
Software Open-Source\glosp per il monitoraggio di sistemi che, ricevuti dati, consente di raccoglierli in grafici che si possono visualizzare, analizzare, misurare e controllare.

\subsubsection*{GUI}
\index{GUI}
Tipo di interfaccia utente che consente l'interazione uomo-macchina in modo visuale utilizzando rappresentazioni grafiche piuttosto che utilizzando una interfaccia a riga di comando.

