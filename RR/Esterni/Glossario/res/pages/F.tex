\subsection*{\quad$F\quad$}
\subsubsection*{Firebase}
\index{Firebase}
Piattaforma di sviluppo di applicazioni mobile e web.

\subsubsection*{Firebase Storage}
\index{Firebase Storage}
Piattaforma offerta da Google che permette a siti web ed applicazioni di salvare i dati dell'utente in diversi formati. Consente upload e download sicuri, anche con una connessione di scarsa qualità.

\subsubsection*{Fragment}
\index{Fragment}
I Fragment costituiscono senz'altro uno dei più importanti elementi per la creazione di una interfaccia utente Android. Un Fragment è una porzione di Activity\glo. Non si tratta solo di un gruppo di controlli o di una sezione del layout. Può essere definito più come una specie di sub-activity con un suo ruolo funzionale molto importante ed un suo ciclo di vita. 

\subsubsection*{Framework}
\index{Framework}
Un framework è un'architettura logica di supporto su cui un software può essere progettato e realizzato. Può includere vari componenti, come librerie, compilatori ed API\glo, tutte atte a migliorare il processo di sviluppo software da parte del programmatore.

\subsubsection*{Freeling}
\index{Freeling}
Software specializzato nel Part of Speech (PoS) tagging\glo. Il software sfrutta delle tecniche di apprendimento automatico supervisionato per tale classificazione.  

\subsubsection*{Front end}
\index{Front end}
Il front end è la parte di un'applicazione con la quale l'utente interagisce direttamente, è responsabile dell'acquisizione dei dati di ingresso e della loro elaborazione. Tali dati sono poi utilizzabili dal back end\glo. 

