\subsection*{\quad$R\quad$}

\subsubsection*{React}
\index{React}
Libreria Open-Source\glosp per JavaScript per la creazione di interfacce grafiche e la gestione delle interazioni in ambito web.

\subsubsection*{Redmine}
\index{Redmine}
Piattaforma Open-Source che permette la gestione di più progetti contemporaneamente, offrendo funzionalità di project management e issue tracking.

\subsubsection*{Redux}
\index{Redux}
Libreria Open-Source JavaScript per la gestione degli stati di React\glo.

\subsubsection*{Refactoring}
\index{Refactoring}
Il refactoring è una tecnica strutturata per modificare la struttura interna di porzioni di codice senza modificarne il comportamento esterno, applicata per migliorare alcune caratteristiche non funzionali del software. I vantaggi che il refactoring persegue riguardano in genere un miglioramento della leggibilità, della manutenibilità, della riusabilità e dell'estendibilità del codice e la riduzione della sua complessità, eventualmente attraverso l'introduzione a posteriori di design pattern. Il refactoring è un elemento importante delle principali metodologie emergenti di sviluppo del software: per esempio delle metodologie agili, dell'extreme programming, e del test driven development (TDD).

\subsubsection*{Release}
\index{Release}
Nell'ambito dello sviluppo software, la release è una specifica versione di un software resa disponibile ai suoi utenti finali. Univocamente identificata da un numero in modo da distinguerla dalle precedenti e future altre release del software.

\subsubsection*{Repository}
\index{Repository}
In generale, locazione di salvataggio dei dati. Nei sistemi di versionamento è una struttura dati più complessa, contenente metadati e operazioni per maneggiarla.

\subsubsection*{REST}
\index{REST}
Representational State Transfer, metodo di sviluppo basato sulla semplice interazione client-server che usa il protocollo HTTP per le comunicazioni. Il concetto di base del metodo REST è che una risorsa è  identificabile tramite un URI. Tutte le interazioni con le risorse si basano su quattro azioni HTTP: POST, GET, PUT e DELETE.

\subsubsection*{Rete Bayesiana}
\index{Rete Bayesiana}
Modello grafico probabilistico che rappresenta un insieme di variabili stocastiche con le loro dipendenze condizionali.

