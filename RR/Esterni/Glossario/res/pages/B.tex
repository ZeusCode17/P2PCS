\subsection*{\quad$B\quad$}
\subsubsection*{Back end}
\index{Back end}
Back end è la parte che permette ad un utente di interagire indirettamente, in generale attraverso l’utilizzo dell’interfaccia di un’applicazione front end.

\subsubsection*{Blockchain}
\index{Blockchain}
Struttura dati condivisa ed immutabile, definita da un registro digitale raggruppate in pagine, dette blocchi, concatenate in ordine cronologico. I blocchi una volta inseriti non sono più modificabili. Questa tecnologia, assimilabile ad un database distribuito, è gestito da una rete di nodi, ognuno dei quali ne possiede una copia privata. Ogni transazione è regolata da un protocollo che
ne verifica l’autenticità e la approva o meno. Una transazione approvata non è più modificabile e viene aggiunta alla blockchain. Con questo sistema non è necessaria la presenza di un’autorità esterna alla transazione che faccia da garante.

\subsubsection*{Body of knowledge}
\index{Body of knowledge}
Insieme completo di concetti, termini e attività di uno stesso dominio. Questo insieme fa parte del dominio e allo stesso tempo lo costituiscono.

\subsubsection*{Boilerplate}
\index{Boilerplate}
Pratica di riuso del codice in cui un frammento di codice viene replicato in più punti di un file, o in più file nello stesso punto. Solitamente rappresenta la base da cui cominciare per la stesura di un'applicazione.

\subsubsection*{Bootstrap}
\index{Bootstrap} 
Raccolta di strumenti liberi per la creazione di siti e applicazioni per il Web. Essa contiene modelli di progettazione basati su HTML e CSS, sia per la tipografia, che per le varie componenti dell'interfaccia, come moduli, pulsanti e navigazione, così come alcune estensioni opzionali di JavaScript.
