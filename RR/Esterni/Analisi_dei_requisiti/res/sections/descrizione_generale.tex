\section{Descrizione generale} 
\subsection{Obiettivi del prodotto}
Il progetto \textit{P2PCS} si pone come obiettivo lo sviluppo di una piattaforma di car sharing Peer-to-Peer\glosp che la distingua dalla concorrenza tramite l'utilizzo di almeno 5 core-drive\glosp del framework\glosp Octalysis\glo.

\subsection{Funzioni del prodotto}
Per il servizio di car sharing l'applicativo deve offrire le seguenti funzionalità:
\begin{itemize}
	\item registrazione e login degli utenti;
	\item inserimento della propria auto per la condivisione;
	\item gestione del proprio profilo;
	\item prenotazione di un'auto;
	\item visualizzazione dei veicoli disponibili;
	\item visualizzazione distanza (lista con distanza in km o possibilità di vederle su una mappa);
	\item confermare l’avvenuto scambio delle chiavi;
	\item chiusura del viaggio;
	\item visualizzazione dei dati del viaggio appena trascorso.
	\newline
\end{itemize}
L'applicativo, oltre a offrire il servizio di car sharing, deve invogliare l'utente a condividere la propria auto a un prezzo basso. 
I core-drive\glosp e le game mechanics che il gruppo \textit{Zeus Code} intende utilizzare per motivare i singoli utenti sono incentrati sui seguenti core-drive di Octalysis\glo: 
\begin{itemize}
	\item \textbf{accomplishment}: con questo ci assicureremo che gli utenti stiano superando le sfide di cui possono essere orgogliosi, per esempio implementeremo la Leaderboard [UC22] e la Progress Bar [UC16];
	\item \textbf{ownership}: con questo si tende ad invogliare l'utente a migliore/proteggere qualcosa che già possiede o ad ottenere di più, per esempio implementando l'Avatar;
	\item \textbf{unpredictability \& curiosity}: con questo possiamo stimolare la curiosità dell'utente e indirizzarlo verso una ricompensa o evento inaspettato, implementando la Lucky Spin [UC20] e non solo, anche i Random, Daily Rewards e gli Easter Egg;
	\item \textbf{social influence}: ovvero le attività ispirate da ciò che gli altri pensano, fanno o dicono (questo Core-Drive è il motore di temi come mentorship, competizione, invidia, ricerche di gruppo, tesori sociali e compagnia), per esempio faremo sì che un utente condivida con amici un codice per invitarli ad utilizzare l’applicazione [UC23];
	\item \textbf{empowerment}: permettere all'utente di sbloccare potenziamenti, ovvero le Milestone Unlock [UC18].
\end{itemize}
\subsection{Caratteristiche degli utenti}
Si evidenziano sin da subito due categorie di utenti:
\begin{itemize}
	\item utente non autenticato;
	\item utente autenticato con registrazione incompleta (nome, email e password) che potrà essere:
		\begin{itemize}
			\item usufruente: utente autenticato con registrazione completa che necessita di un'auto;
			\item proprietario: utente autenticato con registrazione completa che condivide la propria auto.
		\end{itemize}
		Usufruente e proprietario non sono due categorie mutuamente esclusive: un'usufruente può diventare proprietario semplicemente inserendo un veicolo nell'applicazione e un proprietario può diventare usufruente prendendo in prestito un veicolo che non è di sua proprietà.
\end{itemize}
Dall'analisi del capitolato e dal confronto con la proponente, non si è riscontrata la necessità di distinguere gli utenti in altre sotto-categorie perciò ogni utente, proprietario o usufruente, sarà rappresentato allo stesso modo.