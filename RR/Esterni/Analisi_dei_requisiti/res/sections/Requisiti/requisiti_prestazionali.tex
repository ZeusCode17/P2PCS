\subsection{Requisiti prestazionali}

%\rowcolors{2}{pari}{dispari}
%
%\begin{longtable}{ >{\centering}p{0.10\textwidth} >{\centering}p{0.25\textwidth}
%		>{\raggedright}p{0.35\textwidth} >{\centering}p{0.14\textwidth}}
%	\caption{Tabella dei requisiti prestazionali}\\
%	\rowcolorhead 
%	\textbf{\color{white}Requisito} 
%	& \textbf{\color{white}Classificazione} 
%	& \centering\textbf{\color{white}Descrizione}
%	& \textbf{\color{white}Fonti} 
%	\endfirsthead
%	\rowcolor{white}\caption[]{(continua)}\\
%	\rowcolorhead 
%	\textbf{\color{white}Requisito} 
%	& \textbf{\color{white}Classificazione} 
%	& \centering\textbf{\color{white}Descrizione}
%	& \textbf{\color{white}Fonti} 
%	\endhead	
%	
%	
%
%\end{longtable}
bisogna scriverci qualcosa per dire che non abbiamo i requisiti prestazionali
