\subsubsection{UC12 - Gestione Profilo}
%\begin{figure}[h]
	%\includegraphics[width=10cm]{res/images/UC7-profilepage.png}
	%\centering
	%\caption{UC7 - Gestione Profilo}
%\end{figure}
\begin{itemize}
	\item \textbf{Attori Primari}: utente autenticato;
	\item \textbf{Descrizione}: all'utente è permesso modificare i dati del proprio account oppure eliminarlo;
	\item \textbf{Scenario principale}: 
	\begin{enumerate}[label=\alph*.]
		\item l'utente sceglie di modificare i dati[UC7.1];
		\item l'utente sceglie di eliminare l'account[UC7.3];
	\end{enumerate}
	
	\item \textbf{Precondizione}: l'utente è in possesso di un account all'interno del sistema. Deve quindi essersi registrato e non aver eliminato l'account;
	\item \textbf{Postcondizione}:l'utente ha effettuato l'operazione di modifica dati oppure l'eliminazione dell'account e il processo è stato confermato dal sistema.
	\item \textbf{Estensioni}:
	\begin{enumerate}
	\item visualizzazione di errore sui dati in input[UC7.2]
	\end{enumerate}
\end{itemize} 
\subsubsection{UC7.1 - Modifica dati account}
\begin{itemize}
	\item \textbf{Attori Primari}: utente autenticato;
	\item \textbf{Descrizione}: l'utente ha la possibilità di modificare i propri dati;
	\item \textbf{Scenario principale}:
	\begin{enumerate}
	\item modifica  password[UC7.1.1];
	\item conferma modifica[UC7.1.3].
	\end{enumerate}
	\item \textbf{Inclusioni}:
	\begin{enumerate}
	\item inserimento vecchia password[7.1.2].
	\end{enumerate}
	\item \textbf{Scenari alternativi}:
	\begin{enumerate}
	\item l'utente interrompe la modifica dei dati senza confermare il salvataggio di essi. Il sistema non salverà le modifiche parziali apportate dall'utente me lo riporterà alla schermata di visualizzazione dell'account.
	\end{enumerate}	 
	\item \textbf{Precondizione}: l'utente è in possesso di un account all'interno del sistema. Deve quindi essersi registrato e non aver eliminato l'account;
	\item \textbf{Postcondizione}: il sistema ha memorizzato le modifiche apportate ai dati da parte dell’utente.
\end{itemize}

\subsubsection{UC7.1.1 - Modifica password}
\begin{itemize}
	\item \textbf{Attori Primari}: utente autenticato;
	\item \textbf{Descrizione}: l'utente ha la possibilità di modificare la password inserita in precedenza;
	\item \textbf{Precondizione}: il sistema fornisce una schermata nella quale è possibile inserire la nuova password;
	\item \textbf{Postcondizione}: l'utente ha inserito la nuova password.
\end{itemize}

\subsubsection{UC7.1.2 - Inserimento vecchia password}
\begin{itemize}
	\item \textbf{Attori Primari}: utente autenticato;
	\item \textbf{Descrizione}: l'utente deve inserire la password attuale per poter la aggiornare;
	\item \textbf{Precondizione}: il sistema fornisce una schermata nella quale è possibile inserire la vecchia password;
	\item \textbf{Postcondizione}: l'utente ha inserito la vecchia password.
\end{itemize}

\subsubsection{UC7.1.3 - Modifica patente}
\begin{itemize}
	\item \textbf{Attori Primari}: utente autenticato;
	\item \textbf{Descrizione}: l'utente ha la possibilità di aggiornare la patente inserita in precedenza;
	\item \textbf{Precondizione}: il sistema fornisce una schermata nella quale è possibile inserire la nuova patente;
	\item \textbf{Postcondizione}: l'utente ha inserito la nuova patente.
\end{itemize}

\subsubsection{UC7.1.4 - Conferma modifica dati}
\begin{itemize}
	\item \textbf{Attori Primari}: utente autenticato;
	\item \textbf{Descrizione}: l'utente deve 
	confermare la modifica apportata;
	\item \textbf{Precondizione}: l'utente ha inserito tutti i dati richiesti e desiderati. Si trova dunque davanti ad una schermata con la possibilità di confermare la modifica effettutata;
	\item \textbf{Postcondizione}: l'utente ha confermato di voler rendere effettivo il cambiamento all'interno del sistema.
\end{itemize}

\subsubsection{UC7.2 - Errore nei dati in input}
\begin{itemize}
	\item \textbf{Attori Primari}: utente autenticato;
	\item \textbf{Descrizione}: durante la fase di modifica dei dati, l'utente può aver commesso uno dei seguenti errori:
	\begin{itemize}[label=$-$]
	\item la password inserita in UC7.1.2 non corrisponde con la vecchia password;
	\item la password nuova inserita non è conforme ai vincoli di sicurezza imposti dal sistema;
	\item la nuova patente inserita non è valida.	
	\end{itemize}
	\item \textbf{Precondizione}: l'utente ha effettuato la conferma dei dati inseriti;
	\item \textbf{Postcondizione}: viene notificato all'utente un errore nell'inserimento dei dati. Bisogna specificare l'errore commesso e su quali dati.
\end{itemize}

\subsubsection{UC7.3 - Eliminazione account}
\begin{itemize}
	\item \textbf{Attori Primari}: utente autenticato;
	\item \textbf{Descrizione}: all'utente viene fornita la possibilità di eliminare il proprio account e di conseguenza i propri dati all'interno del sistema.
	\item \textbf{Precondizione}: l'utente è in possesso di un account all'internodel sistema. Deve quindi aver effettuato la registrazione e non avere mai efettuato la procedura di eliminazione account.
	\item \textbf{Postcondizione}: l'utente ha cancellato il proprio account e viene riportato alla schermata iniziale dell'app[UC1].Il sistema non dovrà avere più taccia di tale utente.
\end{itemize}








