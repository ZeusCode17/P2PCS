\subsubsection{UC4 - Login}
\begin{itemize}
	\item \textbf{Attori Primari}: utente non autenticato;
	%\item \textbf{Attori Secondari}: Movens\glo;
	\item \textbf{Descrizione}: per effettuare il procedimento di autenticazione, l'utente deve compilare i campi necessari ovvero e-mail e password;
	\item \textbf{Scenario principale}: l'applicazione rende disponibili i campi da compilare per l'autenticazione. Dunque l'utente dovrà inserire tutti i dati necessari.
	
	\item \textbf{Precondizione}: l'utente ha inserito correttamente tutti i dati necessari nei campi.
	\item \textbf{Postcondizione}: dopo aver controllato che i campi sono stati compilati correttamente attraverso la piattaforma Movens, l'utente viene autenticato nell'applicazione.	
\end{itemize}

\subsubsection{UC3.1 - Compilazione campi per il login}
\begin{figure}[h]
	%\includegraphics[width=9cm]{res/images/UC3-1Login.png}
	\centering
	\caption{UC2 - Registrazione}
\end{figure}
\begin{itemize}
	\item \textbf{Attori Primari}: utente non autenticato;
	%\item \textbf{Attori Secondari}: Movens\glo;
	\item \textbf{Descrizione}: l'utente compila i campi richiesti per l'autenticazione;
	\item \textbf{Scenario principale}: l'utente compila i campi necessari all'autenticazione ovvero: 
	\newline
	a. l'utente inserisce l'email associata al proprio account [UC3.1.1];
	\newline
	b. l'utente inserisce la password associata la proprio account [UC3.1.2].	
	\item \textbf{Precondizione}: l'utente si trova nell'activity di autenticazione.
	\item \textbf{Postcondizione}: l'utente ha compilato tutti i campi necessari all'autenticazione.	
\end{itemize}

\subsubsection{UC3.1.1 - Inserimento e-mail}
\begin{itemize}
	\item \textbf{Attori Primari}: utente non autenticato;
	\item \textbf{Descrizione}: al fine di portare a termine il processo di autenticazione l'utente deve inserire l'indirizzo e-mail associato all'account, campo ritenuto obbligatorio;
	\item \textbf{Scenario principale}: l'utente compila il campo relativo all'indirizzo e-mail;	
	\item \textbf{Precondizione}: l'applicazione ha reso disponibile il campo per l'inserimento dell'indirizzo e-mail;
	\item \textbf{Postcondizione}: l'utente ha compilato il campo con l'indirizzo e-mail associato al proprio account.
\end{itemize}

\subsubsection{UC3.1.2 - Inserimento password}
\begin{itemize}
	\item \textbf{Attori Primari}: utente non autenticato;
	\item \textbf{Descrizione}: al fine di portare a termine il processo di autenticazione l'utente deve inserire la password associata al proprio account, campo ritenuto obbligatorio;
	\item \textbf{Scenario principale}: l'utente compila il campo relativo alla password;	
	\item \textbf{Precondizione}: l'applicazione ha reso disponibile il campo per l'inserimento della password;
	\item \textbf{Postcondizione}: l'utente ha compilato il campo con la password associata al proprio account.
\end{itemize}

\subsubsection{UC3.1.2 - Invio dati}
\begin{itemize}
	\item \textbf{Attori Primari}: utente non autenticato;
	\item \textbf{Descrizione}: l'utente preme il pulsante per la conferma e l'invio dei dati; se e-mail e password risulteranno corrette l'utente verrà autenticato;
	\item \textbf{Scenario principale}: l'utente preme il pulsante di verifica ed invio dei dati;	
	\item \textbf{Precondizione}: i campi dati necessari per l'autenticazione sono compilabili. È presente il pulsante per la conferma dei dati;
	\item \textbf{Postcondizione}: l'utente viene autenticato e rimandato all'activity per la gestione dei veicoli.
\end{itemize}

%INSERIRE ERRORE LOGIN UC5


%INSERIRE RECUPERO PASSWORD UC6

\subsubsection{UC7 - Logout}
\begin{itemize}
	\item \textbf{Attori Primari}:
	utente autenticato;
	\item \textbf{Descrizione}: l'utente dal fragment\glosp Area Personale richiede il logout dall'applicazione;
	\item \textbf{Scenario principale}: l'utente è autenticato nell'applicazione e richiede di effettuare il logout, premendo sull'apposito pulsante;
	\item \textbf{Precondizione}: l'utente ha effettuato il login all'applicazione e richiede di essere disconnesso dall'applicazione;
	\item \textbf{Postcondizione}: l'utente viene disautenticato e rimandato all'activity introduttiva. 
\end{itemize}

