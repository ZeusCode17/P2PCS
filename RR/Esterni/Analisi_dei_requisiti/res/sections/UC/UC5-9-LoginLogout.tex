\begin{figure}[h]
	%\includegraphics[width=9cm]{res/images/UC2Registrazione.png}
	\centering
	\caption{Schema generale: login ed errori annessi}
\end{figure}
\subsubsection{UC5 - Login}
\begin{itemize}
	\item \textbf{Attori Primari}: utente non autenticato;
	\item \textbf{Descrizione}: per effettuare il procedimento di autenticazione, l'utente deve compilare i campi necessari ovvero e-mail e password;
	\item \textbf{Scenario principale}: l'applicazione rende disponibili i campi da compilare per l'autenticazione. Dunque l'utente dovrà inserire tutti i dati necessari;
	\item \textbf{Precondizione}: l'utente ha inserito correttamente tutti i dati necessari nei campi;
	\item \textbf{Postcondizione}: dopo aver controllato che i campi sono stati compilati correttamente attraverso la piattaforma Movens\glo, l'utente viene autenticato nell'applicazione.
	\item \textbf{Estensioni}:
		\begin{enumerate}
			\item Visualizzazione errore campi vuoti [UC6];
			\item Visualizzazione errore combinazione e-mail e password errata [UC7].
		\end{enumerate}	
\end{itemize}
\begin{figure}[h]
	%\includegraphics[width=9cm]{res/images/UC2Registrazione.png}
	\centering
	\caption{UC5 - Login}
\end{figure}
\subsubsection{UC5.1 - Compilazione campi per il login}
\begin{figure}[h]
	%\includegraphics[width=9cm]{res/images/UC3-1Login.png}
	\centering
	\caption{UC5.1 - Compilazione campi per il login}
\end{figure}
\begin{itemize}
	\item \textbf{Attori Primari}: utente non autenticato;
	\item \textbf{Descrizione}: l'utente compila i campi richiesti per l'autenticazione;
	\item \textbf{Scenario principale}: l'utente compila i campi necessari all'autenticazione ovvero:
		\begin{itemize}
			\item l'utente inserisce l'email associata al proprio account [UC5.1.1];
			\item l'utente inserisce la password associata la proprio account [UC5.1.2].
		\end{itemize}	
	\item \textbf{Precondizione}: l'utente si trova nell'activity\glosp di autenticazione;
	\item \textbf{Postcondizione}: l'utente ha compilato tutti i campi necessari all'autenticazione.	
\end{itemize}

\subsubsection{UC5.1.1 - Inserimento e-mail}
\begin{itemize}
	\item \textbf{Attori Primari}: utente non autenticato;
	\item \textbf{Descrizione}: al fine di portare a termine il processo di autenticazione l'utente deve inserire l'indirizzo e-mail associato all'account, campo ritenuto obbligatorio;
	\item \textbf{Scenario principale}: l'utente compila il campo relativo all'indirizzo e-mail;	
	\item \textbf{Precondizione}: l'applicazione ha reso disponibile il campo per l'inserimento dell'indirizzo e-mail;
	\item \textbf{Postcondizione}: l'utente ha compilato il campo con l'indirizzo e-mail associato al proprio account.
\end{itemize}

\subsubsection{UC5.1.2 - Inserimento password}
\begin{itemize}
	\item \textbf{Attori Primari}: utente non autenticato;
	\item \textbf{Descrizione}: al fine di portare a termine il processo di autenticazione l'utente deve inserire la password associata al proprio account, campo ritenuto obbligatorio;
	\item \textbf{Scenario principale}: l'utente compila il campo relativo alla password;	
	\item \textbf{Precondizione}: l'applicazione ha reso disponibile il campo per l'inserimento della password;
	\item \textbf{Postcondizione}: l'utente ha compilato il campo con la password associata al proprio account.
\end{itemize}

\subsubsection{UC5.2 - Invio dati}
\begin{itemize}
	\item \textbf{Attori Primari}: utente non autenticato;
	\item \textbf{Descrizione}: l'utente preme il pulsante per la conferma e l'invio dei dati; se e-mail e password risulteranno corrette l'utente verrà autenticato;
	\item \textbf{Scenario principale}: l'utente preme il pulsante di verifica ed invio dei dati;	
	\item \textbf{Precondizione}: i campi dati necessari per l'autenticazione sono compilabili. È presente il pulsante per la conferma dei dati;
	\item \textbf{Postcondizione}: l'utente viene autenticato e rimandato all'activity\glosp per la gestione dei veicoli.
\end{itemize}

\subsubsection{UC6 - Visualizzazione errore campi vuoti}
\begin{itemize}
	\item \textbf{Attori Primari}: utente non autenticato;
	\item \textbf{Descrizione}: l'utente visualizza un messaggio d'errore in quanto non sono stati riempiti i campi di inserimento e-mail e password;
	\item \textbf{Scenario principale}: l'utente non ancora autenticato tenta di accedere non inserendo un indirizzo e-mail e password;	
	\item \textbf{Precondizione}: l'utente non autenticato non inserisce nessun dato nei campi di inserimento;
	\item \textbf{Postcondizione}: l'applicazione fa visualizzare un messaggio d'errore.
\end{itemize}
\subsubsection{UC7 - Visualizzazione errore combinazione e-mail e password errata}
\begin{itemize}
	\item \textbf{Attori Primari}: utente non autenticato;
	\item \textbf{Descrizione}: l'utente visualizza un messaggio d'errore in quanto i campi di inserimento e-mail e password sono stati riempiti in modo errato;
	\item \textbf{Scenario principale}: l'utente non ancora autenticato tenta di accedere inserendo un indirizzo e-mail e password che insieme risultano non validi;	
	\item \textbf{Precondizione}: l'utente non autenticato inserisce i dati che combinati risultano non validi;
	\item \textbf{Postcondizione}: l'applicazione fa visualizzare un messaggio d'errore.
\end{itemize}

\subsubsection{UC8 - Recupero password}
\begin{itemize}
	\item \textbf{Attori Primari}: utente non autenticato;
	\item \textbf{Descrizione}: l'utente non autenticato non si ricorda la propria password per accedere all'account e ne richiede il recupero che avviene inserendo un'indirizzo e-mail dove verrà inviata una nuova password;
	\item \textbf{Scenario principale}: l'utente non ancora autenticato inserisce un'indirizzo e-mail dove potrà ricevere la nuova password da utilizzare per accedere al proprio account; 
	\item \textbf{Precondizione}: l'utente non autenticato non ricorda la password di accesso e indica un'indirizzo e-mail di recupero;
	\item \textbf{Postcondizione}: l'utente riceve la nuova password tramite posta all'indirizzo e-mail inviato.
	\item \textbf{Estensione}:
		\begin{itemize}
			\item Visualizzazione errore e-mail non valida [UC2].
		\end{itemize}
\end{itemize}
\begin{figure}[h]
	%\includegraphics[width=9cm]{res/images/UC3-1Login.png}
	\centering
	\caption{UC8 - Recupero password}
\end{figure}

\subsubsection{UC9 - Logout}
\begin{itemize}
	\item \textbf{Attori Primari}:
	utente autenticato;
	\item \textbf{Descrizione}: l'utente dal fragment\glosp Area Personale richiede il logout dall'applicazione;
	\item \textbf{Scenario principale}: l'utente è autenticato nell'applicazione e richiede di effettuare il logout, premendo sull'apposito pulsante;
	\item \textbf{Precondizione}: l'utente ha effettuato il login all'applicazione e richiede di essere disconnesso dall'applicazione;
	\item \textbf{Postcondizione}: l'utente viene disautenticato e rimandato all'activity\glosp di login. 
\end{itemize}

