\subsubsection{UC7 - Gestione Profilo}
%\begin{figure}[h]
	%\includegraphics[width=10cm]{res/images/UC7-profilepage.png}
	%\centering
	%\caption{UC7 - Gestione Profilo}
%\end{figure}
\begin{itemize}
	\item \textbf{Attori Primari}: utente autenticato;
	\item \textbf{Attori Secondari}: Movens\glo;
	\item \textbf{Descrizione}: all'utente è permesso modificare i dati del proprio account oppure eliminarlo;
	\item \textbf{Scenario principale}: 
	\begin{enumerate}[label=\alph*.]
		\item l'utente sceglie di modificare i dati[UC7.1];
		\item l'utente sceglie di eliminare l'account[UC7.3];
	\end{enumerate}
	
	\item \textbf{Precondizione}: l'utente è in possesso di un account all'interno del sistema. Deve quindi essersi registrato e non aver eliminato l'account;
	\item \textbf{Postcondizione}:l'utente ha effettuato l'operazione di modifica dati oppure l'eliminazione dell'account e il processo è stato confermato dal sistema.
	\item \textbf{Estensioni}:
	\begin{enumerate}
	\item visualizzazione di errore sui dati in input[UC7.2]
	\end{enumerate}
	\end{itemize}
\end{itemize} 
\subsubsection{UC7.1 - Modifica dati account}
\begin{itemize}
	\item \textbf{Attori Primari}: utente autenticato;
	\item \textbf{Descrizione}: l'utente ha la possibilità di modificare i propri dati;
	\item \textbf{Scenario principale}:
	\begin{enumerate}
	\item modifica  password[UC7.1.1];
	\item conferma modifica[UC7.1.3].
	\end{enumerate}
	\item \textbf{Inclusioni}:
	\begin{enumerate}
	\item inserimento vecchia password[7.1.2]
	\end{enumerate}
	\item \textbf{Scenari alternativi}:
	\begin{enumerate}
	\item l'utente interrompe la modifica dei dati senza confermare il salvataggio di essi. Il sistema non salverà le modifiche parziali apportate dall'utente me lo riporterà alla schermata di visualizzazione dell'account.
	\end{enumerate}	 
	\item \textbf{Precondizione}: l'utente è in possesso di un account all'interno del sistema. Deve quindi essersi registrato e non aver eliminato l'account;
	\item \textbf{Postcondizione}: il sistema ha memorizzato le modifiche apportate ai dati da parte dell’utente.
\end{itemize}

\subsubsection{UC7.2 - Visualizzazione errore carrello vuoto}
\begin{itemize}
	\item \textbf{Attori Primari}: azienda, cittadino;
	\item \textbf{Descrizione}:
	l'utente visualizza un messaggio di errore relativo al fatto che non è presente alcun prodotto nel proprio carrello e che quindi non è possibile procedere con la procedura di checkout;
	\item \textbf{Scenario principale}: l'utente tenta di procedere con il checkout senza aver inserito alcun bene/servizio nel carrello;
	\item \textbf{Precondizione}: il sistema ha reso disponibile all'utente il carrello ed il pulsante di checkout. L'utente ha premuto sul pulsante di checkout ed il carrello risulta vuoto; 
	\item \textbf{Postcondizione}: viene visualizzato un messaggio d'errore per informare l'utente del fatto che e non è presente alcun prodotto nel proprio carrello e che quindi non è possibile procedere con la procedura di checkout.
\end{itemize}

\subsubsection{UC7.3 - Scelta indirizzo spedizione}
\begin{itemize}
	\item \textbf{Attori Primari}: azienda, cittadino;
	\item \textbf{Descrizione}:
	l'utente, dopo aver effettuato il checkout, deve selezionare l'indirizzo di spedizione. Gli vengono presentate due possibilità:
	\begin{itemize}
		\item utilizzare l'indirizzo inserito durante la registrazione;
		\item inserire un nuovo indirizzo da utilizzare per questa spedizione, inserendo le informazioni relative ad un indirizzo [UC2.2.2].
	\end{itemize}
	\item \textbf{Scenario principale}: l'utente deve selezionare un indirizzo di spedizione;
	\item \textbf{Precondizione}: l'utente ha eseguito il checkout;
	\item \textbf{Postcondizione}:
	l'utente ha selezionato come indirizzo di spedizione, l'indirizzo di 
	residenza oppure ne ha inserito uno differente. Può dunque concludere il 
	procedimento di acquisto. Tale indirizzo verrà utilizzato per la creazione 
	della relativa fattura.
\end{itemize}

\subsubsection{UC7.4 - Conferma ordine e pagamento}
\begin{itemize}
	\item \textbf{Attori Primari}: azienda, cittadino;
	\item \textbf{Attori Secondari}: azienda, MetaMask\glo;
	\item \textbf{Descrizione}: l'utente, dopo aver selezionato l'indirizzo di spedizione, procede con la conferma ed il pagamento dell'ordine;
	\item \textbf{Scenario principale}: l'utente conferma l'ordine premendo l'apposito pulsante. Segue dunque la procedura di pagamento attraverso l'utilizzo di MetaMask\glo;
	\item \textbf{Estensioni}: 
	\begin{itemize}
		\item \textbf{UC7.4}: l'esito del pagamento da parte di MetaMask\glosp risulta negativo, l'utente riceve un messaggio di errore che lo invita a controllare la causa dello stesso direttamente dal plug-in\glo. 
	\end{itemize}
	\item \textbf{Precondizione}: l'utente ha effettuato il checkout e la selezione dell'indirizzo di spedizione;
	\item \textbf{Postcondizione}: il pagamento è avvenuto con successo. 
	L'importo della transazione momentaneamente è trattenuto dal sistema a 
	causa del meccanismo di escrow\glo. Il cliente riceve nella pagina dedicata 
	la conferma d'acquisto\glo.
\end{itemize}


\subsubsection{UC7.5 - Visualizzazione errore pagamento fallito}
\begin{itemize}
	\item \textbf{Attori Primari}: azienda, cittadino;
	\item \textbf{Attori Secondari}: MetaMask\glo;
	\item \textbf{Descrizione}:
	l'utente visualizza un messaggio di errore relativo al fatto che il tentativo di pagamento non è andato a buon fine, e che quindi l'ordine è stato annullato. L'utente viene invitato ad informarsi sulla causa del fallimento dell'operazione all'interno del plug-in;
	\item \textbf{Scenario principale}: l'utente tenta pagare attraverso MetaMask\glosp la somma dovuta al venditore per l'acquisto corrente;
	\item \textbf{Precondizione}: il sistema permette all'utente di procedere con il pagamento, ovvero l'ordine è stato confermato da parte dell'utente;
	\item \textbf{Postcondizione}: viene visualizzato un messaggio d'errore per informare l'utente del fatto che l'acquisto non è andato a buon fine, e che per ottenere informazioni più precise dovrà riferirsi al messaggio di errore riportato nel plug-in. 
\end{itemize} 






