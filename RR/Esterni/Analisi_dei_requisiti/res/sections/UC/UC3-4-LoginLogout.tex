
\subsubsection{UC3 - Login}
\begin{itemize}
	\item \textbf{Attori Primari}: utente non autenticato;
	\item \textbf{Attori Secondari}: Movens\glo;
	\item \textbf{Descrizione}: per effettuare il procedimento di autenticazione, l'utente deve compilare i campi necessari ovvero e-mail e password;
	\item \textbf{Scenario principale}: l'applicazione rende disponibili i campi da compilare per l'autenticazione. Dunque l'utente dovrà inserire tutti i dati necessari.
	
	\item \textbf{Precondizione}: l'utente ha inserito correttamente tutti i dati necessari nei campi.
	\item \textbf{Postcondizione}: dopo aver controllato che i campi siano stati compilati correttamente, l'utente viene autenticato nell'applicazione.	
\end{itemize}

\subsubsection{UC3.1 - Compilazione campi per il login}
\begin{itemize}
	\item \textbf{Attori Primari}: utente non autenticato;
	\item \textbf{Attori Secondari}: Movens\glo;
	\item \textbf{Descrizione}: l'utente compila i campi richiesti per l'autenticazione;
	\item \textbf{Scenario principale}: l'utente compila i campi necessari all'autenticazione ovvero: a.l'utente inserisce l'email associata al proprio account [UC3.1.1];
	\newline
	b.l'utente inserisce la password associata la proprio account [UC3.1.2].	
	\item \textbf{Precondizione}: l'utente si trova nell'activity di autenticazione.
	\item \textbf{Postcondizione}: l'utente ha compilato tutti i campi necessari all'autenticazione.	
\end{itemize}

\subsubsection{UC3.1.1 - Inserimento e-mail}
\begin{itemize}
	\item \textbf{Attori Primari}: utente non autenticato;
	\item \textbf{Descrizione}: al fine di portare a termine il processo di autenticazione l'utente deve inserire l'indirizzo e-mail associato all'account, campo ritenuto obbligatorio;
	\item \textbf{Scenario principale}: l'utente compila il campo relativo all'indirizzo e-mail;	
	\item \textbf{Precondizione}: l'applicazione ha reso disponibile il campo per l'inserimento dell'indirizzo e-mail;
	\item \textbf{Postcondizione}: l'utente ha compilato il campo con l'indirizzo e-mail associato al proprio account.
\end{itemize}

\subsubsection{UC3.1.2 - Inserimento password}
\begin{itemize}
	\item \textbf{Attori Primari}: utente non autenticato;
	\item \textbf{Descrizione}: al fine di portare a termine il processo di autenticazione l'utente deve inserire la password associata al proprio account, campo ritenuto obbligatorio;
	\item \textbf{Scenario principale}: l'utente compila il campo relativo alla password;	
	\item \textbf{Precondizione}: l'applicazione ha reso disponibile il campo per l'inserimento della password;
	\item \textbf{Postcondizione}: l'utente ha compilato il campo con la password associata al proprio account.
\end{itemize}


\subsubsection{UC3.1 - Login automatico}
\begin{itemize}
	\item \textbf{Attori Primari}:
	utente non autenticato;
	\item \textbf{Attori Secondari}:
	MetaMask\glo;
	\item \textbf{Descrizione}:
	in modo automatico, il sistema procede all'identificazione dell'utente;
	\item \textbf{Scenario principale}:
	l'utente non ancora autenticato richiede il login;
	\item \textbf{Estensioni}:
	\begin{itemize}
		\item \textbf{UC2.5}: se l'utente non dispone di MetaMask\glosp o ha disabilitato l'estensione, viene visualizzato un messaggio di errore a riguardo;
		\item \textbf{UC2.6}: se l'utente non possiede una chiave\glosp su MetaMask\glo, esso viene avvisato tramite l'apposito messaggio di errore;
		\item \textbf{UC3.2}: se l'utente tenta di accedere al sito tramite MetaMask\glosp senza aver mai provveduto a registrarsi, riceverà un messaggio di errore a riguardo;
		
		\item \textbf{UC3.3}: se l'utente si è registrato ma il suo account è stato disabilitato, riceverà un messaggio di errore a riguardo.
	\end{itemize}
	\item \textbf{Precondizione}:
	l'utente tenta di autenticarsi alla piattaforma;
	\item \textbf{Postcondizione}:
	l'utente viene individuato attraverso l'utilizzo del plug-in MetaMask\glo.
\end{itemize}
\subsubsection{UC3.2 - Visualizzazione messaggio di errore relativo a chiave non registrata}
\begin{itemize}
	\item \textbf{Attori Primari}:
	utente non autenticato;
	\item \textbf{Descrizione}:
	l'utente visualizza un messaggio di errore dovuto al fatto che ha tentato il login senza essersi registrato in precedenza;
	\item \textbf{Scenario principale}:
	l'utente tenta di eseguire la procedura di login alla piattaforma senza essere registrato;
	\item \textbf{Precondizione}:
	l'utente tenta di autenticarsi nella piattaforma;
	\item \textbf{Postcondizione}: viene visualizzato un messaggio d'errore per informare l'utente del fatto che è necessario registrarsi alla piattaforma prima di poter poi effettuare la procedura di login.
\end{itemize}
\subsubsection{UC3.3 - Visualizzazione schermata relativa a utente non abilitato}
\begin{itemize}
	\item \textbf{Attori Primari}: utente non autenticato;
	\item \textbf{Descrizione}:
	l'utente tenta di autenticarsi alla piattaforma, tuttavia, a causa della disabilitazione del suo account, il login viene interrotto, e l'utente visualizza il messaggio personalizzato lasciato dal governo che illustra la causa della disabilitazione dell'account;
	\item \textbf{Scenario principale}:
	l'utente non autenticato con account disabilitato tenta di autenticarsi. La procedura di autenticazione viene bloccata a causa dello stato dell'account;
	\item \textbf{Precondizione}:
	un utente non autenticato, registrato alla piattaforma e con account disabilitato tenta di effettuare il login automatico;
	\item \textbf{Postcondizione}:  viene visualizzato un messaggio d'errore per informare l'utente del fatto che il proprio account è stato disabilitato dal governo. Se quest'ultimo, durante la procedura di disabilitazione, ha inserito un messaggio contenente la causa di tale azione, allora tale messaggio viene visualizzato.
\end{itemize}
\subsubsection{UC4 - Logout}
\begin{itemize}
	\item \textbf{Attori Primari}:
	utente autenticato;
	\item \textbf{Attori Secondari}:
	MetaMask\glo;
	\item \textbf{Descrizione}: l'utente richiede il logout dalla piattaforma web. Vengono visualizzate le informazioni necessarie per procedere alla procedura di logout, che deve essere effettuata attraverso l'utilizzo del plug-in MetaMask\glo;
	\item \textbf{Scenario principale}: l'utente è autenticato dal sito e richiede di effettuare il logout, premendo sull'apposito pulsante;
	\item \textbf{Precondizione}: l'utente ha effettuato il login alla piattaforma web e richiede di essere disconnesso dal sito;
	\item \textbf{Postcondizione}: vengono visualizzate le istruzioni necessarie per eseguire il logout tramite MetaMask\glo. 
\end{itemize}

