\subsubsection{UC26 - Garage veicolo}
\begin{figure}[h]
	%\includegraphics[width=9cm]{res/images/UC23Codiceamico.png}
	\centering
	\caption{UC26 - Garage veicolo}
\end{figure}
\begin{itemize}
	\item \textbf{Attori Primari}: proprietario del veicolo;
	\item \textbf{Descrizione}: agli utenti autenticati è reso disponibile un Minigioco composto da un garage dove si possono fare modifiche ad un'auto base di partenza tramite rewards ottenuti da sblocchi di obbiettivi e al superamento di livelli d'esperienza personale grazie ad un'utilizzo dell'applicazione in modo continuo. \\ Ogni pezzo di ogni tipo di modifica, installato al veicolo, visualizza in punteggio le seguenti informazioni:
	\begin{itemize}
		\item velocità;
		\item accelerazione;
		\item peso;
		\item maneggevolezza.
	\end{itemize}
	punteggi che poi verranno sommati e attribuiti come statistiche al veicolo e visualizzate nella schermata principale dell'auto; 
	\item \textbf{Scenario principale}: l'utente accede al Minigioco e visualizza un'auto base con la possibilità di modificarla attraverso due categorie:
	\begin{itemize}
		\item Prestazione [UC26.1];
		\item Estetica [UC26.2];
	\end{itemize}
	e successivamente se le modifiche apportate vanno bene all'utente, può installarle e confermarle;
	\item \textbf{Estensioni}:
	\begin{itemize}
		\item annulla tutte le modifiche [UC27].
	\end{itemize}
	\item \textbf{Precondizione}: l'utente autenticato ha selezionato la voce \textit{Garage} dal menu dell'applicazione;
	\item \textbf{Post-condizione}: l'utente autenticato ha visualizzato la sua auto e installato le modifiche se attuate. 
\end{itemize}
\subsubsection{UC26.1 - Prestazione}
\begin{figure}[h]
	%\includegraphics[width=9cm]{res/images/UC23Codiceamico.png}
	\centering
	\caption{UC26.1 - Prestazione}
\end{figure}
\begin{itemize}
	\item \textbf{Attori Primari}: utente autenticato;
	\item \textbf{Descrizione}: l'utente accede alla categorie di modifiche sulla prestazione per la propria auto. Avrà a disposizione un tot di pezzi per ogni singola sottocategoria;
	\item \textbf{Scenario principale}: l'utente sta visualizzando la sua auto con una serie di modifiche da attuare per la categoria \textit{Prestazione} quali:
	\begin{itemize}
		\item motore [UC26.1.1];
		\item centralina [UC26.1.2];
		\item trasmissione [UC26.1.3];
		\item sospensioni [UC26.1.4];
		\item gomme [UC26.1.5].
	\end{itemize}
	\item \textbf{Precondizione}: L'utente ha intenzione di modificare la parte prestazionale della sua auto;
	\item \textbf{Postcondizione}: l'utente ha modificato elementi prestazionali della sua auto e potrà confermare l'installazione di tali modifiche.
\end{itemize}
\subsubsection{UC26.1.1 - Motore}
\begin{itemize}
	\item \textbf{Attori Primari}: utente autenticato;
	\item \textbf{Descrizione}: in questa sezione l'utente può modificare il motore della propria auto se in possesso di premi ricevuti da completamento di qualche obbiettivo o altro e di eventuali punti esperienza guadagnati col l'utilizzo continuo dell'applicazione.
	All'inizio viene messo a disposizione il modello base;
	\item \textbf{Scenario principale}: l'utente vuole modificare il motore della propria auto e verifica la presenza di motori migliori da poter installare.
	\item \textbf{Precondizione}: 
	\item \textbf{Postcondizione}:
\end{itemize}
\subsubsection{UC26.1.2 - Ritiro del Daily Rewards}
\begin{itemize}
	\item \textbf{Attori Primari}: utente autenticato;
	\item \textbf{Descrizione}: l'applicazione rende disponibile dei premi giornalieri, illustrati e ritirabili tramite la tabella Daily Rewards;
	\item \textbf{Scenario principale}: l'utente sta visualizzando la tabella dei premi giornalieri e preme la casella del premio del giorno corrente per ritirarlo;
	\item \textbf{Precondizione}: l'utente non ha ancora ritirato il premio del giorno corrente e apre la tabella dei premio da cui preme la casella del giorno corrente per ritirare il premio;
	\item \textbf{Postcondizione}: l'utente ha ricevuto il premio del giorno corrente e l'applicazione disabilita il relativo pulsante, segnalando il premio come ritirato. 
\end{itemize}
\subsubsection{UC26.1.3 - Ritiro del Daily Rewards}
\begin{itemize}
	\item \textbf{Attori Primari}: utente autenticato;
	\item \textbf{Descrizione}: l'applicazione rende disponibile dei premi giornalieri, illustrati e ritirabili tramite la tabella Daily Rewards;
	\item \textbf{Scenario principale}: l'utente sta visualizzando la tabella dei premi giornalieri e preme la casella del premio del giorno corrente per ritirarlo;
	\item \textbf{Precondizione}: l'utente non ha ancora ritirato il premio del giorno corrente e apre la tabella dei premio da cui preme la casella del giorno corrente per ritirare il premio;
	\item \textbf{Postcondizione}: l'utente ha ricevuto il premio del giorno corrente e l'applicazione disabilita il relativo pulsante, segnalando il premio come ritirato. 
\end{itemize}
\subsubsection{UC26.1.4 - Ritiro del Daily Rewards}
\begin{itemize}
	\item \textbf{Attori Primari}: utente autenticato;
	\item \textbf{Descrizione}: l'applicazione rende disponibile dei premi giornalieri, illustrati e ritirabili tramite la tabella Daily Rewards;
	\item \textbf{Scenario principale}: l'utente sta visualizzando la tabella dei premi giornalieri e preme la casella del premio del giorno corrente per ritirarlo;
	\item \textbf{Precondizione}: l'utente non ha ancora ritirato il premio del giorno corrente e apre la tabella dei premio da cui preme la casella del giorno corrente per ritirare il premio;
	\item \textbf{Postcondizione}: l'utente ha ricevuto il premio del giorno corrente e l'applicazione disabilita il relativo pulsante, segnalando il premio come ritirato. 
\end{itemize}
\subsubsection{UC26.1.5 - Ritiro del Daily Rewards}
\begin{itemize}
	\item \textbf{Attori Primari}: utente autenticato;
	\item \textbf{Descrizione}: l'applicazione rende disponibile dei premi giornalieri, illustrati e ritirabili tramite la tabella Daily Rewards;
	\item \textbf{Scenario principale}: l'utente sta visualizzando la tabella dei premi giornalieri e preme la casella del premio del giorno corrente per ritirarlo;
	\item \textbf{Precondizione}: l'utente non ha ancora ritirato il premio del giorno corrente e apre la tabella dei premio da cui preme la casella del giorno corrente per ritirare il premio;
	\item \textbf{Postcondizione}: l'utente ha ricevuto il premio del giorno corrente e l'applicazione disabilita il relativo pulsante, segnalando il premio come ritirato. 
\end{itemize}

\subsubsection{UC26.2 - Estetica}
\begin{itemize}
	\item \textbf{Attori Primari}: utente autenticato;
	\item \textbf{Descrizione}: l'applicazione rende disponibile dei premi giornalieri, illustrati e ritirabili tramite la tabella Daily Rewards;
	\item \textbf{Scenario principale}: l'utente sta visualizzando la tabella dei premi giornalieri e preme la casella del premio del giorno corrente per ritirarlo;
	\item \textbf{Precondizione}: l'utente non ha ancora ritirato il premio del giorno corrente e apre la tabella dei premio da cui preme la casella del giorno corrente per ritirare il premio;
	\item \textbf{Postcondizione}: l'utente ha ricevuto il premio del giorno corrente e l'applicazione disabilita il relativo pulsante, segnalando il premio come ritirato. 
\end{itemize} 
