\subsubsection{UC14 - Gestione Profilo}
%\begin{figure}[h]
	%\includegraphics[width=10cm]{res/images/UC7-profilepage.png}
	%\centering
	%\caption{UC14 - Gestione Profilo}
%\end{figure}
\begin{itemize}
	\item \textbf{Attori Primari}: utente autenticato;
	\item \textbf{Descrizione}: all'utente è permesso modificare i dati del proprio account oppure eliminarlo;
	\item \textbf{Scenario principale}: 
	\begin{enumerate}[label=\alph*.]
		\item l'utente sceglie di modificare i dati[UC7.1];
		\item l'utente sceglie di eliminare l'account[UC7.3];
	\end{enumerate}
	
	\item \textbf{Precondizione}: l'utente è in possesso di un account all'interno del sistema. Deve quindi essersi registrato e non aver eliminato l'account;
	\item \textbf{Postcondizione}:l'utente ha effettuato l'operazione di modifica dati oppure l'eliminazione dell'account e il processo è stato confermato dal sistema.
	\item \textbf{Estensioni}:
	\begin{enumerate}
	\item visualizzazione di errore password non conforme[UC3];
	\item visualizzazione di errore password non corrispondente[UC15].
	\end{enumerate}
\end{itemize} 
\subsubsection{UC7.1 - Modifica dati account}
\begin{itemize}
	\item \textbf{Attori Primari}: utente autenticato;
	\item \textbf{Descrizione}: l'utente ha la possibilità di modificare i propri dati;
	\item \textbf{Scenario principale}:
	\begin{enumerate}
	\item modifica  password[UC7.1.1];
	\item conferma modifica[UC7.1.3].
	\end{enumerate}
	\item \textbf{Inclusioni}:
	\begin{enumerate}
	\item inserimento vecchia password[7.1.2].
	\end{enumerate}
	\item \textbf{Scenari alternativi}:
	\begin{enumerate}
	\item l'utente interrompe la modifica dei dati senza confermare il salvataggio di essi. Il sistema non salverà le modifiche parziali apportate dall'utente me lo riporterà alla schermata di visualizzazione dell'account.
	\end{enumerate}	 
	\item \textbf{Precondizione}: l'utente è in possesso di un account all'interno del sistema. Deve quindi essersi registrato e non aver eliminato l'account;
	\item \textbf{Postcondizione}: il sistema ha memorizzato le modifiche apportate ai dati da parte dell’utente.
\end{itemize}

\subsubsection{UC7.1.1 - Modifica nome}
\begin{itemize}
	\item \textbf{Attori Primari}: utente autenticato;
	\item \textbf{Descrizione}: l'utente ha la possibilità di modificare il nome inserito in precedenza;
	\item \textbf{Precondizione}: il sistema fornisce una schermata nella quale è possibile inserire il nuovo nome;
	\item \textbf{Postcondizione}: l'utente ha inserito il nuovo nome.
\end{itemize}

\subsubsection{UC7.1.2 - Modifica cognome}
\begin{itemize}
	\item \textbf{Attori Primari}: utente autenticato;
	\item \textbf{Descrizione}: l'utente ha la possibilità di modificare il cognome inserito in precedenza;
	\item \textbf{Precondizione}: il sistema fornisce una schermata nella quale è possibile inserire il nuovo cognome;
	\item \textbf{Postcondizione}: l'utente ha inserito il nuovo cognome.
\end{itemize}

\subsubsection{UC7.1.3 - Modifica numero telefonico}
\begin{itemize}
	\item \textbf{Attori Primari}: utente autenticato;
	\item \textbf{Descrizione}: l'utente ha la possibilità di modificare il numero telefonico inserito in precedenza;
	\item \textbf{Precondizione}: il sistema fornisce una schermata nella quale è possibile inserire il nuovo numero telefonico;
	\item \textbf{Postcondizione}: l'utente ha inserito il nuovo numero telefonico.
\end{itemize}

\subsubsection{UC7.1.4 - Modifica email}
\begin{itemize}
	\item \textbf{Attori Primari}: utente autenticato;
	\item \textbf{Descrizione}: l'utente ha la possibilità di modificare l'email inserita in precedenza;
	\item \textbf{Precondizione}: il sistema fornisce una schermata nella quale è possibile inserire la nuova email;
	\item \textbf{Postcondizione}: l'utente ha inserito la nuova email.
\end{itemize}

\subsubsection{UC7.1.5 - Modifica data di nascita}
\begin{itemize}
	\item \textbf{Attori Primari}: utente autenticato;
	\item \textbf{Descrizione}: l'utente ha la possibilità di modificare la data di nascita inserita in precedenza;
	\item \textbf{Precondizione}: il sistema fornisce una schermata nella quale è possibile inserire la nuova data di nascita;
	\item \textbf{Postcondizione}: l'utente ha inserito la nuova data di nascita.
\end{itemize}

\subsubsection{UC7.1.6 - Modifica residenza}
\begin{itemize}
	\item \textbf{Attori Primari}: utente autenticato;
	\item \textbf{Descrizione}: l'utente ha la possibilità di modificare la residenza inserita in precedenza;
	\item \textbf{Precondizione}: il sistema fornisce una schermata nella quale è possibile inserire la nuova residenza;
	\item \textbf{Postcondizione}: l'utente ha inserito la nuova residenza.
\end{itemize}

\subsubsection{UC7.1.7 - Modifica password}
\begin{itemize}
	\item \textbf{Attori Primari}: utente autenticato;
	\item \textbf{Descrizione}: l'utente ha la possibilità di modificare la password inserita in precedenza;
	\item \textbf{Precondizione}: il sistema fornisce una schermata nella quale è possibile inserire la nuova password;
	\item \textbf{Postcondizione}: l'utente ha inserito la nuova password.
\end{itemize}

\subsubsection{UC7.1.8 - Inserimento vecchia password}
\begin{itemize}
	\item \textbf{Attori Primari}: utente autenticato;
	\item \textbf{Descrizione}: l'utente deve inserire la password attuale per poter la aggiornare;
	\item \textbf{Precondizione}: il sistema fornisce una schermata nella quale è possibile inserire la vecchia password;
	\item \textbf{Postcondizione}: l'utente ha inserito la vecchia password.
\end{itemize}

\subsubsection{UC7.1.3 - Conferma modifica dati}
\begin{itemize}
	\item \textbf{Attori Primari}: utente autenticato;
	\item \textbf{Descrizione}: l'utente deve 
	confermare la modifica apportata;
	\item \textbf{Precondizione}: l'utente ha inserito tutti i dati richiesti e desiderati. Si trova dunque davanti ad una schermata con la possibilità di confermare la modifica effettutata;
	\item \textbf{Postcondizione}: l'utente ha confermato di voler rendere effettivo il cambiamento all'interno del sistema.
\end{itemize}

\subsubsection{UC7.2 - Eliminazione account}
\begin{itemize}
	\item \textbf{Attori Primari}: utente autenticato;
	\item \textbf{Descrizione}: all'utente viene fornita la possibilità di eliminare il proprio account e di conseguenza i propri dati all'interno del sistema.
	\item \textbf{Precondizione}: l'utente è in possesso di un account all'internodel sistema. Deve quindi aver effettuato la registrazione e non avere mai efettuato la procedura di eliminazione account.
	\item \textbf{Postcondizione}: l'utente ha cancellato il proprio account e viene riportato alla schermata iniziale dell'app[UC1].Il sistema non dovrà avere più taccia di tale utente.
\end{itemize}




