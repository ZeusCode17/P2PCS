\section{Casi d'uso} 
\subsection{Attori dei casi d'uso}
\subsubsection{Attori primari}
\begin{figure}[h]
	\includegraphics[width=7cm]{res/images/attori_primari.png}
	\centering
	\caption{Gerarchia degli attori primari}
\end{figure}
\begin{description}[style=nextline]
	\item[Utente generico]
	Si riferisce ad un utente generico che entra nell'applicazione.
	\item[Utente non autenticato]
	Si riferisce ad un utente generico che non ha ancora effettuato l'autenticazione nell'applicazione.
	\item[Utente autenticato]
	Si riferisce ad un utente generico che si è autenticato nell'applicazione con la procedura di login.
\end{description}

\subsection{Elenco dei casi d'uso}
In questa sezione vi sono elencati tutti i casi d'uso individuati. Ogni caso d'uso rappresenta uno scenario per uno o più attori, ovviamente applicabile anche ad eventuali attori derivati. Ogni caso d'uso, inoltre, viene descritto tramite diagrammi dei casi d'uso e possiede una precondizione seguita da una postcondizione.
\subsubsection*{Operazioni utenti}
Di seguito sono riportati tutti i casi d'uso che coinvolgono come attore primario l'utente generico, l'utente non autenticato, l'utente autenticato.
\newpage
\input{res/sections/UC/UC1-4-GuidaRegistrazione.tex}
\begin{figure}[h]
	%\includegraphics[width=9cm]{res/images/UC2Registrazione.png}
	\centering
	\caption{Schema generale: login ed errori annessi}
\end{figure}
\subsubsection{UC5 - Login}
\begin{itemize}
	\item \textbf{Attori Primari}: utente non autenticato;
	\item \textbf{Descrizione}: per effettuare il procedimento di autenticazione, l'utente deve compilare i campi necessari ovvero e-mail e password;
	\item \textbf{Scenario principale}: l'applicazione rende disponibili i campi da compilare per l'autenticazione. Dunque l'utente dovrà inserire tutti i dati necessari;
	\item \textbf{Precondizione}: l'utente ha inserito correttamente tutti i dati necessari nei campi;
	\item \textbf{Postcondizione}: dopo aver controllato che i campi sono stati compilati correttamente attraverso la piattaforma Movens\glo, l'utente viene autenticato nell'applicazione.
	\item \textbf{Estensioni}:
		\begin{enumerate}
			\item Visualizzazione errore campi vuoti [UC6];
			\item Visualizzazione errore combinazione e-mail e password errata [UC7].
		\end{enumerate}	
\end{itemize}
\begin{figure}[h]
	%\includegraphics[width=9cm]{res/images/UC2Registrazione.png}
	\centering
	\caption{UC5 - Login}
\end{figure}
\subsubsection{UC5.1 - Compilazione campi per il login}
\begin{figure}[h]
	%\includegraphics[width=9cm]{res/images/UC3-1Login.png}
	\centering
	\caption{UC5.1 - Compilazione campi per il login}
\end{figure}
\begin{itemize}
	\item \textbf{Attori Primari}: utente non autenticato;
	\item \textbf{Descrizione}: l'utente compila i campi richiesti per l'autenticazione;
	\item \textbf{Scenario principale}: l'utente compila i campi necessari all'autenticazione ovvero:
		\begin{itemize}
			\item l'utente inserisce l'email associata al proprio account [UC5.1.1];
			\item l'utente inserisce la password associata la proprio account [UC5.1.2].
		\end{itemize}	
	\item \textbf{Precondizione}: l'utente si trova nell'activity\glosp di autenticazione;
	\item \textbf{Postcondizione}: l'utente ha compilato tutti i campi necessari all'autenticazione.	
\end{itemize}

\subsubsection{UC5.1.1 - Inserimento e-mail}
\begin{itemize}
	\item \textbf{Attori Primari}: utente non autenticato;
	\item \textbf{Descrizione}: al fine di portare a termine il processo di autenticazione l'utente deve inserire l'indirizzo e-mail associato all'account, campo ritenuto obbligatorio;
	\item \textbf{Scenario principale}: l'utente compila il campo relativo all'indirizzo e-mail;	
	\item \textbf{Precondizione}: l'applicazione ha reso disponibile il campo per l'inserimento dell'indirizzo e-mail;
	\item \textbf{Postcondizione}: l'utente ha compilato il campo con l'indirizzo e-mail associato al proprio account.
\end{itemize}

\subsubsection{UC5.1.2 - Inserimento password}
\begin{itemize}
	\item \textbf{Attori Primari}: utente non autenticato;
	\item \textbf{Descrizione}: al fine di portare a termine il processo di autenticazione l'utente deve inserire la password associata al proprio account, campo ritenuto obbligatorio;
	\item \textbf{Scenario principale}: l'utente compila il campo relativo alla password;	
	\item \textbf{Precondizione}: l'applicazione ha reso disponibile il campo per l'inserimento della password;
	\item \textbf{Postcondizione}: l'utente ha compilato il campo con la password associata al proprio account.
\end{itemize}

\subsubsection{UC5.2 - Invio dati}
\begin{itemize}
	\item \textbf{Attori Primari}: utente non autenticato;
	\item \textbf{Descrizione}: l'utente preme il pulsante per la conferma e l'invio dei dati; se e-mail e password risulteranno corrette l'utente verrà autenticato;
	\item \textbf{Scenario principale}: l'utente preme il pulsante di verifica ed invio dei dati;	
	\item \textbf{Precondizione}: i campi dati necessari per l'autenticazione sono compilabili. È presente il pulsante per la conferma dei dati;
	\item \textbf{Postcondizione}: l'utente viene autenticato e rimandato all'activity\glosp per la gestione dei veicoli.
\end{itemize}

\subsubsection{UC6 - Visualizzazione errore campi vuoti}
\begin{itemize}
	\item \textbf{Attori Primari}: utente non autenticato;
	\item \textbf{Descrizione}: l'utente visualizza un messaggio d'errore in quanto non sono stati riempiti i campi di inserimento e-mail e password;
	\item \textbf{Scenario principale}: l'utente non ancora autenticato tenta di accedere non inserendo un indirizzo e-mail e password;	
	\item \textbf{Precondizione}: l'utente non autenticato non inserisce nessun dato nei campi di inserimento;
	\item \textbf{Postcondizione}: l'applicazione fa visualizzare un messaggio d'errore.
\end{itemize}
\subsubsection{UC7 - Visualizzazione errore combinazione e-mail e password errata}
\begin{itemize}
	\item \textbf{Attori Primari}: utente non autenticato;
	\item \textbf{Descrizione}: l'utente visualizza un messaggio d'errore in quanto i campi di inserimento e-mail e password sono stati riempiti in modo errato;
	\item \textbf{Scenario principale}: l'utente non ancora autenticato tenta di accedere inserendo un indirizzo e-mail e password che insieme risultano non validi;	
	\item \textbf{Precondizione}: l'utente non autenticato inserisce i dati che combinati risultano non validi;
	\item \textbf{Postcondizione}: l'applicazione fa visualizzare un messaggio d'errore.
\end{itemize}

\subsubsection{UC8 - Recupero password}
\begin{itemize}
	\item \textbf{Attori Primari}: utente non autenticato;
	\item \textbf{Descrizione}: l'utente non autenticato non si ricorda la propria password per accedere all'account e ne richiede il recupero che avviene inserendo un'indirizzo e-mail dove verrà inviata una nuova password;
	\item \textbf{Scenario principale}: l'utente non ancora autenticato inserisce un'indirizzo e-mail dove potrà ricevere la nuova password da utilizzare per accedere al proprio account; 
	\item \textbf{Precondizione}: l'utente non autenticato non ricorda la password di accesso e indica un'indirizzo e-mail di recupero;
	\item \textbf{Postcondizione}: l'utente riceve la nuova password tramite posta all'indirizzo e-mail inviato.
	\item \textbf{Estensione}:
		\begin{itemize}
			\item Visualizzazione errore e-mail non valida [UC2].
		\end{itemize}
\end{itemize}
\begin{figure}[h]
	%\includegraphics[width=9cm]{res/images/UC3-1Login.png}
	\centering
	\caption{UC8 - Recupero password}
\end{figure}

\subsubsection{UC9 - Logout}
\begin{itemize}
	\item \textbf{Attori Primari}:
	utente autenticato;
	\item \textbf{Descrizione}: l'utente dal fragment\glosp Area Personale richiede il logout dall'applicazione;
	\item \textbf{Scenario principale}: l'utente è autenticato nell'applicazione e richiede di effettuare il logout, premendo sull'apposito pulsante;
	\item \textbf{Precondizione}: l'utente ha effettuato il login all'applicazione e richiede di essere disconnesso dall'applicazione;
	\item \textbf{Postcondizione}: l'utente viene disautenticato e rimandato all'activity\glosp di login. 
\end{itemize}


 \subsubsection{UC10 - Gestione Veicoli}
  \begin{figure}[H]
 	\includegraphics[width=6cm]{res/images/UC5-Generale.png}
 	\centering
 	\caption{UC10 - Gestione Veicoli}
 \end{figure}
 \begin{itemize}
 	\item \textbf{Attori Primari}: utente autenticato;
 	\item \textbf{Descrizione}: l'utente visualizza i propri veicoli. Per ogni veicolo vengono visualizzate le seguenti informazioni:
 	\begin{itemize}
 		\item marca;
 		\item modello;
 		\item anno di immatricolazione;
 		\item rating;
 	\end{itemize}
 	\item \textbf{Precondizione}: l'utente accede al fragment\glosp per la gestione dei veicoli;
 	\item \textbf{Postcondizione}: l'utente visualizza le informazioni relative ai propri veicoli, con le eventuali operazioni disponibili su ognuno di essi.
 \end{itemize}
 \subsubsection{UC10.1 - Aggiunta Veicolo}
 \begin{itemize}
 	\item \textbf{Attori Primari}: utente autenticato;
 	\item \textbf{Descrizione}: l'utente può aggiungere un mezzo di trasporto al proprio parco macchine;
 	\item \textbf{Scenario principale}: l'utente aggiunge un veicolo compilando gli appositi campi, ovvero:
 	\begin{enumerate}
 		\item l'utente inserisce la marca del veicolo [UC10.1.1];
 		\item l'utente inserisce il modello del veicolo [UC10.1.2];
 		\item l'utente inserisce l'anno d'immatricolazione del veicolo [UC10.1.3];
 	\end{enumerate}
 	e successivamente salverà il nuovo veicolo [UC10.1.4];
 	\item \textbf{Precondizione}: l'utente ha inserito correttamente tutti i campi necessari
 	\item \textbf{Postcondizione}: il nuovo veicolo viene aggiunto ai veicoli posseduti.
 \end{itemize}

\subsubsection{UC10.1.1 - Inserimento marca veicolo}
\begin{itemize}
	\item \textbf{Attori Primari}: utente autenticato;
	\item \textbf{Descrizione}: al fine di portare a termine il processo di inserimento di un nuovo veicolo l'utente deve inserirne la marca, campo ritenuto obbligatorio;
	\item \textbf{Scenario principale}: l'utente compila il campo relativo alla marca del veicolo;	
	\item \textbf{Precondizione}: l'applicazione ha reso disponibile il campo per l'inserimento della marca del veicolo;
	\item \textbf{Postcondizione}: l'utente ha compilato il campo con la marca.	
\end{itemize}

\subsubsection{UC10.1.2 - Inserimento modello veicolo}
\begin{itemize}
	\item \textbf{Attori Primari}: utente autenticato;
	\item \textbf{Descrizione}: al fine di portare a termine il processo di inserimento di un nuovo veicolo l'utente deve inserirne il modello, campo ritenuto obbligatorio;
	\item \textbf{Scenario principale}: l'utente compila il campo relativo alla marca del veicolo;	
	\item \textbf{Precondizione}: l'applicazione ha reso disponibile il campo per l'inserimento il modello del veicolo;
	\item \textbf{Postcondizione}: l'utente ha compilato il campo con il modello.	
\end{itemize}

\subsubsection{UC10.1.3 - Inserimento anno d'immatricolazione veicolo}
\begin{itemize}
	\item \textbf{Attori Primari}: utente autenticato;
	\item \textbf{Descrizione}: al fine di portare a termine il processo di inserimento di un nuovo veicolo l'utente deve inserirne l'anno di immatricolazione, campo ritenuto obbligatorio;
	\item \textbf{Scenario principale}: l'utente compila il campo relativo all'anno d'immatricolazione del veicolo;	
	\item \textbf{Precondizione}: l'applicazione ha reso disponibile il campo per l'inserimento dell'anno d'immatricolazione del veicolo;
	\item \textbf{Postcondizione}: l'utente ha compilato il campo con l'anno d'immatricolazione.	
\end{itemize}

\subsubsection{UC10.1.4 - Invio dati veicolo}
\begin{itemize}
	\item \textbf{Attori Primari}: utente autenticato;
	\item \textbf{Descrizione}: l'utente preme il pulsante per la conferma e l'invio dei dati del veicolo;
	\item \textbf{Scenario principale}: l'utente preme il pulsante di verifica ed invio dei dati;	
	\item \textbf{Precondizione}: i campi dati necessari per l'inserimento di un veicolo sono compilabili. È presente il pulsante per la conferma dei dati;
	\item \textbf{Postcondizione}: il nuovo veicolo viene inserito e l'utente potrà visualizzarlo nel proprio parco macchine.
\end{itemize}

\subsubsection{UC10.2 - Rimozione Veicolo}
\begin{itemize}
	\item \textbf{Attori Primari}: utente autenticato;
	\item \textbf{Descrizione}: l'utente può rimuovere un mezzo di trasporto al proprio parco macchine;
	\item \textbf{Scenario principale}: l'utente può selezionare un veicolo e attraverso l'apposito pulsante rimuoverlo dal proprio parco macchine.
	\item \textbf{Precondizione}: l'applicazione mostra all'utente i propri veicoli e ne permette la selezione.
	\item \textbf{Postcondizione}: il veicolo selezionato viene rimosso dal parco macchine
\end{itemize}
\subsubsection{UC11 - Gestione prenotazioni}
\begin{figure}[h]
	\includegraphics[width=12cm]{res/images/UC11Prenotazione.png}
	\centering
	\caption{UC11 - Gestione prenotazione}
\end{figure}
\begin{itemize}
	\item \textbf{Attori Primari}: utente autenticato;
	\item \textbf{Attori Secondari}:
	usufruente del veicolo, proprietario del veicolo;
	\item \textbf{Descrizione}: agli utenti autenticati è resa disponibile una maschera che presenta la lista di tutte le sue prenotazioni ancora attive, dalla quale l'utente può scegliere di effettuare operazioni di gestione su ognuna di esse;
	per ogni prenotazione presente nella lista saranno visualizzati dei dettagli riassuntivi, che sono:
	\begin{itemize}
		\item nome del proprietario del veicolo o dell'usufruente;
		\item data;
		\item fascia oraria;
		\item marca del veicolo prenotato;
		\item modello del veicolo prenotato;
		\item anno d'immatricolazione del veicolo prenotato.
	\end{itemize}
	\item \textbf{Scenario principale}: l'utente autenticato effettua operazioni di gestione di una prenotazione. Esse comprendono:
	\begin{itemize}
		\item visualizzazione dettagli prenotazione [UC11.1].
	\end{itemize}
	\item \textbf{Scenario alternativo}: Se l'utente autenticato è il proprietario del veicolo, inoltre potrà confermare:
	\begin{itemize}
		\item le prenotazioni ricevute [UC11.3];
		\item la riconsegna del veicolo [UC11.5];
		\item recensire l'usufruente [UC11.6].
	\end{itemize}
	Se l'utente autenticato è l'usufruente del veicolo, inoltre potrà:
	\begin{itemize}
		\item cancellare la prenotazione [UC11.2];
		\item richiesta di conferma per la riconsegna del veicolo [UC11.4];
		\item recensire il proprietario [UC11.6].
	\end{itemize}
	\item \textbf{Precondizione}: il sistema carica correttamente la lista delle prenotazioni effettuate attualmente attive;
	\item \textbf{Post-condizione}: l'utente ha visualizzato le sue prenotazioni correnti ed è riuscito ad effettuare eventuali modifiche.
\end{itemize} 
 \subsubsection{UC11.1 - Visualizzazione dettagli prenotazione}
\begin{itemize}
	\item \textbf{Attori Primari}: utente autenticato;
	\item \textbf{Descrizione}: l'utente visualizza i dettagli della prenotazione scelta dalla maschera di presentazione delle prenotazioni, ciò gli permette di visualizzare:
	\begin{itemize}
		\item nome del proprietario del veicolo o dell'usufruente;
		\item la data;
		\item la fascia oraria;
		\item il veicolo prenotato;
	\end{itemize}
	Inoltre se presenti, verranno visualizzati anche:
	\begin{itemize}		
		\item il luogo d'incontro;
		\item l'orario d'incontro;
	\end{itemize}
	\item \textbf{Scenario principale}: L'utente visualizza i dettagli della prenotazione.	
	\item \textbf{Precondizione}: l'utente ha scelto una prenotazione;
	\item \textbf{Post-condizione}: l'utente ha visualizzato correttamente i dettagli della prenotazione.
\end{itemize}
\begin{comment}

\end{comment}
\subsubsection{UC11.2 - Cancellazione prenotazione}
\begin{itemize}
	\item \textbf{Attori Primari}: usufruente del veicolo;
	\item \textbf{Descrizione}: l'utente usufruente cancella la prenotazione selezionata;
	\item \textbf{Scenario principale}: l'utente si trova all'interno della pagina di visualizzazione dei dettagli della prenotazione precedentemente selezionata. Attraverso l'apposito pulsante l'utente cancellerà la prenotazione effettuata;
	\item \textbf{Precondizione}: l'utente ha premuto il pulsante per la cancellazione delle prenotazione precedentemente selezionata;
	\item \textbf{Post-condizione}: l'utente ha annullato correttamente la prenotazione selezionata.
\end{itemize}

\subsubsection{UC11.3 - Conferma prenotazione, lato proprietario}
\begin{itemize}
	\item \textbf{Attori Primari}: proprietario del veicolo;
	\item \textbf{Descrizione}: il proprietario del veicolo conferma o annulla la richiesta di prenotazione ricevuta;
	\item \textbf{Scenario principale}: l'utente si trova all'interno della pagina di visualizzazione dei dettagli della richiesta di prenotazione appena ricevuta. Attraverso gli appositi campi l'utente potrà specificare:
	\begin{itemize}
		\item luogo d'incontro [UC11.3.1];
		\item orario d'incontro [UC11.3.2].
	\end{itemize} 
	\item \textbf{Precondizione}: l'utente visualizza correttamente la prenotazione che intende confermare o annullare;
	\item \textbf{Post-condizione}: l'utente ha confermato o annullato correttamente la richiesta di prenotazione ricevuta.
\end{itemize}
\begin{figure}[h]
	\includegraphics[width=10cm]{res/images/UC11-3Conferma.png}
	\centering
	\caption{UC11.3 - Conferma prenotazione, lato proprietario}
\end{figure}

\subsubsection{UC11.3.1 - Inserimento luogo d'incontro}
\begin{itemize}
	\item \textbf{Attori Primari}: proprietario del veicolo;
	\item \textbf{Descrizione}: al fine di portare a termine il processo di conferma della prenotazione, l'utente deve inserire il luogo d'incontro, campo ritenuto obbligatorio;
	\item \textbf{Scenario principale}: l'utente compila il campo relativo al luogo d'incontro;	
	\item \textbf{Precondizione}: l'applicazione ha reso disponibile il campo per l'inserimento del luogo d'incontro;
	\item \textbf{Postcondizione}: l'utente ha compilato il campo con il luogo d'incontro.	
\end{itemize}

\subsubsection{UC11.3.2 - Inserimento orario d'incontro}
\begin{itemize}
	\item \textbf{Attori Primari}: proprietario del veicolo;
	\item \textbf{Descrizione}: al fine di portare a termine il processo di conferma della prenotazione, l'utente deve inserire l'orario d'incontro, campo ritenuto obbligatorio;
	\item \textbf{Scenario principale}: l'utente compila il campo relativo all'orario d'incontro;	
	\item \textbf{Precondizione}: l'applicazione ha reso disponibile il campo per l'inserimento dell'orario d'incontro;
	\item \textbf{Postcondizione}: l'utente ha compilato il campo con l'orario d'incontro.	
\end{itemize}

\subsubsection{UC11.3.3 - Invio dati}
\begin{itemize}
	\item \textbf{Attori Primari}: proprietario del veicolo;
	\item \textbf{Descrizione}: l'utente preme il pulsante per la conferma e l'invio dei dati;
	\item \textbf{Scenario principale}: l'utente preme il pulsante di verifica ed invio dei dati;	
	\item \textbf{Precondizione}: l'utente ha compilato i dati necessari per la conferma della prenotazione e preme il pulsante per l'invio dei dati;
	\item \textbf{Postcondizione}: la prenotazione viene confermata.
\end{itemize}

\subsubsection{UC11.4 - Riconsegna del veicolo, lato usufruente}
\begin{itemize}
	\item \textbf{Attori Primari}: usufruente del veicolo;
	\item \textbf{Descrizione}: l'utente riconsegna il veicolo e chiude la prenotazione;
	\item \textbf{Scenario principale}: l'utente si trova all'interno della pagina di visualizzazione dei dettagli della prenotazione precedentemente selezionata. Dopo aver riconsegnato il veicolo e le chiavi, attraverso l'apposito pulsante l'utente chiederà la chiusura della prenotazione (la prenotazione verrà definitivamente chiusa quando anche il proprietario del veicolo confermerà la riconsegna delle chiavi e del mezzo [UC11.5]). Comparirà un pop-up che permetterà di lasciare una recensione all'altro utente [UC11.6];
	\item \textbf{Inclusioni}: 
	\begin{itemize}
		\item recensione utente [UC11.6].
	\end{itemize}
	\item \textbf{Precondizione}: l'utente visualizza correttamente la prenotazione che intende concludere;
	\item \textbf{Post-condizione}: l'utente ha richiesto correttamente la chiusura della prenotazione selezionata.
	
\end{itemize}

\subsubsection{UC11.5 - Riconsegna del veicolo, lato proprietario}
\begin{itemize}
	\item \textbf{Attori Primari}: proprietario del veicolo;
	\item \textbf{Descrizione}: il proprietario del veicolo conferma la riconsegna del mezzo e delle chiavi;
	\item \textbf{Scenario principale}: l'utente si trova all'interno della pagina di visualizzazione dei dettagli della prenotazione precedentemente selezionata. Alla riconsegna del veicolo e delle chiavi, attraverso l'apposito pulsante l'utente confermerà la chiusura della prenotazione in modo definitivo. Comparirà un pop-up che permetterà di lasciare una recensione all'altro utente [UC11.6];
	\item \textbf{Inclusioni}: 
	\begin{itemize}
		\item recensione utente [UC11.6].
	\end{itemize}
	\item \textbf{Precondizione}: l'utente visualizza correttamente la prenotazione che intende concludere;
	\item \textbf{Post-condizione}: l'utente ha concluso correttamente la prenotazione selezionata.
\end{itemize}

\subsubsection{UC11.6 - Recensione}
\begin{itemize}
	\item \textbf{Attori Primari}: utente autenticato;
	\item \textbf{Descrizione}: l'utente recensisce l'altro utente coinvolto nella prenotazione;
	\item \textbf{Scenario principale}: l'utente visualizza il pop-up per inserire la recensione che consiste in una valutazione da una a cinque stelle;
	\item \textbf{Precondizione}: l'utente visualizza correttamente il pop-up;
	\item \textbf{Post-condizione}: l'utente ha inserito correttamente la recensione.
\end{itemize}


\subsubsection{UC12 - Effettua una prenotazione}
 \begin{figure}[h]
	\includegraphics[width=6cm]{res/images/UC6GestioneCarrello.png}
	\centering
	\caption{UC11 - Effettua prenotazione}
\end{figure}
\begin{itemize}
	\item \textbf{Attori Primari}: utente autenticato;
	\item \textbf{Descrizione}: agli utenti autenticati è resa disponibile una maschera che presenta la lista di tutte le sue prenotazioni, dalla quale l'utente può scegliere di effettuare operazioni di gestione su ognuna di esse;
	Per ogni prenotazione presente nella lista saranno visualizzati dei dettagli riassuntivi, che sono:
	\begin{itemize}
		\item \textit{da completare in seguito}.
	\end{itemize}
	\item \textbf{Scenario principale}: l'utente effettua operazioni di gestione di una prenotazione. Esse comprendono:
	\begin{enumerate}[label=\alph*.]
		\item la visualizzazione dei dettagli di una prenotazione [UC11.1];
		\item la modifica di una prenotazione [UC11.2];
		\item la cancellazione di una prenotazione [UC11.3].
	\end{enumerate}
	\item \textbf{Precondizione}: il sistema riconosce l'utente proprietario o usufruente e rende disponibile il servizio di gestione delle prenotazioni;
	\item \textbf{Post-condizione}: l'utente riconosciuto può procedere con le operazioni di gestione rese disponibili.
\end{itemize} 

\subsubsection{UC13 - Storico prenotazioni}
 \begin{figure}[h]
	\includegraphics[width=8cm]{res/images/Schemagenerale4.png}
	\centering
	\caption{UC13 - Storico prenotazioni}
\end{figure}
\begin{itemize}
	\item \textbf{Attori Primari}: utente autenticato;
	\item \textbf{Descrizione}: agli utenti autenticati è resa disponibile una maschera che presenta la lista di tutte le sue prenotazioni concluse dalla quale si possono ricavare le seguenti informazioni:
	\begin{itemize}
		\item marca;
		\item modello;
		\item data;
		\item fascia oraria;
		\item stato della prenotazione;
		\item immagine del veicolo.
	\end{itemize} 
	\item \textbf{Scenario principale}: l'utente visualizza la lista contenente tutte le sue prenotazioni concluse;
	\item \textbf{Precondizione}: l'utente autenticato ha selezionato la voce \textit{Storico prenotazioni} dal menu dell'applicazione;
	\item \textbf{Post-condizione}: l'utente autenticato ha visualizzato lo storico delle sue prenotazioni concluse. 
\end{itemize} 

\input{res/sections/UC/UC14-ModificaDati.tex}
