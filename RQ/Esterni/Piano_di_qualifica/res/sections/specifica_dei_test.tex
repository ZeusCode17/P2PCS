\section{Specifica dei test}
Per assicurare la qualità del software prodotto, il gruppo \textit{Zeus Code} adotta come modello di sviluppo del software il
\textbf{Modello a V\glo}, il quale prevede lo sviluppo dei test in parallelo alle attività di analisi e progettazione. In questo modo i test permetteranno di verificare sia la correttezza delle parti di programma sviluppate, sia che tutti gli aspetti del progetto siano implementati e corretti. Segue quindi l'esito dei test per mezzo di tabelle che ne semplificheranno la consultazione e che potranno fornire una precisa indicazione degli output prodotti, specificando se il risultato ottenuto sia quello atteso, errato oppure non coerente a quanto fissato in precedenza. \\
Per definire lo stato dei test, vengono utilizzate le seguenti sigle:
\begin{itemize}
	\item \textbf{I}: per indicare che il test è stato implementato;
	\item \textbf{NI}: per indicare che il test non è stato implementato.
\end{itemize}
Inoltre per lo stato dei test si usano le seguenti abbreviazioni:
\begin{itemize}
	\item \textbf{S}: per indicare che il test ha soddisfatto la richiesta;
	\item \textbf{NS}: per indicare che il test non ha soddisfatto la richiesta.
\end{itemize}
\subsection{Test di accettazione}
I test di accettazione hanno lo scopo di dimostrare che il software sviluppato, eseguito durante il collaudo finale, soddisfi i requisiti presentati nel capitolato\glosp e concordati con il proponente. Tali test verranno indicati nel seguente modo: \\ \\
	\centerline{\textbf{TA[Codice]}}
dove:
\begin{itemize}
	\item \textbf{Codice}: rappresenta il codice identificativo crescente del test.
\end{itemize}
	Nella seguente tabella vengono anche tracciati i requisiti a cui fanno riferimento. Tali requisiti sono indicati nel documento \textit{Analisi dei requisiti v2.0.0}, sezione §4.
	\renewcommand{\arraystretch}{1.5}
	\rowcolors{2}{pari}{dispari}
	
	\begin{longtable}{ >{\centering}p{0.10\textwidth} >{\centering}p{0.12\textwidth} >{\centering}p{0.650\textwidth}
			>{\centering}p{0.10\textwidth}}% >{\centering}p{0.14\textwidth}}
			
		%\hline
		\caption{Riepilogo test di accettazione}\\	
		\rowcolorhead
		\textbf{\color{white}Test} 
		& \textbf{\color{white}Requisito} 
		& \textbf{\color{white}Descrizione} 
		& \centering\textbf{\color{white}Esito}
	%	& \textbf{\color{white}Fonti} 
		\tabularnewline %\hline 
		\endfirsthead	
		
		\rowcolor{white}\caption[]{(continua)}\\	
		\rowcolorhead
		\textbf{\color{white}Test} 
		& \textbf{\color{white}Requisito} 
		& \textbf{\color{white}Descrizione} 
		& \centering\textbf{\color{white}Esito}
		%	& \textbf{\color{white}Fonti} 
		\tabularnewline %\hline 
		\endhead	
		
		TA1	& RFO2 &	L'utente non ancora autenticato deve poter effettuare la registrazione al sistema. All'utente viene chiesto di:
		\begin{itemize}
			\item accedere alla pagina di registrazione;
			\item completare i campi dati: nome, cognome, email, password ed eventualmente codice amico;
			\item procedere con l'invio dei dati inseriti;
			\item attendere la conferma di successo dell'operazione e relativo avviso di richiesta di conferma dell'account tramite email;
			\item verificare che in seguito alla registrazione sia presentata la guida introduttiva.
		\end{itemize}	&	NI	\tabularnewline
		TA2	& RFD1 &	L'utente deve poter visualizzare la guida introduttiva. All'utente viene chiesto di:
		 \begin{itemize}
		 	\item verificare che sia disponibile la guida introduttiva;
		 	\item verificare che sia disponibile l'opzione "salta la guida";
		 	\item verificare che in seguito alla guida sia presentato il servizio di login. 
		 \end{itemize}	&	NI	\tabularnewline
		 TA3	& RFO3 & L'utente registrato deve poter effettuare il login. All'utente registrato viene chiesto di:
		 \begin{itemize}
		 	\item accedere alla pagina di login;
		 	\item inserire l'email e password;
		 	\item confermare i dati ed essere autenticato;
		 	\item verificare che dopo l'autenticazione l'utente venga rimandato all'activity\glosp \textit{Home} dell'applicazione.
		 \end{itemize}	&	NI	\tabularnewline
		 TA3.1 & RF02.7	& L'utente deve visualizzare un messaggio d'errore quando non compila il campo dati email e password. All'utente viene chiesto di:
		 \begin{itemize}
		 	\item accedere alla pagina di login;
		 	\item non compilare il campo email e password e confermare/inviare i dati;
		 	\item verificare che sia comparso il messaggio d'errore.
		 \end{itemize}  &	NI	\tabularnewline
		 TA3.2 & RF03.3 &	L'utente deve visualizzare un messaggio d'errore quando la combinazione email-password è scorretta. All'utente viene chiesto di:
		 \begin{itemize}
		 	\item accedere alla pagina di login;
		 	\item inserire una combinazione email-password scorretta e confermare/inviare i dati;
		 	\item verificare che sia comparso il messaggio d'errore.
		 \end{itemize}  &	NI	\tabularnewline
		 TA4 & RFO3.4	&	L'utente registrato deve poter recuperare la propria password nel caso in cui ne abbia bisogno. All'utente viene chiesto di:
		 \begin{itemize}
		 	\item selezionare il procedimento di recupero password dalla schermata di login;
		 	\item inserire l'email con cui è registrato nel sistema;
		 	\item verificare che il sistema abbia inviato una nuova password all'indirizzo email specificato;
		 	\item verificare di poter accedere all'account con la nuova password.
		 \end{itemize}	&	NI	\tabularnewline
		 TA5	& RFO4 &	L'utente autenticato deve poter effettuare il logout. All'utente viene chiesto di:
		 \begin{itemize}
		 	\item richiedere il logout con l'apposito pulsante;
		 	\item confermare di voler uscire dall'applicazione;
		 	\item attendere la conferma di logout;
		 	\item verificare di essere rimandato alla pagina di login.
		 \end{itemize}  &	NI	\tabularnewline
	 	 TA6	& RFO5 & L'utente autenticato deve poter capire cosa vuole fare all'interno dell'applicazione. Verrà mandato, dopo aver effettuato l'accesso, alla \textit{Home}. 
	 	 All'utente viene chiesto di:
	 	 \begin{itemize}
	 	 	\item accedere all'applicazione effettuando il login;
	 	 	\item verificare di essere sulla schermata \textit{Home} dell'applicazione;
	 	 	\item verificare che siano disponibili le due sezioni: Viaggia e Prenotazioni;
	 	 	\item verificare che nella sezione "Viaggia" siano presenti i bottoni per la ricerca di una veicolo da prenotare e per la gestione dei propri veicoli;
	 	 	\item verificare che nella sezione "Prenotazioni" sia presente una lista di tutte le prenotazioni attive in quel momento dell'utente. 
	 	 \end{itemize}	&	NI	\tabularnewline
		 TA6.1	& RFO5.1 & L'utente autenticato deve poter gestire e visualizzare i propri veicoli. 
		 All'utente viene chiesto di:
		 \begin{itemize}
		 	\item accedere al fragment\glosp per la gestione dei veicoli;
		 	\item verificare la comparsa della lista dei propri veicoli;
		 	\item per ogni veicolo deve verificare che ci siano indicate marca, modello, anno d'immatricolazione e foto del veicolo;
		 	\item verificare che siano abilitate le opzione di selezione veicolo e aggiunta di un nuovo veicolo. 
		 \end{itemize}	&	NI	\tabularnewline
		 TA6.2 & RFO5.1.1 &	L'utente autenticato deve poter inserire un nuovo veicolo. All'utente viene chiesto di:
		 \begin{itemize}
		 	\item premere sul pulsante di aggiunta veicolo dal fragment\glosp di gestione veicoli;
		 	\item compilare i campi dati richiesti (caricamento foto, specificare la marca, modello, anno d'immatricolazione, posizione e costo orario) e confermare l'inserimento del veicolo;
		 	\item verificare che venga rimandato nel fragment di gestione veicoli;
		 	\item verificare che il mezzo venga inserito nella schermata di gestione veicoli.
		 \end{itemize}	&	NI	\tabularnewline
		 TA6.2.1 & RFO5.1.2 \\ RFO5.1.2.1 \\ RFO5.1.2.2 \\ RFO5.1.2.3 &	L'utente autenticato deve poter visualizzare i dettagli del veicolo e che siano disponibili i bottoni per visualizzare le disponibilità, le statistiche e la rimozione del veicolo. All'utente viene chiesto di:
		 \begin{itemize}
		 	\item selezionare un veicolo dal proprio parco macchine;
		 	\item verificare la comparsa di tutti i dati del veicolo (foto, marca, modello, anno d'immatricolazione, costo orario e rating);
		 	\item verificare che sia disponibile il pulsante di visualizzazione disponibilità:
		 	\begin{itemize}
		 		\item verificare di essere mandati nel fragment\glosp della visualizzazione delle disponibilità;
		 		\item verificare di vedere una lista delle disponibilità se presenti con relative informazioni (data, fascia oraria e ripetizione se presente);
		 		\item verificare che si possa aggiungere una disponibilità:
		 		\begin{itemize}
		 			\item verificare di essere mandati nel fragment dell'aggiunta di una disponibilità;
		 			\item verificare che siano disponibili le due sezioni: Singola e Ripetuta;
		 			\item verificare che nella sezione "Singola" siano presenti i campi da compilare (data prenotazione, ora di inizio e ora di fine);
		 			\item verificare che nella sezione "Ripetuta" siano presenti i campi da compilare (ora di inizio, ora di fine, ripetizione giorni settimana e ripetizione per quanto tempo);
		 			\item confermare i campi inseriti e verificare l'aggiunta della disponibilità nella lista di visualizzazione.
		 		\end{itemize}
		 	\end{itemize}
	 		\item verificare che sia disponibile il pulsante di visualizzazione delle statistiche del veicolo;
	 		\begin{itemize}
	 			\item verificare di vedere dati o grafici per la frequenza dei giorni di utilizzo, il guadagno nei giorni e il guadagno totale.
	 		\end{itemize}
		 	\item verificare che sia disponibile il pulsante di rimozione veicolo;
		 	\begin{itemize}
		 		\item verificare che il veicolo precedentemente selezionato sia stato rimosso dalla schermata di visualizzazione veicoli.
		 	\end{itemize}
		 \end{itemize}	&	NI	\tabularnewline
		 TA6.3 & RFO5.2 \\ RFO5.2.1  & L'utente autenticato deve poter visualizzare la lista di tutte le sue prenotazioni attive.
		 Per ogni prenotazione attiva presente nella lista l'utente deve poter visualizzare i dettagli riassuntivi (marca, modello, stato della prenotazione, data, fascia oraria e foto del veicolo) ed eseguire operazioni di gestione su di esse (visualizzazione dettaglio). All'utente viene chiesto di:
		 \begin{itemize}
		 	\item accedere alla schermata di visualizzazione delle prenotazioni attive nella sezione "Prenotazioni";
		 	\item verificare che per ogni prenotazione presentata siano visibili i dati riassuntivi;
		 	\item selezionare una prenotazione;
		 	\item verificare che per la prenotazione selezionata siano disponibili i dettagli.
		 \end{itemize} & NI	\tabularnewline
		 TA6.3.1 & RFO5.2.2 & L'utente proprietario del veicolo può confermare o annullare una richiesta di prenotazione. All'utente è richiesto di:
		 \begin{itemize}
		 	\item accedere alla schermata di visualizzazione delle prenotazioni attive;
		 	\item verificare che la prenotazione attiva sia in attesa di conferma;
		 	\item selezionare la prenotazione;
		 	\item verificare che nei dettagli della prenotazione sia presente il pulsante di conferma o annulla;
		 	\item confermare o annullare la prenotazione comporterà poi il cambio di stato;
		 	\item verificare che l'utente si ritrovi nella schermata di visualizzazione delle prenotazioni attive.
		 \end{itemize}	&	NI	\tabularnewline
	 	 TA6.3.2 & RFO5.2.3 & L'utente proprietario del veicolo può confermare l'avvenuta consegna delle chiavi del veicolo all'usufruente. All'utente è richiesto di:
	 	 \begin{itemize}
	 	 	\item accedere alla schermata di visualizzazione delle prenotazioni attive;
	 	 	\item verificare che la prenotazione attiva sia confermata;
	 	 	\item selezionare la prenotazione;
	 	 	\item verificare che nei dettagli della prenotazione sia presente il pulsante di chiavi consegnate;
	 	 	\item premere il pulsante qualora si abbia effettivamente consegnato le chiavi all'usufruente;
	 	 	\item verificare che l'utente si ritrovi nella schermata di visualizzazione delle prenotazioni attive e che la prenotazione abbia come stato "in corso".
	 	 \end{itemize}	&	NI	\tabularnewline
 	 	 TA6.3.3 & RFO5.2.4 & L'utente proprietario del veicolo può chiudere la prenotazione nel momento in cui ha ricevuto il veicolo e le chiavi sbloccando così la possibilità di recensire l'utente usufruente. All'utente è richiesto di:
 	 	 \begin{itemize}
 	 	 	\item accedere alla schermata di visualizzazione delle prenotazioni attive;
 	 	 	\item verificare che la prenotazione attiva sia in corso;
 	 	 	\item selezionare la prenotazione;
 	 	 	\item verificare che nei dettagli della prenotazione sia presente il pulsante di termina prenotazione;
 	 	 	\item premere il pulsante qualora si abbia effettivamente avuto indietro le chiavi e il veicolo dall'usufruente;
 	 	 	\item verificare che l'utente si ritrovi nella schermata di visualizzazione delle prenotazioni attive e che la prenotazione non sia presente nella lista di prenotazioni attive.
 	 	 \end{itemize}	&	NI	\tabularnewline
		 TA6.3.4 & RFO5.2.5 &	L'utente usufruente deve poter annullare una prenotazione in attesa di conferma da parte del proprietario. All'utente viene chiesto di:
		 \begin{itemize}
		 	\item accedere alla schermata di visualizzazione delle prenotazioni attive;
		 	\item verificare che la prenotazione attiva sia ancora in attesa di conferma;
		 	\item selezionare la prenotazione che si intende annullare;
		 	\item premere l'opzione di cancellazione prenotazione;
		 	\item verificare che la prenotazione precedentemente selezionata e cancellata sia stata rimossa dalla lista delle prenotazioni attive.
		 \end{itemize}	&	NI	\tabularnewline
	 	 TA6.3.5 & RFO5.2.6 & L'utente usufruente deve poter chiudere la prenotazione nel momento in cui riconsegni il veicolo e le chiavi sbloccando così la possibilità di recensire l'utente proprietario del veicolo. All'utente viene chiesto di:
	 	 \begin{itemize}
	 		\item accedere alla schermata di visualizzazione delle prenotazioni attive;
	 		\item verificare che la prenotazione attiva sia in corso;
	 		\item selezionare la prenotazione;
	 		\item verificare che nei dettagli della prenotazione sia presente il pulsante di termina prenotazione;
	 		\item premere il pulsante, che viene sbloccato solo dopo il termine del periodo di prenotazione, qualora si abbia effettivamente ridato le chiavi e il veicolo al proprietario del veicolo;
	 		\item verificare che l'utente si ritrovi nella schermata di visualizzazione delle prenotazioni attive e che la prenotazione non sia presente nella lista di prenotazioni attive.
	 	 \end{itemize}	&	NI	\tabularnewline 				
		 TA6.4	& RFO5.3 \\ RFD5.3.1 \\ RFD5.3.2 \\ RFD5.3.3 \\ RFD5.3.4 \\ RFO5.3.5 &	L'utente autenticato deve poter effettuare una ricerca specifica sui veicoli disponibili e poter proseguire con la prenotazione di uno di essi. All'utente viene chiesto di:
		 \begin{itemize}
		 	\item accedere al fragment\glosp della ricerca di un veicolo da prenotare;
		 	\item effettuare una ricerca dei veicoli disponibili, impostando i filtri obbligatori (data, ora di inizio, ora di fine e posizione);
		 	\item attendere i risultati e verificare che siano presenti due sezioni: Lista veicoli e Mappa;
		 	\item verificare che per default venga portato nella sezione "lista veicoli";
		 	\begin{itemize}
			 	\item verificare che nella lista dei veicoli sia disponibile la selezione di uno di questi;
			 	\item selezionare uno dei veicoli proposti;
			 	\item procedere con la prenotazione del veicolo selezionato tramite apposito pulsante di "prenota";
			 	\item verificare che la prenotazione effettuata sia stata registrata.
			\end{itemize} 
		 	\item verificare che possa spostarsi nella sezione "mappa";
		 	\begin{itemize}
		 		\item verificare che nella mappa dei veicoli sia disponibile la selezione di uno di questi entro un raggio di km di distanza dalla posizione segnata nei filtro;
		 		\item selezionare uno dei veicoli proposti;
		 		\item procedere con la prenotazione del veicolo selezionato tramite apposito pulsante di "prenota";
		 		\item verificare che la prenotazione effettuata sia stata registrata.
		 	\end{itemize}
		 \end{itemize}	&	NI	\tabularnewline
		 TA7 &	RFD6 & L'utente autenticato deve poter visualizzare una lista contente tutte le sue prenotazioni concluse. All'utente viene chiesto di:
		 \begin{itemize}
		 	\item accedere alla schermata di visualizzazione delle prenotazioni concluse;
		 	\item verificare che la lista contenga tutte le sue prenotazioni concluse.
		 \end{itemize}	&	NI	\tabularnewline
		 TA8	& RFO7 & L'utente autenticato deve poter visualizzare, modificare e cancellare il proprio account con i relativi dati. All'utente viene chiesto di:
		 \begin{itemize}
		 	\item selezionare l'opzione di visualizzazione dati profilo;
		 	\item verificare che sia disponibile il servizio di modifica dati account;
		 	\item verificare che sia disponibile il servizio di eliminazione account.
		 \end{itemize}	&	NI	\tabularnewline
		 TA8.1 & RFO7.2 & L'utente autenticato deve poter aggiornare i dati del proprio account. All'utente viene chiesto di:
		 \begin{itemize}
		 	\item accedere alla schermata di visualizzazione profilo;
		 	\item verificare di poter inserire/modificare patente;
		 	\item verificare di poter modificare il nome;
		 	\item verificare di poter modificare il cognome;
		 	\item verificare di poter inserire/modificare il numero telefonico;
		 	\item verificare di poter modificare il email;
		 	\item verificare di poter inserire/modificare la data di nascita;
		 	\item verificare di poter inserire/modificare l'indirizzo di residenza;
		 	\item verificare di poter modificare la password;
		 	\item verificare di poter confermare/salvare i dati modificati.
		 \end{itemize}	&	NI	\tabularnewline
		 TA8.1.1	& RFO7.1 & L'utente autenticato deve poter aggiornare la patente di guida del proprio account. All'utente viene chiesto di:
		 \begin{itemize}
		 	\item verificare di poter inserire il numero della patente;
		 	\item verificare di poter inserire la data di rilascio della patente;
		 	\item verificare di poter inserire la data di scadenza della patente;
		 	\item verificare di poter inserire l'immagine fronte e retro della patente;
		 	\item verificare che i dati inseriti siano stati aggiornati nella schermata di gestione del profilo.
		 \end{itemize}	&	NI	\tabularnewline
		 TA8.1.2	& RFO7.2.8 &	L'utente autenticato deve poter aggiornare la password del proprio account. All'utente viene chiesto di:
		 \begin{itemize}
		 	\item verificare che sia richiesto l'inserimento della password attuale;
		 	\item verificare che sia richiesto l'inserimento della nuova password;
		 	\item verificare che non sia accettata una password uguale a quella attuale;
		 	\item verificare che la password inserita rispetti i vincoli generali per la password;
		 	\item verificare che dopo il completamento del procedimento di cambio password, essa sia aggiornata.
		 \end{itemize}	&	NI	\tabularnewline
		 TA8.2 & RFO7.3 	&	L'utente autenticato deve poter cancellare il proprio profilo e i relativi dati. All'utente viene chiesto di:
		 \begin{itemize}
		 	\item di accedere alla schermata di visualizzazione profilo;
		 	\item premere sul pulsante Elimina account;
		 	\item verificare che dopo la conferma dell'eliminazione dell'account, non sia più possibile accedere all'account.
		 \end{itemize}	&	NI	\tabularnewline
		 TA9 & RFF8	&	L'utente autenticato che non ha ancora compilato tutti i dati della sezione profilo deve visualizzare la Progress Bar. All'utente autenticato che non ha compilato tutti i dati viene chiesto di:
		 \begin{itemize}
		 	\item aprire l'applicazione;
		 	\item verificare che nella schermata di gestione profilo sia presente la Progress Bar;
		 	\item inserire qualche dato nella sezione profilo e accertarsi che la Progress Bar sia aumentata del/dei livelli corrispondenti e abbia ricevuto i dovuti punti esperienza per il loro completamento.
		 \end{itemize}	&	NI	\tabularnewline
		 TA9.1	& RFF8.1 &	L'utente autenticato che compila tutti i dati della sezione profilo, completando così la Progress Bar, deve ricevere una ricompensa. All'utente autenticato che ha completato la Progress Bar viene chiesto di:
		 \begin{itemize}
		 	\item verificare che l'applicazione abbia consegnato un premio per il completamento della Progress Bar.
		 \end{itemize}	&	NI	\tabularnewline
		 TA10	& RFF9 &	L'utente autenticato deve poter accedere alla pagina \textit{Gioca} dell'applicazione. All'utente e richiesto di:
		 \begin{itemize}
		 	\item verificare di accedere alla pagina;
		 	\item verificare che all'interno della pagina sia presente il totale di punti esperienza fino ad ora ottenuti;
		 	\item verificare la presenza dei bottoni per la mileston unlock, la lucky spin, i daily rewards e la leaderboard;
		 \end{itemize}	&	NI	\tabularnewline
		 TA10.1	& RFF9.1 &	L'utente autenticato deve visualizzare la tabella Milestone Unlock\glo, che illustra i premi ottenibili dal raggiungimento di determinati livelli d'esperienza. All'utente autenticato viene chiesto di:
		 \begin{itemize}
		 	\item verificare che il bottone presente nella pagina \textit{Gioca} sia disponibile ad essere premuto;
		 	\item verificare che venga visualizzata la tabella Milestone Unlock.
		 \end{itemize}	&	NI	\tabularnewline		 
		 TA10.1.1	& RFF9.1.1 \\ RFF9.1.2 &	L'utente autenticato che ha raggiunto un nuovo livello esperienza, superiore al 5, deve ricevere il rispettivo premio illustrato nella tabella Milestone Unlock\glo. All'utente autenticato che raggiunto un nuovo livello superiore al quinto viene chiesto di:
		 \begin{itemize}
		 	\item verificare che l'applicazione abbia consegnato il relativo premio illustrato nella tabella Milestone Unlock.
		 \end{itemize}	&	NI	\tabularnewline
		 TA10.2 & RFF9.2 &	L'utente autenticato deve visualizzare la ruota della fortuna Lucky Spin\glo. All'utente autenticato viene chiesto di:
		 \begin{itemize}
		 	\item verificare che il bottone presente nella pagina \textit{Gioca} sia disponibile ad essere premuto;
		 	\item verificare che si possa visualizzare la Lucky Spin.
		 \end{itemize}	&	NI	\tabularnewline		 
		 TA10.2.1	& RFF9.2.1 \\ RFF9.2.2 &	L'utente autenticato che ha appena concluso una prenotazione deve ricevere un tentativo per la Lucky Spin\glosp che può riprendere anche in un secondo momento. All'utente autenticato che ha concluso una prenotazione viene chiesto di:
		 \begin{itemize}
		 	\item verificare di aver ricevuto un tentativo per la ruota della fortuna;
		 	\item usare il tentativo girando la ruota;
		 	\item verificare di aver ricevuto il premio promesso.
		 \end{itemize}	&	NI	\tabularnewline	
		 TA9.3	& RFF9.3 &	L'utente generico deve poter visualizzare la classifica degli utenti migliori. All'utente viene chiesto di:
		 \begin{itemize}
		 	\item verificare che il bottone presente nella pagina \textit{Gioca} sia disponibile ad essere premuto;
		 	\item verificare che le classifiche siano visualizzate correttamente.
		 \end{itemize}	&	NI	\tabularnewline	
		 TA9.3.1	& RFF9.3.1 & Dopo un tempo prestabilito, i primi tre utente della classifica devono ricevere un premio. All'utente che si è classificato tra le prime tre posizioni, nella data prestabilita, viene chiesto di:
		 \begin{itemize}
		 	\item verificare di aver ricevuto il premio promesso.
		 \end{itemize}	&	NI	\tabularnewline	
	 	 TA9.4	& RFF9.4	& L'utente autenticato deve poter visualizzare la tabella Daily Rewards che illustra i premi giornalieri del mese corrente. All'utente viene chiesto di :
	 	 \begin{itemize}
	 	 	\item verificare che il bottone presente nella pagina \textit{Gioca} sia disponibile ad essere premuto;
	 	 	\item verificare che si possa accedere e visualizzare la tabella dei Daily Rewards.
	 	 \end{itemize}	&	NI	\tabularnewline	
	 	 TA9.4.1	& RFF9.4.1 \\ RFF9.4.1.1 &	L'utente autenticato deve poter ritirare il premio del giorno corrente dalla tabella Daily rewards. \\All'utente viene chiesto di:
	 	 \begin{itemize}
	 	 	\item ritirare il premio del giorno dalla tabella dei premi giornalieri tramite l'apposito pulsante "Ritira premio";
	 	 	\item verificare di non poter ritirare di nuovo il premio del giorno attuale;
	 	 	\item verificare di non poter ritirare il premio dei giorni successivi o di quelli precedenti.
	 	 \end{itemize}	&	NI	\tabularnewline
		 TA10	& RFF10 &	L'utente autenticato deve avere un codice personale visualizzabile nella schermata di gestione del profilo, da inviare ad amici che non sono ancora registrati al servizio offerto da \textit{GaiaGo}.  All'utente autenticato viene chiesto di:
		 \begin{itemize}
		 	\item accedere alla sezione \textit{Gestione Profilo};
		 	\item verificare che sia presente il suo codice personale.
		 \end{itemize}	&	NI	\tabularnewline
		 TA11	& RFF2.6 & 	L'utente che si registra inserendo un codice amico deve ricevere un premio dopo la registrazione. Tale premio deve essere corrisposto anche al proprietario del codice. All'utente che si è registrato inserendo un codice amico viene chiesto di:
		 \begin{itemize}
		 	\item visualizzare la schermata relativa alla registrazione;
		 	\item inserire, se a conoscenza, i dati nel campo del codice amico;
		 	\item confermare/inviare i dati della registrazione;
		 	\item verificare di aver ricevuto i punti esperienza per la registrazione tramite codice amico;
		 	\item verificare, chiedendo all'amico se ha ricevuto lo stesso premio in seguito alla sua registrazione.
		 \end{itemize}	&	NI	\tabularnewline	
	 	 TA12	& RFF11 &	L'utente autenticato deve poter accedere al Minigioco che consiste in un garage che dà la possibilità di effettuare modifiche ad un'auto. All'utente viene chiesto di:
		 \begin{itemize}
		 	\item accedere alla sezione Minigioco dal menu dell'applicazione;
		 	\item verificare di poter visualizzare l'auto base e le sue statistiche (velocità, accelerazione, peso, maneggevolezza);
		 	\item verificare di poter entrare nella modalità di modifica auto;
		 	\item apportare qualche modifica;
		 	\item verificare che le modifiche siano state effettivamente installate.
		 \end{itemize}	&	NI	\tabularnewline
		 TA12.1	& RFF11.1\\ RFF11.1.1 \\ RFF11.1.2 \\ RFF11.1.3 \\ RFF11.1.4 \\ RFF11.1.5  \\RFF11.3  &	L'utente autenticato deve poter accedere al Minigioco del garage in cui potrà modificare le prestazioni dell'auto base. All'utente viene chiesto di:
		 \begin{itemize}
		 	\item accedere alla sezione Minigioco dal menu dell'applicazione ed entrare in modalità modifica prestazioni auto;
		 	\item verificare di poter sostituire i seguenti pezzi:
		 		\begin{itemize}
		 			\item motore;
		 			\item centralina;
		 			\item trasmissione;
		 			\item sospensioni;
		 			\item gomme.
		 		\end{itemize}
		 	\item confermare l'installazione della modifica.
		 \end{itemize}	&	NI	\tabularnewline
		 TA12.2	& RFF11.2\\ RFF11.2.1 \\ RFF11.2.2 \\ RFF11.2.3 \\ RFF11.2.4 \\ RFF11.2.5  \\ 	RFF11.2.6 \\ RFF11.2.7 \\RFF11.3 & L'utente autenticato deve poter accedere al Minigioco del garage in cui potrà modificare l'estetica dell'auto base. All'utente viene chiesto di:
		 \begin{itemize}
		 	\item accedere alla sezione Minigioco dal menu dell'applicazione ed entrare in modalità modifica estetica auto;
		 	\item verificare di poter sostituire i seguenti pezzi:
		 		\begin{itemize}
		 			\item colore;
		 			\item adesivi;
		 			\item paraurti;
		 			\item fari;
		 			\item scarichi;
		 			\item cerchioni;
		 			\item alettoni.
		 		\end{itemize}
		 	\item confermare l'installazione della modifica.
		 \end{itemize}	&	NI	\tabularnewline
\end{longtable}
\newpage
\subsection{Test di sistema}
I test di sistema sono impiegati per garantire il corretto funzionamento delle componenti dell'intero sistema. Tali test verranno indicati nel seguente modo:\\
	\centerline{\textbf{TS[Codice]}}
dove:
\begin{itemize}
	\item \textbf{Codice}: rappresenta il codice identificativo crescente del test.
\end{itemize}
	Nella seguente tabella vengono anche tracciati i requisiti a cui fanno riferimento. Tali requisiti sono indicati nel documento \textit{Analisi dei requisiti v2.0.0}, sezione §4.

	\renewcommand{\arraystretch}{1.5}
	\rowcolors{2}{pari}{dispari}
	
	\begin{longtable}{ >{\centering}p{0.10\textwidth} >{\centering}p{0.10\textwidth} >{\centering}p{0.650\textwidth}
			>{\centering}p{0.10\textwidth}}% >{\centering}p{0.14\textwidth}}
			
		%\hline
		\caption{Riepilogo test di sistema}\\	
		\rowcolorhead
		\textbf{\color{white}Test} 
		& \textbf{\color{white}Requisito} 
		& \textbf{\color{white}Descrizione} 
		& \centering\textbf{\color{white}Esito}
	%	& \textbf{\color{white}Fonti} 
		\tabularnewline %\hline 
		\endfirsthead	
		
		\rowcolor{white}\caption[]{(continua)}\\	
		\rowcolorhead
		\textbf{\color{white}Test} 
		& \textbf{\color{white}Requisito} 
		& \textbf{\color{white}Descrizione} 
		& \centering\textbf{\color{white}Esito}
		%	& \textbf{\color{white}Fonti} 
		\tabularnewline %\hline 
		\endhead	
		
		TS1	& RFO2 & Verificare che il sistema permetta all'utente di effettuare la registrazione all'applicazione qualora non lo fosse. &	NI	\tabularnewline
		
		 TS2	& RFD1 & Verificare che in seguito alla registrazione dell'utente, il sistema renda disponibile la guida introduttiva.	&	NI	\tabularnewline
		 
		 TS3	& RFO3 & Verificare che il sistema renda disponibile all'utente la possibilità di effettuare il login e che in seguito gli si presenti la schermata \textit{Home}. &	NI	\tabularnewline
		 
		 TS3.1 & RF02.7 & Verificare che il sistema faccia apparire un messaggio d'errore nel caso in cui l'utente non compili il campo email e password. &	NI	\tabularnewline
		 
		 TS3.2 & RF03.3 &	Verificare che il sistema faccia apparire un messaggio d'errore nel caso in cui l'utente inserisca una combinazione email-password errata. &	NI	\tabularnewline
		 
		 TS4 & RFO3.4	& Verificare che il sistema permetta all'utente registrato di recuperare la propria password dalla schermata di login.	&	NI	\tabularnewline
		 
		 TS5	& RFO4 & Verificare che il sistema permetta all'utente autenticato di effettuare il logout. &	 NI \tabularnewline
		 
		 TS6	& RFO5 & Verificare che il sistema permetta all'utente appena autenticato di visualizzare la \textit{Home} dell'applicazione. &	NI	\tabularnewline
		 
		 TS6.1	& RFO5.1 & Verificare che il sistema permetta all'utente autenticato di poter gestire e visualizzare i propri veicoli oltre che a selezionarli o aggiungerli. 	&	NI	\tabularnewline 
		 
		 TS6.2 & RFO5.1.1 &	Verificare che il sistema permetta all'utente autenticato di poter inserire un nuovo veicolo. 	&	NI	\tabularnewline
		
		 TS6.2.1 & RFO5.1.2 \\ RFO5.1.2.1 \\ RFO5.1.2.2 \\ RFO5.1.2.3 &	Verificare che il sistema permetta all'utente autenticato di poter visualizzare i dettagli del veicolo e che siano disponibili i bottoni per visualizzare le disponibilità, le statistiche e la rimozione del veicolo . 	&	NI	\tabularnewline
		 
		 TS6.3 & RFO5.2 \\ RFO5.2.1  & Verificare che il sistema permetta all'utente autenticato di poter visualizzare la lista di tutte le sue prenotazioni attive.
		 Per ogni prenotazione attiva presente nella lista l'utente deve poter visualizzare i dettagli riassuntivi (marca, modello, stato della prenotazione, data, fascia oraria e foto del veicolo) ed eseguire operazioni di gestione su di esse (visualizzazione dettaglio). 	&	NI	\tabularnewline
		 
		 TS6.3.1 & RFO5.2.2 & Verificare che il sistema permetta all'utente proprietario del veicolo di confermare o annullare una richiesta di prenotazione. 	&	NI	\tabularnewline
		 
		 TS6.3.2 & RFO5.2.3 & Verificare che il sistema permetta all'utente proprietario del veicolo di confermare l'avvenuta consegna delle chiavi del veicolo all'usufruente. 	&	NI	\tabularnewline
		 
		 TS6.3.3 & RFO5.2.4 & Verificare che il sistema permetta all'utente proprietario del veicolo di chiudere la prenotazione nel momento in cui ha ricevuto il veicolo e le chiavi sbloccando così la possibilità di recensire l'utente usufruente. 	&	NI	\tabularnewline
		 
		 TS6.3.4 & RFO5.2.5 &	Verificare che il sistema permetta all'utente usufruente di poter annullare una prenotazione in attesa di conferma da parte del proprietario. 	&	NI	\tabularnewline
		 
		 TS6.3.5 & RFO5.2.6 & Verificare che il sistema permetta all'utente usufruente di poter chiudere la prenotazione nel momento in cui riconsegni il veicolo e le chiavi sbloccando così la possibilità di recensire l'utente proprietario del veicolo. 	&	NI	\tabularnewline
		  				
		 TS6.4	& RFO5.3 \\ RFD5.3.1 \\ RFD5.3.2 \\ RFD5.3.3 \\ RFD5.3.4 \\ RFO5.3.5 &	Verificare che il sistema permetta all'utente autenticato di poter effettuare una ricerca specifica sui veicoli disponibili, applicando dei filtri, e poter proseguire con la prenotazione di uno di essi. 	&	NI	\tabularnewline
		  
		 TS7 &	RFD6 &  Verificare che il sistema permetta all'utente autenticato di poter visualizzare una lista contente tutte le sue prenotazioni concluse. 	&	NI	\tabularnewline
		 
		TS8	& RFO7 & Verificare che il sistema permetta all'utente autenticato di poter visualizzare, modificare e cancellare il proprio account con i relativi dati. 	&	NI	\tabularnewline
		 
		TS8.1 & RFO7.2 & Verificare che il sistema permetta all'utente autenticato di poter aggiornare i dati del proprio account. 	&	NI	\tabularnewline
		 
		 TS8.1.1	& RFO7.1 & Verificare che il sistema permetta all'utente autenticato di poter aggiornare la patente di guida del proprio account. 	&	NI	\tabularnewline
		 
		TS8.1.2	& RFO7.2.8 &	Verificare che il sistema permetta all'utente autenticato di poter aggiornare la password del proprio account. 	&	NI	\tabularnewline 
		
		TS8.2 & RFO7.3 	&	Verificare che il sistema permetta all'utente autenticato di poter cancellare il proprio profilo e i relativi dati. 	&	NI	\tabularnewline
		
		 TS9 & RFF8	&	Verificare che il sistema permetta all'utente autenticato, che non ha ancora compilato tutti i dati della sezione profilo, di visualizzare la Progress Bar. 	&	NI	\tabularnewline
		 
		 TS9.1	& RFF8.1 & Verificare che il sistema permetta all'utente autenticato, che compila tutti i dati della sezione profilo completando così la Progress Bar, di ricevere una ricompensa. 	&	NI	\tabularnewline
		 
		 TS10	& RFF9 &	Verificare che il sistema permetta all'utente autenticato di poter accedere alla pagina \textit{Gioca} dell'applicazione. 	&	NI	\tabularnewline
		 
		 TS10.1	& RFF9.1 &	Verificare che il sistema permetta all'utente autenticato di visualizzare la tabella Milestone Unlock\glo, che illustra i premi ottenibili dal raggiungimento di determinati livelli d'esperienza. 	&	NI	\tabularnewline	
		 	 
		 TS10.1.1	& RFF9.1.1 \\ RFF9.1.2 &	Verificare che il sistema permetta all'utente autenticato, che ha raggiunto un nuovo livello esperienza superiore al 5, di ricevere il rispettivo premio illustrato nella tabella Milestone Unlock\glo. 	&	NI	\tabularnewline
		 
		 TS10.2 & RFF9.2 &	Verificare che il sistema permetta all'utente autenticato di visualizzare la ruota della fortuna Lucky Spin\glo. 	&	NI	\tabularnewline		
		  
		 TS10.2.1	& RFF9.2.1 &	Verificare che il sistema permetta all'utente autenticato, che ha appena concluso una prenotazione, di ricevere un tentativo per la Lucky Spin\glo. 	&	NI	\tabularnewline	
		 
		  TS10.3	& RFF9.3 &	Verificare che il sistema permetta all'utente generico di poter visualizzare la classifica degli utenti migliori. 	&	NI	\tabularnewline	
		  
		 TS10.3.1	& RFF9.3.1 & Verificare che, dopo un tempo prestabilito, il sistema permetta ai primi tre utenti della classifica di ricevere un premio.	&	NI	\tabularnewline
		 
		 TS10.4	& RFF9.4	& Verificare che il sistema permetta all'utente autenticato la visualizzazione della tabella Daily Rewards che illustra i premi giornalieri del mese corrente. 	&	NI	\tabularnewline	
		 
		 TS10.4.1	& RFF9.4.1 &	Verificare che il sistema permetta all'utente autenticato all'utente autenticato di poter ritirare il premio del giorno corrente dalla tabella Daily rewards. 	&	NI	\tabularnewline	
		 
		 TS11	& RFF10 &	Verificare che il sistema renda disponibile all'utente autenticato un codice personale visualizzabile nella schermata di gestione del profilo, da inviare ad amici che non sono ancora registrati al servizio offerto da \textit{GaiaGo}.  	&	NI	\tabularnewline
		 
		 TS12	& RFF2.6 & In seguito ad una registrazione con l'inserimento di un codice amico, verificare che il sistema abbia corrisposto ad entrambi gli utenti i punti esperienza spettanti. 	&	NI	\tabularnewline	

	 TS13	& RFF11 &	Verificare che il sistema permetta agli utenti autenticati di poter accedere al Minigioco che consiste in un garage che dà la possibilità di effettuare modifiche ad un'auto. &	NI	\tabularnewline
	 
	TS13.1	& RFF11.1\\ RFF11.1.1 \\ RFF11.1.2 \\ RFF11.1.3 \\ RFF11.1.4 \\ RFF11.1.5  \\RFF11.3  &	Verificare che il sistema permetta all'utente autenticato all'utente autenticato di poter accedere al Minigioco del garage in cui potrà modificare le prestazioni dell'auto base. 	&	NI	\tabularnewline
	
		TS13.2	& RFF11.2\\ RFF11.2.1 \\ RFF11.2.2 \\ RFF11.2.3 \\ RFF11.2.4 \\ RFF11.2.5  \\ RFF11.2.6 \\ RFF11.2.7 \\RFF11.3 & Verificare che il sistema permetta all'utente autenticato all'utente autenticato di poter accedere al Minigioco del garage in cui potrà modificare l'estetica dell'auto base. 	&	NI	\tabularnewline
\end{longtable}
\newpage
\subsection{Test di integrazione}
I test di integrazione sono usati per verificare il corretto funzionamento tra le varie unità dell'architettura. Tali test verranno indicati nel seguente modo:\\
	\centerline{\textbf{TI[Codice]}}
dove:
\begin{itemize}
	\item \textbf{Codice}: rappresenta il codice identificativo crescente del test.
\end{itemize}
Nella seguente tabella sono rappresentati i test di integrazione:
\renewcommand{\arraystretch}{1.5}
	\rowcolors{2}{pari}{dispari}

\begin{longtable}{ >{\centering}p{0.10\textwidth} >{\centering}p{0.650\textwidth} >{\centering}p{0.10\textwidth}
			>{\centering}p{0.10\textwidth}}% >{\centering}p{0.14\textwidth}}
			
		%\hline
		\caption{Riepilogo test di integrazione}\\	
		\rowcolorhead
		\textbf{\color{white}Test}  
		& \textbf{\color{white}Descrizione} 
		& \textbf{\color{white}Stato}
		& \centering\textbf{\color{white}Esito}
	%	& \textbf{\color{white}Fonti} 
		\tabularnewline %\hline 
		\endfirsthead	
		
		\rowcolor{white}\caption[]{(continua)}\\	
		\rowcolorhead
		\textbf{\color{white}Test} 
		& \textbf{\color{white}Descrizione}
		& \textbf{\color{white}Stato} 
		& \centering\textbf{\color{white}Esito}
		%	& \textbf{\color{white}Fonti} 
		\tabularnewline %\hline 
		\endhead	
TI1&Verificare che l'utente possa registrarsi all'interno del sistema, ricevendo un messaggio di conferma se l'operazione va a buon fine, altrimenti un messaggio di errore specifico per l'errore commesso. & I & S \tabularnewline

TI2&Verificare che l'utente già registrato possa effettuare il login, mostrando un messaggio di successo se l'operazione va a buon fine, altrimenti un messaggio di errore specifico per l'errore commesso. & I & S \tabularnewline

TI3&Verificare che l'utente possa inserire informazioni aggiuntive all'interno del profilo. & I & S \tabularnewline 

TI4&Verificare che l'utente dopo aver effettuato la registrazione visualizzi correttamente l'activity introduttiva. & I & S \tabularnewline

TI5&Verificare che l'utente possa inserire la patente, mostrando un messaggio di successo se l'operazione va a buon fine, altrimenti un messaggio di errore specifico per l'errore commesso. & I & S \tabularnewline

TI6&Verificare che l'utente che ha effettuato l'accesso possa visualizzare i propri veicoli. & I & S \tabularnewline

TI7&Verificare che l'utente che ha effettuato l'accesso possa gestire la disponibilità dei proprio veicoli. & I & S \tabularnewline


TI8&Verificare che l'utente che ha effettuato l'accesso possa ricercare un veicolo adatto alle sue esigenze. & I & S \tabularnewline

TI9&Verificare che l'utente che ha effettuato l'accesso possa visualizzare le prenotazioni che lo riguardano. & I & S \tabularnewline

TI9&Verificare che l'utente che ha effettuato l'accesso possa gestire le prenotazioni che devono ancora avvenire. & I & S \tabularnewline

\end{longtable}		
		
\subsection{Test di unità}
I test di unità hanno l'obiettivo di verificare il corretto funzionamento della parte più piccola autonoma del lavoro realizzato. Tali test verranno indicati nel seguente modo:\\
	\centerline{\textbf{TU[Codice]}}
dove:
\begin{itemize}
	\item \textbf{Codice}: rappresenta il codice identificativo crescente del test.
\end{itemize}
Tale tipologia di test verrà sviluppata in un immediato futuro, in seguito alla richiesta della sua istanziazione.



\renewcommand{\arraystretch}{1.5}
	\rowcolors{2}{pari}{dispari}
	
	\begin{longtable}{ >{\centering}p{0.10\textwidth}  >{\centering}p{0.650\textwidth} >{\centering}p{0.10\textwidth}
			>{\centering}p{0.10\textwidth}}% >{\centering}p{0.14\textwidth}}
			
		%\hline
		\caption{Riepilogo test di accettazione}\\	
		\rowcolorhead
		\centering\textbf{\color{white}Test} 
		& \centering\textbf{\color{white}Descrizione} 
		& \centering\textbf{\color{white}Stato}
		& \centering\textbf{\color{white}Esito}
	%	& \textbf{\color{white}Fonti} 
		\tabularnewline %\hline 
		\endfirsthead	
		
		\rowcolor{white}\caption[]{(continua)}\\	
		\rowcolorhead
		\centering\textbf{\color{white}Test} 
		& \centering\textbf{\color{white}Descrizione} 
		& \centering\textbf{\color{white}Stato}
		& \centering\textbf{\color{white}Esito}
		%	& \textbf{\color{white}Fonti} 
		\tabularnewline %\hline 
		\endhead	
	\end{longtable}
