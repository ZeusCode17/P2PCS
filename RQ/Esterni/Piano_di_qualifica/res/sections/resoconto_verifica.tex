\section{Resoconto attività di verifica}

%%%%%%%%%%%%%%%%%%%%%%%%%%%%%%%%%%%%%%%%%%%%%%%%%%%%%%%%%%%%%%%%%%%%
%%%%%%%%%%%%%%%%%%%%%%%%%%%%%%%%%%%%%%%%%%%%%%%%%%%%%%%%%%%%%%%%%%%%
\subsection{Riassunto delle attività di verifica}
%%%%%%%%%%%%%%%%%%%%%%%%%%%%%%%%%%%%%%%%%%%%%%%%%%%%%%%%%%%%%%%%%%%%
%%%%%%%%%%%%%%%%%%%%%%%%%%%%%%%%%%%%%%%%%%%%%%%%%%%%%%%%%%%%%%%%%%%%

	\subsubsection{Revisione dei requisiti}
	
	Nel periodo precedente alla consegna di tale revisione sono stati verificati i documenti ed i processi. \\
	La verifica svolta sui documenti è avvenuta seguendo le indicazioni delle Norme di Progetto v3.0.0 e misurando la metrica descritta in §2.3.3.3.2. \\
	L'attività di walkthrough\glosp sui documenti ha evidenziato una serie di anomalie, in questo modo è stato possibile stilare la lista di anomalie frequenti (§3.3.3.1.1, Norme di Progetto v3.0.0) che si potrà controllare tramite Inspection\glosp. \\
	Il tracciamento dei requisiti è stato effettuato tramite l'utilizzo del software PragmaDB. \\
	In questa revisione non è stato possibile valutare la qualità del prodotto, data la sua non implementazione che avverrà nelle fasi successive.
	
	\subsubsection{Revisione di progettazione}
	
	Nel periodo precedente alla consegna di tale revisione sono stati verificati i documenti ed i processi. \\
	La verifica svolta sui documenti è avvenuta seguendo le indicazioni delle Norme di Progetto v3.0.0 e misurando le metriche adottate. È stata effettuata anche la misurazione delle metriche mancanti nella fase di RR. \\
	In assenza di strumenti automatici, i \textit{Verificatori} si sono impegnati nella correzione degli errori di stile tipografico segnalati nell'esito della revisione dei requisiti.
	
	\subsubsection{Revisione di qualifica}
	\paragraph{Product Baseline}
	In vista dell'incontro di presentazione della Product Baseline, i \textit{Verificatori} hanno svolto le attività di verifica effettuando le misurazioni delle metriche di prodotto e verificando i documenti prodotti per la Product Baseline.
	
			
	\paragraph{Considerazioni generali RQ}
	Nel periodo precedente alla consegna della revisione di qualifica sono stati verificati i documenti, processi e qualità del prodotto. \\
	L'attività di verifica svolta dai \textit{Verificatori} è avvenuta come determinato dal Piano di Progetto v3.0.0, seguendo le indicazioni delle Norme di Progetto v3.0.0 e misurando le metriche indicate in §2 e §3.  Successivamente sono state effettuate le misurazioni delle metriche relative ai documenti. \\
	Fa eccezione un'ulteriore attività di verifica, richiesta inseguito alla attuazione delle modifiche richieste dalla proponente. \\
	Per velocizzare l'attività di misurazione, ai \textit{Programmatori} è stato chiesto di tenere conto delle metriche riguardanti lo sviluppo del codice, aggiornando ad ogni incremento lo stato di calcolo delle metriche relative alla porzione di codice implementata.
	Durante l'attività di test, sono emersi dei problemi di integrazione delle componenti per la tra le componenti che sono stati prontamente risolti dai \textit{Programmatori}. 
	\newpage
%%%%%%%%%%%%%%%%%%%%%%%%%%%%%%%%%%%%%%%%%%%%%%%%%%%%%%%%%%%%%%%%%%%%
%%%%%%%%%%%%%%%%%%%%%%%%%%%%%%%%%%%%%%%%%%%%%%%%%%%%%%%%%%%%%%%%%%%%
\subsection{Dettaglio dell'esito delle revisioni e relative modifiche}
%%%%%%%%%%%%%%%%%%%%%%%%%%%%%%%%%%%%%%%%%%%%%%%%%%%%%%%%%%%%%%%%%%%
%%%%%%%%%%%%%%%%%%%%%%%%%%%%%%%%%%%%%%%%%%%%%%%%%%%%%%%%%%%%%%%%%%%%
	\subsubsection{Revisione dei requisiti}
		In generale, risulta che i documenti abbiano un buona struttura ma che siano scarsi per contenuti. \\
		Il gruppo si è impegnato a integrare i contenuti ritenuti insoddisfacenti e ad aggiungere maggiore dettaglio nella loro descrizione.\\ 
		 Di seguito vengono descritte brevemente le modifiche apportate in base alle segnalazioni ricevute:
	\begin{itemize}
		\item corrette le difformità nei titoli dei vari documenti;
		\item \textbf{Norme di progetto}: 
			\begin{itemize}	
				\item inserito lo standard ISO/IEC 12207 come riferimento informativo;
				\item modificato §2.1 del Processo di Fornitura, descrivendo le attività previste dallo standard ISO/IEC 12207 di interesse per il gruppo;
				\item aggiunto maggiore dettaglio a §2.2 del Processo di Sviluppo;
				\item aggiunto §4.2 che tratta il processo di Fornitura.
			\end{itemize}
		\item \textbf{Analisi dei Requisiti}: 
			\begin{itemize}
				\item corretti tutti i casi d'uso secondo le indicazioni ricevute;
				\item aggiunto UC24, UC25, UC26 e UC27 per completare il tracciamento requisiti - casi d'uso.
			\end{itemize} 
		\item \textbf{Piano di Progetto}: 
			\begin{itemize}
				\item modificata la pianificazione delle fasi successive all'RR, cercando di seguire il modello incrementale;
				\item modificato il titolo di §6 in "Consuntivo di periodo" e inserito paragrafo "Conclusioni" con relativa analisi e considerazioni del periodo a precedere.
			\end{itemize}
		\item \textbf{Piano di Qualifica}: 
			\begin{itemize}
				\item stabiliti obiettivi e strategie per garantire la qualità del processo di Fornitura, in seguito all'analisi dello standard ISO/IEC 12207:1995 §5.2;
				\item inserita tabella riportante gli esiti delle metriche calcolabile in fase RR.
			\end{itemize}
	\end{itemize}
\newpage
	\subsubsection{Revisione di progettazione}
	
		Si riscontrano vari problemi di pianificazione e controllo della qualità. \\
		Il gruppo si è impegnato a integrare i contenuti ritenuti insoddisfacenti.\\ 
		 Di seguito vengono descritte brevemente le modifiche apportate in base alle segnalazioni ricevute:
	\begin{itemize}
		\item rivisti comandi \LaTeX \space che generano la struttura dei documenti;
		\item richiesto ai membri del gruppo di aggiungere i riferimenti a testi e l'attribuzione delle fonti per le parti redatte;
		\item inconsistenze nei titoli: il gruppo si è reso conto di non avere gli strumenti necessari per correggere l'errore in modo automatico e a causa delle risorse limitate non può impegnarsi nello sviluppo di tali strumenti. Tuttavia, per rimediare al problema i \textit{Verificatori} si sono impegnati in un'ulteriore verifica inspection\glosp, seguendo la lista di controllo redatta dagli \textit{Amministratori} nel \textit{Norme di Progetto v.3.0.0}. 
		\item \textbf{Norme di progetto}: 
			\begin{itemize}	
				\item rivisti e ampliati le sezioni ritenute insoddisfacenti.
			\end{itemize}
		\item \textbf{Analisi dei Requisiti}: 
			\begin{itemize}
				\item corretti tutti i casi d'uso secondo le indicazioni ricevute;
				\item riguardati i diagrammi dei casi d'uso.
			\end{itemize} 
		\item \textbf{Piano di Progetto}: 
			\begin{itemize}
				\item rivisti i diagrammi di Gantt;
				\item rivisto il consuntivo.
			\end{itemize}
		\item \textbf{Piano di Qualifica}: 
			\begin{itemize}
				\item rivista la struttura dell'appendice secondo le indicazioni ricevute;
				\item aumentata la frequenza delle misurazioni di qualità.
			\end{itemize}
	\end{itemize}
	
	
	
	\newpage
%%%%%%%%%%%%%%%%%%%%%%%%%%%%%%%%%%%%%%%%%%%%%%%%%%%%%%%%%%%%%%%%%%%%
%%%%%%%%%%%%%%%%%%%%%%%%%%%%%%%%%%%%%%%%%%%%%%%%%%%%%%%%%%%%%%%%%%%%
\subsection{Dettaglio delle verifiche tramite analisi}
%%%%%%%%%%%%%%%%%%%%%%%%%%%%%%%%%%%%%%%%%%%%%%%%%%%%%%%%%%%%%%%%%%%%
%%%%%%%%%%%%%%%%%%%%%%%%%%%%%%%%%%%%%%%%%%%%%%%%%%%%%%%%%%%%%%%%%%%%

%%%%%%%%%%%%%%%%%%%%%%%%%%%%%%%%%%%%%%%%%%%%%%%%%%%%%%%%%%%%%%%%%%%%
	\subsubsection{Revisione dei requisiti}
%%%%%%%%%%%%%%%%%%%%%%%%%%%%%%%%%%%%%%%%%%%%%%%%%%%%%%%%%%%%%%%%%%%%
		\paragraph{Qualità di processo}

		Di seguito vengono riportati gli esiti delle metriche derivanti dalla gestione di qualità di processi, seguiti dalla tabella degli indici di Gulpease e degli indici di Gunning Fog che riporta gli esiti di tutti i documenti prodotti finora. \\
\textbf{Legenda}:
\begin{itemize}
	\item \textbf{N.C.} - Non Calcolabile;
	\item \textbf{S} - Superato;
	\item \textbf{N.S.} - Non Superato;
	\item \textbf{\%} - Percentuale;
	\item \textbf{V} - Valore numerico;
	\item \textbf{\euro{}} - Euro.
\end{itemize}
	\rowcolors{2}{pari}{dispari}	
	\begin{longtable}{ >{\centering}p{0.25\textwidth} >{\centering}p{0.10\textwidth}
			 >{\centering}p{0.10\textwidth} >{\centering}p{0.07\textwidth} >{\centering}p{0.31\textwidth}}
		\caption{ Valutazione della qualità di processo - RR} \\
		%\hline
		\rowcolorhead
		
		\centering\textbf{\color{white}Nome metrica} 
		& \centering\textbf{\color{white}Unità di misura} 
		& \centering\textbf{\color{white}Valore} 
		& \centering\textbf{\color{white}Esito}
		& \centering\textbf{\color{white}Accettazione}
		\tabularnewline %\hline 
		\endfirsthead
		
		\rowcolor{white}\caption[]{(continua)}\\	
		\rowcolorhead
		\centering\textbf{\color{white}Nome metrica} 
		& \centering\textbf{\color{white}Unità di misura} 
		& \centering\textbf{\color{white}Valore} 
		& \centering\textbf{\color{white}Esito}
		& \centering\textbf{\color{white}Accettazione}
		\tabularnewline %\hline 
		\endhead
		
		Requisiti obbligatori soddisfatti & \% & N.C. & N.S. & 100
		\tabularnewline 
		
		Coupling Between Object classes & V & N.C. & N.S. & $0 \leq CBO \leq 6$
		\tabularnewline
		
		Planned Value & \euro{} & 4.688,00 & S & $ \geq 0$
		\tabularnewline
		
		Actual Cost & \euro{} & 4.833,00 & S & $0 \leq AC \leq 11.689,00 $
		\tabularnewline
		
		Earned Value & \euro{} & 4.688,00 & S & $ \geq 0$
		\tabularnewline
		
		Budget at Completion & \euro{} & 4.688,00 & S & $4.591,35 \leq BAC \leq 5074,65 $
		\tabularnewline
		
		Cost Variance & \euro{} & -145,00 & N.S. & $ \geq 0$
		\tabularnewline
		
		Schedule Variance & \euro{} & 0,00 & S & $ \geq 0$
		\tabularnewline
		
		Code Coverage & \% & N.C. & N.S. & $ \geq 75\%$
		\tabularnewline
		
		Indice Gunning fog (media) & V & 13,71 & S & $ \leq 16$
		\tabularnewline
		
		Indice di Gulpease (media) & V & 68,54 & S & $40 < I_G < 100$
		\tabularnewline
		
		Correttezza ortografica & V & 0 & S & 0
		\tabularnewline
		
		Percentuale di metriche soddisfatte & \% & 88,89 & S & 100
		\tabularnewline
		
	\end{longtable}
	
	\paragraph*{Considerazioni}
	Ci sono delle metriche non calcolabili in questa fase. Queste ultime si riferiscono all'analisi del codice e alla sua progettazione di dettaglio che verrà svolta nelle fasi successive.
	Risulta invece negativa la metrica di Cost Variance, in conseguenza dell'impiego di 8 ore di lavoro aggiuntive rispetto a quanto previsto. \\
	La percentuale di metriche soddisfatte tiene conto solo delle metriche calcolabili, quindi di un totale di 9 tra cui 8 soddisfatte; se si ritiene che le metriche non calcolabili siano da considerare non superate, la percentuale di metriche soddisfatte risulta comunque superata, per un valore di 66.67.

	\rowcolors{2}{pari}{dispari}
	\begin{longtable}{ >{\centering}p{0.40\textwidth} >{\centering}p{0.25\textwidth}
			 >{\centering}p{0.2075\textwidth} }
		\caption{ Verifiche automatizzate indice di Gulpease - RR} \\
		%\hline
		\rowcolorhead
		\centering\textbf{\color{white}Documento} 
		& \centering\textbf{\color{white}Indice Gulpease} 
		& \centering\textbf{\color{white}Esito}
		\tabularnewline %\hline 
		\endfirsthead
			
		\rowcolor{white}\caption[]{(continua)}\\	
		\rowcolorhead
		\centering\textbf{\color{white}Documento} 
		& \centering\textbf{\color{white}Indice Gulpease} 
		& \centering\textbf{\color{white}Esito}
		\tabularnewline %\hline 
		\endhead	
			
	
		\textit{Analisi dei Requisiti v1.0.0} & 52,32 & Superato
		
		\tabularnewline 
		\textit{Glossario v1.0.0} & 100 & Superato
				
		\tabularnewline 
		\textit{Norme di Progetto v1.0.0} & 57,61 & Superato
		
		\tabularnewline 
		\textit{Piano di Progetto v1.0.0} & 53,39 & Superato
		
		\tabularnewline 
		\textit{Piano di Qualifica v1.0.0} & 56,87 & Superato	
		
		\tabularnewline 
		\textit{Studio di Fattibilità v1.0.0} & 54,93 & Superato
		
		\tabularnewline 
		\textit{Verbale Esterno 2019-03-14 v1.0.0} & 80 & Superato
		
		\tabularnewline 
		\textit{Verbale Esterno 2019-03-25 v1.0.0} & 72 & Superato
		
		\tabularnewline 
		\textit{Verbale Esterno 2019-04-10 v1.0.0} & 69 & Superato
		
		\tabularnewline 
		\textit{Verbale Interno 2019-03-06 v1.0.0} & 79 & Superato
		
		\tabularnewline 
		\textit{Verbale Interno 2019-03-13 v1.0.0} & 77 & Superato
		
		\tabularnewline 
		\textit{Verbale Interno 2019-03-18 v1.0.0} & 71 & Superato
	\end{longtable}
	\paragraph*{Considerazioni} 
	Tutti i documenti hanno un esito positivo. La media dei valori è di 68.54, un valore intermedio tra il limite inferiore stabilito di 40 e il limite inferiore preferibile di 80. \\
	I valori più alti sono riscontrati nei verbali e glossario, mentre i valori più bassi sono presenti nell'\texttt{AdR} e nel \texttt{PdP} ma data la loro natura tecnica, riteniamo soddisfacente l'esito della verifica.
	
	
	
	\rowcolors{2}{pari}{dispari}
	\begin{longtable}{ >{\centering}p{0.40\textwidth} >{\centering}p{0.25\textwidth}
			 >{\centering}p{0.2075\textwidth}}
		\caption{ Verifiche automatizzate indice di Gunning Fog  - RR} \\
		%\hline
		\rowcolorhead
		\centering\textbf{\color{white}Documento} 
		& \centering\textbf{\color{white}Indice Gunning Fog} 
		& \centering\textbf{\color{white}Esito}
		\tabularnewline %\hline 
		\endfirsthead
		
		\rowcolor{white}\caption[]{(continua)}\\	
		\rowcolorhead
		\centering\textbf{\color{white}Documento} 
		& \centering\textbf{\color{white}Indice Gunning Fog} 
		& \centering\textbf{\color{white}Esito}
		\tabularnewline %\hline 
		\endhead	
		
			
		\textit{Analisi dei Requisiti v1.0.0} & 13,06 & Superato
		
		\tabularnewline 
		\textit{Glossario v1.0.0} & 12,32 & Superato
				
		\tabularnewline 
		\textit{Norme di Progetto v1.0.0} & 11,70  & Superato
		
		\tabularnewline 
		\textit{Piano di Progetto v1.0.0} & 12,29 & Superato
		
		\tabularnewline 
		\textit{Piano di Qualifica v1.0.0} & 12,71 & Superato	
		
		\tabularnewline 
		\textit{Studio di Fattibilità v1.0.0} & 11,28 & Superato
		
		\tabularnewline 
		\textit{Verbale Esterno 2019-03-14 v1.0.0} & 14,81 & Superato
		
		\tabularnewline 
		\textit{Verbale Esterno 2019-03-25 v1.0.0} & 15,59 & Superato
		
		\tabularnewline 
		\textit{Verbale Esterno 2019-04-10 v1.0.0} & 15,77  & Superato
		
		\tabularnewline 
		\textit{Verbale Interno 2019-03-06 v1.0.0} & 13,96 & Superato
		
		\tabularnewline 
		\textit{Verbale Interno 2019-03-13 v1.0.0} & 15,44 & Superato
		
		\tabularnewline 
		\textit{Verbale Interno 2019-03-18 v1.0.0} & 15,62 & Superato
	\end{longtable}
	\paragraph*{Considerazioni} 
	Tutti i documenti hanno superato la verifica con un esito inferiore al limite massimo imposto di 16. 
	Con una media di 13.71, in cui i documenti più rilevanti hanno un esito vicino al valore preferibile di 12, riteniamo che la verifica abbia riportato dei risultati soddisfacenti. 
	
	\paragraph{Qualità di prodotto}
	Data la non implementazione del prodotto software, nella fase attuale le metriche derivanti dalla gestione della qualità di prodotto non possono essere calcolate.
\newpage
%%%%%%%%%%%%%%%%%%%%%%%%%%%%%%%%%%%%%%%%%%%%%%%%%%%%%%%%%%%%%%%%%%%%
	\subsubsection{Revisione di progettazione}
%%%%%%%%%%%%%%%%%%%%%%%%%%%%%%%%%%%%%%%%%%%%%%%%%%%%%%%%%%%%%%%%%%%%
	\paragraph{Qualità di processo}

Di seguito vengono riportati gli esiti delle metriche derivanti dalla gestione di qualità di processi, seguiti dalla tabella degli indici di Gulpease e degli indici di Gunning Fog che riporta gli esiti di tutti i documenti prodotti finora. \\
\textbf{Legenda}:
\begin{itemize}
	\item \textbf{N.C.} - Non Calcolabile;
	\item \textbf{S} - Superato;
	\item \textbf{N.S.} - Non Superato;
	\item \textbf{\%} - Percentuale;
	\item \textbf{V} - Valore numerico;
	\item \textbf{\euro{}} - Euro.
\end{itemize}

	\rowcolors{2}{pari}{dispari}	
	\begin{longtable}{ >{\centering}p{0.25\textwidth} >{\centering}p{0.10\textwidth}
			 >{\centering}p{0.10\textwidth} >{\centering}p{0.07\textwidth} >{\centering}p{0.31\textwidth}}
		\caption{ Valutazione della qualità di processo - RP} \\
		%\hline
		\rowcolorhead
		
		\centering\textbf{\color{white}Nome metrica} 
		& \centering\textbf{\color{white}Unità di misura} 
		& \centering\textbf{\color{white}Valore} 
		& \centering\textbf{\color{white}Esito}
		& \centering\textbf{\color{white}Accettazione}
		\tabularnewline %\hline 
		\endfirsthead
		
		\rowcolor{white}\caption[]{(continua)}\\	
		\rowcolorhead
		\centering\textbf{\color{white}Nome metrica} 
		& \centering\textbf{\color{white}Unità di misura} 
		& \centering\textbf{\color{white}Valore} 
		& \centering\textbf{\color{white}Esito}
		& \centering\textbf{\color{white}Accettazione}
		\tabularnewline %\hline 
		\endhead	
		
		
		Requisiti obbligatori soddisfatti & \% & N.C. & N.S. & 100
		\tabularnewline 
		
		Coupling Between Object classes & V & N.C. & N.S. & $0 \leq CBO \leq 6$
		\tabularnewline
		
		Planned Value & \euro{} & 3.992,00 & S & $ \geq 0$
		\tabularnewline
		
		Actual Cost & \euro{} & 3.852,00 & S & $0 \leq AC \leq 6.856,00 $
		\tabularnewline
		
		Earned Value & \euro{} & 3.992,00 & S & $ \geq 0$
		\tabularnewline
		
		Budget at Completion & \euro{} & 3.992,00 & S & $3.659,40 \leq BAC \leq 4.044,60 $
		\tabularnewline
		
		Cost Variance & \euro{} & +140,00 & N.S. & $ \geq 0$
		\tabularnewline
		
		Schedule Variance & \euro{} & 0,00 & S & $ \geq 0$
		\tabularnewline
		
		Code Coverage & \% & N.C. & N.S. & $ \geq 75\%$
		\tabularnewline
		
		Indice Gunning fog (media) & V & 13,20 & S & $ \leq 16$
		\tabularnewline
		
		Indice di Gulpease (media) & V & 67,25 & S & $40 < I_G < 100$
		\tabularnewline
		
		Correttezza ortografica & V & 0 & S & 0
		\tabularnewline
		
		Percentuale di metriche soddisfatte & \% & 88,89 & S & 100
		\tabularnewline
		
	\end{longtable}
	
	\paragraph*{Considerazioni}
	Ci sono delle metriche non calcolabili in questa fase. Queste ultime si riferiscono all'analisi del codice e alla sua progettazione di dettaglio che verrà svolta nelle fasi successive.
	La variazione di costo CV è positiva, per un valore di 140 dovuta a vari cambiamenti rispetto a quanto preventivato: 
	\begin{itemize}
		\item si sono dedicate più ore del previsto per il ruolo di Amministratore, dovute ai vari problemi di configurazione dei software;
		\item aumento delle ore per il ruolo di Programmatore per vari errori legati alla comprensione del linguaggio Kotlin;
		\item una diminuzione delle ore dei Verificatori in conseguenza della diminuzione del lavoro di progettazione. 
	\end{itemize}
	Tutto sommato, anche se ci sono stati notevoli cambiamenti la fase di progettazione si è conclusa con un risparmio di 140,00 euro. 
	La percentuale di metriche soddisfatte tiene conto solo delle metriche calcolabili, quindi di un totale di 9 tra cui 8 soddisfatte; se si ritiene che le metriche non calcolabili siano da considerare non superate, la percentuale di metriche soddisfatte risulta comunque superata, per un valore di 66.67.
	\rowcolors{2}{pari}{dispari}
	\begin{longtable}{ >{\centering}p{0.40\textwidth} >{\centering}p{0.25\textwidth}
			 >{\centering}p{0.2075\textwidth} }
		\caption{ Verifiche automatizzate indice di Gulpease - RP} \\
		%\hline
		\rowcolorhead
		\centering\textbf{\color{white}Documento} 
		& \centering\textbf{\color{white}Indice Gulpease} 
		& \centering\textbf{\color{white}Esito}
		\tabularnewline %\hline 
		\endfirsthead
		
		
		\rowcolor{white}\caption[]{(continua)}\\	
		\rowcolorhead
		\centering\textbf{\color{white}Documento} 
		& \centering\textbf{\color{white}Indice Gulpease} 
		& \centering\textbf{\color{white}Esito}
		\tabularnewline %\hline 
		\endhead
		
			
		\textit{Analisi dei Requisiti v2.0.0} & 52,72 & Superato
		
		\tabularnewline 
		\textit{Glossario v2.0.0} & 100 & Superato
				
		\tabularnewline 
		\textit{Norme di Progetto v2.0.0} & 56,93  & Superato
		
		\tabularnewline 
		\textit{Piano di Progetto v2.0.0} & 54,36 & Superato
		
		\tabularnewline 
		\textit{Piano di Qualifica v2.0.0} & 55,78 & Superato	
		
		\tabularnewline 
		\textit{Verbale Interno 2019-05-03 v1.0.0} & 73 & Superato
		
		\tabularnewline 
		\textit{Verbale Esterno 2019-05-10 v1.0.0} & 80 & Superato
	\end{longtable}
	\paragraph*{Considerazioni} 
	Tutti i documenti hanno un esito positivo. La media dei valori è di 67.25, un valore intermedio tra il limite inferiore stabilito di 40 e il limite inferiore preferibile di 80. \\
	I valori più alti sono riscontrati nei verbali e glossario, mentre i valori più bassi sono presenti nell'\texttt{AdR} e nel \texttt{PdP} ma data la loro natura tecnica, riteniamo soddisfacente l'esito della verifica.
	
	
	
	\rowcolors{2}{pari}{dispari}
	\begin{longtable}{ >{\centering}p{0.40\textwidth} >{\centering}p{0.25\textwidth}
			 >{\centering}p{0.2075\textwidth}}
		\caption{ Verifiche automatizzate indice di Gunning Fog - RP} \\
		%\hline
		\rowcolorhead
		\centering\textbf{\color{white}Documento} 
		& \centering\textbf{\color{white}Indice Gunning Fog} 
		& \centering\textbf{\color{white}Esito}
		\tabularnewline %\hline 
		\endfirsthead
		
		\rowcolor{white}\caption[]{(continua)}\\	
		\rowcolorhead
		\centering\textbf{\color{white}Documento} 
		& \centering\textbf{\color{white}Indice Gunning Fog} 
		& \centering\textbf{\color{white}Esito}
		\tabularnewline %\hline 
		\endhead
			
		\textit{Analisi dei Requisiti v2.0.0} & 13,05 & Superato
		
		\tabularnewline 
		\textit{Glossario v2.0.0} & 12,32 & Superato
				
		\tabularnewline 
		\textit{Norme di Progetto v2.0.0} & 11,84  & Superato
		
		\tabularnewline 
		\textit{Piano di Progetto v2.0.0} & 12,52 & Superato
		
		\tabularnewline 
		\textit{Piano di Qualifica v2.0.0} & 12,85 & Superato	
				
		\tabularnewline 
		\textit{Verbale Interno 2019-05-03 v1.0.0} & 14,64 & Superato
		
		\tabularnewline 
		\textit{Verbale Esterno 2019-05-10 v1.0.0} & 15,21 & Superato
		
	\end{longtable}
	\paragraph*{Considerazioni} 
	Tutti i documenti hanno superato la verifica con un esito inferiore al limite massimo imposto di 16. 
	Con una media di 13.20, in cui i documenti più rilevanti hanno un esito vicino al valore preferibile di 12, riteniamo che la verifica abbia riportato dei risultati soddisfacenti. 
	
	\paragraph{Qualità di prodotto}
	Data la non implementazione del prodotto software, nella fase attuale le metriche derivanti dalla gestione della qualità di prodotto non possono essere calcolate.
\newpage
%%%%%%%%%%%%%%%%%%%%%%%%%%%%%%%%%%%%%%%%%%%%%%%%%%%%%%%%%%%%%%%%%%%%
	\subsubsection{Revisione di qualifica}
%%%%%%%%%%%%%%%%%%%%%%%%%%%%%%%%%%%%%%%%%%%%%%%%%%%%%%%%%%%%%%%%%%%%
 		\paragraph{Product Baseline}
 		Di seguito sono riportati i valori delle metriche misurate durante la verifica del lavoro per la Product Baseline.\\
\textbf{Legenda}:
 		\begin{itemize}
 			\item \textbf{N.C.} - Non Calcolabile;
 			\item \textbf{S} - Superato;
 			\item \textbf{N.S.} - Non Superato;
 			\item \textbf{\%} - Percentuale;
 			\item \textbf{V} - Valore numerico;
 			\item \textbf{\euro{}} - Euro.
 		\end{itemize}
 		
 		\rowcolors{2}{pari}{dispari}	
	\begin{longtable}{ >{\centering}p{0.25\textwidth} >{\centering}p{0.10\textwidth}
			 >{\centering}p{0.10\textwidth} >{\centering}p{0.07\textwidth} >{\centering}p{0.31\textwidth}}
		\caption{ Valutazione della qualità di processo - RQ} \\
		%\hline
		\rowcolorhead
		
		\centering\textbf{\color{white}Nome metrica} 
		& \centering\textbf{\color{white}Unità di misura} 
		& \centering\textbf{\color{white}Valore} 
		& \centering\textbf{\color{white}Esito}
		& \centering\textbf{\color{white}Accettazione}
		\tabularnewline %\hline 
		\endfirsthead
		
		\rowcolor{white}\caption[]{(continua)}\\	
		\rowcolorhead
		\centering\textbf{\color{white}Nome metrica} 
		& \centering\textbf{\color{white}Unità di misura} 
		& \centering\textbf{\color{white}Valore} 
		& \centering\textbf{\color{white}Esito}
		& \centering\textbf{\color{white}Accettazione}
		\tabularnewline %\hline 
		\endhead
		
		
		Requisiti obbligatori soddisfatti & \% & 100 & S & 100
		\tabularnewline 
		
		Densità dei guasti nei casi di test & \% & 0,23 & S & $ \leq 10$
		\tabularnewline
		
		Risoluzione dei guasti & \% & 100 & S & 100
		\tabularnewline
		
		Facilità di utilizzo & V & 8 & S & $0 \leq 15 $
		\tabularnewline
		
		Facilità di apprendimento & V & 2 & S & $ \leq 5$
		\tabularnewline
		
		Profondità della gerarchia & V & 3 & S & $ \leq 7 $
		\tabularnewline
		
		Comprensione delle funzioni & \% & 100 & S & $ \geq 95$
		\tabularnewline
		
		Facilità di comprensione & V & 0,11 & S & $ \geq 0.10$
		\tabularnewline
		
		Semplicità delle funzioni & V & 2.4 & S & $\leq 6$
		\tabularnewline
		
		Semplicità delle classi & V & 8 & S & $ \leq 15$
		\tabularnewline
		
		SFIN: Structural Fan-in (media) & V & 1.3 & S & $ \geq 0 $
		\tabularnewline
		
		SFOUT: Structural Fan-out (media) & V & 3.6 & S & $ \leq 6$
		\tabularnewline
		\end{longtable}	
		\paragraph{Qualità di processo}
		Di seguito vengono riportati gli esiti delle metriche derivanti dalla gestione di qualità di processi, seguiti dalla tabella degli indici di Gulpease e degli indici di Gunning Fog che riporta gli esiti di tutti i documenti prodotti finora. \\
\textbf{Legenda}:
\begin{itemize}
	\item \textbf{N.C.} - Non Calcolabile;
	\item \textbf{S} - Superato;
	\item \textbf{N.S.} - Non Superato;
	\item \textbf{\%} - Percentuale;
	\item \textbf{V} - Valore numerico;
	\item \textbf{\euro{}} - Euro.
\end{itemize}
	\rowcolors{2}{pari}{dispari}	
	\begin{longtable}{ >{\centering}p{0.25\textwidth} >{\centering}p{0.10\textwidth}
			 >{\centering}p{0.10\textwidth} >{\centering}p{0.07\textwidth} >{\centering}p{0.31\textwidth}}
		\caption{ Valutazione della qualità di processo - RQ} \\
		%\hline
		\rowcolorhead
		
		\centering\textbf{\color{white}Nome metrica} 
		& \centering\textbf{\color{white}Unità di misura} 
		& \centering\textbf{\color{white}Valore} 
		& \centering\textbf{\color{white}Esito}
		& \centering\textbf{\color{white}Accettazione}
		\tabularnewline %\hline 
		\endfirsthead
		
		\rowcolor{white}\caption[]{(continua)}\\	
		\rowcolorhead
		\centering\textbf{\color{white}Nome metrica} 
		& \centering\textbf{\color{white}Unità di misura} 
		& \centering\textbf{\color{white}Valore} 
		& \centering\textbf{\color{white}Esito}
		& \centering\textbf{\color{white}Accettazione}
		\tabularnewline %\hline 
		\endhead
		
		Requisiti obbligatori soddisfatti & \% & 100 & S & 100
		\tabularnewline 
		
		Coupling Between Object classes & V & 1 & S & $0 \leq CBO \leq 6$
		\tabularnewline
		
		Planned Value & \euro{} & 5.293,00 & S & $ \geq 0$
		\tabularnewline
		
		Actual Cost & \euro{} & 6.258,00 & S & $0 \leq AC \leq 6.856,00 $
		\tabularnewline
		
		Earned Value & \euro{} & 5.293,00 & S & $ \geq 0$
		\tabularnewline
		
		Budget at Completion & \euro{} & 5.293,00 & S & $3.659,40 \leq BAC \leq 4.044,60 $
		\tabularnewline
		
		Cost Variance & \euro{} & -615,00 & N.S. & $ \geq 0$
		\tabularnewline
		
		Schedule Variance & \euro{} & 0,00 & S & $ \geq 0$
		\tabularnewline
		
		Code Coverage & \% & 73 & N.S. & $ \geq 75\%$
		\tabularnewline
		
		Indice Gunning fog (media) & V & 12,27 & S & $ \leq 16$
		\tabularnewline
		
		Indice di Gulpease (media) & V & 69,08 & S & $40 < I_G < 100$
		\tabularnewline
		
		Correttezza ortografica & V & 0 & S & 0
		\tabularnewline
		
		Percentuale di metriche soddisfatte & \% & 84,61 & S & 100
		\tabularnewline
		
	\end{longtable}
	
	\rowcolors{2}{pari}{dispari}
	\begin{longtable}{ >{\centering}p{0.40\textwidth} >{\centering}p{0.25\textwidth}
			 >{\centering}p{0.2075\textwidth} }
		\caption{ Verifiche automatizzate indice di Gulpease - RQ} \\
		%\hline
		\rowcolorhead
		\centering\textbf{\color{white}Documento} 
		& \centering\textbf{\color{white}Indice Gulpease} 
		& \centering\textbf{\color{white}Esito}
		\tabularnewline %\hline 
		\endfirsthead
		
		\rowcolor{white}\caption[]{(continua)}\\	
		\rowcolorhead
		\centering\textbf{\color{white}Documento} 
		& \centering\textbf{\color{white}Indice Gulpease} 
		& \centering\textbf{\color{white}Esito}
		\tabularnewline %\hline 
		\endhead
			
		\textit{Analisi dei Requisiti v3.0.0} & 52,34 & Superato
		
		\tabularnewline 
		\textit{Glossario v3.0.0} & 100 & Superato
				
		\tabularnewline 
		\textit{Norme di Progetto v3.0.0} & 57,03  & Superato
		
		\tabularnewline 
		\textit{Piano di Progetto v3.0.0} & 57,36 & Superato
		
		\tabularnewline 
		\textit{Piano di Qualifica v3.0.0} & 58,63 & Superato	
		
		\tabularnewline 
		\textit{Manuale sviluppatore} & 60,52 & Superato	
		
		\tabularnewline 
		\textit{Manuale utente} & 63,47 & Superato	
		
		\tabularnewline 
		\textit{Verbale esterno 2019-06-06 v1.0.0} & 78,01 & Superato
		
		\tabularnewline 
		\textit{Verbale interno 2019-05-28 v1.0.0} & 77,63 & Superato
		
		\tabularnewline 
		\textit{Verbale interno 2019-06-01 v1.0.0} & 77,93 & Superato
		
		\tabularnewline 
		\textit{Verbale interno 2019-06-06 v1.0.0} & 77,03 & Superato
		
	\end{longtable}
	
	
	
	
	\rowcolors{2}{pari}{dispari}
	\begin{longtable}{ >{\centering}p{0.40\textwidth} >{\centering}p{0.25\textwidth}
			 >{\centering}p{0.2075\textwidth}}
		\caption{  Verifiche automatizzate indice di Gunning Fog - RQ} \\
		%\hline
		\rowcolorhead
		\centering\textbf{\color{white}Documento} 
		& \centering\textbf{\color{white}Indice Gunning Fog} 
		& \centering\textbf{\color{white}Esito}
		\tabularnewline %\hline 
		\endfirsthead
		
		
		\rowcolor{white}\caption[]{(continua)}\\	
		\rowcolorhead
		\centering\textbf{\color{white}Documento} 
		& \centering\textbf{\color{white}Indice Gunning Fog} 
		& \centering\textbf{\color{white}Esito}
		\tabularnewline %\hline 
		\endhead
			
		\textit{Analisi dei Requisiti v3.0.0} & 13,05 & Superato
		
		\tabularnewline 
		\textit{Glossario v3.0.0} & 12,32 & Superato
				
		\tabularnewline 
		\textit{Norme di Progetto v3.0.0} & 11,84  & Superato
		
		\tabularnewline 
		\textit{Piano di Progetto v3.0.0} & 12,52 & Superato
		
		\tabularnewline 
		\textit{Piano di Qualifica v3.0.0} & 12,85 & Superato	
				
		\tabularnewline 
		\textit{Manuale sviluppatore} & 14,03 & Superato	
		
		\tabularnewline 
		\textit{Manuale utente} & 15,01 & Superato	
		
		\tabularnewline 
		\textit{Verbale esterno 2019-06-06 v1.0.0} & 10,52 & Superato
		
		\tabularnewline 
		\textit{Verbale interno 2019-05-28 v1.0.0} & 11,19 & Superato
		
		\tabularnewline 
		\textit{Verbale interno 2019-06-01 v1.0.0} & 10,68 & Superato
		
		\tabularnewline 
		\textit{Verbale interno 2019-06-06 v1.0.0} & 10,98 & Superato
		
	\end{longtable}

	
	
		\paragraph{Qualità di prodotto}
		Di seguito viene riportata la misurazione delle metriche descritte in §3, derivanti dalla gestione di qualità di prodotto.
		\rowcolors{2}{pari}{dispari}	
	\begin{longtable}{ >{\centering}p{0.25\textwidth} >{\centering}p{0.10\textwidth}
			 >{\centering}p{0.10\textwidth} >{\centering}p{0.07\textwidth} >{\centering}p{0.31\textwidth}}
		\caption{  Valutazione della qualità di processo - RQ} \\
		%\hline
		\rowcolorhead
		
		\centering\textbf{\color{white}Nome metrica} 
		& \centering\textbf{\color{white}Unità di misura} 
		& \centering\textbf{\color{white}Valore} 
		& \centering\textbf{\color{white}Esito}
		& \centering\textbf{\color{white}Accettazione}
		\tabularnewline %\hline 
		\endfirsthead
		
		\rowcolor{white}\caption[]{(continua)}\\	
		\rowcolorhead
		\centering\textbf{\color{white}Nome metrica} 
		& \centering\textbf{\color{white}Unità di misura} 
		& \centering\textbf{\color{white}Valore} 
		& \centering\textbf{\color{white}Esito}
		& \centering\textbf{\color{white}Accettazione}
		\tabularnewline %\hline 
		\endhead
		
		Requisiti obbligatori soddisfatti & \% & 100 & S & 100
		\tabularnewline 
		
		Densità dei guasti nei casi di test & \% & 0,24 & S & $ \leq 10$
		\tabularnewline
		
		Risoluzione dei guasti & \% & 100 & S & 100
		\tabularnewline
		
		Facilità di utilizzo & V & 7 & S & $0 \leq 15 $
		\tabularnewline
		
		Facilità di apprendimento & V & 2 & S & $ \leq 5$
		\tabularnewline
		
		Profondità della gerarchia & V & 3 & S & $ \leq 7 $
		\tabularnewline
		
		Comprensione delle funzioni & \% & 100 & S & $ \geq 95$
		\tabularnewline
		
		Facilità di comprensione & V & 0,12 & S & $ \geq 0.10$
		\tabularnewline
		
		Semplicità delle funzioni & V & 2.4 & S & $\leq 6$
		\tabularnewline
		
		Semplicità delle classi & V & 8 & S & $ \leq 15$
		\tabularnewline
		
		SFIN: Structural Fan-in (media) & V & 1.4 & S & $ \geq 0 $
		\tabularnewline
		
		SFOUT: Structural Fan-out (media) & V & 3.5 & S & $ \leq 6$
		\tabularnewline
		
	\end{longtable}
	\newpage
		\paragraph{Esiti dei test}
	
		Di seguito viene riportata la tabella indicante l'esito dei test effettuati. Ogni test è stato eseguito più volte nel periodo, in base alle esigenze. Viene riportato solo l'esito finale.
		
		\subparagraph{Test di unità} {\color{white}.}
		
\renewcommand{\arraystretch}{1.5}
	\rowcolors{2}{pari}{dispari}
	
	\begin{longtable}{ >{\centering}p{0.10\textwidth}  >{\centering}p{0.50\textwidth} >{\centering}p{0.10\textwidth}
			>{\centering}p{0.10\textwidth}}
			
		%\hline
		\caption{   Esito test di unità - RQ}\\	
		\rowcolorhead
		\centering\textbf{\color{white}Test} 
		& \centering\textbf{\color{white}Metodo} 
		& \centering\textbf{\color{white}N. prove}
		& \centering\textbf{\color{white}Esito finale}
		
	%	& \textbf{\color{white}Fonti} 
		\tabularnewline %\hline 
		\endfirsthead	
		
		\rowcolor{white}\caption[]{(continua)}\\	
		\rowcolorhead
		\centering\textbf{\color{white}Test} 
		& \centering\textbf{\color{white}Metodo} 
		& \centering\textbf{\color{white}N. prove}
		& \centering\textbf{\color{white}Esito finale}
		
		%	& \textbf{\color{white}Fonti} 
		\tabularnewline %\hline 
		\endhead	
		
		TU1 & \texttt{setMinutePicker()}  & 2 & Superato \tabularnewline		
		TU2 & \texttt{getMinute()}  & 2 & Superato \tabularnewline	
		TU3 & \texttt{authenticateUser()}  & 2 & Superato \tabularnewline	
		TU4 & \texttt{show(fragment)}  & 15 & Superato\tabularnewline	
		TU5 & \texttt{extractCredentials()}  & 2 & Superato\tabularnewline	
		TU6 & \texttt{onRegistrationFail(error)}  & 1 & Superato\tabularnewline	
		TU7.1 & \texttt{onRegistrationSubmitted(credentials)}  & 2 & Superato \tabularnewline	
		TU7.2 & \texttt{onRegistrationSubmitted(credentials)}  & 2 & Superato\tabularnewline	
		TU7.3 & \texttt{onRegistrationSubmitted(credentials)}  & 1 & Superato  \tabularnewline	
		TU8 & \texttt{onRegistrationSuccess()} &  2 & Superato\tabularnewline	
		TU9 & \texttt{extractUserInformation()} & 3 & Superato \tabularnewline	
		TU10 & \texttt{onUserInformationSaved()} &  3  & Superato \tabularnewline	
		TU11 & \texttt{addUserInformation(user)} &  3 & Superato \tabularnewline	
		TU12 & \texttt{addUserInformation(user)} &  2 & Superato \tabularnewline	
		TU13 & \texttt{onAuthenticationSuccess()} &  1  & Superato \tabularnewline	
		TU14 & \texttt{onAuthenticationError()} &  1  & Superato \tabularnewline	
		TU15 & \texttt{login()} &  4 & Superato \tabularnewline	
		TU16 & \texttt{validate(credentials)} &  5  & Superato \tabularnewline	
		TU17 &  \texttt{EmailValidator.isValid()} &  2  & Superato \tabularnewline	
		TU18 & \texttt{EmailValidator.validate(user)} & 2 & Superato \tabularnewline	
		TU19 & \texttt{NameValidator.validate(user)} & 1 & Superato \tabularnewline	
		TU20 &  texttt{PasswordValidator.isValid()}   & 2 & Superato \tabularnewline	
		TU21 &   \texttt{PasswordValidator.validate(user)}   & 3 & Superato\tabularnewline	
		TU22 & \texttt{SurnameValidator.validate()} & 1 & Superato\tabularnewline	
		TU23 & \texttt{UsernameValidator.validate()} & 1  & Superato\tabularnewline	
		TU24 & \texttt{calendarFromDatePicker()} & 3 & Superato
		\tabularnewline	
		TU25 & \texttt{showEmptyLicenseDateExpirationDialog()} & 1 & Superato
		\tabularnewline	
		TU26 & \texttt{showEmptyLicenseReleaseDateDialog()}  & 1 & Superato
		\tabularnewline	
		TU27 & \texttt{displayedDateFrom(date)} & 1 & Superato
		\tabularnewline	
		TU28 & \texttt{showDialogWith(info)} & 5 & Superato
		\tabularnewline	
		TU30 & \texttt{onDatesSubmitted(releaseDate,expirationDate)}  & 1 & Superato
		\tabularnewline	
		TU31 & \texttt{onDatesSubmitted(releaseDate,expirationDate)} & 2 & Superato
		\tabularnewline	
		TU32 & \texttt{showEmptyLicenseNumberDialog()} & 1 & Superato
		\tabularnewline	
		TU33 & \texttt{showUncheckedLicenseCategoryHeldDialog()} & 1 & Superato
		\tabularnewline	
		TU34 &  \texttt{onLicenseNumberSubmitted()}  & 2 & Superato
		\tabularnewline	
		TU35 &  \texttt{onUploadError()}  & 2 & Superato
		\tabularnewline	
		TU36 & \texttt{showNoneLicenseBackSidePictureDialog()} & 1 & Superato
		\tabularnewline	
		TU37 & \texttt{showNoneLicenseFrontSidePictureDialog()}  & 1 & Superato
		\tabularnewline	
		TU38 & \texttt{createImageFile()} & 4 & Superato
		\tabularnewline	
		TU39 & \texttt{dispatchTakePictureIntent()} & 4 & Superato
		\tabularnewline	
		TU40 & \texttt{onPicturesSubmitted(frontPicturePath, backPicturePath, listener)} & 2 & Superato
		\tabularnewline	
		TU41 & \texttt{onPicturesSubmitted(frontPicturePath, backPicturePath, listener)} & 2 & Superato
		\tabularnewline	
		TU42 &\texttt{saveFront()} & 2 & Superato
		\tabularnewline	
		TU43 &\texttt{saveBack()} & 2 & Superato
		\tabularnewline	
		TU44 & \texttt{showLicenseDatesRegistration()} & 2 & Superato
		\tabularnewline	
		TU45 & \texttt{showLicensePicturesRegistration()} 2 & & Superato
		\tabularnewline	
		TU46 & \texttt{showInitialFragment()}  & 2 & Superato
		\tabularnewline	
		TU47 & \texttt{onVehiclesSearch(reservation,distance)} & 2 & Superato
		\tabularnewline	
		TU48 & \texttt{askLogOutConfirmation()} & 1 & Superato
		\tabularnewline	
		TU49 & \texttt{showLoginView()} & 3 & Superato
		\tabularnewline	
		TU50 & \texttt{showDialog(dialogFragment)} & 6 & Superato
		\tabularnewline	
		TU51 & \texttt{showProfile(user)} &3& Superato
		\tabularnewline	
		TU52 & \texttt{onProfileLoaded(profile)} & 3 & Superato
		\tabularnewline	
		TU53 & \texttt{loadProfile()}  & 2 & Superato
		\tabularnewline	
		TU54 & \texttt{updateNameSurname()}  & 2 & Superato
		\tabularnewline	
		TU55 & \texttt{updatePhoneNumber()}  & 2 & Superato
		\tabularnewline	
		TU56 & \texttt{updateEmail()} & 2 & Superato
		\tabularnewline	
		TU57 & \texttt{updateBirthDate()}  & 2 & Superato
		\tabularnewline	
		TU58 & \texttt{updateAddress()}  & 2 & Superato
		\tabularnewline	
		TU59 &   \texttt{intToHour(interval)}  & 2 & Superato
		\tabularnewline	
		TU60 &   \texttt{onStart()} & 2& Superato
		\tabularnewline	
		TU61 & \texttt{initRecycler()}  & 3 & Superato
		\tabularnewline	
		TU62 &   \texttt{showBookingConfirmation()} & 2& Superato
		\tabularnewline	
		TU63 &   \texttt{showBooking()} & 2 & Superato
		\tabularnewline	
		TU64 &   \texttt{loadPersonalReservations()}  & 3 & Superato
		\tabularnewline	
		TU65 &   \texttt{loadOwnerReservations()}& 3 & Superato
		\tabularnewline	
		TU66 &   \texttt{SelectedVehicleActivity.initView()} & 3& Superato
		\tabularnewline	
		TU67 &   \texttt{bookVehicle(dayAvailability,startHour,endHour)}  & 2 & Superato
		\tabularnewline	
		TU68 &   \texttt{onReservationInserted()}  & 3 & Superato
		\tabularnewline	
		TU69 &   \texttt{onBookingConfirmed(reservation, vehicleID, newAvailability)}  & 3& Superato
		\tabularnewline	
		TU70 &   \texttt{loadRecycler()} & 3 & Superato
		\tabularnewline	
		TU71 &   \texttt{AvailableVehicleFragment.loadVehicles()} & 6 & Superato
		\tabularnewline	
		TU72.1 &   \texttt{loadUserLocation()} & 6 & Superato
		\tabularnewline	
		TU72.2 &   \texttt{loadUserLocation()} & 2 & Superato
		\tabularnewline	
		TU73 &   \texttt{getLocationPermission()}  & 2 & Superato
		\tabularnewline	
		TU74 &   \texttt{onRequestPermissionsResult()}  & 5 & Superato
		\tabularnewline	
		TU75 &   \texttt{AvailableVehiclesPresenter.loadVehicles()} & 3 & Superato
		\tabularnewline	
		TU76 &   \texttt{searchVehicles()} & 2 & Superato
		\tabularnewline	
		TU77.1 &   \texttt{checkUserAuthentication()} & 2 & Superato
		\tabularnewline	
		TU77.2 &   \texttt{checkUserAuthentication()} & 2 & Superato
		\tabularnewline	
		TU78 &   \texttt{saveCar()}  & 5 & Superato
		\tabularnewline	
		TU79 &   \texttt{uploadVehicle()} & 4 & Superato
		\tabularnewline	
		TU80 &   \texttt{onVehicleUploaded()} & 2 & Superato
		\tabularnewline	
		TU81.1 &   \texttt{initRegisterButton()} & 2& Superato
		\tabularnewline	
		TU81.2 &   \texttt{initRegisterButton()} & 3 & Superato
		\tabularnewline	
		TU82 &   \texttt{initFabImagePicker()} & 3& Superato
		\tabularnewline	
		TU83 &   \texttt{initBrandEditText()} & 3 & Superato
		\tabularnewline	
		TU84 &   \texttt{initModelEditText()}  & 3 & Superato
		\tabularnewline	
		TU85 &   \texttt{initPlaceAutocomplete()} & 3& Superato
		\tabularnewline	
		TU86 &   \texttt{fetchPlace()} & 4 & Superato
		\tabularnewline	
		TU87.1 &   \texttt{onAvailabilityUpdated()} & 3 & Superato
		\tabularnewline	
		TU87.2 &   \texttt{onAvailabilityUpdated()} & 2& Superato
		\tabularnewline	
		TU87.3 &   \texttt{onAvailabilityUpdated()}  & 2& Superato
		\tabularnewline	
		TU88 &   \texttt{addFixedAvailability(vehicle, date, startHour, endHour)} & 4 & Superato
		\tabularnewline	
		TU89 &   \texttt{addRepeatedAvailability(vehicle, startHour, endHour, week, repeatFor)} & 5 & Superato
		\tabularnewline	
		TU90 &   \texttt{setDayAvailability(dayAvailability, startHour, endHour)} & 3 & Superato
		\tabularnewline	
		TU91 &   \texttt{showCheckBoxDialog()} & 2 & Superato
		\tabularnewline	
		TU92 &   \texttt{selectedDays()} & 3 & Superato
		\tabularnewline	
		TU93 &   \texttt{askRemoveConfirmation()} & 2& Superato
		\tabularnewline	
		TU94 &   \texttt{showAddAvailability()}  & 3 & Superato
		\tabularnewline	
		TU95 &   \texttt{onVehicleRemoved()} & 2& Superato
		\tabularnewline	
		TU96 &   \texttt{removeVehicle(key)} & 4 & Superato
		\tabularnewline	
		TU97 &   \texttt{VehicleListPresenter.loadVehicles()} & 3& Superato
		\tabularnewline	
		TU98 &   \texttt{verifyTime()} & 5 & Superato
		\tabularnewline	
		TU99 &   \texttt{countDown()} & 4 & Superato
		\tabularnewline	
		TU100 &   \texttt{saveReward()} & 5 & Superato
		\tabularnewline	
		TU101.1 &   \texttt{ritira()} & 4 & Superato
		\tabularnewline	
		TU101.2 &   \texttt{ritira()} & 5 & Superato
		\tabularnewline	
		TU101.3 &   \texttt{ritira()} & 4 & Superato
		\tabularnewline
		TU102 &   \texttt{millisToString()} & 3 & Superato
		\tabularnewline	
		TU103 &   \texttt{checkStatus()} & 3 & Superato
		\tabularnewline	
		TU104 &   \texttt{showPopUp()} & 4 & Superato
		\tabularnewline	
		TU105 &   \texttt{givePrize()} & 4 & Superato
		\tabularnewline	
		TU106 &   \texttt{loadMilestoneList()} & 6 & Superato
		
		
		\end{longtable}
		\newpage
		\subparagraph{Test di integrazione} {\color{white}.}
\renewcommand{\arraystretch}{1.5}
	\rowcolors{2}{pari}{dispari}

\begin{longtable}{ >{\centering}p{0.10\textwidth} >{\centering}p{0.10\textwidth} >{\centering}p{0.10\textwidth}
			}% >{\centering}p{0.14\textwidth}}
			
		%\hline
		\caption{   Esito test di integrazione - RQ}\\	
		\rowcolorhead
		\textbf{\color{white}Test}  
		& \textbf{\color{white}N. prove} 
		& \textbf{\color{white}Esito finale}
		\tabularnewline 
		\endfirsthead	
		
		\rowcolor{white}\caption[]{(continua)}\\	
		\rowcolorhead
		\textbf{\color{white}Test} 
		& \textbf{\color{white}N. prove}
		& \textbf{\color{white}Esito finale}  
		\tabularnewline  
		\endhead	
TI1& 5 & Superato  \tabularnewline

TI2& 4 & Superato  \tabularnewline

TI3& 4 & Superato  \tabularnewline 

TI4& 3 & Superato  \tabularnewline

TI5& 4 & Superato  \tabularnewline

TI6& 4 & Superato  \tabularnewline

TI7& 5 & Superato \tabularnewline

TI8& 5 & Superato \tabularnewline

TI9& 6 & Superato  \tabularnewline

TI10& 5 & Superato  \tabularnewline

TI10.1& 6 & Superato  \tabularnewline

TI10.2& 3 & Superato  \tabularnewline

TI11& 4 & Superato  \tabularnewline

TI12.1& 5 & Superato  \tabularnewline
TI12.2 & 4 & Superato  \tabularnewline

\end{longtable}		