\section{Introduzione}
\subsection{Scopo del documento}
Questo documento ha l'obiettivo di identificare e dettagliare la pianificazione del gruppo \textit{Zeus Code} relativa allo sviluppo del progetto \textit{P2PCS}. In particolare il documento tratta i seguenti argomenti:
\begin{itemize}
	\item analisi dei rischi;
	\item descrizione del modello di sviluppo adottato;
	\item ripartizione dei compiti tra i membri del gruppo;
	\item stima dei costi e delle risorse necessarie.
\end{itemize}
\subsection{Scopo del prodotto}
Lo scopo del prodotto è quello di sviluppare una piattaforma di Car Sharing Peer-to-Peer\glosp per l'applicazione Android\glosp sviluppata da \textit{GaiaGo}, un servizio di Car Sharing condominiale. L'obiettivo del capitolato\glosp consiste nel realizzare quest'applicazione sfruttando i meccanismi di Gamification\glo; a questo fine
verrà utilizzato il framework\glosp \textit{Octalysis}\glo.
\subsection{Riferimenti}
\subsubsection{Normativi}
\begin{itemize}
	\item \textbf{Norme di Progetto}: \textit{Norme di Progetto v3.0.0};
	\item \textbf{Regolamento organigramma e specifica tecnico-economica}: \\
	\url{https://www.math.unipd.it/~tullio/IS-1/2018/Progetto/RO.html}.
\end{itemize}

\subsubsection{Informativi}
\begin{itemize}
	\item \textbf{Capitolato\glosp d'appalto C5 - \textit{P2PCS}: piattaforma di Peer-to-Peer\glosp car sharing}: \\
	\url{https://www.math.unipd.it/~tullio/IS-1/2018/Progetto/C5.pdf};
	\item \textbf{Slide L05 del corso Ingegneria del Software - Ciclo di vita 
		del software}:\\
	\url{https://www.math.unipd.it/~tullio/IS-1/2018/Dispense/L05.pdf};
	\item \textbf{Slide L06 del corso Ingegneria del Software - Gestione di 
	Progetto}: \\
	\url{https://www.math.unipd.it/~tullio/IS-1/2018/Dispense/L06.pdf};
	\item \textbf{Software Engineering - Ian Sommerville - 10$^{th}$ Edition, 
	2010}.
\end{itemize}

\hypertarget{scadenze}{\subsection{Scadenze}}
Il gruppo \textit{ZeusCode} si impegna a rispettare le seguenti scadenze per lo 
sviluppo del progetto \textit{P2PCS}:

\begin{itemize}
	\item \textbf{Revisione dei Requisiti}: 2019-04-19;
	\item \textbf{Revisione di Progettazione}: 2019-05-17;
	\item \textbf{Revisione di Qualifica}: 2019-06-17;
	\item \textbf{Revisione di Accettazione}: 2019-07-15.
\end{itemize}