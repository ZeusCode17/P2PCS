\section{Analisi dei rischi}

Durante lo sviluppo di un progetto avente grandi dimensioni è indispensabile prevenire 
futuri problemi tramite l'analisi dei rischi. Un'approfondita valutazione dei rischi permette la prevenzione e l'arginamento di problemi che intaccherebbero la buona riuscita del prodotto finale.\\
Sarà adottata la seguente procedura durante lo sviluppo del progetto:

\begin{itemize}
	\item \textbf{Individuazione dei rischi}: fase di identificazione dei fattori di rischio che possono interferire con l'avanzamento regolare del progetto;
	\item \textbf{Analisi dei rischi}: fase di studio dei rischi individuati, procede con l'assegnazione di una probabilità e di una pericolosità individuali;
	\item \textbf{Pianificazione di controllo}: fase in cui viene indicato come affrontare e prevenire
	i problemi derivati dai fattori di rischio;
	\item \textbf{Monitoraggio dei rischi}: processo ciclico al fine di evitare gli errori arginando i 
	fattori di rischio.
\end{itemize}
Ogni problema deriva da rischi diversi, ogni rischio viene rappresentato da uno dei seguenti codici:

\begin{itemize}
	\item \textbf{RT}: Rischi Tecnologici;
	\item \textbf{RO}: Rischi Organizzativi;
	\item \textbf{RI}: Rischi Interpersonali;
	\item \textbf{RR}: Rischi dei Requisiti.
\end{itemize}

\counterwithin{table}{section}
\renewcommand{\arraystretch}{1.5}
\rowcolors{2}{dispari}{pari}
	\arrayrulecolor{white}
	\begin{longtable}{ 
			>{\centering}p{0.17\textwidth} 
			>{\raggedright}p{0.28\textwidth}
			>{\raggedright}p{0.29\textwidth} 
			>{\centering}p{0.15\textwidth}
		}

	
	\caption{Tabella dei rischi di progetto}\\
	\rowcolorhead
	\colorhead\textbf{Codice \\ Nome} & \centering\colorhead\textbf{Descrizione} & 
	\centering\colorhead\textbf{Rilevamento} & 
	\colorhead\textbf{Grado di rischio} 
	\tabularnewline
	\endfirsthead
	\rowcolor{white}\caption[]{(continua)}\\
	\rowcolorhead
	\colorhead\textbf{Nome \\ Codice} & \centering\colorhead\textbf{Descrizione} & 
	\centering\colorhead\textbf{Rilevamento} & 
	\colorhead\textbf{Grado di rischio} 
	\tabularnewline
	\endhead
	
	%RT1---------------------------------------------------------
	 \rowcolordark \textbf{RT1} \\ Inesperienza Tecnologica & 
	 Alcune delle tecnologie adottate sono nuove ad alcuni membri
	 del team, pertanto è possibile incorrere in problemi durante lo svolgimento delle attività che le coinvolgono. &
	 Ciascun membro del gruppo è incaricato di far presente al \textit{Responsabile} 
	 le proprie lacune e difficoltà. &
	 Occorrenza: \textbf{Alta} \\
	 Pericolosità: \textbf{Alta} 
	 \tabularnewline
	\rowcolordark \multicolumn{1}{p{0.17\textwidth}}{\centering{Piano di contingenza}}& 
	 \multicolumn{3}{p{0.7775\textwidth}}{Verranno suddivisi i membri in piccoli gruppi, in modo da
	 	affrontare insieme i compiti più onerosi. }
	 \tabularnewline 
	\pagebreak
	
		%RT2---------------------------------------------------------
	\textbf{RT2} \\ Hardware personale & 
	Esiste la possibilità che i supporti hardware utilizzati da uno o più membri del gruppo siano soggetti a guasti e di conseguenza ci sia una perdita locale dei dati. &
	Ciascun membro del gruppo è responsabile del proprio materiale e dovrà far fronte ad eventuali guasti tramite l'utilizzo di software di versionamento. &
	Occorrenza: \textbf{Basso} \\
	Pericolosità: \textbf{Media} 
	\tabularnewline
	\rowcolorlight \multicolumn{1}{p{0.17\textwidth}}{\centering{Piano di contingenza}}& 
	\multicolumn{3}{p{0.7775\textwidth}}{In caso di perdita definitiva di dati i membri del gruppo si adoperano alla loro immediata riformazione. }
	\tabularnewline  	
	 	
	 %RO1---------------------------------------------------------
	\rowcolordark \textbf{RO1} \\ Tempistiche  &
	Essendo un metodo di lavoro nuovo per il team, non sempre il calcolo preventivo in termini di tempo 
	per svolgere un'attività sarà rispettato.&
	Ogni task ha la propria data di inizio e di fine, ogni variazione sulla data di scadenza dovrà
	essere comunicata e seguita da una descrizione.&	
	Occorrenza: \textbf{Alta} \\
	Pericolosità: \textbf{Alta}
	\tabularnewline
	\rowcolordark\multicolumn{1}{p{0.17\textwidth}}{\centering{Piano di contingenza}}& 
	\multicolumn{3}{p{0.7775\textwidth}}{All'insorgere di tali problematiche, 
	il \textit{Responsabile} gestirà le risorse in modo da ridurre i ritardi nel modo
	più efficiente possibile.}
	\tabularnewline	
	
	%R02------------------------------------------------------------
	\rowcolorlight	\textbf{RO2} \\ Calcolo dei \\costi &
	A causa dell'inesperienza del team in ambito lavorativo, la possibilità di incorrere in errori
	di valutazione è plausibile. &
	Il \textit{Responsabile} dispone delle schedule\glosp di lavoro di ogni membro del gruppo così da verificare 
	l'effettivo monte ore.&
	Occorrenza: \textbf{Media} \\
	Pericolosità: \textbf{Alta}
	\tabularnewline
	\rowcolorlight\multicolumn{1}{p{0.17\textwidth}}{\centering{Piano di contingenza}}& 
	\multicolumn{3}{p{0.7775\textwidth}}{Verrà rivalutata una differente distribuzione del lavoro in caso di costi eccessivi.}
	\tabularnewline	
	
	%R03------------------------------------------------------------
	\rowcolordark \textbf{RO3} \\ Impegni Accademici e Personali& 
	Si possono creare dei momenti in cui uno o più membri del gruppo non siano disponibili
	a causa di impegni scolastici o personali. &
	Viene predisposto un calendario condiviso accessibile a tutti i membri del gruppo, in modo da indicare
	le date e gli orari disponibili di ognuno.&
	Occorrenza: \textbf{Alta} \\
	Pericolosità: \textbf{Bassa}
	\tabularnewline
	\rowcolordark \multicolumn{1}{p{0.17\textwidth}}{\centering{Piano di contingenza}}& 
	\multicolumn{3}{p{0.7775\textwidth}}{ Il carico di lavoro sarà distribuito, nel modo più efficiente possibile, 
		in base agli impegni dei membri durante tutto l'arco di sviluppo.}
	\tabularnewline	
	
	%R04------------------------------------------------------------
	\rowcolorlight
	\textbf{RO4} \\ Ritardi  &
	Le problematiche sopracitate possono causare ritardi.&
	Ogni componente segnalerà tempestivamente l'eventuale ritardo.&
	Occorrenza: \textbf{Media} \\
	Pericolosità: \textbf{Bassa}
	\tabularnewline
	\rowcolorlight \multicolumn{1}{p{0.17\textwidth}}{\centering{Piano di contingenza}}& 
	\multicolumn{3}{p{0.7775\textwidth}}{ Il \textit{Responsabile}, se 
	necessario, 
	riassegnerà le risorse al fine di evitare rallentamenti.}
	\tabularnewline	
	
	%RI1------------------------------------------------------------
	\rowcolordark
	\textbf{RI1} \\ Comunicazione Interna  & 
	Uno o più membri del gruppo possono risultare irreperibili. &
	Tutti i membri sono tenuti a segnalare eventuali periodi di irraggiungibilità. &
	Occorrenza: \textbf{Bassa} \\
	Pericolosità: \textbf{Alta}
	\tabularnewline
	\rowcolordark\multicolumn{1}{p{0.17\textwidth}}{\centering{Piano di contingenza}}& 
	\multicolumn{3}{p{0.7775\textwidth}}{Solitamente questo rischio può essere causato da un emergenza e i soliti canali di comunicazione
		potrebbero essere insufficienti.
		Per ovviare al precedente problema vengono organizzati incontri settimanali in concordanza tra tutti membri.}
	\tabularnewline	
	
	%RI2------------------------------------------------------------
	\rowcolorlight
	\textbf{RI2} \\ Comunicazione Esterna &
	Il proponente, essendo una realtà imprenditoriale, non sempre è disposto alla comunicazione immediata con il gruppo.
	Potrebbero verificarsi casi di impossibilità di comunicazione per vari periodi di tempo. &
	Verranno definite delle date per le riunioni con largo anticipo in comune accordo con il proponente.&
	Occorrenza: \textbf{Bassa} \\
	Pericolosità: \textbf{Media}
	\tabularnewline
	\rowcolorlight \multicolumn{1}{p{0.17\textwidth}}{\centering{Piano di contingenza}}& 
	\multicolumn{3}{p{0.7775\textwidth}}{In caso di ritardi il gruppo procederà seguendo i canoni imposti dal capitolato\glo,
		in attesa di una futura relazione col proponente.}
	\tabularnewline	
	
	%RI3------------------------------------------------------------
	\rowcolordark
	\textbf{RI3} \\ Contrasti interni &
	Essendo un gruppo di persone alle prime armi è possibile incorrere in contrasti e tensioni tra i membri. &
	Si fa affidamento sulla maturità dei vari membri, i quali cercheranno di mantenere un clima di reciproco rispetto ed 
	eviteranno eventuali contrasti. &
	Occorrenza: \textbf{Bassa} \\
	Pericolosità: \textbf{Alta}
	\tabularnewline
	\rowcolordark \multicolumn{1}{p{0.17\textwidth}}{\centering{Piano di contingenza}}& 
	\multicolumn{3}{p{0.7775\textwidth}}{In caso di controversie riguardanti aspetti del progetto sarà compito del \textit{Responsabile} 
		decidere tra le varie alternative proposte.}
	\tabularnewline	
	
	%RR1------------------------------------------------------------
	\rowcolorlight
	\textbf{RR1} \\ Variazione dei requisiti &
	È possibile che alcuni requisiti individuati dagli analisti siano errati o che non coincidano con quanto richiesto dal proponente. Ciò prevede una rimozione e/o correzione di eventuali mancanze. &
	Si argina questo rischio tramite il continuo confronto col proponente sia in via telematica che tramite incontri in sede universitaria. &
	Occorrenza: \textbf{Bassa} \\
	Pericolosità: \textbf{Alta}
	\tabularnewline
	\rowcolorlight\multicolumn{1}{p{0.17\textwidth}}{\centering{Piano di contingenza}}& 
	\multicolumn{3}{p{0.7775\textwidth}}{Nel caso alcuni requisiti risultino errati sarà necessario provvedere alla loro immediata rimozione o modifica.}
	\tabularnewline	
	
	%RR2------------------------------------------------------------
	\rowcolorlight
	\textbf{RR2} \\ Variazione dei casi d'uso &
	Alcuni dei casi d'uso possono necessitare di modifiche per una corretta integrazione con la parte funzionale dell'applicazione. &
	Durante la codifica di alcuni elementi di Gamification\glosp ci siamo resi conto che alcuni casi d'uso necessitano di una rivisitazione perchè siano ben integrati con il resto dell'applicazione. &
	Occorrenza: \textbf{Bassa} \\
	Pericolosità: \textbf{Media}
	\tabularnewline
	\rowcolorlight\multicolumn{1}{p{0.17\textwidth}}{\centering{Piano di contingenza}}& 
	\multicolumn{3}{p{0.7775\textwidth}}{Nel caso alcuni casi d'uso risultino errati sarà necessario provvedere alla loro modifica.}
	\tabularnewline	
		
	%RT3------------------------------------------------------------
	\rowcolordark
	\textbf{RT3} \\ Interruzione dei servizi Google &
	L'applicazione per funzionare necessita di usufruire di alcuni servizi offerti da Google, i quali possono essere soggetti ad interruzioni. &
	Essendo servizi offerti da terzi il gruppo non può prevederne l'interruzione, si fa affidamento sulla solidità di tali aziende per una riattivazione veloce. &
	Occorrenza: \textbf{Bassa} \\
	Pericolosità: \textbf{Molto Alta}
	\tabularnewline
	\rowcolordark\multicolumn{1}{p{0.17\textwidth}}{\centering{Piano di contingenza}}& 
	\multicolumn{3}{p{0.7775\textwidth}}{Si possono cercare strade alternative anche se spostare l'applicazione su altri servizi non risolve la problematica sopracitata.}
	\tabularnewline		
		
	\end{longtable}
\counterwithin{table}{subsection}	
\renewcommand{\arraystretch}{1}
