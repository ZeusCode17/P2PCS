\section{Glossario}
\subsection*{A}
\addcontentsline{toc}{subsection}{A}
\subsubsection*{Activity}
Sono quelle classi scritte in linguaggio Java che compongono una applicazione Android e subiscono
una interazione diretta con l'utente. All'avvio di ogni applicazione viene eseguita una activity, la
quale può eseguire delle operazioni e può anche aprire/eseguire altre activity. Le attività create
estendono la classe Activity da cui ereditano proprietà e metodi.

\subsubsection*{Android}
Sistema operativo per dispositivi mobili sviluppato da Google Inc. e basato sul kernel Linux; è
un sistema embedded progettato principalmente per smartphone e tablet, con interfacce utente
specializzate per televisori (Android TV), automobili (Android Auto), orologi da polso (Wear
OS), occhiali (Google Glass), e altri.

\subsection*{B}
\addcontentsline{toc}{subsection}{B}
\subsubsection*{BaaS}
L'acronimo sta per Back end as a Service ed è un modello per fornire a sviluppatori di app web o mobili un modo per collegare le loro applicazioni a un back end\glosp cloud storage e API\glosp esposte da applicazioni back end\glosp, fornendo allo stesso tempo funzioni quali la gestione degli utenti, le notifiche push, e l'integrazione con servizi di social networking.

\subsection*{D}
\addcontentsline{toc}{subsection}{D}
\subsubsection*{Design Pattern}
In informatica, per Design Pattern si intende un modello logico che rappresenta una soluzione progettuale generale ad un problema ricorrente. Durante le fasi di progettazione del software può essere utile applicare uno di questi modelli logici per risolvere un particolare
problema che può presentarsi in più di una situazione.

\subsection*{F}
\addcontentsline{toc}{subsection}{F}
\subsubsection*{Fragment}
Un fragment rappresenta un comportamento o una porzione di un'interfaccia utente di un'activity. Più fragment possono essere combinati insieme per creare un'interfaccia multipannello. Essendo componenti indipendenti, i possono essere integrati in più activities, agevolando così il riuso.


\subsection*{K}
\addcontentsline{toc}{subsection}{K}
\subsubsection*{Kotlin}
Linguaggio di programmazione general purpose, multi-paradigma, Open-Source a tipizzazione statica e forte, ed è particolarmente orientato verso la programmazione a oggetti permettendo
peraltro un pieno uso dell'approccio funzionale.

\subsection*{M}
\addcontentsline{toc}{subsection}{D}
\subsubsection*{MVP}
Sta per Model-View-Presenter ed è un pattern architetturale.
\subsubsection*{Model}
Il Model fornisce i metodi per accedere ai dati utili all'applicazione.

\subsection*{P}
\addcontentsline{toc}{subsection}{P}
\subsubsection*{Peer to Peer}
Un sistema distribuito dove non c'è distinzione tra client e server. I computer nel sistema possono operare sia come cliente sia come server. 
\subsubsection*{Presenter}
Il Presenter è il mediatore tra Model e View ed elabora i dati del Model in modo che possano essere visualizzati nella View.

\subsection*{V}
\addcontentsline{toc}{subsection}{T}
\subsubsection*{View}
La View visualizza i dati contenuti nel model e si occupa dell'interazione con utenti e agenti;

\subsection*{X}
\addcontentsline{toc}{subsection}{T}
\subsubsection*{XML}
eXtensible Markup Language è un metalinguaggio per la definizione di linguaggi markup. Esso permette di creare un qualsiasi insieme di marcatori ed attributi che identifica un linguaggio.