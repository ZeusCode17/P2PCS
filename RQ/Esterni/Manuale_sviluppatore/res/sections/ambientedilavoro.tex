\section{Ambiente di lavoro}
In questa sezione vengono descritte le procedure e i requisiti necessari per impostare il proprio ambiente di lavoro in modo da emulare quello del gruppo \textit{Zeus Code}.  
\subsection{Requisiti per lo sviluppo}
In questa sezione tutti i requisiti necessari sono descritti.
\subsubsection{Smartphone Android}
Per utilizzare il prodotto è necessario disporre di uno smartphone dotato di sistema operativo Android\glo. Come versione si consiglia di utilizzare Android Pie 9.
\subsubsection{Requisiti di sistema}
\begin{itemize}
	\item \textbf{Sistema operativo:} Windows 7/8/10 oppure Ubuntu 12.04 o superiore;
	\item \textbf{RAM:} minimo 3GB, 8 GB consigliati;
	\item \textbf{Spazio su disco:} minimo 2GB, 4GB consigliati;
	\item \textbf{Risoluzione schermo:} minimo 1280*800px.
\end{itemize}
\subsubsection{Strumenti}
Sono richiesti i seguenti strumenti:
\begin{itemize}
	\item \textbf{Android Studio} Android Studio è un ambiente di sviluppo integrato per lo sviluppo per la piattaforma Android\glosp. Viene utilizzato per la stesura del codice Kotlin\glosp e per i layout in XML\glosp. Android Studio integra autonomamente Gradle, un sistema per l'automazione dello sviluppo;
	\item \textbf{Git} è un software di controllo di versione. La repository di P2PCS è ospitata su GitLab;
	\item \textbf{Browser Web} necessario per accedere a Firebase\glo. Per sfruttarne a pieno le potenzialità si consiglia di utilizzare la versione più recente di Google Chrome.
\end{itemize}
\subsection{Installazione e configurazione}
\subsubsection{Android Studio} 
Android Studio è reperibile a questo sito: \url{https://developer.android.com/studio}
\subsubsection{Git}
\begin{itemize}
	\item Su Ubuntu è necessario digitare nella shell il seguente comando: \textbf{apt-get install git-core}
	\item Su Windows è possibile scaricare Git\glosp da: \url{https://git-scm.com/download/win}.
	Al termine del download, eseguire il file e seguire le istruzioni fino al termine.
\end{itemize}
\subsubsection{Firebase}
Per impostare Firebase\glosp è necessario recarsi a questo sito: \url{https://console.firebase.google.com} e seguire questi passaggi:
\begin{itemize}
	\item creare un nuovo progetto;
	\item accedere alla voce sviluppo;
	\item alla voce authentication, abilitare il metodo di accesso "email/password";
	\item alla voce database, creare un nuovo database in modalità di prova;
	\item alla voce storage, cliccare su inizia e confermare.
\end{itemize}
A questo punto il back-end sarà attivo. Ora è necessario collegarlo all'applicazione tramite Android Studio:
\begin{itemize}
	\item accedere alla voce "Tools";
	\item accedere alla voce "Firebase"
	\item tramite la voce "Authentication" connettere l'app a Firebase con l'apposito pulsante "Connetti".
\end{itemize}
 

