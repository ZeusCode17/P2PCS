
\subsubsection{UC24 - Daily rewards}
\begin{itemize}
	\item \textbf{Attori Primari}: utente autenticato;
	\item \textbf{Descrizione}: l'applicazione rende disponibile una tabella che illustra i premi giornalieri del mese corrente. La tabella è organizzata sotto forma di calendario, in cui associa ad ogni giorno del mese corrente un premio.
	Sotto alla tabella è presente un bottone "Ritira premio" che si disattiva dopo il primo utilizzo.\\
	Inoltre, i premi ottenibili possono includere:
	\begin{itemize}
		\item sconti su e-commerce o buoni spesa;
		\item sconto sul prossimo viaggio con \textit{GaiaGo};
		\item viaggio gratuito con \textit{GaiaGo};
		\item accessori per l'auto del minigioco;
		\item bonus di punti esperienza;
		\item oggetti virtuali esclusivi per il minigioco.
	\end{itemize}
	\item \textbf{Scenario principale}: l'utente preme l'icona relativa alla tabella dei Daily Rewards da cui può visualizzare i premi disponibili per il mese corrente e decidere di ritirare il premio del giorno corrente [UC25].
	Ogni premio può essere ritirato solo una volta;
	\item \textbf{Estensioni}: 
		\begin{itemize}
			\item ritiro del premio giornaliero [UC25].
		\end{itemize}
	\item \textbf{Precondizione}: l'utente ha premuto l'icona Daily Rewards;
	\item \textbf{Postcondizione}: l'utente ha visualizzato i premi disponibili per il mese corrente e può aver ritirato il premio del giorno corrente se non lo ha precedentemente fatto [UC25]. 
\end{itemize} 
\begin{figure}[h]
	\includegraphics[width=13.2cm]{res/images/UC24Daily.png}
	\centering
	\caption{UC24 - Daily rewards}
\end{figure}
\subsubsection{UC25 - Ritiro del daily rewards}
\begin{itemize}
	\item \textbf{Attori Primari}: utente autenticato;
	\item \textbf{Descrizione}: l'applicazione rende disponibile dei premi giornalieri, illustrati e ritirabili tramite l'apposito pulsante di raccoglimento premi (una sola volta);
	\item \textbf{Scenario principale}: l'utente sta visualizzando la tabella dei premi giornalieri e preme il pulsante "Ritira premio" per ritirarlo;
	\item \textbf{Precondizione}: l'utente apre la tabella dei premi e preme il pulsante "Ritira premio" per ritirare il premio;
	\item \textbf{Postcondizione}: l'utente riceve il premio se non lo ha già ritirato precedentemente e l'applicazione disabilita il relativo pulsante mostrando, tramite un pop-up, il relativo premio.  
\end{itemize} 