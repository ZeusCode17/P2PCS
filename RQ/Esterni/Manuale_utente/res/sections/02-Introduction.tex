\section{Introduzione}
Questo documento è il manuale utente dell'applicazione \textit{GaiaGo} sviluppata da \textit{Zeus Code}.
\subsection{Scopo del documento}
Il presente documento è stato redatto per l’utente finale con lo scopo di illustrare tutte le funzionalità presenti all'interno dell'applicazione \textit{GaiaGo} e spiegarne l’utilizzo.

\subsection{Scopo del prodotto}
Lo scopo di \textit{GaiaGo} è quello di fornire una piattaforma di car sharing condominiale. L'applicazione coinvolge tutti gli utenti che necessitano di utilizzare un veicolo, in una determinata fascia oraria, al prezzo imposto dall'utente che mette a disposizione la propria auto. Per utilizzare \textit{GaiaGo} sarà necessario registrarsi dopodiché saranno disponibili tutte le funzionalità. L'applicazione sfrutta inoltre dei meccanismi di Gamification\glo, estratti dal framework\glosp \textit{Octalysis}\glo, utili per invogliare l'utente finale ad utilizzare spesso l'applicazione anche nel caso in cui non necessiti di sfruttare la parte funzionale di quest'ultima. Come già anticipato l'applicazione ha una sottile suddivisione in due parti:
\begin{itemize}
	\item \textbf{funzionale}: ovvero tutte le funzioni principali utilizzabili dall'utente ad esempio la prenotazione di un veicolo o per il suo inserimento; 
	\item \textbf{Gamification}: ovvero tutte le funzioni secondarie utili ad invogliare l'utente ad utilizzare in modo frequente \textit{GaiaGo}, che si relazionano con la parte funzionale fornendo ad esempio buoni sconto per la prenotazione.
\end{itemize}
 Si parla di suddivisione a livello teorico e mentale in quanto la parte di Gamification interagisce ed è integrata allo stesso modo nell'applicazione della parte funzionale.
 Con questo meccanismo \textit{GaiaGo} attira a se un gran numero di utenti, fornendo la possibilità di divertirsi utilizzandola, oppure di sfruttarla solo se si necessita di fare un viaggio.
\subsection{Glossario}
Con l'obiettivo di evitare ridondanze e ambiguità di linguaggio, i termini tecnici e gli acronimi
utilizzati nel documento verranno definiti e descritti riportandoli nell'appendice A.
Visto che \textit{GaiaGo} è destinata a un qualsiasi livello di utenza si cerca di inserire quanti più termini tecnici possibili nel glossario.
I vocaboli riportati vengono indicati con una 'G' a pedice.