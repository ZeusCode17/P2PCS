\section{Glossary}
\subsection*{A}
\addcontentsline{toc}{subsection}{A}
\subsubsection*{Activity}
Sono quelle classi scritte in linguaggio Java che compongono una applicazione Android e subiscono
una interazione diretta con l'utente. All'avvio di ogni applicazione viene eseguita una activity, la
quale può eseguire delle operazioni e può anche aprire/eseguire altre activity. Le attività create
estendono la classe Activity da cui ereditano proprietà e metodi.

\subsubsection*{Android}
Sistema operativo per dispositivi mobili sviluppato da Google Inc. e basato sul kernel Linux; è
un sistema embedded progettato principalmente per smartphone e tablet, con interfacce utente
specializzate per televisori (Android TV), automobili (Android Auto), orologi da polso (Wear
OS), occhiali (Google Glass), e altri

\subsection*{D}
\addcontentsline{toc}{subsection}{D}
\subsubsection*{Dialog}
Piccola finestra che mostra all'utente ulteriori informazioni. Non riempie lo schermo e richiede la pressione di un tasto per chiudersi.

\subsection*{F}
\addcontentsline{toc}{subsection}{F}
\subsubsection*{Framework}
Un framework è un'architettura logica di supporto su cui un software può essere progettato e
realizzato. Può includere vari componenti, come librerie, compilatori ed API, tutte atte a
migliorare il processo di sviluppo software da parte del programmatore.

\subsection*{G}
\addcontentsline{toc}{subsection}{G}
\subsubsection*{Gamification}
Riutilizzo di concetti ed elementi tipici dei giochi in applicazioni di diverso contesto. Utilizzando
questi principi si ottengono sia maggiore coinvolgimento degli utenti dell'applicazione, sia maggior
produttività ed organizzazione nel lavoro

\subsection*{O}
\addcontentsline{toc}{subsection}{O}
\subsubsection*{Octalisys}
Gamification framework, progettato a forma ottagonale con 8 core driveG che rappresentano le varie funzionalità.

\subsection*{P}
\addcontentsline{toc}{subsection}{P}
\subsubsection*{Placeholder}
Viene utilizzato nello sviluppo front-end per inserire una foto o un testo generico che andrà successivamente sostituito.

\subsection*{R}
\addcontentsline{toc}{subsection}{R}
\subsubsection*{Rating}
Valutazione di un elemento dell'applicazione, può essere in simboli quali stelle o pallini oppure in numeri, il tutto basato su una scala di valutazione univoca per tutti gli elementi dello stesso tipo.

\subsection*{T}
\addcontentsline{toc}{subsection}{T}
\subsubsection*{Tool}
Strumento utile a sostegno del programmatore, si avvale della capacità di risolvere un determinato insieme di problematiche.
