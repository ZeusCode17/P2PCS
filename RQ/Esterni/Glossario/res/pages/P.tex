\subsection*{\quad$P\quad$}
\subsubsection*{Package}
\index{Package}
Collezione di classi utile all'organizzazione logica e gerarchica di esse. Inoltre è utile anche per l'information hiding.

\subsubsection*{Peer-to-peer}
\index{Peer-to-peer}
Un sistema distribuito dove non c'è distinzione tra client e server. I computer nel sistema possono operare sia come cliente sia come server. 

\subsubsection*{Plug-in}
\index{Plug-in}
Componente aggiuntivo che può essere aggiunto a un'altro software per ampliarne le funzionalità. Di solito può essere eseguito in modo indipendente.

\subsubsection*{PragmaDb}
\index{Python}
Strumento utilizzato per il tracciamento dei requisiti.


\subsubsection*{Processo}
\index{Processo}
L'attività di esecuzione di un programma in modo sequenziale, ovvero un compito che il processore dell'elaboratore deve portare a termine su richiesta dell'utente. Più precisamente è un'attività controllata da un programma che si svolge su un processore in genere sotto la gestione o supervisione del rispettivo sistema operativo.

\subsubsection*{Product Baseline}
\index{Product Baseline}
Indica un punto d’arrivo tecnico dal quale non si retrocede e serve da base per gli avanzamenti futuri. La baseline\glosp è fatta di elementi di configurazione versionati. Può essere cambiata solo tramite procedure di controllo di cambiamento.

\subsubsection*{Proof of Concept}
\index{Proof of Concept}
È un prototipo software che dimostra che il progetto è fattibile conformemente alle richieste.

\subsubsection*{Python}
\index{Python}
Linguaggio di programmazione general purpose, interpretato, ad alto livello. Gode di ampia diffusione per la sua semplicità e l'ampia collezione di librerie che lo supportano.

