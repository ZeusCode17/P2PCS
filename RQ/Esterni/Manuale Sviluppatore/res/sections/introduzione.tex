\section{Introduzione} 
\subsection{Scopo del documento}
Questo documento è il manuale sviluppatore per il progetto P2PCS, sviluppato da \textit{Zeus Code} per la proponente \textit{GaiaGo}
In questo manuale è possibile trovare:
\begin{itemize}
	\item le tecnologie utilizzate per lo sviluppo;
	\item gli strumenti software utilizzati e suggeriti;
	\item l'architettura software;
	\item i design pattern\glosp utilizzati;
	\item le funzionalità fornite da P2PCS.
\end{itemize}
\subsection{Scopo del prodotto}
Lo scopo del prodotto è sviluppare una piattaforma di Car Sharing Peer-to-Peer\glosp per l'applicazione Android\glosp sviluppato da GaiaGo per il servizio di Car Sharing condominiale. La funzionalità principale consiste nel noleggio di veicoli tra privati.
\subsection{Glossario}
Con l'obiettivo di evitare ridondanze e ambiguità di linguaggio, i termini tecnici e gli acronimi
utilizzati nel documento verranno definiti e descritti riportandoli nell'appendice A.
Visto che \textit{Gaiago} è destinata a un qualsiasi livello di utenza si cerca di inserire quanti più termini tecnici possibili nel glossario.
I vocaboli riportati vengono indicati con una 'G' a pedice.
\subsection{Riferimenti}
\begin{itemize}
	\item \textbf{Android} \url{https://developer.android.com/}
	\item \textbf{Firebase} \url{https://firebase.google.com/}	
\end{itemize}