\section{Librerie esterne}
In questa sezione del manuale verranno elencate e brevemente descritte le librerie esterne utilizzate nel progetto.
\subsection{Codifica}
\begin{itemize}
	\item \textbf{Picasso:} libreria che permette di scaricare immagini tramite URL e di inserirle in un'ImageView\glosp in una sola linea di codice. Si occupa inoltre di gestire in automatico la memoria e il caching, cancellando il file scaricato al termine del suo utilizzo.
	\item \textbf{Places Auto Complete:} libreria che fornisce un EditText\glosp con autocompletamento già integrato per la ricerca di luoghi.
	\item \textbf{Places:} libreria di Google che fornisce i metodi per la geocodifica dei luoghi.
\end{itemize}
\subsection{Test}
\begin{itemize}
	\item \textbf{KotlinTest:} libreria che fornisce strumenti per il testing in Kotlin;
	\item \textbf{Firebase Test Lab:} strumento esterno fornito che permette di eseguire
	test di integrazione e test di sistema;

	\item \textbf{Espresso:} libreria per il testing della UI;
\end{itemize}
\subsection{Elenco dei collegamenti alla documentazione}
\begin{itemize}
	\item \textbf{Picasso:} \url{https://square.github.io/picasso/}
	\item \textbf{Places Auto Complete:}
	\url{https://github.com/seatgeek/android-PlacesAutocompleteTextView}
	\item \textbf{Places:}
	\url{https://developers.google.com/places/android-sdk/client-migration}
	\item \textbf{Kotlin Test:}
	\url{https://github.com/kotlintest/kotlintest}
	\item \textbf{Firebase Test Lab:}
	\url{https://firebase.google.com/docs/test-lab}
	\item \textbf{Espresso:}
	\url{https://developer.android.com/training/testing/espresso}
\end{itemize}