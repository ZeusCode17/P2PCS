\section{Introduzione}
\subsection{Scopo del documento}
Il seguente documento si propone di fissare le strategie di verifica e validazione\glosp che \textit{Zeus Code} ha deciso di adottate al fine di garantire la qualità di processo e di prodotto desiderata. \newline
Questo consentirà di rilevare e correggere eventuali anomalie, riducendo la possibilità che possano creare ulteriori problemi e minimizzando l'utilizzo di risorse. 
\subsection{Scopo del prodotto}
Lo scopo del prodotto è quello di realizzare una applicazione Android\glosp che fornisca un servizio Peer-to-Peer\glosp per la condivisione delle auto. L'applicazione intende utilizzare concetti di gamification\glosp per invogliare gli utenti ad usufruire del servizio offerto.
% PLACEHOLDER espandere
\subsection{Glossario}
Al fine di evitare possibili ambiguità relative al linguaggio utilizzato nei documenti formali, viene fornito il \textit{Glossario v2.0.0}. In questo documento vengono definiti e/o descritti tutti i termini con un significato particolare. Tali termini sono contrassegnati da una 'G' a pedice.
\subsection{Riferimenti}
\subsubsection{Riferimenti normativi}
\begin{itemize}

\item \textbf{Capitolato\glosp d'appalto C5 - P2PCS: Peer-to-peer\glosp car sharing}: \\ \url{https://www.math.unipd.it/~tullio/IS-1/2018/Progetto/C5.pdf};
\item \textbf{Norme di Progetto} \textit{Norme di progetto v2.0.0}.
%verbali normativi
\end{itemize}
\subsubsection{Riferimenti informativi}
\begin{itemize}
% Guide?(vedi Pro-tech)
\item \textbf{Ingegneria del software - Ian Sommerville decima edizione italiana}:
	\begin{itemize}	
		\item §8.1 Test di sviluppo;
		\item §8.1.1 Test delle unità;
		\item §8.1.3 Test dei componenti;
		\item §8.4 Test degli utenti (Parte finale che tratta dei test di accettazione);
		\item §21.1 Qualità del software;
		\item §21.2 Standard del software;
		\item §21.3 Revisioni e ispezioni;
		\item §21.3.2 Ispezioni dei programmi;
		\item §21.5.1 Metriche di prodotto.
	\end{itemize}
\item \textbf{ISO/IEC 9126}: \\* \url{https://en.wikipedia.org/wiki/ISO/IEC_9126};
\item \textbf{ISO/IEC 12207}: \\* 
\url{https://www.math.unipd.it/~tullio/IS-1/2009/Approfondimenti/ISO\_12207-1995.pdf};
\item \textbf{Approfondimento Fan-in e Fan-out}: \\*
\url{https://www.math.unipd.it/~tullio/IS-1/2004/Approfondimenti/Fan-in_Fan-out.html};
\item \textbf{Indice di Gulpease}: \\* \url{https://it.wikipedia.org/wiki/Indice_Gulpease};
\item \textbf{Metriche di pianificazione}: \\* \url{https://it.wikipedia.org/wiki/Metriche_di_progetto}.
\end{itemize}