
\section{Requisiti} 
Ogni requisito è composto dalla seguente struttura:
\begin{itemize}
	\item \textbf{codice identificativo}: ogni codice identificativo è univoco e conforme alla seguente codifica: \\
	\centerline{\textbf{R[Importanza][Tipologia][Codice]}} \\ \\
	Il significato delle cui voci è:
	\begin{itemize}
		\item \textbf{Tipologia}: ogni requisito può assumere uno dei seguenti valori:
		\begin{itemize}
			\item \textit{F}: funzionale;
			\item \textit{P}: prestazionale;
			\item \textit{Q}: qualitativo;
			\item \textit{V}: vincolo.
		\end{itemize}
		\item \textbf{Importanza}: ogni requisito può assumere uno dei seguenti valori:
		\begin{itemize}
			\item \textit{O}: requisito obbligatorio: irrinunciabili per qualcuno degli stakeholder;
			\item \textit{D}: requisito desiderabile: non strettamente necessari ma  a valore aggiunto riconoscibile;
			\item \textit{F}: requisito facoltativo: relativamente utili oppure contrattabili più avanti nel progetto.	
		\end{itemize}
		\item \textbf{Codice}: è un identificatore univoco del requisito in forma gerarchica padre/figlio.
	\end{itemize}
	\item \textbf{classificazione}: viene riportata l'importanza del requisito. Sebbene questa sia un'informazione ridondante ne facilita la lettura;
	\item \textbf{descrizione}: descrizione breve ma completa del requisito, meno ambigua possibile;
	\item \textbf{fonti}: ogni requisito può derivare da una o più tra le seguenti opzioni:
	\begin{itemize}
		\item \textit{capitolato\glo}: si tratta di un requisito individuato dalla lettura del capitolato;
		\item \textit{interno}: si tratta di un requisito che gli analisti hanno ritenuto opportuno aggiungere;
		\item \textit{caso d'uso}: il requisito è estrapolato da uno o più casi d'uso. In questo caso è riportato il codice univoco del caso d'uso;
		\item \textit{verbale}: si tratta di un requisito individuato in seguito ad una richiesta di chiarimento con il proponente. Tali informazioni sono riportate nei verbali in cui ogni requisito individuato è segnato da un codice presente nella tabella dei tracciamenti.
	\end{itemize}
\end{itemize}
\renewcommand{\arraystretch}{1.5}

\newpage
\subsection{Requisiti funzionali}

\rowcolors{2}{pari}{dispari}

\begin{longtable}{ >{\centering}p{0.15\textwidth} >{\centering}p{0.20\textwidth}
		>{\raggedright}p{0.35\textwidth} >{\centering}p{0.14\textwidth}}
	\caption{Tabella dei requisiti funzionali}\\
	\rowcolorhead 
	\textbf{\color{white}Requisito} 
	& \textbf{\color{white}Classificazione} 
	& \centering\textbf{\color{white}Descrizione}
	& \textbf{\color{white}Fonti} 
	\endfirsthead
	\rowcolor{white}\caption[]{(continua)}\\
	\rowcolorhead 
	\textbf{\color{white}Requisito} 
	& \textbf{\color{white}Classificazione} 
	& \centering\textbf{\color{white}Descrizione}
	& \textbf{\color{white}Fonti} 
	\endhead	
	
	RFD1	&	Desiderabile	&	L'utente visualizza una guida introduttiva.	&	Interno\\  UC4 \\ VE\_3.4 \tabularnewline
	RFO2	&	Obbligatorio	&	Un utente non registrato può effettuare la registrazione.	&	Capitolato \\ UC1	\tabularnewline
	RFO2.1	&	Obbligatorio	&	La registrazione richiede l'email. &	Interno \\ UC1.1.1	\tabularnewline
	RFO2.2	&	Obbligatorio	&	La registrazione richiede il nome. &	Interno \\ UC1.1.2	\tabularnewline
	RFO2.3	&	Obbligatorio	&	La registrazione richiede il cognome.	&	Interno \\ UC1.1.3	\tabularnewline
	RFO2.4	&	Obbligatorio	&	La registrazione richiede una password.	&	Interno \\ UC1.1.4	\tabularnewline
	RFO2.5	&	Obbligatorio	&	Nella registrazione la password inserita deve avere: almeno 8 caratteri, almeno un carattere maiuscolo, almeno un carattere speciale, in caso contrario visualizza un messaggio d'errore.	&	Interno \\ UC3	\tabularnewline
	RFF2.6 & Facoltativo & La registrazione prevede l’inserimento di un codice amico. & UC1.1.5 \tabularnewline
	RFO2.7 & Obbligatorio & l'utente non autenticato non inserisce nessun dato nei campi di inserimento. & Interno \\ UC5 \tabularnewline
	RFO3	&	Obbligatorio	&	L'utente registrato può effettuare il login.	&	Capitolato \\ UC6	\tabularnewline
	RFO3.1	&	Obbligatorio	&	Il login deve richiedere l'email.	&	Interno \\ UC6.1.1	\tabularnewline
	RFO3.2	&	Obbligatorio	&	Il login deve richiedere la password.	&	Interno \\ UC6.1.2	\tabularnewline
	RFO3.3	&	Obbligatorio	&	Durante il login se la combinazione mail/password è scorretta, l'utente visualizza un messaggio d'errore.	&	Interno \\ UC7	\tabularnewline
	RFO3.4	&	Obbligatorio	&	Durante il login se l'utente non ricorda la password ha la possibilità di recuperarla.	&	Interno \\ UC8	\tabularnewline
	RFO3.4.1	&	Obbligatorio	&	Per recuperare la password l'utente deve inserire l'email di recupero associata all'account.	&	Interno \\ UC8	\tabularnewline
	RFO4	&	Obbligatorio	&	L'utente autenticato può effettuare il logout. 	&	Capitolato \\ UC9	\tabularnewline
	RFO5	&	Obbligatorio	&	L'utente autenticato può gestire e visualizzare i propri veicoli.	&	Capitolato \\ UC10	\tabularnewline
	RFO5.1	&	Obbligatorio	&	L'utente autenticato può inserire un nuovo veicolo.		&	Interno \\ UC10.1	\tabularnewline
	RFO5.1.1	&	Obbligatorio	&	L'utente deve inserire un'immagine del suo veicolo.	&	Interno \\ UC10.1.1	\tabularnewline
	RFO5.1.2	&	Obbligatorio	& L'utente deve inserire la marca del suo veicolo.	&	Interno \\ UC10.1.2	
	\tabularnewline
	RFO5.1.3	&	Obbligatorio	& L'utente deve inserire il modello del suo veicolo.	&	Interno \\ UC10.1.3	\tabularnewline
	RFO5.1.4	&	Obbligatorio	&	L'utente deve inserire l'anno di immatricolazione del suo veicolo.	&	Interno \\ UC10.1.4	\tabularnewline
	RFO5.1.5	&	Obbligatorio	&	L'utente deve inserire la posizione del veicolo.	&	Interno \\ UC10.1.5	\tabularnewline
	RFO5.1.6	&	Obbligatorio	&	L'utente deve inserire il costo orario del veicolo.	&	Interno \\ UC10.1.6	\tabularnewline
	RFO5.2	&	Obbligatorio	& L'utente può visualizzare i dettagli del veicolo.	&	Capitolato \\ UC10.2	\tabularnewline
	RFO5.2.1	&	Obbligatorio	& L'utente può visualizzare le disponibilità del proprio veicolo.	&	Capitolato \\ UC10.3	\tabularnewline
	RFO5.2.1.1	&	Obbligatorio	& L'utente può decidere di aggiungere una disponibilità al veicolo.	&	Capitolato \\ UC10.6	\tabularnewline
	RFO5.2.1.1.1 & Obbligatorio	& L'utente decide di rendere disponibile il veicolo per un singolo giorno. & Capitolato \\ UC10.6.1 \tabularnewline
	RFO5.2.1.1.1.1 & Obbligatorio	& L'utente inserisce la data di prenotazione. & Capitolato \\ UC10.6.1.1 \tabularnewline
	RFO5.2.1.1.1.2 & Obbligatorio	& L'utente inserisce l'ora di inizio. & Capitolato \\ UC10.6.1.2 \tabularnewline
	RFO5.2.1.1.1.3 & Obbligatorio	& L'utente inserisce l'ora di fine. & Capitolato \\ UC10.6.1.3 \tabularnewline
	RFO5.2.1.1.2 & Obbligatorio	& L'utente decide di rendere disponibile il veicolo per più giorni ripetuti. & Capitolato \\ UC10.6.1 \tabularnewline
	RFO5.2.1.1.2.1 & Obbligatorio	& L'utente inserisce l'ora di inizio. & Capitolato \\ UC10.6.1.2 \tabularnewline
	RFO5.2.1.1.2.2 & Obbligatorio	& L'utente inserisce l'ora di fine. & Capitolato \\ UC10.6.1.3 \tabularnewline
	RFO5.2.1.1.2.3 & Obbligatorio	& L'utente definisce quali giorni della settimana deve essere ripetuta la disponibilità. & Capitolato \\ UC10.6.1.4 \tabularnewline
	RFO5.2.1.1.2.4 & Obbligatorio	& L'utente definisce per quanto tempo deve essere ripetuta la disponibilità. & Capitolato \\ UC10.6.1.5 \tabularnewline
	RFO5.2.2	&	Obbligatorio	& L'utente visualizza le statistiche del proprio veicolo.	&	Capitolato \\ UC10.4	\tabularnewline
	RFO5.2.3	&	Obbligatorio	& L'utente può rimuove il veicolo.	&	Capitolato \\ UC10.5	\tabularnewline
	RFO6	&	Obbligatorio	&	L'utente può gestire le prenotazioni.	&	Capitolato \\ UC11	\tabularnewline
	RFO6.1	&	Obbligatorio	&	L'utente può visualizzare i dettagli delle prenotazioni attive.	&	Capitolato \\ UC11.1	\tabularnewline
	RFO6.2	&	Obbligatorio	&	L'utente proprietario del veicolo può confermare o annullare una richiesta di prenotazione.	&	Capitolato \\ UC11.2	\tabularnewline
	RFO6.3	&	Obbligatorio	&	L'utente proprietario del veicolo può confermare l'avvenuta consegna delle chiavi del veicolo.	&	Capitolato \\ UC11.3	\tabularnewline
	RFO6.4	&	Obbligatorio	&	L'utente proprietario del veicolo può chiudere la prenotazione nel momento in cui ha ricevuto il veicolo e le chiavi e recensisce l'utente usufruente.	&	Capitolato \\ UC11.4 \\ UC11.7	\tabularnewline
	RFO6.5	&	Obbligatorio	&	L'utente usufruente del veicolo può annullare una prenotazione in attesa di conferma da parte del proprietario.	&	Capitolato \\ UC11.5	\tabularnewline
	RFO6.6 & Obbligatorio & L'utente usufruente del veicolo può chiudere la prenotazione nel momento in cui riconsegna il veicolo e le chiavi e recensisce il proprietario & Capitolato \\ UC11.6 \\ UC11.7 \tabularnewline	
	RFO7	&	Obbligatorio	&	L'utente può prenotare un veicolo.	&	Capitolato \\ UC12	\tabularnewline
	RFD7.1	&	Desiderabile	&	L'utente può filtrare i veicoli secondo l'inserimento di una data.	&	Interno\\ UC12.1	\tabularnewline
	RFD7.2	&	Desiderabile	&	L'utente può filtrare i veicoli secondo l'inserimento di un'ora di inizio.	&	Interno\\ UC12.2	\tabularnewline
	RFD7.3	&	Desiderabile	&	L'utente può filtrare i veicoli secondo l'inserimento di un'ora di fine.	&	Interno\\ UC12.3	\tabularnewline
	RFD7.4	&	Desiderabile	&	L'utente può filtrare i veicoli secondo l'inserimento di una posizione.	&	Interno\\ UC12.4	\tabularnewline
	RFO7.5	&	Obbligatorio	&	L'utente può selezionare un veicolo tra quelli filtrati e prenotarlo.	&	Capitolato\\ UC12.5	\tabularnewline
	RFD8	&	Desiderabile	&	L'utente può visualizzare uno storico delle sue prenotazioni concluse.		&	Interno \\ UC13	\tabularnewline
	RFO9	&	Obbligatorio	&	L'utente può visualizzare il proprio profilo utente.	&	Capitolato \\ UC14	\tabularnewline
	RFO9.1	&	Obbligatorio	&	L'utente può inserire la propria patente di guida.	&	Capitolato \\ UC14.1	\tabularnewline
	RFO9.1.1	&	Obbligatorio	&	L'utente deve inserire l'immagine della patente fronte e retro.	&	Capitolato \\ UC14.1.3	\tabularnewline
	RFO9.1.2	&	Obbligatorio	&	L'utente deve inserire il numero della patente.	&	Capitolato \\ UC14.1.1	\tabularnewline
	RFO9.1.3	&	Obbligatorio	&	L'utente deve inserire la data di rilascio e di scadenza della patente.	&	Capitolato \\ UC14.1.2	\tabularnewline
	RFO9.2	&	Obbligatorio	&	L'utente può modificare i dati del profilo	&	Capitolato \\ UC14.2	\tabularnewline
	RFO9.2.1	&	Obbligatorio	&	L'utente può modificare il nome.	&	Interno \\ UC14.2.1	\tabularnewline
	RFO9.2.2	&	Obbligatorio	&	L'utente può modificare il cognome.	&	Interno \\ UC14.2.2	\tabularnewline
	RFO9.2.3	&	Obbligatorio	&	L'utente può modificare il numero di telefono. 	&	Interno \\ UC14.2.3	\tabularnewline
	RFO9.2.4	&	Obbligatorio	&	L'utente può modificare l'email.	&	Interno \\ UC14.2.4	\tabularnewline
	RFO9.2.5	&	Obbligatorio	&	L'utente può modificare la data di nascita.	&	Interno \\ UC14.2.5	\tabularnewline
	RFO9.2.6	&	Obbligatorio	&	L'utente può modificare la residenza.	&	Interno  \\ UC14.2.6	\tabularnewline
	RFO9.2.7	&	Obbligatorio	&	L'utente visualizzerà un messaggio d'errore se l'email inserita è scorretta.	&	Interno \\ UC2	\tabularnewline
	RFO9.2.8	&	Obbligatorio	&	L'utente può modificare la password.	&	Interno  \\ UC14.2.7	\tabularnewline
	RFO9.2.8.1	&	Obbligatorio	&	L'utente deve inserire la vecchia password.	&	Interno  \\ UC14.2.7.1	\tabularnewline
	RFO9.2.8.2	&	Obbligatorio	&	L'utente visualizza un messaggio d'errore se la password digitata non corrisponde a quella vecchia.	&	Interno  \\ UC15	\tabularnewline
	RFO9.2.8.3	&	Obbligatorio	&	L'utente deve inserire la nuova password.	&	Interno  \\ UC14.2.7.2	\tabularnewline
	RFO9.3	&	Obbligatorio	&	L'utente può eliminare il proprio account.	&	Interno  \\ UC14.3	\tabularnewline
	RFF10	& Facoltativo &	L'utente autenticato entra nella pagina Gioca dell'applicazione. &	UC16 \tabularnewline
	RFF10.1	& Facoltativo &	L'utente autenticato preme il pulsante della Mileston unlock. &	UC17 \tabularnewline
	RFF10.1.1 &	Facoltativo &	L'utente possiede livelli d'esperienza e visualizza una tabella con i premi per un certo livello d'esperienza. & UC17 \tabularnewline
	RFF10.1.2	& Facoltativo & L'utente ritira i premi per il raggiungimento di un certo livello d'esperienza. & UC17 \tabularnewline
	RFF10.2	& Facoltativo &	L'utente autenticato preme il pulsante della Lucky Spin. &	UC18 \tabularnewline
	RFF10.2.1 & Obbligatorio & L'utente, se ha concluso una prenotazione, può usare la Lucky Spin per ottenere un premio o può ritornarci quando vuole. & UC18 \tabularnewline
	RFF10.2.2	& Obbligatorio & L'utente vince un premio dalla Lucky Spin. & UC18 \tabularnewline
	RFF10.3	& Facoltativo &	L'utente autenticato preme il pulsante della Leaderboard. &	UC19 \tabularnewline
	RFF10.3.1 & Facoltativo & L'utente visualizza una classifica dei migliori utenti. & UC19 \tabularnewline
	RFF10.3.1.1 & Facoltativo & Se l'utente, dopo un periodo prestabilito, arriva tra le prime tre posizioni della classifica riceve un premio.  & UC19 \tabularnewline
	RFF10.4	& Facoltativo &	L'utente autenticato preme il pulsante della Daily Rewards. &	UC21 \tabularnewline
	RFF10.4.1 & Facoltativo & L'utente visualizza una tabella coi premi che può ricevere nell'arco di un mese. & UC21 \tabularnewline
	RFF10.4.1.1 & Facoltativo & L'utente visualizza la tabella coi premi e può prendere solo quello del giorno corrente. & UC21
	\tabularnewline
	RFF11 & Facoltativo & L'utente può inviare un “codice amico” ad un amico per invitarlo ad usare l'applicazione ricevendo punti esperienza & UC20 \\ VE\_3.5
	\tabularnewline
	RFF12 & Facoltativo & Agli utenti autenticati è reso disponibile un Minigioco che consiste in un garage che dà la possibilità di effettuare modifiche ad un'auto. & UC22 
	\tabularnewline
	RFF12.1 & Facoltativo & L'utente può modificare l'auto sulla base delle prestazioni. & UC22
	\tabularnewline
	RFF12.2 & Facoltativo & L'utente può modificare l'auto sulla base dell'estetica. & UC22
	\tabularnewline
	RFF13	&	Facoltativo	&	Fornire un insieme di social activity per creare una coesione degli utenti che usano l'applicazione &	Interno\\
	UC20\\ 	\tabularnewline
	RFF14	&	Facoltativo	& Assicurare una medio/grande vincita nella prima Lucky Spin\glosp per invogliare l'utente ad utilizzarla ancora in futuro &	VE\_3.3\\
	UC18   	\tabularnewline
	RFF15	&	Facoltativo	& Un utilizzo e/o prestito continuo dei mezzi fornisce rewards più rapidamente &	Interno\\UC18
\end{longtable}


\pagebreak
\subsection{Requisiti di qualità}

\rowcolors{2}{pari}{dispari}
\LTcapwidth=\linewidth
\begin{longtable}{ >{\centering}p{0.10\textwidth} >{\centering}p{0.25\textwidth}
		>{\raggedright}p{0.35\textwidth} >{\centering}p{0.14\textwidth}}
	\caption{Tabella dei requisiti di qualità}\\
	\rowcolorhead 
	\textbf{\color{white}Requisito} 
	& \textbf{\color{white}Classificazione} 
	& \centering\textbf{\color{white}Descrizione}
	& \textbf{\color{white}Fonti} 
	\endfirsthead
	\rowcolor{white}\caption[]{(continua)}\\
	\rowcolorhead 
	\textbf{\color{white}Requisito} 
	& \textbf{\color{white}Classificazione} 
	& \centering\textbf{\color{white}Descrizione}
	& \textbf{\color{white}Fonti} 
	\endhead
	RQO1	&	Obbligatorio	&	Viene fornito il manuale utente	&	Capitolato	\tabularnewline
	RQO1.1	&	Obbligatorio	&	Manuale utente in italiano	&	Interno	\tabularnewline
	RQD1.2	&	Desiderabile	&	Manuale utente in inglese	&	Interno	\tabularnewline
	RQO2	&	Obbligatorio	&	Viene fornito il manuale sviluppatore	&	Capitolato	\tabularnewline
	RQO2.1	&	Obbligatorio	&		Manuale sviluppatore in italiano	&	Interno	\tabularnewline
	RQD2.2	&	Desiderabile	&		Manuale sviluppatore in inglese	&	Interno	\tabularnewline
	RQO3	&	Obbligatorio	&		Devono essere rispettati i criteri definiti nel documento \textit{Norme di Progetto v1.0.0}	&	Interno	\tabularnewline
	RQO4	& Obbligatorio	& 	Devono essere rispettati i processi descritti nel documento \textit{Piano di Qualifica v1.0.0} &	Interno \tabularnewline
	RQO5	&	Obbligatorio	&	Il prodotto deve essere pubblicato e versionato in una repository\glosp di Github\glo.	&	Capitolato
	
\end{longtable}
	


\subsection{Requisiti di vincolo}

	\rowcolors{2}{pari}{dispari}
	
	\begin{longtable}{ >{\centering}p{0.10\textwidth} >{\centering}p{0.25\textwidth}
			>{\raggedright}p{0.35\textwidth} >{\centering}p{0.14\textwidth}}
		\caption{Tabella dei requisiti di vincolo}\\
		\rowcolorhead 
		\textbf{\color{white}Requisito} 
		& \textbf{\color{white}Classificazione} 
		& \centering\textbf{\color{white}Descrizione}
		& \textbf{\color{white}Fonti} 
			\endfirsthead
		\rowcolor{white}\caption[]{(continua)}\\
		\rowcolorhead 
		\textbf{\color{white}Requisito} 
		& \textbf{\color{white}Classificazione} 
		& \centering\textbf{\color{white}Descrizione}
		& \textbf{\color{white}Fonti} 
		\endhead	
		
		
R1V1	&	Obbligatorio	&	L'applicazione deve essere sviluppata per Android\glosp 9, Versione API\glo: 28	&	Capitolato	\tabularnewline
R2V1	&	Desiderabile	&	L'applicazione deve essere scritta in Kotlin	&	Interno	\tabularnewline
R1V2	&	Obbligatorio	&	Il prodotto deve essere pubblicato e versionato in una repository di Github\glo.	&	Capitolato
\tabularnewline
R1V3	&	Obbligatorio	&	Setup di una continuos integration\glosp per verificare che ogni release\glosp soddisfi i requisiti nuovi senza regressioni. 	&	Capitolato	\tabularnewline
R2V2	&	Desiderabile	&	Utilizzo della piattaforma Movens\glosp per il servizio di back-end\glo.	&	Capitolato	\tabularnewline
R1V4	&	Obbligatorio	&	Presenza di almeno cinque core-drive\glosp del framework\glosp di Octalisys\glosp per la gamification\glo.	&	Capitolato	\tabularnewline
	\end{longtable}


\subsection{Requisiti prestazionali}

%\rowcolors{2}{pari}{dispari}
%
%\begin{longtable}{ >{\centering}p{0.10\textwidth} >{\centering}p{0.25\textwidth}
%		>{\raggedright}p{0.35\textwidth} >{\centering}p{0.14\textwidth}}
%	\caption{Tabella dei requisiti prestazionali}\\
%	\rowcolorhead 
%	\textbf{\color{white}Requisito} 
%	& \textbf{\color{white}Classificazione} 
%	& \centering\textbf{\color{white}Descrizione}
%	& \textbf{\color{white}Fonti} 
%	\endfirsthead
%	\rowcolor{white}\caption[]{(continua)}\\
%	\rowcolorhead 
%	\textbf{\color{white}Requisito} 
%	& \textbf{\color{white}Classificazione} 
%	& \centering\textbf{\color{white}Descrizione}
%	& \textbf{\color{white}Fonti} 
%	\endhead	
%	
%	
%
%\end{longtable}
bisogna scriverci qualcosa per dire che non abbiamo i requisiti prestazionali

\pagebreak
\subsection{Tracciamento}  
\subsubsection{Fonte - Requisiti}

	\rowcolors{2}{pari}{dispari}
	
	\begin{longtable}{ >{\centering}p{0.5\textwidth}
			>{\centering}p{0.5\textwidth}}
		\caption{Tabella di tracciamento fonte-requisiti}\\
		\rowcolorhead 
		\textbf{\color{white}Fonte}
		& \textbf{\color{white}Requisiti} 
		\tabularnewline 	
		\endfirsthead
		\rowcolor{white}\caption[]{(continua)} \\
		\rowcolorhead 
		\textbf{\color{white}Fonte}
		& \textbf{\color{white}Requisiti} 
		\tabularnewline 
		\endhead
		
		
	
		
		
Capitolato	&
RFO2 \\
RFO3 \\
RFO4 \\
RFO5 \\
RFO5.2 \\
RFO5.2.1 \\
RFO6 \\
RFO6.1 \\
RFO6.2 \\
RFO7 \\
RFO7.6 \\
RFO7.7 \\
RFO9 \\
RFO9.1 \\
RFO9.1.1 \\
RFO9.1.2 \\
RFO9.1.3 \\
RFO9.1.4 \\
RFO9.2 \\
RQO1 \\
RQO2 \\
RQO5 \\
RVO3 \\
RVD4 \\
RVO5 
\tabularnewline  \rowcolorlight
Interno	&	 RFD1 \\
RFO2.1 \\
RFO2.2 \\
RFO2.3 \\
RFO2.4 \\
RFO2.5 \\
RFO2.7 \\
RFO3.1 \\
RFO3.2 \\
RFO3.3 \\
RFO3.4 \\
RFO3.4.1 \\
RFO5.1 \\
RFO5.1.1 \\
RFO5.1.2 \\
RFO5.1.3 \\
RFO5.1.4 \\
RFD7.1 \\
RFD7.2 \\
RFD7.3 \\
RFD7.4 \\
RFD7.5 \\
RFD8 \\
RFO9.2.1 \\
RFO9.2.2 \\
RFO9.2.3 \\
RFO9.2.4 \\
RFO9.2.5 \\
RFO9.2.6 \\
RFO9.2.7 \\
RFO9.2.8 \\
RFO9.2.8.1 \\
RFO9.2.8.2 \\
RFO9.2.8.3 \\
RFO9.3 \\
RFF17 \\
RFF19 \\
RQO1.1 \\
RQD1.2 \\
RQO2.1 \\
RQD2.2 \\
RQO3 \\
RQO4 \\ \tabularnewline 
UC1	&	RFO2	\tabularnewline
UC1.1.1 & RFO2.1 \tabularnewline
UC1.1.2 & RFO2.2 \tabularnewline
UC1.1.3 & RFO2.3 \tabularnewline
UC1.1.4 & RFO2.4 \tabularnewline
UC2 & RFO9.2.7 \tabularnewline
UC4 & RFD1 \tabularnewline
UC5 & RFO2.7 \tabularnewline
UC6 & RFO3 \tabularnewline
UC6.1.1 & RFO3.1 \tabularnewline
UC6.1.2 & RFO3.2 \tabularnewline
UC7 & RFO3.3 \tabularnewline
UC8 & RFO3.4 \\ RFO3.4.1 \tabularnewline
UC9 & RFO4 \tabularnewline
UC10 & RFO5 \tabularnewline
UC10.1 & RFO5.1 \tabularnewline
UC10.1.1 & RFO5.1.1 \tabularnewline
UC10.1.2 & RFO5.1.2 \tabularnewline
UC10.1.3 & RFO5.1.3 \tabularnewline
UC10.1.4 & RFO5.1.4 \tabularnewline
UC10.2 & RFO5.2 \tabularnewline
UC10.2.1 & RFO5.2.1 \tabularnewline
UC11 & RFO6 \tabularnewline
UC11.1 & RFO6.1 \tabularnewline
UC11.2 & RFO6.2 \tabularnewline
UC11.3 & RFO6.3 \tabularnewline
UC11.3.1 & RFO6.3.1 \tabularnewline
UC11.3.2 & RFO6.3.2 \tabularnewline
UC11.4 & RFO6.4 \tabularnewline
UC11.5 & RFO6.5 \tabularnewline
UC11.6 & RFO6.4 \\ RFO6.5 \tabularnewline
UC12 & RFO7 \tabularnewline
UC12.1 & RFD7.1 \tabularnewline
UC12.2 & RFD7.2 \tabularnewline
UC12.3 & RFD7.3 \tabularnewline
UC12.4 & RFD7.4 \tabularnewline
UC12.5 & RFD7.5 \tabularnewline
UC12.6 & RFD7.6 \tabularnewline
UC12.7 & RFO7.7 \tabularnewline
UC13 & RFD8 \tabularnewline
UC14 & RFO9 \tabularnewline
UC14.1 & RFO9.1 \tabularnewline
UC14.1.1 & RFO9.1.1 \tabularnewline
UC14.1.2 & RFO9.1.2 \tabularnewline
UC14.1.3 & RFO9.1.3 \tabularnewline
UC14.1.4 & RFO9.1.4 \tabularnewline
UC14.2 & RFO9.2 \tabularnewline
UC14.2.1 & RFO9.2.1 \tabularnewline
UC14.2.2 & RFO9.2.2 \tabularnewline
UC14.2.3 & RFO9.2.3 \tabularnewline
UC14.2.4 & RFO9.2.4 \tabularnewline
UC14.2.5 & RFO9.2.5 \tabularnewline
UC14.2.6 & RFO9.2.6 \tabularnewline
UC14.2.7 & RFO9.2.8 \tabularnewline
UC14.2.7.1 & RFO9.2.8.1 \tabularnewline
UC14.2.7.2 & RFO9.2.8.3 \tabularnewline
UC14.3 & RFO9.3 \tabularnewline
UC15 & RFO9.2.8.2 \tabularnewline
UC16 & RFF10 \tabularnewline
UC17 & RFF10.1 \tabularnewline
UC18 & RFF11 \tabularnewline
UC19 & RFF11.1 \tabularnewline
UC20 & RFO12 \tabularnewline
UC21 & RFO12.1 \tabularnewline
UC22 & RFF13 \tabularnewline
UC22 & RFF13.1 \tabularnewline
UC23 & RFF14 \tabularnewline
UC24 & RFF15 \tabularnewline
UC25 & RFF15.1 \tabularnewline
UC26 & RFF16 \tabularnewline
UC26.1 & RFF16.1 \tabularnewline
UC26.1.1 & RFF16.1.1 \tabularnewline
UC26.1.2 & RFF16.1.2 \tabularnewline
UC26.1.3 & RFF16.1.3 \tabularnewline
UC26.1.4 & RFF16.1.4 \tabularnewline
UC26.1.5 & RFF16.1.5 \tabularnewline
UC26.2 & RFF16.2 \tabularnewline
UC26.2.1 & RFF16.2.1 \tabularnewline
UC26.2.2 & RFF16.2.2 \tabularnewline
UC26.2.3 & RFF16.2.3 \tabularnewline
UC26.2.4 & RFF16.2.4 \tabularnewline
UC26.2.5 & RFF16.2.5 \tabularnewline
UC26.2.6 & RFF16.2.6 \tabularnewline
UC26.2.7 & RFF16.2.7 \tabularnewline
UC27 & RFF16.3 \tabularnewline
UC23 & RFF17 \tabularnewline
UC20 & RFF18 \tabularnewline
UC18 & RFF19 \tabularnewline
VE\_1.1 & RVD2 \tabularnewline
VE\_2.5 & RVO1 \tabularnewline
VE\_3.3 & RFF18 \tabularnewline
VE\_3.4 & RFD1 \tabularnewline
VE\_3.5 & RFF14 \tabularnewline
	\end{longtable}

 \pagebreak
\subsubsection{Requisito - fonti}

	\rowcolors{2}{pari}{dispari}
	
\begin{longtable}{ >{\centering}p{0.5\textwidth}
		>{\centering}p{0.5\textwidth}}
	
	\caption{Tabella tracciamento requisito-fonti}\\
	\rowcolorhead 
	\textbf{\color{white}Requisito}
	& \textbf{\color{white}Fonti} 
	\tabularnewline 
	\endfirsthead
	\rowcolor{white}\caption[]{(continua)}\\	
	\rowcolorhead 
	\textbf{\color{white}Requisito}
	& \textbf{\color{white}Fonti} 
	\tabularnewline 
	\endhead
	
RFD1	&	Interno\\  UC4 \\ VE\_3.4 \tabularnewline
RFO2	&	Capitolato \\ UC1	\tabularnewline
RFO2.1	&	Interno \\ UC1.1.1	\tabularnewline
RFO2.2	&	Interno \\ UC1.1.2	\tabularnewline
RFO2.3	&	Interno \\ UC1.1.3	\tabularnewline
RFO2.4	&	Interno \\ UC1.1.4	\tabularnewline
RFO2.5	&	Interno \\ UC3	\tabularnewline
RFF2.6 & UC1.1.5 \tabularnewline
RFO2.7 & Interno \\ UC5 \tabularnewline
RFO3	&	Capitolato \\ UC6	\tabularnewline
RFO3.1	&	Interno \\ UC6.1.1	\tabularnewline
RFO3.2	&	Interno \\ UC6.1.2	\tabularnewline
RFO3.3	&	Interno \\ UC7	\tabularnewline
RFO3.4	&	Interno \\ UC8	\tabularnewline
RFO3.4.1	&	Interno \\ UC8	\tabularnewline
RFO4	&	Capitolato \\ UC9	\tabularnewline
RFO5	&	Capitolato \\ UC10	\tabularnewline
RFO5.1	&	Capitolato \\ UC11	\tabularnewline
RFO5.1.1	&	Interno \\ UC11.1	\tabularnewline
RFO5.1.1.1	&	Interno \\ UC11.1.1	\tabularnewline
RFO5.1.1.2	&	Interno \\ UC11.1.2	\tabularnewline
RFO5.1.1.3	&	Interno \\ UC11.1.3	\tabularnewline
RFO5.1.1.4	&	Interno \\ UC11.1.4	\tabularnewline
RFO5.1.1.5	&	Interno \\ UC11.1.5	\tabularnewline
RFO5.1.1.6	&	Interno \\ UC11.1.6	\tabularnewline
RFO5.1.2	&	Capitolato \\ UC11.2	\tabularnewline
RFO5.1.2.1	&	Capitolato \\ UC11.3	\tabularnewline
RFO5.1.2.1.1	&	Capitolato \\ UC11.3.1	\tabularnewline
RFO5.1.2.1.1.1	 &	 Capitolato \\ UC11.3.1 \tabularnewline
RFO5.1.2.1.1.1.1	 &	 Capitolato \\ UC11.3.1.1 \tabularnewline
RFO5.1.2.1.1.1.2	 &	 Capitolato \\ UC11.3.1.2 \tabularnewline
RFO5.1.2.1.1.1.3	 &	 Capitolato \\ UC11.3.1.3 \tabularnewline
RFO5.1.2.1.1.2	 &	 Capitolato \\ UC11.3.1 \tabularnewline
RFO5.1.2.1.1.2.1	 &	 Capitolato \\ UC11.3.1.2 \tabularnewline
RFO5.1.2.1.1.2.2	 &	 Capitolato \\ UC11.3.1.3 \tabularnewline
RFO5.1.2.1.1.2.3	 &	 Capitolato \\ UC11.3.1.4 \tabularnewline
RFO5.1.2.1.1.2.4	 &	 Capitolato \\ UC11.3.1.5 \tabularnewline
RFO5.1.2.2	&	Capitolato \\ UC11.4	\tabularnewline
RFO5.1.2.3	&	Capitolato \\ UC11.5	\tabularnewline
RFO5.2	&	Capitolato \\ UC12	\tabularnewline
RFO5.2.1	&	Capitolato \\ UC12.1	\tabularnewline
RFO5.2.2	&	Capitolato \\ UC12.2	\tabularnewline
RFO5.2.3	&	Capitolato \\ UC12.3	\tabularnewline
RFO5.2.4	&	Capitolato \\ UC12.4 \\ UC12.7	\tabularnewline
RFO5.2.5	&	Capitolato \\ UC12.5	\tabularnewline
RFO5.2.6 & Capitolato \\ UC12.6 \\ UC12.7 \tabularnewline	
RFO5.3	&	Capitolato \\ UC13	\tabularnewline
RFD5.3.1	&	Interno\\ UC13.1	\tabularnewline
RFD5.3.2	&	Interno\\ UC13.2	\tabularnewline
RFD5.3.3	&	Interno\\ UC13.3	\tabularnewline
RFD5.3.4	&	Interno\\ UC13.4	\tabularnewline
RFO5.3.5	&	Capitolato\\ UC13.5	\tabularnewline
RFD6	&	Interno \\ UC14	\tabularnewline
RFO7	&	Capitolato \\ UC15	\tabularnewline
RFO7.1	&	Capitolato \\ UC15.1	\tabularnewline
RFO7.1.1	&	Capitolato \\ UC15.1.3	\tabularnewline
RFO7.1.2	&	Capitolato \\ UC15.1.1	\tabularnewline
RFO7.1.3	&	Capitolato \\ UC15.1.2	\tabularnewline
RFO7.2	&	Capitolato \\ UC15.2	\tabularnewline
RFO7.2.1	&	Interno \\ UC15.2.1	\tabularnewline
RFO7.2.2	&	Interno \\ UC15.2.2	\tabularnewline
RFO7.2.3	&	Interno \\ UC15.2.3	\tabularnewline
RFO7.2.4	&	Interno \\ UC15.2.4	\tabularnewline
RFO7.2.5	&	Interno \\ UC15.2.5	\tabularnewline
RFO7.2.6	&	Interno  \\ UC15.2.6	\tabularnewline
RFO7.2.7	&	Interno \\ UC2	\tabularnewline
RFO7.2.8	&	Interno  \\ UC15.2.7	\tabularnewline
RFO7.2.8.1	&	Interno  \\ UC15.2.7.1	\tabularnewline
RFO7.2.8.2	&	Interno  \\ UC16	\tabularnewline
RFO7.2.8.3	&	Interno  \\ UC15.2.7.2	\tabularnewline
RFO7.3	&	Interno  \\ UC15.3	\tabularnewline
RFF8 &	UC17 \tabularnewline
RFF8.1	& UC18 \tabularnewline
RFF9	& UC19 \tabularnewline
RFF9.1	& UC20 \tabularnewline
RFF9.1.1 &	UC20 \tabularnewline
RFF9.1.2	& UC21 \tabularnewline
RFF9.2	& UC22 \tabularnewline
RFF9.2.1 & UC22 \tabularnewline
RFF9.2.2	& UC213 \tabularnewline
RFF9.3	& UC24 \tabularnewline
RFF9.3.1 & UC24 \tabularnewline
RFF9.3.1.1 & UC24 \tabularnewline
RFF9.4	& UC26 \tabularnewline
RFF9.4.1 	& 	UC24 \tabularnewline
RFF9.4.1.1 	& 	UC25 \tabularnewline
RFF10 	& 	UC25 \\ VE\_3.5 \tabularnewline
RFF11	 &	 	UC28  \tabularnewline
RFF11.1		 & 		UC28.1 \tabularnewline
RFF11.1.1	 &	 UC281.1 \tabularnewline
RFF11.1.2	 &	 	UC28.1.2 \tabularnewline
RFF11.1.3	 &	 UC28.1.3 \tabularnewline
RFF11.1.4	 &	 UC28.1.4 \tabularnewline
RFF11.1.5	 &	 UC28.1.5 \tabularnewline
RFF11.2	 &	 UC28.2 \tabularnewline
RFF11.2.1	 &	 UC28.2.1 \tabularnewline
RFF11.2.2	 &	 UC28.2.2 \tabularnewline
RFF11.2.3	 &	 UC28.2.3 \tabularnewline
RFF11.2.4	 &	 UC28.2.4 \tabularnewline
RFF11.2.5	 &	 UC28.2.5 \tabularnewline
RFF11.2.6	 &	 UC28.2.6 \tabularnewline
RFF11.2.7	 &	 UC28.2.7 \tabularnewline
RFF12	&	Interno\\ UC25\\ 	\tabularnewline
RFF13	&	VE\_3.3\\ UC22   	\tabularnewline
RFF14	&	Interno\\UC20 \tabularnewline
RQO1	&	Capitolato \tabularnewline
RQO1.1	&	Interno \tabularnewline
RQD1.2	&	Interno \tabularnewline
RQO2	&	Capitolato \tabularnewline
RQO2.1	&	Interno \tabularnewline
RQD2.2	&	Interno \tabularnewline
RQO3	&	Interno \tabularnewline
RQO4	&	Interno \tabularnewline
RVO1	&	Interno \tabularnewline
RVD2	&	Interno \tabularnewline
RVO3	&	Capitolato \tabularnewline
RVO4	&	Capitolato \tabularnewline
RVD5	&	Capitolato \tabularnewline
RVO6	&	Capitolato \tabularnewline
		
\end{longtable}




\subsection{Considerazioni}
I requisiti potranno subire delle variazioni in futuro, per apportare degli aggiornamenti alle voci presenti o delle migliorie. Nel caso in cui le attività pianificate terminassero prima del previsto, e dovessero avanzare delle ore di lavoro, potranno essere presi in carico nuovi requisiti per aggiungere del valore al prodotto. Dunque eventuali espansioni sono lasciate a momenti futuri. 