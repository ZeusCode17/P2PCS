\section{Librerie esterne}
In questa sezione del manuale verranno elencate e brevemente descritte le librerie esterne utilizzate nel progetto.
\subsection{Codifica}
\begin{itemize}
	\item \textbf{Picasso}: libreria che permette di scaricare immagini tramite URL e di inserirle in un'ImageView\glosp in una sola linea di codice. Si occupa inoltre di gestire in automatico la memoria e il caching, cancellando il file scaricato al termine del suo utilizzo.
	\item \textbf{Places Auto Complete}: libreria che fornisce un EditText\glosp con auto completamento già integrato per la ricerca di luoghi.
	\item \textbf{Places}: libreria di Google\glosp che fornisce i metodi per la geo codifica dei luoghi.
\end{itemize}
\subsection{Test}
\begin{itemize}
	\item \textbf{KotlinTest}: libreria che fornisce strumenti per il testing in Kotlin\glo;
	\item \textbf{Firebase Test Lab}: strumento esterno fornito che permette di eseguire test d'integrazione e test di sistema;
	\item \textbf{Espresso}: libreria per il testing della UI.
\end{itemize}