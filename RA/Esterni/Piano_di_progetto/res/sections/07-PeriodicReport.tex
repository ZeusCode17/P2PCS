\section{Consuntivi di periodo}
Di seguito verranno indicate le spese effettivamente sostenute, considerando sia quelle per ruolo sia quelle per persona. Il bilancio potrà risultare:
\begin{itemize}
	\item \textbf{Positivo:} se il preventivo supera il consuntivo;
	\item \textbf{Pari:} se il consuntivo e il preventivo sono pari;
	\item \textbf{Negativo:} se il consuntivo supera il preventivo.
\end{itemize}

\subsection{Periodo di analisi}
Le ore di lavoro sostenute in questa fase sono da considerarsi come ore di investimento per l'approfondimento personale. Esse sono quindi non rendicontate.

\begin{table}[H]
				\centering\renewcommand{\arraystretch}{1.5}
				\caption{Consuntivo di periodo della fase di analisi}
				\vspace{0.2cm}
                \begin{tabular}{c c c}
                               
                \rowcolorhead
                 {\colorhead \textbf{Ruolo}} &
                 {\colorhead \textbf{Ore}} & 
                 {\colorhead \textbf{Costo}} \\
				
                \rowcolorlight
                 {\colorbody Responsabile} & {\colorbody 38 (+2)} & 
                 {\colorbody \EUR{1.140,00} (+\EUR{60,00})}  
				\\
				
				\rowcolordark
                 {\colorbody Amministratore} & {\colorbody 25 (+7)} & 
                 {\colorbody \EUR{500,00} (+\EUR{140,00})}
				\\	
				
				\rowcolorlight
                 {\colorbody Analista} & {\colorbody 71 (-4)} & 
                 {\colorbody \EUR{1.775,00} (-\EUR{100,00})} 
				\\
				
				\rowcolordark
                 {\colorbody Progettista} & {\colorbody 19
                 (+0)} & 
                 {\colorbody \EUR{418,00} (+\EUR{0,00})} 
				\\
				
				\rowcolorlight
                 {\colorbody Programmatore} & {\colorbody -} & 
                 {\colorbody -} 
				\\
				
				\rowcolordark
                 {\colorbody Verificatore} & {\colorbody 57 (+3)} & 
                 {\colorbody \EUR{855,00} (+\EUR{45,00})} 
				\\
				
				\rowcolorlight
                 {\colorbody \textbf{Totale Preventivo}} & {\colorbody \textbf{210}} & 
                 {\colorbody \textbf{\EUR{4.688,00}}} 
				\\
				
				
				\rowcolordark
                 {\colorbody \textbf{Totale Consuntivo}} & {\colorbody \textbf{218}} & 
                 {\colorbody \textbf{\EUR{4.833,00}}} 
				\\
				
				
				\rowcolorlight
                 {\colorbody \textbf{Differenza}} & {\colorbody \textbf{8}} & 
                 {\colorbody \textbf{\EUR{+145,00}}} 
				\\
				
                

                \end{tabular}
                
\end{table}

\subsubsection{Conclusioni}
Come emerge dai dati riportati nella tabella soprastante, che presenta le ore relative al consuntivo della fase di Analisi, è stato necessario investire più tempo del previsto nei ruoli di \textit{Responsabile}, \textit{Amministratore} e \textit{Verificatore}.  Al contempo però sono risultate sufficienti
un numero inferiore di ore per il ruolo di \textit{Analista}. 
Di seguito sono elencate le cause dei ritardi sopracitati:
\begin{itemize}
	\item \textbf{Amministratori:} la ricerca e configurazione dei software atti alla produzione e alla gestione del progetto ha richiesto più tempo del previsto, in particolare la creazione dei template \LaTeX\space e la configurazione di PragmaDb\glo;\
	\item \textbf{Responsabile:} si è reso necessario un monte ore maggiore per la coordinazione generale del progetto, causato dall'inesperienza dei membri del gruppo; 
	\item \textbf{Verificatore:} a causa dell'inesperienza dei membri del gruppo si sono verificati diversi errori e mancanze durante la stesura della documentazione che hanno richiesto un maggiore monte ore di verifica;
	\item \textbf{Analista:} al contrario di quanto preventivato il totale delle ore di analisi è risultato inferiore, questo grazie alla facile comprensione dei requisiti richiesti che ha permesso una rapida stesura di quest'ultimi.
\end{itemize}

\subsubsection{Preventivo a finire}
Il risultato del periodo è complessivamente di 8 ore lavorative oltre il previsto e di una
spesa aggiunta di \EUR{+145,00}, che però facendo parte del periodo di investimento
non influirà sul totale rendicontato.

\subsection{Periodo di consolidamento dei requisiti}
Le ore di lavoro sostenute durante questo periodo sono successive alla fase di analisi. Diverse ore sono state impiegate per lo studio personale delle tecnologie che andremo ad utilizzare e per questo non vengono riportate nella tabella sottostante e non sono rendicontate.


\begin{table}[H]
	\centering\renewcommand{\arraystretch}{1.5}
	\caption{Consuntivo di periodo della fase di consolidamento dei requisiti}
	\vspace{0.2cm}
	\begin{tabular}{c c c}
		
		\rowcolorhead
		{\colorhead \textbf{Ruolo}} &
		{\colorhead \textbf{Ore}} & 
		{\colorhead \textbf{Costo}} \\
		
		\rowcolorlight
		{\colorbody Responsabile} & {\colorbody 5 (+0)} & 
		{\colorbody \EUR{150,00} (+\EUR{0,00})}  
		\\
		
		\rowcolordark
		{\colorbody Amministratore} & {\colorbody 3 (+0)} & 
		{\colorbody \EUR{60,00} (+\EUR{0,00})}
		\\	
		
		\rowcolorlight
		{\colorbody Analista} & {\colorbody 12 (+0)} & 
		{\colorbody \EUR{300,00} (+\EUR{0,00})} 
		\\
		
		\rowcolordark
		{\colorbody Progettista} & {\colorbody -} & 
		{\colorbody \EUR{0,00} (+\EUR{0,00})} 
		\\
		
		\rowcolorlight
		{\colorbody Programmatore} & {\colorbody -} & 
		{\colorbody -} 
		\\
		
		\rowcolordark
		{\colorbody Verificatore} & {\colorbody 10 (+0)} & 
		{\colorbody \EUR{150,00} (+\EUR{0,00})} 
		\\
		
		\rowcolorlight
		{\colorbody \textbf{Totale Preventivo}} & {\colorbody \textbf{30}} & 
		{\colorbody \textbf{\EUR{660,00}}} 
		\\
		
		
		\rowcolordark
		{\colorbody \textbf{Totale Consuntivo}} & {\colorbody \textbf{30}} & 
		{\colorbody \textbf{\EUR{660,00}}} 
		\\
		
		
		\rowcolorlight
		{\colorbody \textbf{Differenza}} & {\colorbody -} & 
		{\colorbody -} 
		\\
		
		
		
	\end{tabular}
	
\end{table}

\subsubsection{Conclusioni}
Come emerge dai dati riportati nella tabella soprastante, che presenta le ore relative al consuntivo della fase di Consolidamento dei requisiti, è stato rispettato il monte ore preventivato. Questo grazie all'esperienza acquisita durante la fase precedente e alla breve durata del periodo, che ha permesso una migliore gestione delle risorse.

\subsubsection{Preventivo a finire}
Il risultato del consuntivo di periodo coincide col monte ore preventivato inoltre, facendo parte del periodo rendicontato, non è necessario eseguire nessuna modifica o accorgimenti ai futuri periodo o al preventivo.

\subsection{Periodo di progettazione e codifica per la Technology Baseline}
Le ore di lavoro sostenute durante questo periodo sono successive alla fase di analisi. Diverse ore sono state impiegate per lo studio personale delle tecnologie che andremo ad utilizzare e per questo non vengono riportate nella tabella sottostante e non sono rendicontate.


\begin{table}[H]
	\centering\renewcommand{\arraystretch}{1.5}
	\caption{Consuntivo di periodo della fase di progettazione e codifica per la Technology Baseline}
	\vspace{0.2cm}
	\begin{tabular}{c c c}
		
		\rowcolorhead
		{\colorhead \textbf{Ruolo}} &
		{\colorhead \textbf{Ore}} & 
		{\colorhead \textbf{Costo}} \\
		
		\rowcolorlight
		{\colorbody Responsabile} & {\colorbody 10 (+0)} & 
		{\colorbody \EUR{300,00} (+\EUR{0,00})}  
		\\
		
		\rowcolordark
		{\colorbody Amministratore} & {\colorbody 17 (+10)} & 
		{\colorbody \EUR{540,00} (+\EUR{200,00})}
		\\	
		
		\rowcolorlight
		{\colorbody Analista} & {\colorbody 29 (-5)} & 
		{\colorbody \EUR{625,00} (-\EUR{100,00})} 
		\\
		
		\rowcolordark
		{\colorbody Progettista} & {\colorbody 41 (-20)} & 
		{\colorbody \EUR{462,00} (-\EUR{440,00})} 
		\\
		
		\rowcolorlight
		{\colorbody Programmatore} & {\colorbody 26 (+35)} & 
		{\colorbody \EUR{915,00} (+\EUR{525,00})} 
		\\
		
		\rowcolordark
		{\colorbody Verificatore} & {\colorbody 45 (-20)} & 
		{\colorbody \EUR{375,00} (-\EUR{300,00})} 
		\\
		
		\rowcolorlight
		{\colorbody \textbf{Totale Preventivo}} & {\colorbody \textbf{168}} & 
		{\colorbody \textbf{\EUR{3332,00}}} 
		\\
		
		
		\rowcolordark
		{\colorbody \textbf{Totale Consuntivo}} & {\colorbody \textbf{168}} & 
		{\colorbody \textbf{\EUR{3192,00}}} 
		\\
		
		
		\rowcolorlight
		{\colorbody \textbf{Differenza}} & {\colorbody -} & 
		{\colorbody \textbf{+\EUR{140,00}}} 
		\\
		
		
		
	\end{tabular}
	
\end{table}

\subsubsection{Conclusioni}
Come emerge dai dati riportati nella tabella soprastante, che presenta le ore relative al consuntivo della fase di Progettazione e Codifica per la Technology Baseline, la progettazione ha subito una sostanziale modifica. Tale scostamento è dovuto all'idea iniziale di presentare una completa progettazione architetturale del prodotto. Tuttavia ci siamo concentrati maggiormente sul creare delle solide fondamenta per lo sviluppo dell'applicazione attraverso la progettazione e codifica del Proof of Concept\glo. Di seguito sono riportate in dettaglio i vari scostamenti orari dei ruoli interessati:
\begin{itemize}
\item \textbf{Amministratori:} successivamente alla fase di testing dell'editor prestabilito sono insorte svariate problematiche riguardanti il sistema di Continuos Integration\glo, che ha richiesto un notevole monte ore per essere operativo;
\item \textbf{Progettista:} visto il cambiamento dell'obbiettivo finale di questo periodo il monte ore per la fase di progettazione è diminuito in quanto il progettista ha dovuto occuparsi solo di alcune parti del prodotto finale; 
\item \textbf{Verificatore:} a seguito della diminuzione del monte ore della fase di progettazione anche la fase di verifica ha subito una diminuzione del monte ore totali preventivate;
\item \textbf{Programmatore:} al contrario di quanto preventivato il totale delle ore di programmazione ha subito un notevole aumento, ciò è dovuto alla necessità di integrare le modifiche da noi apportate con il codice dell'applicazione già fornito dal proponente. La difficoltà di tale integrazione è scaturita dalla difficoltà di comprensione di alcune parti del codice fornito prive di documentazione e/o semplici commenti.
\end{itemize}

\subsubsection{Preventivo a finire}
Il bilancio economico è positivo, sono stati risparmiati \EUR{140,00}. Tali fondi verranno reinvestiti nelle fasi future per la realizzazione di alcuni requisiti opzionali.


\subsection{Periodo di progettazione di dettaglio e codifica}
Le ore sostenute durante questo periodo sono relative alla redazione alla codifica necessaria per la realizzazione della Product Baseline\glo. Tale periodo è da considerarsi rendicontato in quanto il lavoro è svolto con lo scopo di sviluppare il prodotto finale.


\begin{table}[H]
	\centering\renewcommand{\arraystretch}{1.5}
	\caption{Consuntivo di periodo della fase di progettazione di dettaglio e codifica}
	\vspace{0.2cm}
	\begin{tabular}{c c c}
		
		\rowcolorhead
		{\colorhead \textbf{Ruolo}} &
		{\colorhead \textbf{Ore}} & 
		{\colorhead \textbf{Costo}} \\
		
		\rowcolorlight
		{\colorbody Responsabile} & {\colorbody 16 (+0)} & 
		{\colorbody \EUR{480,00} (+\EUR{0,00})}  
		\\
		
		\rowcolordark
		{\colorbody Amministratore} & {\colorbody 21 (-5)} & 
		{\colorbody \EUR{420,00} (-\EUR{100,00})}
		\\	
		
		\rowcolorlight
		{\colorbody Analista} & {\colorbody 0 (+7)} & 
		{\colorbody \EUR{0,00} (+\EUR{125,00})} 
		\\
		
		\rowcolordark
		{\colorbody Progettista} & {\colorbody 64 (+20)} & 
		{\colorbody \EUR{1408,00} (+\EUR{440,00})} 
		\\
		
		\rowcolorlight
		{\colorbody Programmatore} & {\colorbody 117 (+10)} & 
		{\colorbody \EUR{1755,00} (+\EUR{150,00})} 
		\\
		
		\rowcolordark
		{\colorbody Verificatore} & {\colorbody 82 (+0)} & 
		{\colorbody \EUR{1230,00} (+\EUR{0,00})} 
		\\
		
		\rowcolorlight
		{\colorbody \textbf{Totale Preventivo}} & {\colorbody \textbf{300}} & 
		{\colorbody \textbf{\EUR{5293,00}}} 
		\\
		
		
		\rowcolordark
		{\colorbody \textbf{Totale Consuntivo}} & {\colorbody \textbf{332}} & 
		{\colorbody \textbf{\EUR{6258,00}}} 
		\\
		
		
		\rowcolorlight
		{\colorbody \textbf{Differenza}} & {\colorbody \textbf{32}} & 
		{\colorbody \textbf{+\EUR{965,00}}} 
		\\
		\rowcolordark
		{\colorbody \textbf{Totale con risparmio(-\EUR{140,00})}} & & 
		{\colorbody \textbf{\EUR{825,00}}} 
		\\
		
		
	\end{tabular}
	
\end{table}

\subsubsection{Conclusioni}
Come emerge dai dati riportati nella tabella soprastante, che presenta le ore relative al consuntivo della fase di Progettazione di Dettaglio e Codifica, la progettazione ha subito una sostanziale modifica. Di seguito sono riportati in dettaglio i vari scostamenti orari dei ruoli interessati:
\begin{itemize}
	\item \textbf{Amministratori:} sono stati risolti i probelmi sorti nella fase precedente riguardanti la Continuos Integration\glosp ciò ha portato ad una diminuzione del monte ore preventivato;
	\item \textbf{Progettista:} si è reso necessario un monte ore maggiore per questo ruolo a causa della necessità di individuare i corretti design pattern\glosp da applicare, soprattuto per quanto riguarda il back end\glo. Modifica che ci è stata suggerita a seguito della Product Baseline\glo; 
	\item \textbf{Verificatore:} nonostante la variazione del monte ore da programmatore le ore preventivate sono risultate sufficienti;
	\item \textbf{Programmatore:} al contrario di quanto preventivato il totale delle ore di programmazione ha subito un leggero aumento, a causa della difficoltà di integrare alcune parti riguardanti la Gamification\glo, e di alcune modifiche richieste dalla proponente.
\end{itemize}

Di seguito riportiamo la tabella riguardante l'avanzamento degli incrementi individuati in questa fase. Come si può notare lo sviluppo della maggior parte degli incrementi raggiunge la quasi totalità di completamento. Prendiamo come esempio la Gestione Veicoli, la quale è completa e funzionante ma necessita dell'integrazione della parte di Gamification\glosp per essere ultimata. Quest'ultima riguarderà un altro incremento che verrà sviluppato nella fase successiva.

\counterwithin{table}{section}
\renewcommand{\arraystretch}{1.5}
\rowcolors{2}{dispari}{pari}
\arrayrulecolor{white}
\begin{longtable}{ 
		>{\centering}p{0.17\textwidth} 
		>{\raggedright}p{0.28\textwidth}
		>{\raggedright}p{0.29\textwidth} 
		>{\centering}p{0.15\textwidth}
	}
	
	
	\caption{Tabella del completamento degli incrementi}\\
	\rowcolorhead
	\colorhead\textbf{Incremento} & \centering\colorhead\textbf{Completamento(\%)}
	\tabularnewline
	\endfirsthead
	\rowcolor{white}\caption[]{(continua)}\\
	\rowcolorhead
	\colorhead\textbf{Incremento} & \centering\colorhead\textbf{Completamento(\%)}
	\tabularnewline
	\endhead
	
	%R---------------------------------------------------------
	{Lucky Spin} & \centering 100\\
	\tabularnewline
	
	%R---------------------------------------------------------
	{Daily Rewards} & \centering 100\\
	\tabularnewline
	
	%R---------------------------------------------------------
	{Milestone Unlock} & \centering 100\\
	\tabularnewline
	%R---------------------------------------------------------
	{Leaderboard} & \centering 100\\
	\tabularnewline
	%R---------------------------------------------------------
	{Codice Invita Amici}\\ & \centering NI\\
	\tabularnewline
	%R---------------------------------------------------------
	{Visualizzazione Guida Introduttiva}\\ & \centering 50\\
	\tabularnewline
	%R---------------------------------------------------------
	{Minigioco} & \centering 50\\
	\tabularnewline
	%R---------------------------------------------------------
	{Gestione Veicoli}\\ & \centering 90\\
	\tabularnewline
	%R---------------------------------------------------------
	{Gestione Prenotazioni}\\ & \centering 90\\
	\tabularnewline
	
	%R---------------------------------------------------------
	{Storico Prenotazioni}\\ & \centering 90\\
	\tabularnewline
	%R---------------------------------------------------------
	{Progress Bar}\\ & \centering 100\\
	\tabularnewline
	
	
	
\end{longtable}
\counterwithin{table}{subsection}	
\renewcommand{\arraystretch}{1}
\pagebreak

\subsubsection{Preventivo a finire}
Il bilancio economico è negativo, il costo è di \EUR{825,00}. Oltre a questo scostamento dal preventivo iniziale, nella fase seguente, si terrà conto anche del monte ore richiesto in aggiunta. Si prevede di recuperare il debito grazie alla rivisitazione della progettazione effettuata che ci consentirà di risparmiare ore nella fase di testing e codifica in generale che attualmente si trova ad uno stato avanzato. 



\subsection{Periodo di validazione e collaudo}
Le ore di lavoro sostenute durante questo periodo sono successive alla fase di dettaglio e codifica. Il monte ore impiegato in questa fase è relativo alla codifica necessaria per il completamento del prodotto finale e alla sua verifica tramite l'utilizzo dei test. Tale periodo è da considerarsi rendicontato in quanto il lavoro è svolto con lo scopo di sviluppare il prodotto finale.


\begin{table}[H]
	\centering\renewcommand{\arraystretch}{1.5}
	\caption{Consuntivo di periodo della fase di validazione e collaudo}
	\vspace{0.2cm}
	\begin{tabular}{c c c}
		
		\rowcolorhead
		{\colorhead \textbf{Ruolo}} &
		{\colorhead \textbf{Ore}} & 
		{\colorhead \textbf{Costo}} \\
		
		\rowcolorlight
		{\colorbody Responsabile} & {\colorbody 13 (-10)} & 
		{\colorbody \EUR{390,00} (-\EUR{300,00})}  
		\\
		
		\rowcolordark
		{\colorbody Amministratore} & {\colorbody 11 (-7)} & 
		{\colorbody \EUR{220,00} (-\EUR{140,00})}
		\\	
		
		\rowcolorlight
		{\colorbody Analista} & {\colorbody -(+1)} & 
		{\colorbody \EUR{0,00} (+\EUR{25,00})} 
		\\
		
		\rowcolordark
		{\colorbody Progettista} & {\colorbody 9 (-5)} & 
		{\colorbody \EUR{198,00} (-\EUR{110,00})} 
		\\
		
		\rowcolorlight
		{\colorbody Programmatore} & {\colorbody 35 (+0)} & 
		{\colorbody \EUR{525,00} (+\EUR{0,00})} 
		\\
		
		\rowcolordark
		{\colorbody Verificatore} & {\colorbody 52 (-20)} & 
		{\colorbody \EUR{780,00} (-\EUR{300,00})} 
		\\
		
		\rowcolorlight
		{\colorbody \textbf{Totale Preventivo}} & {\colorbody \textbf{120}} & 
		{\colorbody \textbf{\EUR{2113,00}}} 
		\\
		
		
		\rowcolordark
		{\colorbody \textbf{Totale Consuntivo}} & {\colorbody \textbf{80}} & 
		{\colorbody \textbf{\EUR{1278,00}}} 
		\\
		
		
		\rowcolorlight
		{\colorbody \textbf{Differenza}} & {\colorbody -} & 
		{\colorbody \textbf{-\EUR{825,00}}} 
		\\
		\rowcolordark
		{\colorbody \textbf{Totale con costi(\EUR{825,00})}} & & 
		{\colorbody \textbf{\EUR{0,00}}} 
		\\
		
		
	\end{tabular}
	
\end{table}

\subsubsection{Conclusioni}
Come emerge dai dati riportati nella tabella soprastante, che presenta le ore relative al consuntivo della fase di validazione e collaudo, il ruolo di responsabile ha avuto una diminuzione del monte ore seguito dai ruoli di progettista e verificatore. Tale scostamento è dovuto alla realizzazione quasi completa dell'applicazione nella fase precedente. Di seguito sono riportate in dettaglio i vari scostamenti orari dei ruoli interessati:
\begin{itemize}
	\item \textbf{Progettista:} la buona implementazione della fase precedente non ha avuto aspetti negativi nell'implementazione delle ultime componenti quindi questo ruolo ha subito un calo delle ore; 
	\item \textbf{Verificatore:} la realizzazione dei test è risultata più semplice e veloce da effettuare di quanto preventivato, ciò ha causato una drastica diminuzione del monte ore associate alla verifica del prodotto;
	\item \textbf{Responsabile:} questo ruolo non ha avuto molto margine in questa fase in quanto non si sono resi necessari interventi nella pianificazione generale del gruppo.
\end{itemize}

\subsubsection{Preventivo a finire}
A seguito di un ottima codifica del prodotto eseguita grazie all'impiego di un monte ore maggiore nelle fasi precedenti, sono stati risparmiati \EUR{825,00} ai quali vengono sottratti gli \EUR{825,00} di debito della precedente fase, in questo modo è stato rispettato il preventivo destinato alla proponente. Visto la grande percentuale di requisiti desiderabili e facoltativi implementati e la totalità dei requisiti obbligatori, il gruppo si dichiara soddisfatto del consuntivo a finire. Questo anche a seguito di un incontro con la proponente la quale ha accettato di buon grado l'ultima versione presentata.


\begin{table}[H]
	\centering\renewcommand{\arraystretch}{1.5}
	\caption{Distrinuzione delle ore rendicontate} 
	\vspace{0.2cm}
	\begin{tabular}{c c c c c c c c}
		
		\rowcolorhead
		{\colorhead \textbf{Nominativo}} &
		{\colorhead \textbf{Re}} & 
		{\colorhead \textbf{Am}} & 
		{\colorhead\textbf{An}} & 
		{\colorhead \textbf{Pt}} & 
		{\colorhead\textbf{Pr}} & 
		{\colorhead \textbf{Ve}} & 
		{\colorhead \textbf{Ore totali} }\\
		
		\rowcolorlight
		{\colorbody Riccardo Basso} & {\colorbody 7(-1)} & 
		{\colorbody 10} & {\colorbody 15} & {\colorbody 11} & 
		{\colorbody 30(+1)} & {\colorbody 30} & {\colorbody 103} 
		\\
		
		\rowcolordark
		{\colorbody Marco Dalla Bà} & {\colorbody 11(-4)} & 
		{\colorbody 6(+4)} & {\colorbody 11} & {\colorbody 15} & 
		{\colorbody 40(+15)} & {\colorbody 20(-15)} & {\colorbody 103} 
		\\	
		
		\rowcolorlight
		{\colorbody Riccardo Dario} & {\colorbody 5} & 
		{\colorbody 11} & {\colorbody 10} & {\colorbody 15(-4)} & 
		{\colorbody 30(+1)} & {\colorbody 32(+3)} & {\colorbody 103} 
		\\
		
		\rowcolordark
		{\colorbody Irina Hornoiu} & {\colorbody 8(-1)} & 
		{\colorbody 8} & {\colorbody 9} & {\colorbody 15} & 
		{\colorbody 31(+8)} & {\colorbody 32(-7)} & {\colorbody 103} 
		\\
		
		\rowcolorlight
		{\colorbody Diba Meysamiazad} & {\colorbody 6} & 
		{\colorbody 16} & {\colorbody 10} & {\colorbody 12} & 
		{\colorbody 35(+16)} & {\colorbody 24(-16)} & {\colorbody 103} 
		\\
		
		\rowcolordark
		{\colorbody Andrea Pigatto} & {\colorbody 9(-2)} & 
		{\colorbody 6(+2)} & {\colorbody 15(-1)} & {\colorbody 19(-1)} & 
		{\colorbody 30(+4)} & {\colorbody 24(-2)} & {\colorbody 103} 
		\\	
		
		\rowcolorlight
		{\colorbody \textbf{Ore totali ruolo}} & {\colorbody 46(-8)} & 
		{\colorbody 57(+6)} & {\colorbody 70(-1)} & {\colorbody 87(-5)} & 
		{\colorbody 196(+45)} & {\colorbody 162(-37)} & {\colorbody 618} 
		\\
		
	\end{tabular}             
\end{table}
