
\section{Procedure individuate}

Di seguito vengono presentate un insieme di linee guida che i membri del gruppo seguiranno durante lo sviluppo del progetto.
\subsection{Gestione dei rischi} 
			La rilevazione dei rischi è compito del responsabile di progetto, questa attività è documentata nel \textit{Piano di Progetto}.
	        La procedura da seguire per la gestione dei rischi è la seguente:
			\begin{itemize}
				\item individuare possibili problemi e monitorare i rischi in modo preventivo;
				\item tenere traccia dei singoli rischi con eventuali contromisure nel \textit{Piano di Progetto};
				\item ridefinire e implementare le strategie di gestione dei rischi.
			\end{itemize}
			
			\noindent
			\paragraph{Codifica dei rischi}
				Le tipologie di rischi sono così codificate:
				\begin{itemize}
					\item \textbf{RT}: Rischi Tecnologici;
					\item \textbf{RO}: Rischi Organizzativi;
					\item \textbf{RI}: Rischi Interpersonali.
				\end{itemize}
				
				
\subsection{Gestione delle comunicazioni} 
			\paragraph{Comunicazioni interne} 
			Le comunicazioni interne sono gestite utilizzando il sistema di messaggistica multi piattaforma Slack\glo. Questo servizio implementa la suddivisione in canali e permette quindi la comunicazione tra membri interessati alle singole attività. Ogni membro ha accesso a tutti i canali e partecipa alle conversazioni che sono di suo diretto interesse. I canali presenti sono variabili, ad eccezione di alcuni che saranno presenti dall'inizio alla fine del progetto.
			I canali utilizzati sono:
			\begin{itemize}
				\item \textbf{general:} vengono discusse tutte le informazioni off-topic\glo, include la gestione generale del progetto;
				\item \textbf{git:} vengono comunicate le pull eseguite al termine o in esecuzione di un qualsiasi file;
				\item \textbf{documento "...":} più canali nominati in base al nome del documento in fase di redazione e non ancora approvato. Consiste in un insieme di canali in continua variazione in quanto vengono chiusi dopo l'approvazione del documento relativo;
				\item \textbf{trello\glo}: utilizzato per avere un rapporto diretto con le informazioni riguardanti lo stato della documentazione;
				\item \textbf{spam:} utilizzato per discutere liberamente di ciò che è esterno al progetto.
			\end{itemize}
			%\newline \newline
			\paragraph{Comunicazioni esterne} 
			Le comunicazioni con soggetti esterni al gruppo, quali committente e proponente, sono di competenza del responsabile. Gli strumenti predefiniti sono la posta elettronica, utilizzando l'indirizzo di posta elettronica del gruppo, di cui tutti i membri hanno le credenziali di accesso  \url{zeuscode17@gmail.com}.
			Per le comunicazioni con \textit{Gaiago} si utilizza il servizio Google Meet per le riunioni. In caso di assenza di uno o più membri del gruppo un membro presente, a turno, assume l'onere di creare un riassunto scritto per gli altri membri.
			\newline
			\paragraph{Gestione degli incontri}
			\paragraph*{Incontri interni del team} 
			Le riunioni interne del team sono organizzate dal responsabile in accordo con i membri del gruppo. Viene usato Google Calendar se si deve effettuare un incontro importante ed è necessaria la presenza di tutti i membri, in questo modo ogni membro comunica i giorni liberi e si decide. In caso di incontri di minore importanza si utilizza Telegram per organizzarsi con gli orari.
			\paragraph*{Verbali di riunioni interne} 
			Al termine di ogni riunione viene nominato un segretario incaricato di produrre il \textit{Verbale} nel quale vengono riportate le decisioni prese e le varie idee. 
			\paragraph*{Incontri esterni del team} 
			Gli incontri esterni sono gestiti dal responsabile, il quale ha il compito di organizzare le date e le modalità di incontro relazionandosi terze parti. Vengono usati i canali sopra citati per prendere accordi.
			
			\paragraph*{Verbali di riunioni esterne} 
			Anche per le riunioni esterne viene redatto un \textit{Verbale}, il quale presenta una struttura analoga al verbale redatto per le riunioni interne.  
			\paragraph{Gestione degli strumenti di coordinamento} 
			\paragraph*{Tickecting} 
			Strumento utilizzato per la suddivisione dei compiti all'interno del team. Permette al responsabile di progetto di monitorare le attività in corso e di assegnare le risorse disponibili ad una o più schedule\glo.\newline
			Utile e versatile in quanto presenta anche una controparte mobile accessibile in qualunque momento. Ad ogni task sono associati una data di scadenza e un insieme di membri assegnatari. Ogni compito passa attraverso i seguenti stati:
			\begin{itemize}
				\item da fare;
				\item in lavorazione;
				\item in revisione;
				\item completato.
			\end{itemize}

			\noindent{Il gruppo ha deciso di usare Trello\glosp a discapito di altri tool vista la sua semplicità di utilizzo ed apprendimento.}
	\paragraph{Riepilogo degli strumenti di comunicazione}
		\begin{itemize}
			\item \textbf{Telegram\glo}: strumento di messaggistica utilizzato inizialmente per la gestione del gruppo;
			\item \textbf{Slack\glo}: per la comunicazione interna del team, composta anche da chiamate vocali;
			\item \textbf{Trello\glo}: per suddividere le varie task tra i membri;
			\item \textbf{Git}: sistema di controllo di versionamento;
			\item \textbf{GitHub Desktop}: applicazione desktop che aiuta la gestione del versionamento;
			\item \textbf{GitHub}\glo: per il versionamento e il salvataggio in repository\glosp remota dei file;
			\item \textbf{Mega}: utilizzato per contenere i file che hanno un basso reteo di modifica e un alto rateo di visualizzazione;
			\item \textbf{Google Calendar}: utile per pianificare gli incontri soprattutto in caso di comunicazioni esterne;
			\item \textbf{Google Meet}: servizio che offre la possibilità di fare videoconferenze e chiamate VoIP, utilizzato per parlare con il proponente e per alcuni incontri interni;
		\end{itemize}