\section{Processi organizzativi}
	Questa sezione definisce i seguenti processi del ciclo di vita organizzativo:
	\begin{enumerate}
		\item Processo di gestione;
		\item Gestione dell'infrastruttura;
		\item Processo di formazione.
	\end{enumerate}
	Le attività e i compiti del processo organizzativo definiscono responsabilità. L'organizzazione garantisce che i processi esistano e siano funzionali.
	
	
	\subsection{Processo di gestione}
		\subsubsection{Scopo}

		Lo scopo di questo processo è definire le linee guida che sono raccolte nel documento \textit{Piano di Progetto}. In particolare:
		\begin{itemize}
			\item definire un modello di sviluppo\glosp comune a tutti i membri del gruppo; 
			\item creare un modello organizzativo volto alla prevenzione e correzione di errori;
			\item pianificare il lavoro in base alle scadenze;
			\item comprendere le responsabilità derivanti dai processi.
		\end{itemize}

		\subsubsection{Aspettative}
		Le aspettative del processo sono:
		\begin{itemize}
			\item ottenere una pianificazione efficace delle attività da svolgere;
			\item suddividere per ruoli i membri del gruppo così da poter ricoprire tutte le attività;
			\item garantire un controllo diretto su ogni parte del progetto.
		\end{itemize}
		
		\subsubsection{Descrizione}
		Questo processo consiste delle seguenti attività:
		\begin{itemize}
			\item avvio e definizione dell'obiettivo;
			\item pianificazione;
			\item esecuzione e controllo;
			\item revisione e valutazione;
			\item chiusura.
		\end{itemize}
		\subsubsection{Attività}
		\paragraph{Avvio e definizione dell'obiettivo}
		Questa attività consiste nei seguenti compiti:
		\begin{itemize}
			\item il processo di gestione deve essere avviato stabilendo i requisiti del processo da intraprendere;
			\item una volta stabiliti i requisiti, si deve stabilire la fattibilità del processo controllando che le risorse richieste per l'esecuzione e la gestione del processo siano disponibili, adatte e appropriate e che i tempi desiderati per il completamento sono realizzabili;
			\item se necessario, per accordo tra gli stakeholders\glo, i requisiti del processo possono essere modificati in questo punto per garantire il criterio di completamento.
		\end{itemize}
		\paragraph{Pianificazione}
		Questa attività consiste nei seguenti compiti:
		\begin{itemize}
			\item si devono preparare i piani di esecuzione del processo. I piani associati all'esecuzione del processo devono contenere descrizioni delle attività e compiti associati e  identificazione dei prodotti software che saranno forniti;
			\item la pianificazione deve includere i seguenti:
			\begin{itemize}
				\item pianificazione dell'esecuzione temporale dei compiti;
				\item stima degli sforzi;
				\item risorse necessarie per l'esecuzione dei task;
				\item allocazione dei compiti;
				\item assegnazione delle responsabilità;
				\item quantificazione dei rischi associati ai compiti o al processo stesso;
				\item misure per il controllo della qualità da impiegare durante il processo;
				\item costi associati all'esecuzione del processo;
				\item previsioni sull'ambiente e sull'infrastruttura. 
			\end{itemize}
		\end{itemize}
		\paragraph{Esecuzione e controllo}
		L'attività consiste dei seguenti compiti:
		\begin{itemize}
			\item avvio dell'implementazione dei piani stabiliti per soddisfare gli obiettivi e criteri scelti, ed esercitare il controllo del processo;
			\item i verificatori devono monitorare l'esecuzione del processo, fornendo:
			\begin{itemize}
				\item report interni del progresso del processo;
				\item report esterno per la proponente.
			\end{itemize}
			\item l'amministratore deve investigare, analizzare e risolvere problemi scoperti durante l'esecuzione del processo. La risoluzione dei problemi può comportare cambiamenti alla pianificazione. È responsabilità del responsabile di progetto assicurare che l'impatto dei cambiamenti sia determinato, controllato e monitorato. Problemi nella loro risoluzione devono essere documentati.
		\end{itemize}
		\paragraph{Revisione e valutazione}
		Questa attività consiste nei seguenti compiti:
		\begin{itemize}
			\item gli amministratori devono assicurarsi che il prodotto software e la pianificazione siano valutati per verificare il soddisfacimento dei requisiti;
			\item gli amministratori devono fornire i risultati della valutazione del software, attività e compiti completati durante l'esecuzione del processo.
		\end{itemize}
		\paragraph{Chiusura}
		Questa attività consiste dei seguenti compiti:
		\begin{itemize}
			\item quando tutti i prodotti software, attività e compiti sono completati, i verificatori devono determinare se il processo è stato completato tenendo conto dei criteri specificati nel capitolato e di quelli contrattati con la proponente;
			\item i verificatori devono controllare e documentare il prodotto software, le attività e i compiti assegnati per il completamento.
		\end{itemize}
		
		
		\subsection{Gestione dell'infrastruttura}
		\subsubsection{Scopo}
		Il processo di infrastruttura è un processo per stabilire e mantenere l'infrastruttura necessaria per ogni altro processo. L'infrastruttura può includere hardware, software, strumenti, tecniche, standard e strutture per lo sviluppo, il funzionamento o la manutenzione. 
		
		\subsubsection{Descrizione}
		Il processo consiste nelle seguenti attività:
		\begin{itemize}
			\item implementazione del processo;
			\item istituzione dell'infrastruttura;
			\item manutenzione dell'infrastruttura.
		\end{itemize}
		
		\subsubsection{Attività}
		\paragraph{Implementazione del processo}
		\begin{itemize}
			\item l'infrastruttura deve essere definita e documentata, tenendo conto delle procedure, degli standard, degli strumenti e delle tecniche applicabili;
			\item  la creazione dell'infrastruttura dovrebbe essere pianificata e documentata.
		\end{itemize}
		\paragraph{Istituzione dell'infrastruttura}
		\begin{itemize}
			\item la configurazione dell'infrastruttura deve essere pianificata e documentata. È necessario prendere in considerazione funzionalità, prestazioni, sicurezza, disponibilità, requisiti di spazio, attrezzature, costi e limiti di tempo;
			\item l'infrastruttura deve essere installata in tempo utile per l'esecuzione dei processi rilevanti.
		\end{itemize}
		\paragraph{Mantenimento dell'infrastruttura}
		\begin{itemize}
			\item l'infrastruttura deve essere mantenuta, monitorata e modificata secondo necessità per garantire che continui a soddisfare i requisiti dei processi che impiegano questo processo. 
		\end{itemize}
		
			
			
		
	
		
		
		
		
		\subsection{Processo di formazione}
		
	Il processo di formazione è un processo per fornire e mantenere personale qualificato. L'acquisizione, la fornitura e lo sviluppo  di prodotti software dipende in gran parte da personale esperto e competente. 
	\subsubsection{Scopo}
	Lo scopo di questo processo è assicurare che ogni membro il gruppo abbia le conoscenze e capacità necessarie per lo svolgimento dei compiti assegnatigli. Nell'eventualità in cui un membro ritenga di non avere informazioni e competenze sufficienti allo svolgimento di un task destinatogli, ha il compito di segnalarlo al \textit{Responsabile di Progetto} il quale dovrà poi disporre le attività di apprendimento.\\
	\subsubsection{Aspettative}
	Ogni membro del gruppo provvederà in modo autonomo allo studio individuale delle tecnologie che verranno utilizzate nel corso delle varie fasi del progetto al fine di promuovere un miglioramento continuo. Attraverso il versionamento dei prodotti inoltre si mira all'integrazione delle conoscenze tra i vari componenti del gruppo, migliorando quindi qualità ed efficienza nello svolgere le attività.
	
		
		\subsubsection{Descrizione}
		Questo processo consiste delle seguenti attività:
		\begin{itemize}
			\item implementazione del processo;
			\item sviluppo del materiale di formazione.
		\end{itemize}
		\subsubsection{Attività}
		\paragraph{Implementazione del processo}
		L'attività consiste dei seguenti compiti:
		\begin{itemize}
			\item È necessario condurre una revisione dei requisiti del progetto per stabilire e prevedere tempestivamente l'acquisizione o lo sviluppo delle risorse e competenze richieste dal personale tecnico e di gestione. Devono essere determinati i tipi e livelli di formazione e le persone che necessitano di formazione. 
		\end{itemize}
		\paragraph{Sviluppo del materiale di formazione}
		L'attività consiste nel seguente compito:
		\begin{itemize}
			\item Dovrebbero essere sviluppati manuali di formazione, compresi i materiali di presentazione utilizzati per fornire formazione;
			\item Ogni componente del gruppo è libero di cercare informazioni aggiuntive non indicate nelle \textit{Norme di Progetto} e, nel caso in cui creda che le informazioni consultate siano utili  alla formazioni degli altri membri, è tenuto a condividerle utilizzando i canali comunicativi indicati nella sezione §4.1.6.
		\end{itemize}
		
		
		