
		\section{Ruoli di progetto}%--------start-------------
		Ogni membro del gruppo ricopre un ruolo che viene assegnato a rotazione, le attività svolte da ogni ruolo sono definite in modo chiaro nel documento \textit{Piano di Progetto}.
		I ruoli che ogni componente è tenuto a rappresentare sono descritti in generale di seguito.
			\subsection{Responsabile di progetto} \mbox{}\\ \mbox{}\\
			Il responsabile di progetto è incaricato di gestire le comunicazioni, fa da referente sia per il committente che per il fornitore.\newline
			Il responsabile si occupa di:
			\begin{itemize}
				\item pianificazione delle attività di progetto;
				\item  gestione e coordinamento tra membri del team;
				\item studio ed analisi dei rischi;
				\item approvare la documentazione.
			\end{itemize}
			\subsection{Amministratore di progetto} \mbox{}\\ \mbox{}\\
			L'amministratore coordina l'ambiente di lavoro, assumendosi la responsabilità di gestire la capacità operativa.\newline
			Egli si fa carico dei seguenti aspetti:
			\begin{itemize}
				\item amministra i servizi di supporto, come documentazione e strumenti;
				\item risolvere problemi legati alla gestione dei processi;
				\item effettua controlli volti alla correzione, verifica, aggiornamento e approvazione della documentazione;
				\item controlla versionamento e configurazione dei prodotti;
				\item redige i documenti \textit{Norme di Progetto} e \textit{Piano di Progetto}.
			\end{itemize}
			\subsection{Analista} \mbox{}\\ \mbox{}\\
			L'analista è la figura incaricata di studiare il problema indicato nel modo più approfondito possibile, così da poter fornire eventuali strumenti e metodologie per affrontarlo. 
			Partecipa per un periodo limitato di tempo, è di grande importanza durante la stesura del documento \textit{Analisi dei Requisiti}.\newline
			Le sue responsabilità sono:
			\begin{itemize}
				\item studio del dominio del problema e della sua complessità;
				\item analisi delle richieste implicite ed esplicite;
				\item redige i documenti \textit{Analisi dei Requisiti} e \textit{Studio di Fattibilità}.
			\end{itemize}
			\subsection{Progettista} \mbox{}\\ \mbox{}\\
			Il progettista ha il compito di trovare una soluzione ai problemi rilevati dall'analista, fornendo aspetti tecnici e tecnologici coerenti.\newline
			Il progettista deve:
			\begin{itemize}
				\item applicare soluzioni note ed ottime;
				\item operare scelte che portino ad una soluzione efficiente rispetto ai requisiti, considerando costi e risorse;
				\item sviluppare l'architettura seguendo un insieme di best practice per ottenere un progetto solido e facilmente mantenibile.
			\end{itemize}
			\subsection{Programmatore} \mbox{}\\ \mbox{}\\
			Il programmatore è la figura responsabile delle attività di codifica e delle componenti necessarie per effettuare le prove di verifica.
			Il programmatore si occupa di:
			\begin{itemize}
				\item implementare le decisioni del Progettista;
				\item scrivere codice che rispetti le metriche predefinite, sia versionamento e documentato;
				\item creare e gestire componenti di supporto per la verifica e validazione del codice;
				\item redige i documenti \textit{Manuale Utente} e \textit{Manuale Sviluppatore}.
			\end{itemize}
			\subsection{Verificatore} \mbox{}\\ \mbox{}\\
			Il verificatore ha il compito di supervisionare il prodotto del lavoro degli altri membri del team, sia esso codice o documentazione. Segue delle linee guida volte al controllo e correzione di errori presenti nel documento \textit{Norme di progetto}, nonché alla propria esperienza e capacità di giudizio.\\
			Il verificatore deve:
			\begin{itemize}
				\item controllare la conformità del prodotto in ogni suo stadio di vita;
				\item segnalare al Responsabile di progetto eventuali problemi causati dalla violazione del documento \textit{Norme di progetto}.
				\item segnalare errori meno importanti all'autore dell'oggetto in questione per un eventuale correzione.
							\end{itemize}
		